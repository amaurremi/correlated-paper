\documentclass[runningheads,a4paper]{llncs}

\usepackage{style/packages} 

\newcommand{\keywords}[1]{\par\addvspace\baselineskip
\noindent\keywordname\enspace\ignorespaces#1}

\begin{document}

% inline todo
\newcommand{\comment}[1]{\todo[size=\tiny,noline,backgroundcolor=-red!75!green!50!blue]{#1}}
% side-note todo
\newcommand{\TODO}[1]{\todo[inline,backgroundcolor=Peach]{#1}}

% brackets for denotations
\newcommand{\denote}[1]{\left\llbracket#1\right\rrbracket}

%\renewcommand{\theFancyVerbLine}{\sffamily \textcolor[rgb]{0.5,0.5,0.5}{\scriptsize \oldstylenums{\arabic{FancyVerbLine}}}}

\newcommand{\inputMinted}[2]{\lstinputlisting[language=#1]{code/#2}

}

%\newcommand{\inputMinted}[2]{\par\inputminted[fontsize=%\footnotesize,linenos=true,bgcolor=white,numberblanklines=false,mathescape,samepage=true]{#1}{code/#2}\\ \ \\}
\definecolor{bg}{rgb}{0.95,0.95,0.95}
%%\usemintedstyle{manni}     % comci comca
%%\usemintedstyle{borland}   % pretty strict style
%%\usemintedstyle{vim}       % together with fruity and native only with dark bg!
%\usemintedstyle{bw}         % strict, too

\counterwithout{footnote}{chapter}

\newlength{\dist}
\setlength{\dist}{1cm}

\newlength{\edge}
\setlength{\edge}{1.5cm}

\newlength{\longedge}
\setlength{\longedge}{1.7cm}

\setlength{\parindent}{0cm}

\newlength{\nd}
\setlength{\nd}{.6cm}

\newlength{\ndd}
\setlength{\ndd}{.5cm}

%\newtheorem{theorem}{Theorem}[section]
\newtheorem{sublemma}[theorem]{Sub-Lemma}
%\newtheorem{lemma}[theorem]{Lemma}
%\newtheorem{proposition}[theorem]{Proposition}
%\newtheorem{corollary}[theorem]{Corollary}
%\newtheorem{definition}[theorem]{Definition}
%\newtheorem{example}[theorem]{Example}
\let\oldexample\example
\renewcommand{\example}{\oldexample\normalfont}

\tikzset{>=latex,on grid, auto}

\tikzset{
  ashadow/.style={
    opacity=.25,
    shadow xshift=0.07,
    shadow yshift=-0.07
  }
}
\tikzstyle{supergraph}=[
  fill=greyblue,
  rounded corners,
  font=\small,
  align=right,
  text=charcoal,
  drop shadow={ashadow, color=greyblue}
]

\newcommand{\backEq}{\mathcal U^\equiv}
\newcommand{\backCC}{\mathcal U^\Subset}
\newcommand{\transEq}{\mathcal T^\equiv}
\newcommand{\transCC}{\mathcal T^\Subset}

\newcommand{\rcc}{{R^\Subset}}
\newcommand{\lcc}{L^\Subset}

\newcommand{\result}{\mathcal R}

\newcommand{\resultCC}{\result_\Subset}
\newcommand{\resultEq}{\result_\equiv}

\newcommand{\updmap}[1]{\textsf{update}_{#1}}
\newcommand{\updfun}[1]{\textsf{update}_{#1,\,r}}

\newcommand{\setpair}[2]{\left\langle #1,\,#2\right\rangle}

\newcommand{\highlight}[2]{\colorbox{#2}{$\displaystyle #1$}}

\newcommand{\ione}{I_{f_1,\,r}}
\newcommand{\itwo}{I_{f_2,\,r}}
\newcommand{\uone}{U_{f_1,\,r}}
\newcommand{\utwo}{U_{f_2,\,r}}

\newcommand{\mvp}[1]{\textsf{MVP}_{#1}}
\newcommand{\ivp}{\textsf{VP}}

\newcommand{\startmain}{\textsf{start}_\texttt{main}}

\newcommand{\edgefn}{\textsf{EdgeFn}}
\newcommand{\ccedgefn}[1]{\edgefn^\Subset_{#1}}

\newcommand{\menv}{M_\textsf{Env}}

\newcommand{\topcc}{\top_\Subset}
\newcommand{\botcc}{\bot_\Subset}

\newcommand{\mpddef}[2]{\xi(#1,\,#2)}
\newcommand{\mpd}{\mpddef p d}
\newcommand{\mppd}{\mpddef{p'} d}
\newcommand{\mpdkm}{\mpddef{p}{d_{k-1}}}
\newcommand{\efek}{\ccedgefn R(e_k)(\mpdkm)(r)}

\newcommand{\id}{\textsf{id}}


\pagenumbering{gobble}
\listoftodos[Overview of Comments]

\mainmatter  % start of an individual contribution

\title{Data Flow Analysis\\in the Presence of Correlated Calls}

\author{
 Marianna Rapoport\inst{1} \and
 Ond\v{r}ej Lhot\'ak\inst{1} \and
 Frank Tip\inst{2}
}
\institute{
 University of Waterloo\\
 \email{ \{mrapoport, olhotak\}@uwaterloo.ca }
  \and
 Samsung Research America\\
 \email{ ftip@samsung.com  }
}

\pagenumbering{arabic}

\toctitle{Data Flow Analysis in the Presence of Correlated Calls}
%\tocauthor{Authors' Instructions} % TODO
\maketitle\mtodo{Frank wanted to discuss the title}


\begin{abstract}
We present a technique to improve the precision of data-flow analyses on object-oriented programs in the presence of 	\textit{correlated calls}. Two method calls are correlated if they are polymorphic and are invoked on the same object. Correlated calls are problematic because they can make  existing data-flow analyses consider certain infeasible data-flow paths as valid. This leads to loss in precision of the analysis solution.

We show how infeasible paths can be eliminated for \textit{Inter-procedural Finite Distributive Subset} (IFDS) problems, a large class of data-flow analysis problems. We show how the precision of IFDS problems can be improved in the presence of correlated calls, by using the \textit{Inter-procedural Distributive Environment} (IDE) algorithm to eliminate infeasible paths.  Using IDE, we eliminate the infeasible paths and obtain a more precise result for the original IFDS problem.

Our analysis is implemented in Scala, using the WALA framework for static program analysis on Java bytecode.

%\keywords{We would like to encourage you to list your keywords within
%the abstract section}
\end{abstract}

\section{Introduction}
\otodo{This section should be shrunk a lot (not completely deleted)}
\begin{odelete}
Static program analysis aims to discover properties of computer programs without running them.
Static analysis has applications in compiler optimization, development of programming tools, and computer security, among others.
As an example, we might want to analyze a program to know which variables are constants. 
We could then write a compiler optimization that ensures that the values of those variables are computed only once.
Alternatively, we could use the information about constant variables in an integrated development environment; for instance, to notify the user when an if-expression executes only one of its branches because its test condition has a constant value.

There are demonstrable limits on what information we can obtain about a program without running it.
Rice's theorem states that verifying any non-trivial property of a program is an undecidable problem~\cite{rice1953classes}. However, it is sometimes possible to design an algorithm that \textit{over}- or \textit{underapproximates} the solution that we are seeking.

\textit{Data-flow analysis} is an area of program analysis whose goal is to compute approximations of certain information (for example, which variables must be constants) for each program point. 

Other examples of data-flow analyses are \textit{reaching definitions} (finding out up to which instruction a given assignment of a variable must be valid) and \textit{available expressions} (retrieving the expressions in the program that do not need to be recomputed at a given program point).

Another example of a data-flow analysis is \textit{taint analysis}~\cite{tripp2009taj}. Taint analysis discovers if ``secret'' values, like passwords or other confidential user information, can leak to an external observer. Methods that generate secret values, e.g. those that read user input, are called \textit{sources}. Methods that can leak information, e.g. those that write data to a file or send data through a network, are called \textit{sinks}. The goal of taint analysis is to find out whether data can propagate from sources to sinks.

An important property of a data-flow analysis is \textit{precision}.
Precision reflects how closely a data-flow-analysis result over- or underapproximates the information we are interested in. In the case of taint analysis, let $T$ be the number of sinks that the analysis considers to leak secret information, and $R$ the real number of potential information leaks. The smaller the difference between $T$ and $R$, the greater the precision of the taint analysis.

Data-flow analyses operate on \textit{control-flow graphs} that model the order in which the instructions of a program are executed.
A data-flow-analysis problem defines \textit{flow functions} that represent how data is propagated along the edges of the control-flow graph. The \textit{confluence operator} specifies how the data that has been computed along different paths should be merged when the paths join.

Since a control-flow graph is an overapproximation of the possible flows of control in concrete executions of a program, the graph may contain \textit{infeasible} paths that cannot occur at runtime.

One way to improve the precision of a data-flow analysis is to detect and eliminate infeasible paths.

Our goal is to improve the precision of solutions to problems that can be solved by the \textit{Inter-procedural Finite Distributive Subset} (IFDS) algorithm~\cite{reps1995precise}.
The IFDS algorithm is a general data-flow algorithm that can compute solutions to various data-flow problems, like reaching definitions, available expressions, and taint analysis.

We improve the precision of IFDS problem solutions by eliminating infeasible paths that occur in object-oriented programs in the presence of \textit{correlated method calls}~--- polymorphic calls that are invoked on the same object~\cite{DBLP:journals/scp/Tip15}.
\end{odelete}

\section{Motivating Example}

Consider a call site~$r.m()$ in an object-oriented programming language, where the variable~$r$ is the \textit{receiver} variable of the call site and $m$ is the name of the invoked method\footnote{We assume an internal representation of the program in which for each call site $e_r.m()$, the expression~$e_r$ has been evaluated to the variable~$r$.}. In the rest of the paper, we use the general term \textit{receiver} to mean a receiver variable.
At runtime, the actual method that will be invoked by the call site depends on the runtime type of the object referenced by $r$. If the call site $r.m()$ can be associated with more than one method at compile time, we will say that the call site is \textit{polymorphic}.

For example, in Listing~\ref{list:ccexample}, it is not possible to infer statically whether the runtime type of the variable \verb'a' in the \texttt{main} method is \verb'A' or \verb'B'.
The call \verb'a.foo()' can be dispatched to either \verb'A.foo' or \verb'B.foo', and \verb'a.bar(v)' can be dispatched to either \verb'A.bar' or \verb'B.bar'.
A concrete execution path for the main method might therefore go through \verb'A.foo' and \verb'A.bar', or through \verb'B.foo' and \verb'B.bar'.
However, there cannot be an execution path through \verb'A.foo' and \verb'B.bar' or through \verb'B.foo' and \verb'A.bar'.

\begin{figure}
  \centering
  \begin{minipage}{\textwidth}
    \inputMinted{java}{ccexample.java}
  \end{minipage}
  \caption{Example program containing correlated calls}
  \label{list:ccexample}
\end{figure}

We call the invocations to methods \verb'foo' and \verb'bar' \textit{correlated}.
More generally, correlated calls occur when more than one polymorphic call is invoked on the same receiver variable.

Suppose we wanted to perform a taint analysis on the program in Listing~\ref{list:ccexample}.
Most dataflow-analysis algorithms, including IFDS, would conservatively assume that the call \verb'a.bar' could be dispatched to both \verb'A.bar' and \verb'B.bar', independently of what \verb'a.foo' had been dispatched to in the previous line.

As a result, 
such an analysis would consider a path through \verb'A.foo' and \verb'B.bar' feasible. This means that the variable \verb'v' would be considered secret. We would conclude that a secret value is passed to \verb'B.bar' and printed to the user. In other words, we would consider the program to leak secret information, which it does not do in any concrete execution.

Our technique for improving the precision of an IFDS result is based on transforming the original IFDS problem into a more expressive \textit{Inter-procedural Distributive Environment} (IDE) problem. IDE problems can be solved with the IDE algorithm which is a generalization of IFDS~\cite{sagiv1996precise}. The IDE algorithm can, for instance, solve certain versions of the constant propagation problem that IFDS cannot.

To improve the precision of IFDS results, given an IFDS problem $P$, we convert it into an IDE problem $Q$ that accounts for correlated calls. We then use the IDE algorithm to obtain a solution to $Q$. Finally, we convert the IDE result into a IFDS result. In the presence of correlated calls, the obtained IFDS result can be more precise than the solution that the IFDS algorithm would compute for $P$.

The goal of the correlated-calls analysis presented in this work is to modify the output of an IFDS analysis to account for correlated calls. Specifically, the correlated-calls analysis improves the precision of IFDS problem results by eliminating infeasible execution paths caused by correlated calls. This is done by converting the input-IFDS problem to an IDE problem that detects infeasible paths, and converting the IDE result back to a more precise IFDS result.

The contributions of this work are:
\begin{itemize}
  \item A transformation from IFDS to IDE problems that considers correlated calls.
  \item An implementation in Scala of the correlated-calls transformation and the IDE algorithm which is based on the WALA framework for static analysis on Java bytecode~\cite{fink2012wala}.
\end{itemize}\mtodo{Reference the figure here.}

\begin{figure}
  \centering
    \tikzset{
  ashadow/.style={opacity=.25, shadow xshift=0.07, shadow yshift=-0.07},
}
  \tikzstyle{problem}=[fill=greyblue,text width=2.3cm,rounded corners,font=\small,text=charcoal,drop shadow={ashadow, color=greyblue}]
  \tikzstyle{result}=[fill=bisque,text width=1.8cm,rounded corners,font=\small,text=charcoal,drop shadow={ashadow, color=greyblue}]
\begin{tikzpicture}
    \node [problem] (ifds) {IFDS problem};
    \node [problem] (equiv) [above right=0.9cm and .7\dist of ifds.east] {Equivalent IDE problem};
    \node [problem] (ccide)[below=3cm of equiv.west, anchor=west] {Correlated-calls IDE problem};
    \node [result] (equivres) [right=\dist of equiv.east] {Equivalence-IDE result};
    \node [result] (ccres) [below=3cm of equivres.west,anchor=west] {Correlated-calls result};
    \node [result] (ifdsres) [right=\dist of equivres.east] {IFDS result};
    \node [result] (improved) [right=\dist of ccres.east] {Improved IFDS result};
    \path[->] (equivres) edge node[above]{$\backEq$} (ifdsres);
    \path[->] (ccres) edge node[above] {$\backCC$} (improved);
    \path[->] (ccres) edge[out=30,in=240] node[above] {$\backEq$} (ifdsres);
    \path[->] (ifds) edge[out=40,in=190] node[above] {$\transEq$} (equiv);
    \path[->] (ifds) edge[out=-60,in=165] node[above] {$\transCC$} (ccide);
    \path[->] (equiv) edge node[above]{$\result$} (equivres);
    \path[->] (ccide) edge node[above]{$\result$} (ccres);
\end{tikzpicture}
  \caption{Transformations between IFDS and IDE problems and their results}%
  \label{fig:transformations}%
\end{figure}

  We prove that the solution to an IDE problem that considers correlated calls is more precise than the solution to the original IFDS problem.
  We also show that the correlated-calls analysis is sound, i.e. that it never considers concrete execution paths as infeasible.\mtodo{As discussed on the phone, we want to talk more about the results here.}
 
 Finally, we evaluate the effectiveness of the correlated-calls analysis using an implementation of taint analysis as the source IFDS problem.

The remainder of this paper is organized as follows. In the next section, we describe the IFDS and IDE analyses in detail. In Section~\ref{chapter:cca} we present the correlated-calls analysis as a transformation of IFDS problems into a special kind of IDE problem. Section~\ref{sec:ccdatastr} describes an efficient representation of the data structures that are required to define a correlated-calls IDE transformation.
In Section~\ref{chapter:eval} we address some implementation aspects of the correlated-calls analysis and present an evaluation of its results. Section~\ref{chapter:concl} contains concluding remarks. The proofs for the lemmas in the paper are provided in the Appendix.\mtodo{Reference appendix like this?}

\section{Motivating Example}
  \label{sec:MotivatingExample}
  
In this section, we illustrate our approach using a small motivating example that
shows how our technique can be used to improve the precision of taint analysis.
A taint analysis is a specialized dataflow analysis for computing how string values
may flow from ``sources'', which are typically statements that read untrusted input, 
to ``sinks'', which are typically security-sensitive
operations such as calls to 
a database. In previous research \cite{DBLP:conf/issta/GuarnieriPTDTB11,DBLP:conf/pldi/ArztRFBBKTOM14}, 
taint analysis algorithms have been formulated as IFDS problems.     
  
\begin{figure}
  \centering
  
  \begin{minipage}{\textwidth}
    \inputMinted{java}{ccexample.java}
  \end{minipage}
    \vspace*{-4mm}
  \caption{Example program containing correlated calls}
  \label{list:ccexample}
   
   \begin{minipage}{\textwidth}
    \tikzstyle{supergraph}=[
      fill=greyblue,
      rounded corners,
      font=\footnotesize,
      align=right,
      text=charcoal,
      drop shadow={ashadow, color=greyblue}
    ]
    \tikzstyle{supergraph_y}=[
      fill=bisque,
      rounded corners,
      font=\footnotesize,
      align=left,
      text=charcoal,
      drop shadow={ashadow, color=greyblue}
    ]
    \tikzstyle{supergraph_f}=[
      fill=lightsalmonpink,
      rounded corners,
      font=\footnotesize,
      align=left,
      text=charcoal,
      drop shadow={ashadow, color=greyblue}
    ]
    \tikzstyle{description}=[fill=white]
    \tikzstyle{n}=[fill=black,circle,inner sep=1.5pt]
    \tikzstyle{arrowtext}=[font=\tiny,color=black,above]
    
\hspace*{-10pt}
\begin{tikzpicture}
\scalebox{.8}{
% main method nodes
    \node [supergraph] (st_main) {\textsf{start}$_{\texttt{main}}$};
    \node [supergraph] (asgn_a) [below = \dist of st_main.south] {\texttt{a = args==null\,?}\\\texttt{new\,A()\,:\,new\,B()}};
    \node [supergraph] (call_foo) [below = \dist of asgn_a.south] {\textsf{call}$_\texttt{foo}$};
    \node [supergraph] (return_foo) [below = \dist of call_foo.south] {\textsf{return}$_\texttt{foo}$\\\texttt{v = a.foo()}};
    \node [supergraph] (call_bar) [below = \dist of return_foo.south] {\textsf{call}$_\texttt{bar}$};
    \node [supergraph] (return_bar) [below = \dist of call_bar.south] {\textsf{return}$_\texttt{bar}$};
    \node [supergraph] (end_main) [below = \dist of return_bar.south] {\textsf{end}$_{\texttt{main}}$};
    
% main method edges 
    \path[->,ultra thick] (st_main) edge (asgn_a);
    \path[->,ultra thick] (asgn_a) edge (call_foo);
    \path[->] (call_foo) edge (return_foo);
    \path[->,ultra thick] (return_foo) edge (call_bar);
    \path[->] (call_bar) edge (return_bar);
    \path[->,ultra thick] (return_bar) edge (end_main);
    
%A.foo method nodes
    \node [supergraph_y] (st_a_foo) [right = 5*\dist of st_main] {\textsf{start}$_{\texttt{A.foo}}$};
    \node [supergraph_y] (return_secret) [below = \dist of st_a_foo.south] {\texttt{return secret()}};
    \node [supergraph_y] (end_a_foo) [below = \dist of return_secret.south] {\textsf{end}$_\texttt{A.foo}$};

%A.foo method edges
    \path[->] (st_a_foo) edge (return_secret);
    \path[->] (return_secret) edge (end_a_foo);
    
%A.bar method nodes
    \node [supergraph_y] (st_a_bar) [right = 5*\dist of st_a_foo] {\textsf{start}$_{\texttt{A.bar}}$};
    \node [supergraph_y] (end_a_bar) [below = \dist of st_a_bar.south] {\textsf{end}$_\texttt{A.bar}$};
    
%A.bar method edges
    \path[->] (st_a_bar) edge (end_a_bar);

%B.foo method nodes
    \node [supergraph_f] (st_b_foo) [right = 5*\dist of call_bar] {\textsf{start}$_{\texttt{B.foo}}$};
    \node [supergraph_f] (return_not_secret) [below = \dist of st_b_foo.south] {\texttt{return "not secret"}};
    \node [supergraph_f] (end_b_foo) [below = \dist of return_not_secret.south] {\textsf{end}$_\texttt{B.foo}$};
    
%B.foo method edges
  \path[->] (st_b_foo) edge (return_not_secret);
  \path[->] (return_not_secret) edge (end_b_foo);
    
%B.bar method nodes
    \node [supergraph_f] (st_b_bar) [right = 5*\dist of st_b_foo] {\textsf{start}$_{\texttt{B.bar}}$};
    \node [supergraph_f] (print) [below = \dist of st_b_bar.south] {\texttt{print(s)}};
    \node [supergraph_f] (end_b_bar) [below = \dist of print.south] {\textsf{end}$_\texttt{B.bar}$};
    
%B.bar method edges
  \path[->] (st_b_bar) edge (print);
  \path[->] (print) edge (end_b_bar);
  
%inter-procedural edges
  \path[->,dashed,ultra thick] (call_foo) edge[out=20,in=190] (st_a_foo);
  \path[->,dashed] (call_foo) edge[color=lightsalmonpink,out=-20,in=135] (st_b_foo);
  \path[->,dashed] (call_bar) edge[out=25,in=220] (st_a_bar);
  \path[->,dashed,ultra thick] (call_bar) edge[color=lightsalmonpink,out=18,in=160] (st_b_bar);
  \path[->,dashed,ultra thick] (end_a_foo) edge[out=220,in=15] (return_foo);
  \path[->,dashed] (end_b_foo) edge[color=lightsalmonpink,out=160,in=-60] (return_foo);
  \path[->,dashed] (end_a_bar) edge[out=210,in=40] (return_bar);
  \path[->,dashed,ultra thick] (end_b_bar) edge[color=lightsalmonpink,out=200,in=-40] (return_bar);
    
%arrow description
    \node [description] (intra_arrow_left) [above left = 2*\dist and 3.5*\dist of st_a_bar] {};
    \node [description] (intra_arrow_right) [right = \dist of intra_arrow_left] {};
    \node [description] (inter_arrow_left) [below = .5*\dist of intra_arrow_left] {};
    \node [description] (inter_arrow_right) [below = .5*\dist of intra_arrow_right] {};
    \path[->](intra_arrow_left) edge node[right = .5*\dist]{intra-procedural edge} (intra_arrow_right);
    \path[dashed,->](inter_arrow_left) edge node[right = .5*\dist]{inter-procedural edge} (inter_arrow_right); 
}
\end{tikzpicture}
   \end{minipage}
   \vspace*{-25mm}
  \caption{
    Control flow supergraph for the example program of Figure~\ref{list:ccexample}.
    An infeasible path is shown in bold. 
  }%
  \label{fig:examplesupergraph}%
\end{figure} 
  
Figure~\ref{list:ccexample} shows a small Java program that we will use to illustrate
how our technique can improve the precision of a simple taint analysis. The program
declares a class \code{A} with a subclass \code{B},  where \code{A} defines methods 
\code{foo()} and \code{bar()} that are overridden in \code{B}.  In order to keep the 
example small and self-contained, it is assumed  that secret values are created by 
an unspecified function \code{secret()}, which is called in method \code{A.foo()} on 
line~\ref{line:Afoo}. Furthermore, it is assumed that any write to the standard output 
stream via calls \code{System.out.println()} such as the one in method \code{B.bar()} 
on line~\ref{line:Bbar} is a security-sensitive operation. Depending on the number of 
arguments passed to the program, the \code{main()} method of the example program creates 
either an \code{A}-object or a \code{B}-object on line~\ref{line:createObject}. The 
program then calls \code{foo()} on this object on line~\ref{line:callfoo}, which is 
followed by a call to \code{bar()} on the same object on line~\ref{line:callbar}.  
 
The specific question that we would like to answer for the example program of
Figure~\ref{list:ccexample} is the following: Is it possible for the untrusted value 
that is read on line~\ref{line:Afoo} to flow to the print statement on line~\ref{line:Bbar}? 
%
To determine whether such a flow of tainted data is possible, consider the control-flow
supergraph for the example program that is shown in Figure~\ref{fig:examplesupergraph}.
 The nodes in this graph correspond to statements, method entry points (start nodes) and 
method exit points (end nodes). Note that for each method call, the graph contains a 
distinct call-node and a return-node. Edges in the graph reflect intraprocedural control flow, 
flow of control from a caller to a callee (edges from call-nodes to start-nodes), or
flow of control from a callee back to a caller (edges from end-nodes to return-nodes). 

In the case of our example, the control flow within each of the methods is straightforward and
all interesting issues arise from interprocedural control flow. In particular,
note that, since the variable \code{a} may point to either an \code{A}-object or a \code{B}-object, 
the call on line~\ref{line:callfoo} may dispatch to either \code{A.foo()} or to \code{B.foo()},
and this is reflected by edges  
  from the node labeled $\highlight{\code{call}_\code{foo}}{greyblue}$ to the nodes labeled
  $\highlight{\code{start}_\code{A.foo()}}{bisque}$ and $\highlight{\code{start}_\code{B.foo()}}{lightsalmonpink}$
and by edges
  from the nodes labeled $\highlight{\code{end}_\code{A.foo}}{bisque}$ and $\highlight{\code{end}_\code{B.foo}}{lightsalmonpink}$ 
  to the node labeled  $\highlight{\code{return}_\code{foo}}{greyblue}$.  
Similarly, there are edges from the node labeled $\highlight{\code{call}_\code{bar}}{greyblue}$ to the nodes 
$\highlight{\code{start}_\code{A.bar()}}{bisque}$ and $\highlight{\code{start}_\code{B.bar()}}{lightsalmonpink}$, and 
edges
  from the nodes labeled $\highlight{\code{end}_\code{A.bar}}{bisque}$ and $\highlight{\code{end}_\code{B.bar}}{lightsalmonpink}$ 
  to the node labeled  $\highlight{\code{return}_\code{bar}}{greyblue}$. 
 
Informally speaking, an IFDS-based dataflow analysis propagates dataflow facts along the edges
of a control flow supergraph such as the one in Figure~\ref{fig:examplesupergraph}. The
IFDS algorithm was carefully designed to avoid infeasible paths that may arise in the
presence of multiple calls to the same method. However, in this example all methods are
called in exactly one place, so IFDS is unable to eliminate dataflow along any of the
paths shown in the figure. As a result, IFDS-based taint analysis algorithms such as
\cite{DBLP:conf/issta/GuarnieriPTDTB11,DBLP:conf/pldi/ArztRFBBKTOM14} would report 
that the secret value read on line~\ref{line:Afoo}
might flow to the print statement on line~\ref{line:Bbar}. 

As we discussed previously, the calls to \code{foo()} and \code{bar()} may dispatch
to different implementations of these methods in classes \code{A} and \code{B}, 
because the variable \code{a} that serves as the receiver for the call may be bound to
 objects of type \code{A} or type \code{B} at run time. 
However, consider the fact that the methods \code{foo()} and \code{bar()} are invoked
on \textit{the same object}. This means that the behaviors of the two method calls
are \textit{correlated}: if the call to \code{foo()} dispatches to \code{A.foo()},
then the call to \code{bar()} must dispatch to \code{A.bar()}, and if the call
to \code{foo()} dispatches to \code{B.foo()}, then the call to \code{bar()} must 
dispatch to \code{B.bar()}.  Consequently, paths such as the one highlighted in
Figure~\ref{fig:examplesupergraph} where the call to \code{foo()} dispatches 
to \code{A.foo()} and where the call to \code{bar()} dispatches to \code{B.bar()}
are infeasible.  

The main contribution of this paper is an algorithm for transforming an IFDS problem 
into an IDE problem that expresses the feasibility of paths
in light of correlated calls.
Informally speaking, our approach associates with each edge in an
interprocedural CFG a function that records the types of 
variables that are used as the receiver of correlated method calls. Paths that 
are composed of edges in which the same receiver expression has different types
are infeasible, and the propagation of dataflow facts along such paths is
prevented. Applying our technique to an IFDS-based taint analysis would enable
the resulting IDE-based taint analysis to determine that no secret value can flow from
line~\ref{line:Afoo} to the print statement on line~\ref{line:Bbar}. 

In summary, the example discussed in this section shows how the precision of 
IFDS-based taint analysis can be improved by taking into account how paths
become infeasible in the presence of correlated calls. In particular, we have
shown how a false positive reported by a standard IFDS-based algorithm is eliminated.
While the discussion in this section has focused on the specific problem of taint analysis,
we would like to emphasize that our technique generally applies to \textit{any}
dataflow analysis problem that can be expressed in the IFDS framework. This includes
many common analysis tasks such as \TODO{please add a few examples + citations} 

\subsection{Occurrences of Correlated Calls}
How often do correlated calls occur in practice? To assess the benefit of the correlated-calls analysis, we counted the number of correlated calls that occurred programs of the Dacapo benchmarks~\cite{blackburn2006dacapo}, using the WALA framework for static analysis on Java bytecode~\cite{fink2012wala}.
Our goal was to obtain an upper bound on the number of redundant IFDS-result nodes that could be potentially removed by our analysis. The results are shown in Table~\ref{tab:dacapostat}.

In the table, the number of all call sites in a program is denoted as $C$. 
Polymorphic call sites are denoted as $C_P$, and correlated call sites as $C^\Subset$. 
The first four columns indicate the overall number of various call sites and correlated-call receivers in a program. 
The last three columns indicate the ratio of polymorphic to all call sites, the ratio of correlated to polymorphic call sites, and the ratio of correlated call sites to correlated-call receivers.

\begin{table}
\caption{Frequencies of correlated-call occurrences in the Dacapo benchmarks}\label{tab:dacapostat}
\centering
%\resizebox{\textwidth}{!}{%
\begin{tabular}{@{}lrrrr
>{\columncolor[HTML]{FFFFFF}}r 
>{\columncolor[HTML]{FFFFFF}}l r@{}}
\toprule
\textbf{Benchmark}  &
  \multicolumn{1}{c}{\textbf{$|C|$}} & 
  \multicolumn{1}{c}{\textbf{$|C_P|$}} & 
  \multicolumn{1}{c}{\textbf{$|C^\Subset|$}} & 
  \multicolumn{1}{c}{\textbf{$|\rcc|$}} & 
  {\textbf{\begin{tabular}[c]{@{}l@{}}$\cfrac{|C_P|}{|C|}$\end{tabular}}} & 
  {\textbf{\begin{tabular}[c]{@{}l@{}}$\cfrac{|C^\Subset|}{|C_P|}$\end{tabular}}} & 
  \textbf{\begin{tabular}[c]{@{}l@{}}$\cfrac{|C^\Subset|}{|\rcc|}$\end{tabular}} \\ \midrule
\textbf{antlr}      & 7,610                             & 428                                   & 299                              & 70                                   & {\textbf{6\%}}                                                      & {\textbf{70\%}}                                                     & 4                                                                \\
\textbf{bloat}      & 18,157                            & 933                                   & 429                              & 119                                  & {\textbf{5\%}}                                                      & {\textbf{46\%}}                                                     & 4                                                                \\
\textbf{chart}      & 18,101                            & 466                                   & 195                              & 61                                   & {\textbf{3\%}}                                                      & {\textbf{42\%}}                                                     & 3                                                                \\
\textbf{eclipse}    & 3,222                             & 100                                   & 35                               & 10                                   & {\textbf{3\%}}                                                      & {\textbf{35\%}}                                                     & 4                                                                \\
\textbf{fop}        & 4,831                             & 129                                   & 40                               & 12                                   & {\textbf{3\%}}                                                      & {\textbf{31\%}}                                                     & 3                                                                \\
\textbf{hsqldb}     & 3,573                             & 81                                    & 35                               & 10                                   & {\textbf{2\%}}                                                      & {\textbf{43\%}}                                                     & 4                                                                \\
\textbf{jython}     & 12,149                            & 487                                   & 129                              & 54                                   & {\textbf{4\%}}                                                      & {\textbf{26\%}}                                                     & 2                                                                \\
\textbf{luindex}    & 7,190                             & 188                                   & 79                               & 29                                   & {\textbf{3\%}}                                                      & {\textbf{42\%}}                                                     & 3                                                                \\
\textbf{lusearch}   & 9,043                             & 350                                   & 126                              & 47                                   & {\textbf{4\%}}                                                      & {\textbf{36\%}}                                                     & 3                                                                \\
\textbf{pmd}        & 10,972                            & 219                                   & 68                               & 23                                   & {\textbf{2\%}}                                                      & {\textbf{31\%}}                                                     & 3                                                                \\
\textbf{xalan}      & 3,889                             & 110                                   & 35                               & 10                                   & {\textbf{3\%}}                                                      & {\textbf{32\%}}                                                     & 4                                                                \\
\textbf{Geom. mean} & \textbf{7,572}                    & \textbf{240}                          & \textbf{91}                      & \textbf{29}                          & {\textbf{3\%}}                                                      & {\textbf{38\%}}                                                     & \textbf{3}                                                       \\ \bottomrule
\end{tabular}
%}
\end{table}

We can see that on average, 3\% of all call sites $C$ are polymorphic call sites $C_P$.
\mtodo{Should we explain in a footnote why the number of polymorphic call sites is so low?}
Out of those call sites, 38\% are correlated call sites $C^\Subset$. We also see that for one correlated-call receiver, there are on average three correlated calls. 

 
  


\section{Background}\label{sec:bg}
%The purpose of the correlated-calls analysis is to solve IFDS problems more precisely than using the standard IFDS algorithm by ruling out some infeasible paths. The correlated-calls analysis works by transforming an IFDS problem to an IDE problem, solving the IDE problem, and transforming the IDE result to a solution to the original IFDS problem. 
This section defines basic terminology and presents the IFDS and IDE algorithms.

\subsection{Terminology and Notation}
%We will start by introducing several concepts used by the IFDS and IDE analyses.
The \textit{control-flow graph} of a procedure is a directed graph whose nodes are instructions, and which contains an edge from $n_1$ to $n_2$ whenever
$n_2$ may execute immediately after $n_1$. A control-flow graph has a distinguished \textit{start node} \textsf{start}$_p$ and \textit{end node} \textsf{end}$_p$.
%and edges represent transfer of control between the instructions during an execution of the program. 
%A control-flow graph has a unique start node, $\startmain$, which is the node corresponding to the program entrypoint.
%An \textit{intra-procedural} path is a path in a control-flow graph whose nodes are in the same procedure. By contrast, an \textit{inter-procedural path} is one that contains nodes from different procedures. 
The \textit{control-flow supergraph} of a program contains the control-flow graphs of all of the procedures as
subgraphs. In addition, for each call instruction $c$, the supergraph contains a \textit{call-to-start} edge to the start node of every procedure that
may be called from $c$, and an \textit{end-to-return} edge from the end node of the procedure back to the call instruction.
\otodo{do we really need to extra "return nodes" that the IFDS/IDE papers add?}
%is a control-flow graph
%in which each procedure $p$ is augmented with an additional \textit{start node} \textsf{start}$_p$ and \textit{end node} \textsf{end}$_p$, and for each call $c_q$ to a procedure $q$, there is a \textit{call node} \textsf{call}$_{c_q}$ and subsequent \textit{return node} \textsf{return}$_{c_q}$.

%A control-flow supergraph allows us to model the control flow in inter-procedural paths.
 %The flow from the caller to the callee is represented using an edge \[(\textsf{call}_{c_q},\,\textsf{start}_q)\,.\] The control flow from the callee back to the caller goes through an edge \[(\textsf{end}_q,\,\textsf{return}_{c_q})\,.\]

A call site is \textit{monomorphic} if it always calls the same procedure. In an object-oriented language, 
a call site $r.m(\ldots)$ can dynamically dispatch to multiple methods depending on the runtime
type of the object pointed to by the receiver $r$.
A call site that calls multiple procedures is called
\textit{polymorphic}.
We define a function $\textsf{lookup}$ to specify the dynamic dispatch: 
if $s$ is the signature of $m$ and $t$ is the runtime type of the object
pointed to by $r$, $\textsf{lookup}(s,t)$ gives the procedure that will
be invoked by the call $r.m(\ldots)$. We also define a kind of inverse $\tau$
of $\textsf{lookup}$: given a signature $s$ and a specific invoked procedure
$f$, $\tau(s,f)$ gives the set of all runtime types of $r$ that cause $r.m(\ldots)$
to dispatch to $f$: $\tau(s,f) = \{t\mid \textsf{lookup}(s,t)=f\}$.

\otodo{do we ever use $\src$ and $\target$?}
We denote the source and end nodes of a graph edge $e$ as $\src e$ and $\target e$.

\commentout{
\begin{example}
  Consider the program in Listing~\ref{list:examplejava}. The supergraph corresponding to that program is shown in Figure~\ref{fig:examplesupergraph}.
  \otodo{Can we merge the examples from Figures 1 and 2 into a single example program?}
\otodo{I think we can remove the supergraph Figure 3.}
\begin{figure}
  \centering
  \begin{minipage}{\textwidth}
    \inputMinted{java}{examplesupergraph.java}
  \end{minipage}
  \caption{An example Java program}
  \label{list:examplejava}
\end{figure}
  
  \begin{figure}%
  \centering
  \scalebox{.9}{
      \tikzset{
        ashadow/.style={
          opacity=.25,
          shadow xshift=0.07,
          shadow yshift=-0.07
        },
      }
    \tikzstyle{supergraph}=[
      fill=greyblue,
      rounded corners,
      font=\small,
      align=center,
      text=charcoal,
      drop shadow={ashadow, color=greyblue}
    ]
    \tikzstyle{supergraph_s}=[
      fill=bisque,
      rounded corners,
      font=\small,
      align=center,
      text=charcoal,
      drop shadow={ashadow, color=greyblue}
    ]
    \tikzstyle{supergraph_f}=[
      fill=lightsalmonpink,
      rounded corners,
      font=\small,
      align=center,
      text=charcoal,
      drop shadow={ashadow, color=greyblue}
    ]
    \tikzstyle{description}=[fill=white]
\begin{tikzpicture}

% main method nodes
    \node [supergraph] (st_main) {\textsf{start}$_{\texttt{main}}$};
    \node [supergraph] (call_secret) [below = \dist of st_main.south] {\textsf{call}$_\texttt{secret}$};
    \node [supergraph] (return_secret) [below = \dist of call_secret.south] {\textsf{return}$_\texttt{secret}$\\\texttt{a = secret()}};
    \node [supergraph] (call_f_main) [below = \dist of return_secret.south] {\textsf{call}$_\texttt{A.f}$};
    \node [supergraph] (return_f_main) [below = \dist of call_f_main.south] {\textsf{return}$_\texttt{A.f}$\\\texttt{b = f(a)}};
    \node [supergraph] (end_main) [below = \dist of return_f_main.south] {\textsf{end}$_{\texttt{main}}$};
% main method edges
    \path[->] (st_main) edge (call_secret);
    \path[->] (call_secret) edge (return_secret);
    \path[->] (return_secret) edge (call_f_main);
    \path[->] (call_f_main) edge (return_f_main);
    \path[->] (return_f_main) edge (end_main);
    
%secret method nodes
    \node [supergraph_s] (st_secret) [above right = 2*\dist and 6*\dist of st_main] {\textsf{start}$_{\texttt{secret}}$};
    \node [supergraph_s] (return_secret_string) [below = \dist of st_secret.south] {\texttt{return "secret"}};
    \node [supergraph_s] (end_secret) [below = \dist of return_secret_string.south] {\textsf{end}$_\texttt{secret}$};
%secret method edges
    \path[dashed,->] (call_secret) edge[out=20,in=130,->] (st_secret);
    \path[->] (st_secret) edge (return_secret_string);
    \path[->] (return_secret_string) edge (end_secret);
    \path[dashed,->] (end_secret) edge[out=240,in=5] (return_secret);

%f method nodes
    \node [supergraph_f] (st_f) [below right = 2*\dist and \dist of end_secret.south] {\textsf{start}$_\texttt f$};
    \node [supergraph_f] (if) [below = \dist of st_f.south] {\texttt{if (s == null)}};
    \node [supergraph_f] (return_x) [below right = \dist and 2*\dist of if.south] {\texttt{return s}};
    \node [supergraph_f] (call_f_f) [below left = \dist and 2*\dist of if.south] {\textsf{call}$_\texttt{f}$};
    \node [supergraph_f] (return_f_f) [below = \dist of call_f_f.south] {\textsf{return}$_\texttt{f}$\\\texttt{r = f("not secret")}};
    \node [supergraph_f] (return_r) [below = \dist of return_f_f.south] {\texttt{return r}};
    \node [supergraph_f] (end_f) [below right = \dist and 2*\dist of return_r.south] {\textsf{end}$_\texttt f$};
%f method edges
    \path[dashed,->] (call_f_main) edge[out=0,in=160] (st_f);
    \draw[->] (st_f) edge (if);
    \draw[->] (if) edge [out=210,in=45] (call_f_f);
    \draw[->] (call_f_f) edge (return_f_f);
    \draw[dashed,->] (call_f_f) [out=170,in=200] edge (st_f);
    \draw[->] (return_f_f) edge (return_r);
    \draw[->] (return_r) edge [out=-80,in=175] (end_f);
    \draw[->] (if) edge [out=330,in=135] (return_x);
    \draw[->] (return_x) edge [out=270,in=10] (end_f);
    \draw[dashed,->] (end_f) edge [out=200,in=180] (return_f_f);
    \draw[dashed,->] (end_f) edge [out=215,in=-50] (return_f_main);
    
%arrow description
    \node [description] (intra_arrow_left) [above right = \dist and 2*\dist of st_secret] {};
    \node [description] (intra_arrow_right) [right = \dist of intra_arrow_left] {};
    \node [description] (inter_arrow_left) [below = .5*\dist of intra_arrow_left] {};
    \node [description] (inter_arrow_right) [below = .5*\dist of intra_arrow_right] {};
    \path[->](intra_arrow_left) edge node[right = .5*\dist]{intra-procedural edge} (intra_arrow_right);
    \path[dashed,->](inter_arrow_left) edge node[right = .5*\dist]{inter-procedural edge} (inter_arrow_right);
\end{tikzpicture}
}
  \caption{An example supergraph for Listing~\ref{list:examplejava}}%
  \label{fig:examplesupergraph}%
  \end{figure}
\end{example}
}

%A \textit{flow-sensitive} data-flow analysis is one that takes the order of program instructions into account.

A path through the control-flow supergraph is \textit{valid} if every end-to-return edge on the path returns
to the site of the most recent unmatched call. The set of all valid paths from the program entry point to
a node $n$ is denoted $\ivp(n)$.
\otodo{Should we give a formal definition?}

%Let each call node in a program be labeled with a distinct opening parenthesis and the corresponding return node with the matching closing parenthesis. For a given path $p$, let $s$ be the string that is obtained by concatenating the labels of the nodes in $p$.
%Then $p$ is \textit{valid} if $s$ belongs to the language of substrings of balanced parentheses. The set of all inter-procedurally valid paths from the start node to a node $n$ is denoted as $\ivp(n)$. The set $\ivp(n)$ is a conservative approximation of all concrete execution paths from the start node to $n$.

%A \textit{context-sensitive} data-flow analysis is an analysis that considers only inter-procedurally valid paths.

\commentout{
\begin{mdelete}
\begin{example}
  In the supergraph in Figure~\ref{fig:examplesupergraph}, let us assign $\{,\,\}$ parentheses to $\textsf{call}_\texttt{A.f}$ and $\textsf{return}_\texttt{A.f}$, and $\langle,\,\rangle$ parentheses to $\textsf{call}_\texttt{f}$ and $\textsf{return}_\texttt{f}$. 
  Then the string corresponding to the path 
  \[
    p_1=\left[\highlight{\textsf{call}_\texttt{A.f}}{greyblue},\,
          \highlight{\textsf{start}_\texttt{f}}{lightsalmonpink},\,
          \highlight{\texttt{if (s == null)}}{lightsalmonpink},\,
          \highlight{\texttt{return s}}{lightsalmonpink},\,
          \highlight{\textsf{end}_\texttt{f}}{lightsalmonpink},\,
          \highlight{\textsf{return}_\texttt{A.f}}{greyblue}
    \right]
  \]
  is $\{\}$, which indicates that $p_1$ is valid. Every prefix of $p_1$ is also a valid path.
  
\mtodo{The figure that we deleted was constructed so that there would be a realizable and a non-realizable path. Now that we deleted that figure, let's delete this example as well.}  
  
However, the graph also contains an inter-procedurally invalid path
  \[
    p_2=\left[\highlight{\textsf{call}_\texttt{f}}{lightsalmonpink},\,
          \highlight{\textsf{start}_\texttt{f}}{lightsalmonpink},\,
          \highlight{\texttt{if (s == null)}}{lightsalmonpink},\,
          \highlight{\texttt{return s}}{lightsalmonpink},\,
          \highlight{\textsf{end}_\texttt{f}}{lightsalmonpink},\,
          \highlight{\textsf{return}_\texttt{A.f}}{greyblue}
    \right]
  \]
  with corresponding string $\langle\}$.
\end{example}
\end{mdelete}
}

\otodo{The definitions that Marianna had were for \emph{complete}
lattices/semilattices. All of the lattices in the context of IFDS/IDE
are/must be complete. I'll just say that here.}
A \textit{complete lattice} is a partially ordered set $(S,\sqsubseteq)$ in which every subset has a least upper bound and a greatest lower bound.
A \textit{complete meet semilattice} is a partially ordered set in which every subset need only have a greatest lower bound.
The symbols $\bot$ and $\top$ are used to denote the greatest lower bound of $S$ and of the empty set, respectively.
Since all of the (semi)lattices discussed in this paper are required to be complete, we will henceforth leave out the
\textit{complete} qualifier.


%A \textit{meet semilattice} $L=(S,\,\sqcap)$ is defined by a set $S$ and a meet operation $\sqcap$ that is  associative, commutative, and idempotent.
  %The meet operation induces a partial order $(S,\,\sqsubseteq)$ where every subset contains  a greatest lower bound: For all $x,\,y\in S$, $x\sqsubseteq y$ if $x\sqcap y=x$. The greatest lower bound, or top element, of the semilattice is denoted as $\top$.
  %If $k$ is the length of the longest chains of elements in the semilattice, then the \textit{height} of the semilattice is~$k-1$.

In this paper, we denote a map $m$ as a set of pairs of keys and values, in which each key appears at most once.
For any map $m$,  $m(k)$ is the value paired with the key $k$ in $m$. We denote by $m[x\to y]$ a map that maps $x$ to $y$, and every other key $k$ to $m(k)$.
%To avoid excessive parentheses, we write $\left(m[x_1\to y_1]\right)[x_2\to y_2]$ as $m[x_1\to y_1][x_2\to y_2]$.

%We will denote the identity function $\lambda x\,.x$ by $\textsf{id}$.
%We will use a typed version of this function in various contexts, where the type of $x$ will vary with
%the context.

%OL: we can just as easily write the identity explicitly as \x.x throughout.

\subsection{IFDS}\label{sec:bgifds}
The IFDS framework~\cite{reps1995precise} is a precise and efficient algorithm for data-flow analysis
%. IFDS was developed in 1995 by T.\,Reps, S.\,Horwitz, and M.\,Sagiv at the University of Wisconsin and
that has been used to solve a variety of data-flow analysis problems~\cite{bodden2013spl,naeem2008typestate,DBLP:conf/birthday/KreikerRRSWY13,tripp2009taj}.
%\mtodo{ The four references take away space in the References section. Remove some of them?  }
The IFDS framework is an instance of the \textit{functional approach} to data-flow analysis~\cite{pnueli1981two}
because it constructs summaries of the effects of called procedures.
\commentout{The IFDS analysis is a version of the classic \textit{functional approach} to data-flow analysis proposed by M.\,Sharir and A.\,Pnueli~\cite{pnueli1981two}.}
\commentout{In other data-flow algorithms not based on the functional approach, the result of the analysis at the entry of a procedure ``merges'' the incoming data obtained from all callers of the procedure. As a consequence, there is one global data-flow result computed at the end of the procedure. Context-sensitivity, however, allows an analysis to compute the data-flow result for a given procedure as a \textit{function} of the data-flow value at the start of the procedure. 
In other words, the analysis result for a procedure varies depending on where the procedure was called from. This significantly improves the precision of a data-flow analysis, which is why context-sensitivity is an important advantage of IFDS over classic data-flow algorithms.}
\mtodo{I'm not saying any more that IFDS is a version of Sharir\&Pnueli's functional approach.}
\mtodo{I also removed the comparison of IFDS with other algorithms. I'm not pointing out the benefits of IFDS any more, so we might fail to show the importance of the algorithm.}
The IFDS framework is applicable to \textit{inter-procedural} data flow problems whose domain is \textit{subsets} of a \textit{finite} set $D$,
and whose data-flow functions are \textit{distributive}.
A function $f$ is distributive if 
$f(x_1\sqcap x_2)=f(x_1)\sqcap f(x_2)$.

%The IFDS algorithm is applicable to problems which can be expressed with data-flow functions that satisfy certain restrictions. \textit{Inter-procedural} flow functions specify how data flows from the invocation of a procedure to its start, and from the procedure's end back to its call site. \textit{Distributive} flow functions are those that distribute over the confluence operator. In the context of IFDS, the confluence operator is called meet, and it can be either union or intersection. The data-flow facts on which the analysis operates must be a \textit{finite} set $D$. Each flow function operates on a \textit{subset} of $D$ (for example, the set of variables in the program) which makes the domain of the flow functions the power set of $D$.

\commentout{The purpose of the IFDS framework is to solve a special subset of inter-procedural, flow-sensitive, context-sensitive data-flow-analysis problems. The main idea of IFDS is to encode the data-flow analysis problem into a graph-reachability problem.}

The IFDS algorithm is notable because it computes a meet over valid paths solution in polynomial time.
Most other interprocedural analysis algorithms are either imprecise due to invalid paths,
general but do not run in polynomial time~\cite{knoop1992interprocedural,pnueli1981two}, or handle a very specific set of problems~\cite{knoop1993efficient}.

%\subsubsection{Data-Flow Problems Suitable for IFDS}\label{sec:suitableifds}
\commentout{In this section we describe the data-flow problems that can be solved by an IFDS analysis. We will start with an intuitive definition and later on formalize the notion of an IFDS-suitable problem.}

%Informally, an IFDS analysis can only solve decision problems. An IFDS analysis answers questions of the following kind: ``is property $X$ true at program point~$Y$?''. For example, a taint-analysis problem asks, for each variable $v$ in the program, ``is $v$ secret at a given program point?''. 
\commentout{An available-expressions problem asks, for each expression $e$, ``does $e$ have to be recomputed at a given program point?''.}

The input to the IFDS algorithm is specified as
%Formally, a data-flow analysis problem is suitable for an IFDS analysis if it can be encoded as an IFDS problem
  $(G^*,\,D,\,F,\,M_F,\,\sqcap)$,
where $G^*=(N^*,\,E^*)$ is the supergraph of the input program with nodes $N^*$ and edges $E^*$,
$D$ is a finite set of \textit{data-flow facts},
$F$ is a set of distributive dataflow functions of type $2^D\to2^D$,
$M_F:\,E^*\to F$ assigns a dataflow function to each supergraph edge,
and $\sqcap$ is the \textit{meet operator} on the powerset $2^D$, either union or intersection.
In our presentation, the meet will always be union, but all of the results apply dually when the
meet is intersection.

The output of the IFDS algorithm is, for each node $n$ in the supergraph, the \textit{meet-over-all-valid-paths} solution
 $\mvp{F}(n)=\bigsqcap_{q\in\ivp(n)}M_F(q)(\top)$, where $M_F$ is extended from edges to paths by composition.

\commentout{
is a function that maps supergraph edges to dataflow functions, and $M_F$ is extended to paths by composition\footnote{%
  Let $A$ be a set and $f:\,E^*\to(A\to A)$ a function from supergraph edges to functions on $A$.
  We say that $f$ is extended to  paths by composition to denote that for a path $q$ consisting of the edges $e_1,\,\dots,\,e_k$, $f(q)=f(e_k)\circ\ldots\circ f(e_1)\circ\id$.
}. The \textit{meet operator} $\sqcap$ is either union or intersection.
}

%Without loss of generality, we will take meet to denote union. It can be shown that any problem where meet is defined as intersection can be reformulated into an equivalent one where meet is defined as union~\cite{reps1995precise}.

\subsubsection{Overview of the IFDS Algorithm}\label{sec:overviewifds}
The key idea behind the IFDS algorithm is that it is possible to represent any distributive
function $f$ from $2^D$ to $2^D$ by a \textit{representation relation} $R_f \subseteq 
(D \cup \{0\})
\times
(D \cup \{0\})$. The representation relation can be visualized as a bipartite graph
with edges from one instance of $D \cup \{0\}$ to another instance of $D \cup \{0\}$.
The IFDS algorithm uses such graphs to efficiently represent both the input dataflow functions
and the summary functions that it computes for called procedures.
Specifically, the representation relation $R_f$ of a function $f$ is defined as:
\[
R_f=\{(\mathbf0,\,\mathbf0)\}\cup
      \{(\mathbf0,\,d_j)\,|\,d_j\in f(\varnothing)\}\cup
      \{(d_i,\,d_j)\,|\,d_j\in f(\{d_i\}) \setminus f(\varnothing)\}.
\]
%Each pair $(d_i,\,d_j)\in R_f$ corresponds to an edge $((n_1,\,d_i),\,(n_2,\,d_j))$ in the exploded supergraph.

\commentout{
Note that $R_f$ constructs pairs of dataflow facts so that
\begin{itemize}
  \item there is always an edge $(\mathbf0,\,\mathbf0)$ corresponding to the control-flow-graph edge;
  \item if there is an edge $(\mathbf0,\,d_j)$, then there is no other edge leading to $d_j$; in particular, there is never an edge $(d_i,\,\mathbf0)$ where $d_i\ne\mathbf0$.
\end{itemize}
}

\begin{example}\label{ex:flowfun}
    Given $D=\{u,\,v,\,w\}$ and $f(S) = S\setminus\{v\} \cup \{u\}$,
    the representation relation 
  %The representation relation $R_f$ for a set of data-flow facts 
  %and dataflow function $f=\lambda S\,.\,S\setminus\{v\}\cup\{u\}$ looks as follows:
  $
    R_f=\{(\mathbf0,\,\mathbf0),\,(\mathbf0,\,u),\,(w,\,w)\}
$, which can be visualized with the following graph:
   \begin{center}
    \tikzstyle{n}=[fill=black,circle,inner sep=1.5pt]
\begin{tikzpicture}[>=stealth,on grid, auto]
    \node [n,label=$\mathbf{0}$] (zero){};
    \node [n,] (zero_) [below=\edge of zero] {};
    \node [n,label=$u$] (u) [right=\nd of zero] {};
    \node [n,] (u_) [below=\edge of u] {};
    \node [n,label=$v$] (v) [right=\nd of u] {};
    \node [n,] (v_) [below=\edge of v] {};
    \node [n,label=$w$] (w) [right=\nd of v] {};
    \node [n,] (w_) [below=\edge of w] {};
    \path[->](zero) edge (zero_);
    \path[->](zero) edge (u_);
    \path[->](w) edge (w_);
  \end{tikzpicture}
  \end{center}
\end{example}

The representation relation decomposes a flow function into functions
(edges) that operate on each fact individually. This is possible due
to distributivity: applying the flow function to a set of facts is
equivalent to applying it on each fact individually and
taking the union of the results.

The meet of two functions can be computed as simply the union of their representation
functions: $R_{f\sqcap f'} = R_f \cup R_{f'}$. The composition of two functions can be
computed by combining their representation graphs, merging the range nodes of the
first function with the corresponding domain nodes of the second function, and finding
paths in the resulting graph.

\commentout{
The representation relation allows us to compactly represent the composition and meet operations which are required for the IFDS algorithm.

For two representation relations $R_{f_1},\,R_{f_2}$, the composition and meet operations are defined as follows:
\begin{align}
  R_{f_1}\circ R_{f_2}&=\{(d_1,\,d_3)\ |\ \exists d_2:\ (d_1,\,d_2)\in R_{f_1},\,(d_2,\,d_3)\in R_{f_2}\}\,.
\intertext{and}
  R_{f_1}\sqcap R_{f_2}&=R_{f_1}\cup R_{f_2}\,.
\end{align}

The representation relation distributes over composition and meet:
\begin{align}
  R_{f_1}\circ R_{f_2}=R_{f_1\circ f_2}\,,\qquad
\text{and}\qquad
  R_{f_1}\sqcap R_{f_2}=R_{f_1\sqcap f_2}\,.
\end{align}
}

%On the exploded graph, the composition of two functions is represented by the paths that are formed when the exploded-graph edges are combined.
\otodo{The example used to say $R_f \circ R_g$, but I believe it showed $R_g \circ R_f$. I fixed it, but please check that this is correct.}

\begin{example}
    If $g(S)=S\setminus\{w\}$ and $f(S) = S\setminus\{v\} \cup \{u\}$, then
    $
    R_g\circ R_f=\{(\mathbf0,\,\mathbf0),\,(\mathbf0,\,u)\},
    $
  as illustrated in the following graph:
    \begin{center}
    \tikzstyle{n}=[fill=black,circle,inner sep=1.5pt]
\begin{tikzpicture}[>=stealth,on grid, auto]
    \node [n,label=$\mathbf{0}$] (zero){};
    \node [n] (zero_) [below=\edge of zero] {};
    \node [n,label=$u$] (u) [right=\nd of zero] {};
    \node [n] (u_) [below=\edge of u] {};
    \node [n,label=$v$] (v) [right=\nd of u] {};
    \node [n] (v_) [below=\edge of v] {};
    \node [n,label=$w$] (w) [right=\nd of v] {};
    \node [n] (w_) [below=\edge of w] {};
    \node [n] (zero__) [below=\edge of zero_] {};
    \node [n] (u__) [below=\edge of u_] {};
    \node [n] (v__) [below=\edge of v_] {};
    \node [n] (w__) [below=\edge of w_] {};

    \path[->](zero) edge node[left] {$R_f$} (zero_);
    \path[->](zero) edge (u_);
    \path[->](w) edge (w_);
    \path[->](zero_) edge node[left] {$R_g$} (zero__);
    \path[->](u_) edge (u__);
    \path[->](v_) edge (v__);
    
    \node [n,label=$\mathbf{0}$] (zeroC) [below right= .5\edge and 3.5\dist of w] {};
    \node [n] (zeroC_) [below=\edge of zeroC] {};
    \node [n,label=$u$] (uC) [right=\nd of zeroC] {};
    \node [n] (uC_) [below=\edge of uC] {};
    \node [n,label=$v$] (vC) [right=\nd of uC] {};
    \node [n] (vC_) [below=\edge of vC] {};
    \node [n,label=$w$] (wC) [right=\nd of vC] {};
    \node [n] (wC_) [below=\edge of wC] {};

    \path[->](zeroC) edge node[left] {$R_g\circ R_f$} (zeroC_);
    \path[->](zeroC) edge (uC_);
  \end{tikzpicture}
  \end{center}
\end{example}

Composition of two distributive
functions $f$ and $f'$ corresponds to finding reachable nodes in a graph composed from their representation
relations $R_f$ and $R_{f'}$. Therefore, evaluating the composed dataflow function for a control
flow path corresponds to finding reachable nodes in a graph composed from the representation relations
of the dataflow functions for individual instructions.

It is this graph of representation relations that the IFDS algorithm operates on.
In this graph, called the \textit{exploded supergraph}, each node is a pair $(n,d)$, where
$n\in N^*$ is a node of the control-flow supergraph and $d$ is an element of $D\cup\{0\}$.
For each edge $(n \to n') \in E^*$, the exploded supergraph contains a set of edges
$(n,d_i) \to (n',d_j)$, which form the representation relation of the dataflow function
$M_F(n \to n')$. The IFDS algorithm finds all exploded supergraph edges that are reachable by
\textit{realizable} paths in the exploded supergraph. A path
is \textit{realizable} if its projection to the (non-exploded) supergraph is a valid
path (i.e. if it is of the form $(n_0, d_0) \to (n_1,d_1) \to \cdots \to (n_m,d_m)$
and $n_0 \to n_1 \to \cdots \to n_m$ is a valid path).



\commentout{
To compute the meet-over-all-valid-paths solution, each node in the control-flow supergraph is paired with a \textit{fact} $d\in D\cup\{\mathbf0\},\,\mathbf0\notin D$, yielding the nodes $N^\#$ of the \textit{exploded supergraph} $G^\#=(N^\#,\,E^\#)$.
Roughly, for each node in the program, a fact denotes a binary property whose value (true or false) we want to find out.
The start node of the exploded supergraph is the node $(\startmain,\,\mathbf0)$.

The flow functions $F$ define the edges of the exploded supergraph.
Using the flow functions, the IFDS algorithm computes the inter-procedurally \textit{realizable} paths from the start to the rest of the exploded graph's nodes. A realizable path is a valid path in the exploded supergraph that starts with the entry node $\startmain$.

If there is a realizable path from the node $(\startmain,\,\mathbf0)$ to a given node $(n,\,d),\,d\ne\mathbf0$, then the fact $d$ is considered to hold at node $n$. A path to a node $(n,\,\mathbf0)$ means that in the control-flow supergraph, there is a path from $\startmain$ to $n$.

In this way, the IFDS algorithm reduces the input data-flow problem to a graph-reachability problem.
\begin{example}
  In a taint analysis, $D$ is the set of variables in the program. If a fact $d\in D$ is reachable at a given node, then the variable is considered secret at that node. Otherwise, it is considered not secret. The question ``is $d$ secret at node $n$?'' becomes ``is there a realizable path from (\textsf{start}$_\texttt{main}$,\,\textbf{0}) to $(n,\,d)$?''.
\end{example}
\commentout{\begin{example} 
  In an available-expressions analysis, $D$ is the set of all expressions in the program. If an expression $d\in D$ is reachable at a certain node, it means that it does not need to be recomputed at that node. 
\end{example}}
}
\begin{example}
  The exploded supergraph for Listing~\ref{list:ccexample} is shown in Figure~\ref{fig:cc_edgefn_example}.\mtodo{Tell the reader here to ignore the labels on the edges of the graph?}
  We can see that there is a realizable path
  from the start node of the exploded graph to the variable \verb's' at~the~node~$\textsf{print(s)}$ in the \verb'B.bar' method. This means that at that node, \verb's' is considered secret.

\commentout{  
    \begin{figure}%
      \hspace*{-30pt}
\scalebox{.9}{
      \tikzset{
        ashadow/.style={
          opacity=.25,
          shadow xshift=0.07,
          shadow yshift=-0.07
        },
      }
    \tikzstyle{supergraph}=[
      fill=greyblue,
      rounded corners,
      font=\small,
      align=right,
      text=charcoal,
      drop shadow={ashadow, color=greyblue}
    ]
    \tikzstyle{supergraph_s}=[
      fill=bisque,
      rounded corners,
      font=\small,
      align=left,
      text=charcoal,
      drop shadow={ashadow, color=greyblue}
    ]
    \tikzstyle{supergraph_f}=[
      fill=lightsalmonpink,
      rounded corners,
      font=\small,
      align=left,
      text=charcoal,
      drop shadow={ashadow, color=greyblue}
    ]
    \tikzstyle{description}=[fill=white]
    \tikzstyle{n}=[fill=black,circle,inner sep=1.5pt]
    
\begin{tikzpicture}[>=stealth,on grid, auto]

% main method nodes
    \node [supergraph] (st_main) {\textsf{start}$_{\texttt{main}}$};
    \node [supergraph] (call_secret) [below = \dist of st_main.south east, anchor=east] {\textsf{call}$_\texttt{secret}$};
    \node [supergraph] (return_secret) [below = \dist of call_secret.south east,anchor=east] {\textsf{return}$_\texttt{secret}$\\\texttt{a = secret()}};
    \node [supergraph] (call_f_main) [below = \dist of return_secret.south east, anchor=east] {\textsf{call}$_\texttt{A.f}$};
    \node [supergraph] (return_f_main) [below = \dist of call_f_main.south east, anchor=east] {\textsf{return}$_\texttt{A.f}$\\\texttt{b = f(a)}};
    \node [supergraph] (end_main) [below = \dist of return_f_main.south east, anchor=east] {\textsf{end}$_{\texttt{main}}$};
    
    \node [n,label=$\mathbf{0}$] (st_main_z) [right=\ndd of st_main.east] {};
    \node [n,label=\texttt{a}] (st_main_x) [right=\ndd of st_main_z] {};
    \node [n,label=\texttt{b}] (st_main_y) [right=\ndd of st_main_x] {};
    
    \node [n] (call_secret_z) [right=\ndd of call_secret.east] {};
    \node [n] (call_secret_x) [right=\ndd of call_secret_z] {};
    \node [n] (call_secret_y) [right=\ndd of call_secret_x] {};
    
    \node [n] (return_secret_z) [right=\ndd of return_secret.east] {};
    \node [n] (return_secret_x) [right=\ndd of return_secret_z] {};
    \node [n] (return_secret_y) [right=\ndd of return_secret_x] {};
    
    \node [n] (call_f_main_z) [right=\ndd of call_f_main.east] {};
    \node [n] (call_f_main_x) [right=\ndd of call_f_main_z] {};
    \node [n] (call_f_main_y) [right=\ndd of call_f_main_x] {};
    
    \node [n] (return_f_main_z) [right=\ndd of return_f_main.east] {};
    \node [n] (return_f_main_x) [right=\ndd of return_f_main_z] {};
    \node [n] (return_f_main_y) [right=\ndd of return_f_main_x] {};
    
    \node [n] (end_main_z) [right=\ndd of end_main.east] {};
    \node [n] (end_main_x) [right=\ndd of end_main_z] {};
    \node [n] (end_main_y) [right=\ndd of end_main_x] {};
% main method edges
    \path[->](st_main_z) edge (call_secret_z);
    \path[->](call_secret_z) edge (return_secret_z);
    \path[->](return_secret_z) edge (call_f_main_z);
    \path[->](call_f_main_z) edge (return_f_main_z);
    \path[->](return_f_main_z) edge (end_main_z);  
    \path[->](return_f_main_y) edge (end_main_y);    
    
%secret method nodes
    \node [supergraph_s] (st_secret) [above right = 2*\dist and 6*\dist of st_main] {\textsf{start}$_{\texttt{secret}}$};
    \node [supergraph_s] (return_secret_string) [below = \dist of st_secret.south west, anchor=west] {\texttt{return "secret"}};
    \node [supergraph_s] (end_secret) [below = \dist of return_secret_string.south west,anchor = west] {\textsf{end}$_\texttt{secret}$};
    
    \node [n,label=$\mathbf{0}$] (st_secret_z) [left=\ndd of st_secret.west] {};
    \node [n] (return_secret_string_z) [left=\ndd of return_secret_string.west] {};
    \node [n] (end_secret_z) [left=\ndd of end_secret.west] {};
%secret method edges
    \path[->](st_secret_z) edge (return_secret_string_z); 
    \path[->](return_secret_string_z) edge (end_secret_z); 
    \path[->,dashed](call_secret_z) edge [out=30,in=180] (st_secret_z);
    \path[->,dashed](end_secret_z) edge [out=240,in=40] (return_secret_z);
    \path[->,dashed](end_secret_z) edge [out=260,in=30] (return_secret_x);
    \path[->](return_secret_x) edge (call_f_main_x);
    \path[->](call_f_main_x) edge (return_f_main_x);
    \path[->](return_f_main_x) edge (end_main_x);

%f method nodes
    \node [supergraph_f] (st_f) [below=2*\dist of end_secret.south west,anchor=west] {\textsf{start}$_\texttt f$};
    \node [supergraph_f] (if) [below = \dist of st_f.south west,anchor=west] {\texttt{if (s == null)}};
    \node [supergraph_f] (return_x) [below right = \dist and 2.5*\dist of if.south] {\texttt{return s}};
    \node [supergraph_f] (call_f_f) [below=\dist of if.south west,anchor=west] {\textsf{call}$_\texttt{f}$};
    \node [supergraph_f] (return_f_f) [below = \dist of call_f_f.south west,anchor=west] {\textsf{return}$_\texttt{f}$\\\texttt{r = f("not secret")}};
    \node [supergraph_f] (return_r) [below = \dist of return_f_f.south west,anchor=west] {\texttt{return r}};
    \node [supergraph_f] (end_f) [below=\dist of return_r.south west,anchor=west] {\textsf{end}$_\texttt f$};
   
     \node [n,label=\texttt{s}] (st_f_s) [left=\ndd of st_f.west] {}; 
     \node [n,label=\texttt{r}] (st_f_r) [left=\ndd of st_f_s] {};
     \node [n,label=$\mathbf{0}$] (st_f_z) [left=\ndd of st_f_r] {};
     
     \node [n] (if_s) [left=\ndd of if.west] {};
     \node [n] (if_r) [left=\ndd of if_s] {};
     \node [n] (if_z) [left=\ndd of if_r] {};
     
     \node [n] (call_f_f_s) [left=\ndd of call_f_f.west] {};
     \node [n] (call_f_f_r) [left=\ndd of call_f_f_s] {};
     \node [n] (call_f_f_z) [left=\ndd of call_f_f_r] {};
     
     \node [n] (return_f_f_s) [left=\ndd of return_f_f.west] {};
     \node [n] (return_f_f_r) [left=\ndd of return_f_f_s] {};
     \node [n] (return_f_f_z) [left=\ndd of return_f_f_r] {};
     
     \node [n] (return_r_s) [left=\ndd of return_r.west] {};
     \node [n] (return_r_r) [left=\ndd of return_r_s] {};
     \node [n] (return_r_z) [left=\ndd of return_r_r] {};
     
     \node [n] (end_f_s) [left=\ndd of end_f.west] {};
     \node [n] (end_f_r) [left=\ndd of end_f_s] {};
     \node [n] (end_f_z) [left=\ndd of end_f_r] {};
     
     \node [n,label=$\mathbf0$] (return_x_z) [right=\ndd of return_x.east] {};
     \node [n,label=\texttt{r}] (return_x_r) [right=\ndd of return_x_z] {};
     \node [n,label=\texttt{s}] (return_x_s) [right=\ndd of return_x_r] {};
%f method edges
    \path[->](st_f_z) edge (if_z);
    \path[->](if_z) edge (call_f_f_z);
    \path[->](call_f_f_z) edge (return_r_z);  
    \path[->](return_r_z) edge (end_f_z);
    \path[->](if_z) edge [out=-43,in=127] (return_x_z);
    \path[->](return_x_z) edge [out=270,in=25] (end_f_z);
    \path[->](st_f_s) edge (if_s);
    \path[->](if_s) edge (call_f_f_s);
    \path[->](call_f_f_s) edge (return_r_s);  
    \path[->](return_r_s) edge (end_f_s);
    \path[->](if_s) edge [out=-30,in=110] (return_x_s);
    \path[->](return_x_s) edge [out=270,in=25] (end_f_s);
    \path[->](return_f_f_r) edge (return_r_r);
    \path[->](return_r_r) edge (end_f_r);
    \path[->,dashed](call_f_main_z) edge [out=20,in=160] (st_f_z);
    \path[->,dashed](end_f_z) edge [out=150,in=-20] (return_f_main_z);
    \path[->,dashed](call_f_f_z) edge [out=110,in=250] (st_f_z);
    \path[->,dashed](end_f_z) edge [out=110,in=250] (return_f_f_z);
    \path[->,dashed](call_f_main_x) edge [out=32,in=112] (st_f_s);
    \path[->,dashed](end_f_s) edge [out=155,in=-40] (return_f_main_y);
    \path[->,dashed](end_f_s) edge [out=120,in=-50] (return_f_f_r);
    \path[->,dashed](end_f_r) edge [out=110,in=250] (return_f_f_r);
    \path[->,dashed](end_f_r) edge [out=150,in=-50] (return_f_main_y);
\end{tikzpicture}
}
  \caption{The exploded supergraph corresponding to a taint analysis for the input program in Listing~\ref{list:examplejava}}%
  \label{fig:exampleexploded}%
  \end{figure}
}
\end{example}



\mtodo{We never use the interpretation function, so I deleted it, too}
\commentout{
To convert a representation relation $R_f$ back into the original flow function $f$, we can use the \textit{interpretation} function $\denote {R_f}$:
\begin{align}
  f&=\denote{R_f}\notag\\
  &=\lambda D_1\,.\,\left(\left\{d_2\ |\ \exists d_1\in D_1:\,(d_1,\,d_2)\in R_f\right\}\cup\{d_2\ |\ (\mathbf0,\,d_2)\in R_f\}\right)
  \setminus\{\mathbf0\}\,.
\end{align}
}

A practical implementation of the IFDS algorithm generally takes as input a representation of the
exploded supergraph edges $E^\sharp$ instead of explicit dataflow functions $M_F$. This is convenient because
the two are equivalent in expressiveness, and because the IFDS algorithm works internally with the exploded
supergraph. More specifically, the input generally provides a function
$f: N^* \times D \times N^* \to 2^D$. Given a (non-exploded) supergraph edge $n \to n'$ and a dataflow fact $d$,
$f(n,d,n')$ returns the set of all $d'$ such that the exploded supergraph contains the edge $(n,d) \to (n',d')$.
For convenience, the function $f$ can be split up into separate functions that handle the cases when
the $n\to n'$ is an intraprocedural edge, a call-to-start edge, or an end-to-return edge.

%We presented an overview of the IFDS analysis. IFDS problems are transformed into IDE problems by the correlated-calls analysis. The IDE framework is described in the next section.

\subsection{IDE}\label{sec:bgide}
\commentout{
There exists an entire class of data-flow problems that cannot be formulated as IFDS problems. Informally, the problems cannot be formulated as decision problems. For instance, a constant-propagation problem asks, for each variable $v$ in the program, ``if $v$ is a constant at a given program point, what is $v$'s value?''.
The questions asked by constant propagation are of the form ``if property $X$ \textit{($v$ being a constant)} is true at program point~$Y$, what is the value of some property~$Z$ \textit{(the value of the constant)} corresponding to~$X$?''. It turns out that problems with such questions can often be solved by the IDE algorithm.

The IDE framework is an expressive extension to IFDS that was created by the same authors in 1996.
The problems that IDE can solve include, but are not limited to, IFDS problems~\cite{reps1995precise}. Just as the IFDS algorithm, the IDE algorithm is suitable for data-flow analyses that can be encoded with inter-procedural, distributive flow functions. However, in IDE, the domain of the flow functions is not restricted to sets $D$ of data-flow facts. The IDE domain of a flow function consists of \textit{environments} that map data-flow facts from the set $D$ to lattice elements.

As an example, in a constant propagation problem, an IDE environment would map each variable to the (possibly) constant value that it is bound to. To illustrate the distinction between IFDS and IDE we could say that IFDS can find out which variables in a program are constants, whereas IDE can additionally retrieve the values of the constant variables.
Instead of just telling us whether a fact holds or not, the IDE analysis can provide us with additional information about facts. 

Just as in the IFDS analysis, the IDE algorithm reduces a data-flow problem to a graph-reachability problem. Additionally, for each program point, the algorithm computes an \textit{environment} $\textsf{Env}(D,\,L)$, where data-flow facts are mapped to values of a lattice $L$.
}

The IDE algorithm~\cite{sagiv1996precise} extends IFDS to
\textit{inter-procedural} \textit{distributive} \textit{environment}
problems. An \textit{environment} problem is an analysis whose dataflow
lattice is the lattice $\textsf{Env}(D,L)$ of maps from a finite set $D$ to a meet semilattice
$L$ of finite height, ordered pointwise. Like IFDS, IDE requires the dataflow functions to be distributive.


\commentout{
\begin{mdelete}
For example, using the IDE analysis, we can encode a restricted version of a constant-propagation analysis\footnote{%
  In the general case, constant propagation cannot be encoded with distributive flow functions and is therefore not suitable for an IDE analysis~\cite{muller2001complexity}.
}. The data-flow facts  correspond to program variables, and the lattice incorporates all possible values for constants. 
If a fact $d$ in the exploded supergraph is reachable at node $n$, and $\textsf{Env}(d)\notin\{\bot,\,\top\}$, it means that the variable associated with $d$ is a constant. Furthermore, the value of the constant can be inferred from the environment for the corresponding node and is equal to $\textsf{Env}(d)$.
\end{mdelete}
}

The input to the IDE algorithm is
  $(G^*,\,D,\,L,\,M_\textsf{Env})$
where $G^*$ is a control-flow supergraph,
$D$ is a set of data-flow facts,
$L$ is a meet semilattice of finite height,
and $M_\textsf{Env}:\,E^*\to(\textsf{Env}(D,\,L)\to \textsf{Env}(D,\,L))$ assigns a dataflow function
to each supergraph edge.

The output of the IDE algorithm is, for each node $n$ in the supergraph,
the \textit{meet-over-all-valid-paths} solution
  $\mvp{\textsf{Env}}(n)=\bigsqcap_{q\in\ivp(n)}M_\textsf{Env}(q)(\top)$,
where $M_\textsf{Env}$ is extended from edges to paths by composition.

\commentout{
and
\begin{equation}
  \Omega=\lambda d\,.\,\top
\end{equation}
is the top element in the environment lattice~$\mathsf{Env}(D,\,L)$.
}

\commentout{
The IDE analysis is a generalization of the IFDS analysis: every IFDS problem can be converted into an equivalent IDE problem~\cite{reps1995precise}. The equivalent problem can be solved by the IDE algorithm, and the result converted into an IFDS result. In~an~IFDS-equivalent IDE problem, the~graph~$G^*$ and the~set~$D$ of~data-flow facts remain the~same. The $L$ lattice is a two-point lattice: if a fact is mapped to the top (bottom) element, then it is reachable (unreachable). The conversion between IFDS and IDE problems is discussed in detail in Section~\ref{seq:transIfdsIde}.
}

\commentout{
\subsubsection{Environment Transformers}
For each node in the control-flow graph, the result of an IDE analysis computes an environment $\textsf{Env}(D,\,L)$, which is a map from data-flow facts to lattice elements.

Instead of flow functions that show how to propagate facts, the IDE framework uses distributive environment transformers to propagate environments. For each edge $(n_1,\,n_2)$ in the control-flow supergraph, an environment transformer indicates how the environment at node $n_1$ is modified at node $n_2$.

From Section~\ref{sec:overviewifds} we know that flow functions can be represented with exploded-graph edges. To represent environment transformers, we will construct \textit{labeled} exploded-graph edges, where each edge is associated with a distributive \textit{micro function}%
\footnote{See~Sagiv et al.~\cite{sagiv1996precise} for a formal definition of the representation relation for environment transformers.}
$f:\,L\to L$. A~micro function shows how to change a lattice element for a given node and fact.

For instance, if an IDE problem is equivalent to an IFDS problem, the edges of the exploded supergraph are the same for both problems. In the IDE problem, the edges of the exploded supergraph are labeled with identity micro functions.

We extend the meet operator to work on micro functions by defining
\begin{equation}
  (f_1\sqcap f_2)(l)=f_1(l)\sqcap f_2(l)
\end{equation}
for all $l\in L$.
}

%In IDE problems, the auxiliary fact analogous to $\mathbf0$ in IFDS is denoted as $\Lambda$.

\commentout{
\begin{mdelete}
\begin{example}\label{ex:constprop}
  One version of the constant propagation analysis that can be encoded with IDE is \textit{linear constant propagation}. A linear constant propagation analysis can detect constants of the form $a\cdot x+b$, where $a$ and $b$ are integers and $x$ is a variable. In particular, a variable can only be considered constant if it depends on at most one other constant variable: even if $y$ and $z$ are variables that are considered constant, the variable $x=y+z$ will be considered not constant. If we encoded the analysis in a way to handle non-linear constant assignments, we would have to use non-distributive flow functions, which would violate the requirements of the IDE algorithm.
  
  For linear constant propagation, the $L$ lattice consists of the set of integers $\mathbb Z$, a top element denoting ``not a constant'', and a bottom element denoting an unknown value.
  The meet of two lattice elements is defined as follows: for any lattice element $l\in L$,
  \[
    \top\sqcap l=\top\qquad\text{and}\qquad\bot\sqcap l=l.
  \]
  For two lattice elements $l_1,\,l_2\in\mathbb Z$, $$l_1\sqcap l_2=\top.$$
  
  We define the addition and multiplication operations on lattice elements $l\in L$ and integers $c\in\mathbb Z$ as follows:
  \[
    l+c=\begin{cases}
      \bot&\text{if }l=\bot;\\
      \top&\text{if }l=\top;\\
      l+c&\text{otherwise.}
    \end{cases}
    \qquad
    c\cdot l=\begin{cases}
      \bot&\text{if }l=\bot;\\
      \top&\text{if }l=\top;\\
      c\cdot l&\text{otherwise.}
    \end{cases}
  \]
   Let the function $M$ that maps supergraph edges to environment transformers be defined in the following way:
  \[
    M=\lambda((n_1,\,n_2))\,.\,
    \begin{cases}
      \lambda\textsf{env}\,.\,\textsf{env}[x\to a\cdot\textsf{env}(y)+c\,]
        &\text{if $n_1$ contains an assignment}\\&\text{$x=a\cdot y+c$, where $y$ is a variable}\\&\text{and $a,\,c$ are constants;}\\
      \id
        &\text{otherwise.}
    \end{cases}
  \]
  Here, we denote with $\textsf{env}[x\to a]$ an environment \textsf{env} in which the key $x$ is mapped to $a$, and all other keys $y\ne x$ are mapped to their old values $\textsf{env}(y)$.
  When $M$ is applied to an edge whose source node contains an assignment for a variable $x$, $M$ returns an environment transformer that updates the argument environment with a new value for $x$. 

  Consider the following program:

\inputMinted{java}{constprop.java}
  For the edges $e_1,\,e_2,\,e_3$, and $e_4$ that start at the first, second, third, and fourth instruction, $M$ creates the following environment transformers:
  \begin{align*}
    M(e_1)
    &=\lambda\textsf{env}\,.\,\textsf{env}[\texttt u\to 1]\\
    M(e_2)
    &=\lambda\textsf{env}\,.\,\textsf{env}[\texttt v\to\textsf{env}(\texttt u)+2]\\
    M(e_3)
    &=\lambda\textsf{env}\,.\,\textsf{env}[\texttt w\to\top]\\
    M(e_4)
    &=\lambda\textsf{env}\,.\,\textsf{env}[\texttt u\to 5].
  \end{align*}
  
  The corresponding labeled exploded supergraph is shown in Figure~\ref{fig:constprop}.
  
  The result of the analysis yields a map from nodes to environments. Each environment maps variables to elements of the constant-propagation lattice. The environment at the last node will look as follows:
  \[
    \{(\texttt u,\,5),\ (\texttt v,\,3),\ (\texttt w,\,\top)\}.
  \]
\end{example}

\begin{figure}
  \centering
    \tikzstyle{n}=[fill=black,circle,inner sep=1.5pt,sm]
    \tikzstyle{sm}=[font=\footnotesize]
\begin{tikzpicture}[>=stealth,on grid, auto]
    \node [n,label=$\Lambda$] (zero){};
    \node [n] (zero_) [below=\edge of zero] {};
    \node [n,label={\texttt u}] (u) [right=\dist of zero] {};
    \node [n] (u_) [below=\edge of u] {};
    \node [n,label={\texttt v}] (v) [right=\dist of u] {};
    \node [n] (v_) [below=\edge of v] {};
    \node [n,label=\texttt{w}] (w) [right=\dist of v] {};
    \node [n] (w_) [below=\edge of w] {};
    \node [n] (zero__) [below=\edge of zero_]{};
    \node [n] (u__) [below=\edge of u_]{};
    \node [n] (v__) [below=\edge of v_]{};
    \node [n] (w__) [below=\edge of w_]{};
    \node [n] (zero___) [below=\edge of zero__]{};
    \node [n] (u___) [below=\edge of u__]{};
    \node [n] (v___) [below=\edge of v__)]{};
    \node [n] (w___) [below=\edge of w__]{};  
    \node [n] (zero____) [below=\edge of zero___]{};
    \node [n] (u____) [below=\edge of u___]{};
    \node [n] (v____) [below=\edge of v___)]{};
    \node [n] (w____) [below=\edge of w___]{};     
    
    \path[->](zero) edge node[left,sm]{$\id$} (zero_);
    \path[->](zero) edge node[right,sm]{$\lambda l.1$} (u_);
    \path[->](zero_) edge node[left,sm]{$\id$} (zero__);
    \path[->](u_) edge node[right,sm]{$\lambda l.l+2$} (v__);
    \path[->](u_) edge node[left,sm]{$\id$} (u__);
    \path[->](zero__) edge node[left,sm]{$\id$} (zero___);
    \path[->](u__) edge node[left,sm]{$\id$} (u___);
    \path[->](v__) edge node[left,sm]{$\id$} (v___);
    
    \path[->,dashed,color=gray](v) edge node[left,sm]{$\id$} (v_);
    \path[->,dashed,color=gray](w) edge node[left,sm]{$\id$} (w_);
    \path[->,dashed,color=gray](w_) edge node[right,sm]{$\id$} (w__);
    
    \path[->] (zero___) edge node[left,sm]{$\id$} (zero____);
    \path[->] (u___) edge node[left,sm]{$\lambda l.5$} (u____);
    \path[->] (v___) edge node[left,sm]{$\id$} (v____);
    \path[->,dashed,color=gray] (w___) edge node[left,sm]{$\id$} (w____);
  \end{tikzpicture}
  \caption[A labeled exploded supergraph for a constant-propagation analysis described in Example~\ref{ex:constprop}]{A labeled exploded supergraph for a constant-propagation analysis described in Example~\ref{ex:constprop}. The dashed edges are edges not reachable from the entry node.}%
  \label{fig:constprop}%
  \end{figure}
\end{mdelete}
}
  
\commentout{
In this way, each edge in the exploded graph is labeled with a micro function. The mapping from exploded-graph edges to the corresponding micro functions is stored in \textit{edge functions},  denoted as \textsf{EdgeFn}$:\,E^\#\to(L\to L)$.
}

\subsubsection{Overview of the IDE Algorithm}\label{sec:ideoverview}
Just as any distributive function from $2^D$ to $2^D$ can be represented with a
representation relation, it is also possible to represent any distributive functions from
$\textsf{Env}(D,L)$
to
$\textsf{Env}(D,L)$
with a \textit{pointwise representation}. A pointwise representation is a bipartite graph
with the same nodes~\footnote{The IDE literature uses the symbol $\Lambda$ for the node that
    is denoted $\mathbf0$ in the IFDS literature.}
and edges as a representation relation, except that each edge is labelled
with a \textit{micro-function}, which is a function from $L$ to $L$. 
\commentout{
The pointwise representation of an environment transformer 
$t : \textsf{Env}(D,L) \to \textsf{Env}(D,L)$
is defined as
\newcommand{\microedge}[3]{#1\xrightarrow{#3}#2}
\begin{align*}
    R_t =& \{ \microedge{\Lambda}{\Lambda}{\lambda l.l} \} \cup
    \{\microedge{\Lambda}{d_j}{\lambda l.t(\Omega)(d_j)} \mid d_j \in D\}\\
    &\cup
    \{\microedge{d_i}{d_j}{\lambda l.t(\Omega[d_i\to l])(d_j)} \mid d_i, d_j \in D\}.
\end{align*}
}
Let $\Omega = \lambda d.\top$ be the environment that maps every element of $D$ to $\top$.
Thanks to distributivity, every environment transformer 
$t : \textsf{Env}(D,L) \to \textsf{Env}(D,L)$
can be decomposed into its effect on $\Omega$ and on a set of environments $\Omega[d_i\to l]$
that map every element except one ($d_i$) to $\top$:
\[
    t(m)(d_j) = \lambda l. t(\Omega)(d_j) \sqcap \bigsqcap_{d_i\in D} \lambda l. t(\Omega[d_i\to l])(d_j).
\]
The functions $\lambda l. \cdots$ in this decomposition are the micro-functions that
appear on the edges of the pointwise representation edges from $\Lambda$ to each $d_j$ and from
each $d_i$ to each $d_j$.%
\footnote{The IDE paper defines a more complicated but equivalent set of micro-functions
that eliminate some duplication of computation.}
The absence of an edge in the pointwise representation from some $d_i$ to some $d_j$ is
equivalent to an edge with micro-function $\lambda l.\top$.

\otodo{Discuss the micro-functions in the example exploded supergraph in Figure 3.}

The meet of two environment transformers $t_1, t_2$ can be computed by taking the union of the edges
in their pointwise representations. When the same edge appears in the pointwise representations of
both $t_1$ and $t_2$, the micro-function for that edge in $t_1 \sqcap t_2$ is the meet of the
micro-functions for that same edge in $t_1$ and in $t_2$.

The composition of two environment transformers can be computed by combining their pointwise
representation graphs in the same way as IFDS representation relations, and 
computing the composition of the micro-functions appearing along each path in the resulting graph.

The IDE algorithm operates on the same exploded supergraph as the IFDS algorithm
(except that the edges are labelled with micro-functions). For each pair $(n,d)$ of
node and fact, the algorithm computes a micro-function equal to the meet of the micro-functions
of all the realizable paths from the program entry point to the pair.

In order to do this efficiently, the IDE algorithm requires a representation of micro-functions
that is general enough to express the basic micro-functions of the dataflow functions for individual
instructions, and that supports computing the meet and composition of micro-functions.


Similar to the IFDS algorithm, a practical implementation of the IDE algorithm requires
the input dataflow functions to be provided in their pointwise representation
as exploded supergraph edges labelled with micro-functions. Specifically,
the input is generally provided as a function $f: N^* \times D \times N^* \to (D \to F)$,
where $F$ is the set of representations of micro-functions from $L$ to $L$.
Given a (non-exploded) supergraph edge $n \to n'$ and a dataflow fact $d$,
$f(n,d,n')$ returns a map that gives for each fact $d'$ the micro-function $f$
that appears on the exploded supergraph edge $(n,d) \to (n',d')$.
Like in the IFDS algorithm, the function $f$ can be split up into separate functions that handle the cases when
the $n\to n'$ is an intraprocedural edge, a call-to-start edge, or an end-to-return edge.


\otodo{In the next section, we talk about \textsf{EdgeFn}, which is inconsistent with how
    we define the functions to be implemented here. We need to make the two sections consistent,
    one way or the other.}

\commentout{
Given a labeled exploded supergraph, the IDE algorithm computes the environments for all nodes in the control-flow graph.

The algorithm first computes the lattice elements $l_{n,\,d}$ that correspond to each reachable node $(n,\,d)$ in the exploded supergraph. The union of the exploded nodes $(n,\,d)$ for a given control-flow node $n$, mapped to the corresponding lattice elements $l_{n,\,d}$, form the environment $\textsf{Env}_n$  for that node:
\begin{equation}
  \textsf{Env}_n=\left\{(d,\,l_{n,\,d})\ |\ (n,\,d)\in N^\#\right\}\,.
\end{equation}

\mtodo{I think the rest of this section will not make sense to anyone who does not already understand it. Should I try to replace it with one paragraph that tries to give an intuitive understanding of what the IDE algorithm is doing?}
The overall idea behind computing the lattice elements $l_{n,\,d}$ is the following. For each inter-procedurally realizable path
\[
  p=\left[(\startmain,\,\Lambda),\,(n_1,\,d_1),\,\dots,\,(n_k,\,d_k)\right]
\]
 that starts with the entrypoint of the exploded supergraph, we compute the micro function $f_p$ that corresponds to $p$. The micro function consists of the composition of all individual micro functions with which the edges of $p$ are labeled:
\begin{align}
  f_p=\textsf{EdgeFn}((n_{k-1},\,d_{k-1}),\,(n_k,\,d_k))\,\circ\ldots\circ
      \textsf{EdgeFn}(\startmain,\,\Lambda),\,(n_1,\,d_1))\,.
\end{align}

Let the lattice element that $(n,\,d)$ is mapped to according to path $p$ be denoted as $l_{n,\,d}^p$. As shown in~Sagiv et al.~\cite{sagiv1996precise}, the lattice element can be obtained by applying $f_p$ to the bottom element:
\begin{equation}
 l^p_{n,\,d}=f_p(\bot)\,.
\end{equation}

Let $Q$ be the set of paths that start at the entry point and end at the given node $(n,\,d)$.
The lattice element $l_{n,\,d}$ is the meet of the lattice elements corresponding to all the paths in $Q$:
\[
  l_{n,\,d}=\bigsqcap_{q\in Q}l^q_{n,\,d}\,.
\]

This is a general outline of the IDE analysis. We use the IDE framework to improve the precision of IFDS problems in the presence of correlated calls.
The next section describes how this is done.
}

\section{Correlated Calls Analysis}\label{chapter:cca}
The correlated-calls analysis is presented as a transformation from an arbitrary IFDS problem to a corresponding IDE problem.

After solving the generated IDE problem, its result can be converted to an IFDS result. If the input program contains correlated calls, the converted IFDS result can be more precise than the original IFDS result.

In this section, we first discuss what is necessary to define IFDS and IDE problems. Next we describe how to convert any IFDS problem into an equivalent IDE problem, and, given a solution to the generated IDE problem, how to obtain the result of the original IFDS problem. We then show how to transform an IFDS problem into an IDE problem using the correlated-calls transformation, and how to convert the solution to the latter IDE problem into a more precise IFDS result.

\subsection{Defining IFDS and IDE Problems}
In Section~\ref{sec:bg}, we defined what IFDS and IDE problems are, their applications, and their constraints. In this section, we describe how to create instances of IFDS and IDE problems.

\subsubsection{Defining an IFDS Problem}\label{sec:ifdsdef}
Recall that an IFDS problem instance is defined as a five-tuple
\[
    (G^*,\ D,\ F,\ M_F,\,\sqcap)\,,
\]
where $G^*=(N^*,\,E^*)$ is the control-flow supergraph of the program, $D$ is the set of dataflow facts, $F\subseteq2^D\to2^D$ is a set of distributive dataflow functions, and the function $$M_F:\,E^*\to(2^D\to2^D)$$ maps the supergraph edges to dataflow functions, and is extended to paths by composition.

In practice, an IFDS problem can be defined by providing an exploded supergraph
$G^\#=(N^\#,\,E^\#)$. Each node of $G^\#$ is a pair $(n,\ d)$, where $n\in N^*$ is a node in the control-flow supergraph and $d\in (D\cup\{\mathbf{0}\}),\ \mathbf0\notin D$, where $\mathbf0$ is an auxiliary fact that is necessary for the IFDS algorithm.

The meaning of an edge in the exploded supergraph is the following.
Let $(n_1,\,d_1)$ and $(n_2,\,d_2)$ be two nodes in the exploded supergraph $G^\#$. Furthermore, assume that if fact $d_1$ at node~$n_1$ holds, then the fact~$d_2$ at node~$n_2$ also holds. Then there is an edge $(n_1,\,d_1),\,(n_2,\,d_2)\in E^\#$.

\subsubsection{Defining an IDE Problem}\label{sec:defide}
An IDE problem instance is a four-tuple
\[
    (G^*,\ D,\ L,\ \menv),
\]
where $G^*$ and $D$ are defined in the same way as for IFDS. $L$ is a finite-height lattice that represents the values to which dataflow facts are mapped in an IDE problem. An environment $Env(D, L)$ maps dataflow facts to lattice elements. Finally, the map $$M_{\textsf{Env}}:\,E^*\to(\textsf{Env}(D, L)\to \textsf{Env}(D, L))$$ is a function from the control-flow-supergraph edges to environment transformers, extended to paths by composition.

An IDE problem can be defined with a labeled exploded supergraph\footnote{
    The exploded supergraph in an IDE problem is defined in the same way as in an IFDS problem. The only difference is that the $\mathbf0$ fact is denoted as $\Lambda$~\cite{reps1995precise,sagiv1996precise}.
}, in which an edge function
\begin{equation}
  \edgefn:\ E^\#\to(L\to L)
\end{equation}
pairs edges with \textit{micro functions}, and is extended to paths by composition.

The set of micro functions of an IDE problem is a subset of $L\to L$ that is closed under function meet and composition.

The meaning of an edge in the labeled exploded supergraph is the following. Let $e=((n_1,\,d_1),\,(n_2,\,d_2))\in E^\#$ be an edge in the exploded supergraph with label $f=\mathsf{EdgeFn}(e)$. Then
\begin{itemize}
%	\item the fact that $d_1$ holds at node $n_1$ implies that $d_2$ holds at $n_2$;
  \item if at node $n_1$ the fact $d_1$ was mapped to a lattice element $l_1$ by an environment $Env(D,\,L)$, then the fact $d_2$ at node $n_2$ should be mapped to $f(l_1)$.
\end{itemize}

As shown in~Sagiv et al.~\cite{sagiv1996precise}, the relationship between environment transformers and edge functions can be described with the following equations. For individual edges $(n_1,\,n_2)\in E^*$,
\begin{align}\label{eq:envTransToEdgeFnEdge}
  \menv&((n_1,\,n_2))(\textsf{env})(d)\notag\\
  &=\edgefn((n_1,\,\Lambda),\,(n_2,\,d))(\top)\sqcap\bigsqcap_{d'\in D}\edgefn((n_1,\,d'),\,(n_2,\,d))(\textsf{env}(d'))\,,
\end{align}
where $\textsf{env}$ is an environment $\textsf{Env}(D,\,L)$. Informally, for a given control-flow-supergraph edge $e$ and data-flow fact $d$, the $M_\textsf{Env}$ function captures the meet of the edge function applied to all possible exploded-graph edges along $e$.

For paths $p$ that start with the entry point~$\startmain$,
\begin{equation}\label{eq:envTransToEdgefn}
  \menv(p)(\Omega)(d)=\bigsqcap_{r\in\mathsf{RP}(p,\,d)}\edgefn(r)(\top)\,,
\end{equation}
where $n\in N^*$, $d\in D$, $p\in\ivp(n)$, and $\mathsf{RP}$ is the set of all inter-procedurally realizable paths.

To summarize, an IDE problem can be defined by a labeled exploded supergraph 
\begin{equation}(G^\#,\,L,\,\mathsf{EdgeFn})\,,\end{equation}
where each edge of the exploded supergraph corresponds to a micro function.

\subsection{Transformations Between IFDS and IDE}\label{seq:transIfdsIde}

The correlated-call analysis transforms an existing IFDS problem into a special kind of IDE problem. We described what is necessary to define IFDS and IDE problems independently.

Let $P=(G^\#)$ be an IFDS problem and $Q=(G^\#,\,\edgefn)$ an IDE problem obtained by a conversion from $P$.

We will look at two kinds of transformations
\begin{equation}
  \mathcal T:\ (G^\#)\to (G^\#,\,\edgefn)
\end{equation}
from IFDS to IDE problems:
\begin{itemize}
	\item an equivalence transformation $\transEq$ (pronounced as ``t-equiv''), in which we show how to transform IFDS problems into equivalent IDE problems;
  \item a correlated-call transformation $\transCC$ (pronounced as ``t-c-c''), where we show how to convert IFDS problems into a special form of IDE problems that help eliminate infeasible paths.
\end{itemize}
In each case we also show how to convert the result of the generated IDE problem to a result of the original IFDS problem.

An overview of the transformations is shown in Figure~\ref{fig:transformations}.
\begin{figure}
  \centering
    \tikzset{
  ashadow/.style={opacity=.25, shadow xshift=0.07, shadow yshift=-0.07},
}
  \tikzstyle{problem}=[fill=greyblue,text width=2.3cm,rounded corners,font=\small,text=charcoal,drop shadow={ashadow, color=greyblue}]
  \tikzstyle{result}=[fill=bisque,text width=1.8cm,rounded corners,font=\small,text=charcoal,drop shadow={ashadow, color=greyblue}]
\begin{tikzpicture}
    \node [problem] (ifds) {IFDS problem};
    \node [problem] (equiv) [above right=0.9cm and .7\dist of ifds.east] {Equivalent IDE problem};
    \node [problem] (ccide)[below=3cm of equiv.west, anchor=west] {Correlated-calls IDE problem};
    \node [result] (equivres) [right=\dist of equiv.east] {Equivalence-IDE result};
    \node [result] (ccres) [below=3cm of equivres.west,anchor=west] {Correlated-calls result};
    \node [result] (ifdsres) [right=\dist of equivres.east] {IFDS result};
    \node [result] (improved) [right=\dist of ccres.east] {Improved IFDS result};
    \path[->] (equivres) edge node[above]{$\backEq$} (ifdsres);
    \path[->] (ccres) edge node[above] {$\backCC$} (improved);
    \path[->] (ccres) edge[out=30,in=240] node[above] {$\backEq$} (ifdsres);
    \path[->] (ifds) edge[out=40,in=190] node[above] {$\transEq$} (equiv);
    \path[->] (ifds) edge[out=-60,in=165] node[above] {$\transCC$} (ccide);
    \path[->] (equiv) edge node[above]{$\result$} (equivres);
    \path[->] (ccide) edge node[above]{$\result$} (ccres);
\end{tikzpicture}
  \caption{Transformations between IFDS and IDE problems and their results}%
  \label{fig:transformations}%
\end{figure}

\subsubsection{Equivalence Transformation}\label{sec:equivtrans}
We start with an equivalence transformation $\transEq$ to present a simple IFDS-to-IDE conversion that does not change the result of the original IFDS problem. We will compare the correlated-calls transformation with the equivalence transformation, and use the latter to show  that the correlated-calls analysis results in a
precision improvement of the original IFDS problem result.

\paragraph{Converting IFDS problems to IDE problems}
Since IDE is a generalization of IFDS, any IFDS problem can be converted into an equivalent IDE problem~\cite{sagiv1996precise}.
For an equivalence transformation $\transEq$, the generated lattice $L^\equiv$ consists of two elements, bottom and top:
\[
    L^\equiv=\{\bot,\ \top\}\,,
\]
where $\bot$ means ``reachable'', and $\top$ means ``not reachable''. 

 All micro functions are identity functions.

 
Given an exploded supergraph $G^\#$ provided by an IFDS problem, we want to create an edge function $\edgefn^\equiv$ that maps $G^\#$'s edges $E^\#$ to micro functions $L^\equiv\to L^\equiv$.

The edge functions $\edgefn^\equiv$ are defined as
\begin{equation}
    \edgefn^\equiv=
    \begin{cases}
      \lambda e\,.\,\lambda m\,.\,\bot&\text{if $d_1(e)=\Lambda$ and $d_2(e)\ne\Lambda$;}\\
      \lambda e\,.\,\id&\text{otherwise,}
    \end{cases}
\end{equation}
where $d_1(e)$ is the source fact of an edge $e$ and $d_2(e)$ is its target fact. At a ``diagonal'' edge from a $\Lambda$-fact to a non-$\Lambda$-fact $d$, the micro function is a constant function that returns $\bot$, which makes it a bottom element in the $L\to L$ lattice. Since the initial lattice element passed to the micro function at the start node is the top element (see~\eqref{eq:envTransToEdgefn}), the bottom function at the diagonal edge swaps the top element to bottom to make the fact $d$ reachable.

The resulting equivalence transformation looks as follows:
\begin{equation}
  \transEq((G^\#))=(G^\#,\,L^\equiv,\,\mathsf{EdgeFn}^\equiv).
\end{equation}

Thus, in $\transEq$, all non-diagonal edges in the original IFDS problem are mapped to identity functions.

\paragraph{Converting IDE Results to IFDS Results}

The output of an IFDS analysis states whether a node is reachable in the exploded supergraph. This means that for an IFDS problem $P$, the IFDS-analysis result $\result_{\text{IFDS}}(P):\ N^*\to 2^D$ is a map from nodes of the control-flow supergraph to sets of facts:
\begin{equation}
  \result_{\text{IFDS}}(P)=\{(n,\,\mvp F(n))\,|\,n\in N^*\}\,.
\end{equation}

\begin{example}\label{ex:ifdsresult}
  The solution to the taint-analysis IFDS problem $\mathcal P$ in Listing~\ref{list:examplejava} whose exploded supergraph is presented in Figure~\ref{fig:exampleexploded} looks as follows:
  \small\begin{align*}
    \result_\text{IFDS}(\mathcal P)=\{&(\highlight{\textsf{return}_\texttt{secret}}{greyblue},\,\{\texttt a\}),
      &(\highlight{\textsf{start}_\texttt f}{lightsalmonpink},\,\{\texttt s\}),
      &\qquad\qquad(\highlight{\textsf{return}_\texttt f}{lightsalmonpink},\,\{\texttt r,\,\texttt s\}),\\
%      
      &(\highlight{\textsf{call}_\texttt{A.f}}{greyblue},\,\{\texttt a\}),
      &(\highlight{\texttt{if(s==null)}}{lightsalmonpink},\,\{\texttt s\}),
      &\qquad\qquad(\highlight{\texttt{return r}}{lightsalmonpink},\,\{\texttt r,\,\texttt s\}),\\
%
      &(\highlight{\textsf{return}_\texttt{A.f}}{greyblue},\,\{\texttt a,\,\texttt b\}),
      &(\highlight{\textsf{call}_\texttt f}{lightsalmonpink},\,\{\texttt s\}),
      &\qquad\qquad(\highlight{\textsf{end}_\texttt f}{lightsalmonpink},\,\{\texttt r,\,\texttt s\})\}.\\
%
      &(\highlight{\textsf{end}_\texttt{main}}{greyblue},\,\{\texttt a,\,\texttt b\}),&(\highlight{\texttt{return s}}{lightsalmonpink},\,\{\texttt s\}),
  \end{align*}\normalsize
  All other nodes of the control-flow supergraph are mapped to the empty set.
\end{example}

 The IDE analysis associates a lattice element with each node in the exploded supergraph. For an IDE problem $Q$, the result $\result(Q):\ N^\#\to L$ maps nodes of the exploded supergraph to lattice elements (see~\eqref{eq:mvpdef}):
\begin{equation}\label{eq:ideresult}
  \result(Q)=\{((n,\,d),\,\mvp{\textsf{Env}}(n,\,d))\ |\ n\in N^*,\,d\in D\}\,.
\end{equation}
In other words, for each fact $d\in D$ at a given node $n\in N^*$, $\result(Q)(n,\,d)$ returns a lattice element. If a fact $d\in D$ is unreachable, $\result(Q)(n,\,d)=\top$.

In the case of an equivalence transformation from IFDS to IDE, if a node in the IFDS result is reachable, it will be also reachable in the IDE result, and it will be mapped to the bottom lattice element. For an exploded node in the IDE result, being mapped to the top element means being not reachable.

The~domain of an equivalence-IDE result 
\begin{equation}
  \resultEq=\result(\transEq(P))
\end{equation}
consists of pairs of control-flow-supergraph nodes and data-flow facts. The range of the result is the set of lattice elements. To transform an IDE result to an IFDS result, we need to map each control-flow-supergraph node to the set of facts with which it is paired, provided that the pair is mapped to the bottom lattice element.

\begin{example}\label{ex:ideresult}
  Converting the IFDS problem $\mathcal P$ from Example~\ref{ex:ifdsresult} into an equivalent IDE problem and solving it will yield the following result:
    \small\begin{align*}
    \result(\transEq(\mathcal P))=\{&((\highlight{\textsf{return}_\texttt{secret}}{greyblue},\,\texttt a),\,\bot),\\
      &((\highlight{\textsf{call}_\texttt{A.f}}{greyblue},\,\texttt a),\,\bot),\\
      &((\highlight{\textsf{return}_\texttt{A.f}}{greyblue},\,\texttt a),\,\bot),\\
      &((\highlight{\textsf{return}_\texttt{A.f}}{greyblue},\,\texttt b),\,\bot),\\
      &\dots\}.\\
    \end{align*}\normalsize
  Suppose that for a pair $(n,\,d)$, where $n\in N^*$ and $d\in D$, there is no corresponding result in ~$\result_\text{IFDS}(\mathcal P)$ (see Example~\ref{ex:ifdsresult}). Then $(n,\,d)$ appears in $\result(\transEq(\mathcal P))$ as $((n,\,d),\,\top)$.
\end{example}

Let $\rho$ be the result of an equivalence-IDE analysis for an IFDS problem $P$:
\[
  \rho=\result(\transEq(P)).
\]
For a node $n\in N^*$, let $D_n^\equiv(\rho)$ be a set of data-flow facts such that
\begin{equation}
  D_n^\equiv(\rho)=\{d\ |\ d\in D\,\wedge\,\rho(n,\,d)\ne\top\}\,.
\end{equation}
Then the transformation function
$\backEq:\,(N^\#\to L)\to(N^*\to2^D)$
from an IDE result to~an~IFDS result looks as follows:
\begin{equation}
  \backEq\left(\rho\right)=
    \left\{(n,\,D_n^\equiv(\rho))\ |\ n\in N^*\right\}\,.
\end{equation}

Obviously, if applied to the result of an equivalence-IDE problem, $\backEq$ returns a result equivalent to the original IFDS problem result. In other words, for any IFDS problem $P$ with supergraph $N^*$, and any node $n\in N^*$,
\begin{equation}
  \backEq\left(\result(\transEq(P))\right)(n)=\result_{\text{IFDS}}(P)(n)\,.
\end{equation}

\begin{example}
  Converting the result in Example~\ref{ex:ideresult} with the equivalence-transformation from an IDE result to an IFDS result $\backEq$ will yield the same result as in Example~\ref{ex:ifdsresult}.
\end{example}
\subsubsection{Correlated-Call Transformation}\label{sec:cctrans}

To improve the precision of an IFDS problem, we can convert it to a special type of IDE problem, and use lattice elements to provide us with additional information about a node.
When converting the IDE result to an IFDS result, lattice elements will tell us whether to make the corresponding exploded nodes reachable. This is the idea of the correlated-calls analysis.

\paragraph{Lattice Elements}
Just like in the equivalence transformation $\transEq$, the exploded supergraph for $\transCC$ is the same as in the original IFDS problem. The elements of the correlated-calls lattice $\lcc$ are functions that map receivers to sets of types:
\[
    \lcc=\left\{\,m:\ R\to2^T\right\},
\]
where $R$ is the set of receivers and $T$ is the set of all types in the program. 
The type power set $2^T$ is also a lattice with a bottom element 
\begin{align*}
\bot_T&=T
\intertext{and top element}
\top_T&=\varnothing.
\intertext{The top element of the function lattice}
\topcc&=\lambda r.\top_T
\intertext{is a function that maps any receiver to the empty set\footnote{%
  We prove that $\lcc$ is a lattice in the Proof of Lemma~\ref{lem:efficient}%
}. 
The bottom element}
\botcc&=\lambda r.\bot_T
\end{align*}
maps any receiver to all types in the program.

To understand the meaning of lattice elements in a correlated-call analysis, suppose that an IFDS problem has been converted to an IDE problem using the correlated-calls transformation. 
Assume also that $s$ is the entrypoint of the program, $n$ is a node in the exploded supergraph, and that in the IDE result, $n$ is mapped to a lattice element $l\in \lcc$. 
Then the purpose of $l$ is to provide information about the set of types of the objects that may be referenced by each receiver at runtime at a path from $s$ to $n$. 
If a receiver is mapped to the empty set $\top_T$, it means that for the given program point, the receiver cannot reference an object of any type.
In other words, the corresponding data-flow fact is considered not reachable.

\paragraph{Micro Functions}\label{sec:micro}
Unlike in the equivalence transformation, the micro functions returned by the edge function $\edgefn^\Subset$ are not always identity functions.

Let $e=(n_1,\,n_2)\in E^\#$ be an edge in the exploded supergraph.
$\edgefn^\Subset(e)$ returns a micro function $f\subset\lcc\to \lcc$.
Given a micro function (a map from receivers to sets of types) $m\in \lcc$, $f(m)$ returns a new map from receivers to sets of types.
In other words, $f$ shows how to update the map from receivers to sets of types when we encounter program point $n_1$.

Let $f_1$ and $f_2$ be two micro functions such that ${f_1=\lambda m\,.\,\lambda r\,.\,t_1(r)}$ and ${f_2={\lambda m\,.\,\lambda r\,.\,t_2(r)}}$. We define the meet operation on micro-functions as follows:
\begin{equation}\label{eq:micromeet}
  \lambda m\,.\,\lambda r\,.\,t_1(r)\sqcap\lambda m\,.\,\lambda r\,.\,t_2(r)
  =\lambda m\,.\,\lambda r\,.\,t_1(r)\cup t_2(r)\,.
\end{equation}

The composition of micro functions is defined as ordinary function composition.

\paragraph{Edge Functions}\label{sec:ef}

Let $\mathcal F$ be the set of methods in a program with a signature $s_\mathcal F$.
\begin{definition}
  Let $r.c()$ be a call site on a receiver $r\in R$ with runtime type $t\in T$.
  Let~$s_\mathcal F$ be the method signature corresponding to the call $c()$.
  For $s_\mathcal F$ and $t$, a \textit{lookup function} returns the method implementation $f\in\mathcal F$ to which the call $r.c()$ is dispatched:
  \begin{equation}
    \textsf{lookup}(s_\mathcal F,\,t)=f.
  \end{equation}
\end{definition}

\begin{definition}
  For a method signature $s_\mathcal F$ and a method implementation $f\in\mathcal F$, the static-type function $\tau$ returns the set of types for which the lookup function yields $f$:
  \begin{equation}
    \tau(s_\mathcal F,\,f)=\{\,t\ |\ \textsf{lookup}(s_\mathcal F,\,t)=f\}\,.
  \end{equation}
\end{definition}
In other words, $\tau$ computes the set of types for which calls to methods
with signatures~$s_\mathcal F$ are dispatched to~$f$.

If there is a supergraph path from a method call with signature $s_\mathcal F$ to the start of $f$, then the set $\tau(s_\mathcal F,\,f)$ is always non-empty.

\begin{definition}\label{def:momopoly}
  A call site is called monomorphic if it can be dispatched to only one method. If a call site can be dispatched to more than one method it is called polymorphic.
\end{definition}

  Let $r.c()$ be a call on a receiver $r\in R$ with a method signature $s_\mathcal F$ to a function $f\in\mathcal F$.
  If the call site is monomorphic, then $\tau(s_\mathcal F,\,f)$ contains all types $T'\subseteq T$ that are compatible with the static type of $r$.
  If the call site is polymorphic, then $\tau(s_\mathcal F,\,f)\subset T'$, since some types $t\in T'$ cause dispatch to a method other than $f$.

\begin{definition}\label{def:edgefn}
  For an edge $e$, let $n_1(e)$ and $n_2(e)$ be the source and target nodes of $e$, and $d_1(e)$ and $d_2(e)$ be its source and target facts. A correlated-call edge function for the set $S\subseteq R$ is defined as follows:
  \begin{equation}\label{eq:edgefndef}
    \ccedgefn S=\lambda e\,.\,
        \begin{cases}
          \id  &\text{if $d_1(e)=d_2(e)=\Lambda$},\\
          \lambda m\,.\,\varepsilon_S(e)(\botcc) &\text{if $d_1(e)=\Lambda$ and $d_2(e)\ne\Lambda$},\\
          \lambda m\,.\,\varepsilon_S(e)(m)  &\text{otherwise,}
        \end{cases}
  \end{equation}
  where $\varepsilon_S:\,E\to(L\to L)$ is a function defined as
  \begin{equation}\label{eq:varepsilon}
    \varepsilon_S=\lambda e\,.\,
        \begin{cases}
            \lambda m\,.\,m[r\to m(r)\cap \tau(s_\mathcal F,\,f)],
                &\text{if $e$ is a call-start edge. $r.c()$ is}\\
                &\text{the call site at $n_1(e)$, $f$ is the called}\\
                &\text{procedure with signature $s_\mathcal F$,}\\
                &\text{and $r\in S$;}\\
            \lambda m\,.\,m[r\to m(r)\cap \tau(s_\mathcal F,\,f)]
                &\text{if $e$ is an end-return edge.}\\
            \textcolor{white}{\lambda m\,.\,m}[v_1\to\bot_T]
                &\text{$v_1,\,\dots,\,v_k\in S$ are the local variables}\\
            \textcolor{white}{\lambda m\,.\,m}\dots
                &\text{in the callee method, $r.c()$ is the call}\\
            \textcolor{white}{\lambda m\,.\,m}[v_k\to\bot_T],
                &\text{corresponding to the return node}\\
                &\text{at $n_2(e)$, $f$ is the called method with}\\
                &\text{signature $s_\mathcal F$, and $r\in S$;}\\
            \lambda m\,.\,m\left[r\to \bot_T\right],
                &\text{if $n_1(e)$ contains an assignment}\\
                &\text{for $r\in S$;}\\
            \id
                &\text{otherwise.}
        \end{cases}
  \end{equation}
  We define both $\ccedgefn S$ and $\varepsilon_S$ to be extended to paths by composition.
\end{definition}
In the above definition, the purpose of the set $S$ is to limit the set of considered receivers. We will use $S$ in Section~\ref{sec:ccreceivers}.

The micro functions returned by a correlated-calls edge function can be described as follows. Along $\Lambda$-edges, the micro functions are identity functions. All other functions can be described with $\varepsilon_S$. On ``diagonal'' edges from $\Lambda$ facts to non-$\Lambda$ facts, $\varepsilon_S$ creates edge-specific mappings for a set of receivers, and maps all the other receivers to the set of all types $\bot_T$. On all other edges, $\varepsilon_S$ modifies the mappings for a set of receivers and leaves the mappings for the other receivers unchanged.

\begin{example}
  Consider the program Listing~\ref{list:ccexample}. The exploded supergraph for that program is shown in Figure~\ref{fig:cc_edgefn_example}.

  Returning a secret value in method \verb'A.foo' creates a ``diagonal'' edge from the $\Lambda$-fact to the secret fact $\psi$. 
  The diagonal edge is labeled with the micro function $\lambda m\,.\,\botcc$. Thus, at the end node of the method, every receiver is mapped to the set of all types~$\bot_T$.
  
On the end-return edge from \verb'A.foo' to \verb'main', we need to restrict the set of types for the receiver \verb'a' by labeling the end-return edge from the fact $\psi$ to the fact \verb'v' with the micro function $\lambda m\,.\,m[\texttt a\to m(\texttt a)\cap\{\texttt A\}]$.

Similarly, on the call-start edge from method \verb'main' to method \verb'B.bar', from fact \verb'v' to \verb's', we restrict the type of the receiver \verb'a' to the set \{\texttt B\} with the micro function~$\lambda m\,.\,m[\texttt a\to m(\texttt a)\cap\{\texttt B\}]$.

After we have shown the definitions for the meet and composition operations, we will show in Example~\ref{ex:cc} how the correlated-calls analysis uses the presented micro functions to detect infeasible paths.
  
\begin{figure}%
  \centering
    \tikzstyle{supergraph}=[
      fill=greyblue,
      rounded corners,
      font=\footnotesize,
      align=right,
      text=charcoal,
      drop shadow={ashadow, color=greyblue}
    ]
    \tikzstyle{supergraph_y}=[
      fill=bisque,
      rounded corners,
      font=\footnotesize,
      align=left,
      text=charcoal,
      drop shadow={ashadow, color=greyblue}
    ]
    \tikzstyle{supergraph_f}=[
      fill=lightsalmonpink,
      rounded corners,
      font=\footnotesize,
      align=left,
      text=charcoal,
      drop shadow={ashadow, color=greyblue}
    ]
    \tikzstyle{description}=[fill=white]
    \tikzstyle{n}=[fill=black,circle,inner sep=1.5pt]
    \tikzstyle{arrowtext}=[font=\tiny,color=black,above]
    
\hspace*{-10pt}
\begin{tikzpicture}
\scalebox{.8}{
% main method nodes
    \node [supergraph] (st_main) {\textsf{start}$_{\texttt{main}}$};
    \node [supergraph] (asgn_a) [below = \dist of st_main.south east, anchor=east] {\texttt{a = args==null\,?}\\\texttt{new\,A()\,:\,new\,B()}};
    \node [supergraph] (call_foo) [below = \dist of asgn_a.south east, anchor=east] {\textsf{call}$_\texttt{foo}$};
    \node [supergraph] (return_foo) [below = \dist of call_foo.south east,anchor=east] {\textsf{return}$_\texttt{foo}$\\\texttt{v = a.foo()}};
    \node [supergraph] (call_bar) [below = \dist of return_foo.south east, anchor=east] {\textsf{call}$_\texttt{bar}$};
    \node [supergraph] (return_bar) [below = \dist of call_bar.south east, anchor=east] {\textsf{return}$_\texttt{bar}$};
    \node [supergraph] (end_main) [below = \dist of return_bar.south east, anchor=east] {\textsf{end}$_{\texttt{main}}$};
    
    \node [n,label=$\Lambda$] (st_main_z) [right=\nd of st_main.east] {};
    \node [n,label=\texttt{v}] (st_main_v) [right=\nd of st_main_z] {};
    \node [n] (asgn_a_z) [right=\nd of asgn_a.east] {};
    \node [n] (call_foo_z) [right=\nd of call_foo.east] {};
    \node [n] (return_foo_z) [right=\nd of return_foo.east] {};
    \node [n] (call_bar_z) [right=\nd of call_bar.east] {};
    \node [n] (return_bar_z) [right=\nd of return_bar.east] {};
    \node [n] (end_main_z) [right=\nd of end_main.east] {};
    \node [n] (asgn_a_v) [right=\nd of asgn_a_z] {};
    \node [n] (call_foo_v) [right=\nd of call_foo_z] {};
    \node [n] (return_foo_v) [right=\nd of return_foo_z] {};
    \node [n] (call_bar_v) [right=\nd of call_bar_z] {};
    \node [n] (return_bar_v) [right=\nd of return_bar_z] {};
    \node [n] (end_main_v) [right=\nd of end_main_z] {};
% main method edges
    \path[->](st_main_z) edge (asgn_a_z);
    \path[->](asgn_a_z) edge (call_foo_z);
    \path[->](call_foo_z) edge (return_foo_z);
    \path[->](return_foo_z) edge (call_bar_z);
    \path[->](call_bar_z) edge (return_bar_z);
    \path[->](return_bar_z) edge (end_main_z);  
    \path[->](st_main_v) edge (asgn_a_v);
    \path[->](asgn_a_v) edge (call_foo_v);
    \path[->](call_foo_v) edge (return_foo_v);
    \path[->](return_foo_v) edge (call_bar_v);
    \path[->](call_bar_v) edge (return_bar_v);
    \path[->](return_bar_v) edge (end_main_v);  
    
%A.foo method nodes
    \node [supergraph_y] (st_a_foo) [right = 6*\dist of st_main] {\textsf{start}$_{\texttt{A.foo}}$};
    \node [supergraph_y] (return_secret) [below = \dist of st_a_foo.south west, anchor=west] {\texttt{return secret()}};
    \node [supergraph_y] (end_a_foo) [below = \dist of return_secret.south west,anchor = west] {\textsf{end}$_\texttt{A.foo}$};
    
    \node [n,label=$\psi$] (st_a_foo_p) [left=\nd of st_a_foo.west] {};
    \node [n] (return_secret_p) [left=\nd of return_secret.west] {};
    \node [n] (end_a_foo_p) [left=\nd of end_a_foo.west] {};
    \node [n,label=$\Lambda$] (st_a_foo_z) [left=\nd of st_a_foo_p] {};
    \node [n] (return_secret_z) [left=\nd of return_secret_p] {};
    \node [n] (end_a_foo_z) [left=\nd of end_a_foo_p] {};
%A.foo method edges
    \path[->](st_a_foo_z) edge (return_secret_z); 
    \path[->](return_secret_z) edge node[arrowtext,near start,above right=0.4cm and -.15cm,rotate=-60]{$\lambda m.\botcc$} (end_a_foo_z); 
    \path[->,dashed](call_foo_z) edge [out=38,in=238]  (st_a_foo_z);
    \path[->,dashed](end_a_foo_z) edge [out=200,in=43] (return_foo_z);
    \path[->,dashed](end_a_foo_p) edge  [out=200,in=43] node[arrowtext,below,rotate=30]{$\lambda m.m[\texttt a\to m(\texttt a)\cap\{\texttt A\}]$} (return_foo_v);
    \path[->](st_a_foo_p) edge (return_secret_p); 
    \path[->](return_secret_p) edge (end_a_foo_p);
    \path[->](return_secret_z) edge (end_a_foo_p);
    
%A.bar method nodes
    \node [supergraph_y] (st_a_bar) [right = 6*\dist of st_a_foo] {\textsf{start}$_{\texttt{A.bar}}$};
    \node [supergraph_y] (end_a_bar) [below = \dist of st_a_bar.south west,anchor = west] {\textsf{end}$_\texttt{A.bar}$};
    
    \node [n,label=\texttt{s}] (st_a_bar_s) [left=\nd of st_a_bar.west] {};
    \node [n,label=$\Lambda$] (st_a_bar_z) [left=\nd of st_a_bar_s] {};
    \node [n] (end_a_bar_s) [left=\nd of end_a_bar.west] {};    
    \node [n] (end_a_bar_z) [left=\nd of end_a_bar_s] {};
%A.bar method edges
    \path[->](st_a_bar_z) edge (end_a_bar_z);
    \path[->,dashed](call_bar_z) edge [out=20,in=229] (st_a_bar_z);
    \path[->,dashed](end_a_bar_z) edge [out=220,in=35] (return_bar_z);
    \path[->](st_a_bar_s) edge (end_a_bar_s);
    \path[->,dashed](call_bar_v) edge [out=20,in=230] node[arrowtext,near end,above left=-.1cm and -.15cm,rotate=36]{$\lambda m.m[\texttt a\to m(\texttt a)\cap\{\texttt A\}]$} (st_a_bar_s);


%B.foo method nodes
    \node [supergraph_f] (st_b_foo) [right = 6*\dist of call_bar] {\textsf{start}$_{\texttt{B.foo}}$};
    \node [supergraph_f] (return_not_secret) [below = \dist of st_b_foo.south west, anchor=west] {\texttt{return "not secret"}};
    \node [supergraph_f] (end_b_foo) [below = \dist of return_not_secret.south west,anchor = west] {\textsf{end}$_\texttt{B.foo}$};
    
    \node [n,label=$\Lambda$] (st_b_foo_z) [left=\nd of st_b_foo.west] {};
    \node [n] (return_not_secret_z) [left=\nd of return_not_secret.west] {};
    \node [n] (end_b_foo_z) [left=\nd of end_b_foo.west] {};
    
%B.foo method edges
    \path[->](st_b_foo_z) edge (return_not_secret_z); 
    \path[->](return_not_secret_z) edge (end_b_foo_z); 
    \path[->,dashed](call_foo_z) edge [out=-60,in=160,color=lightsalmonpink] (st_b_foo_z);
    \path[->,dashed](end_b_foo_z) edge [out=145,in=-65,color=lightsalmonpink] (return_foo_z);
    
%B.bar method nodes
    \node [supergraph_f] (st_b_bar) [right = 6*\dist of st_b_foo] {\textsf{start}$_{\texttt{B.bar}}$};
    \node [supergraph_f] (print) [below = \dist of st_b_bar.south west,anchor = west] {\texttt{print(s)}};
    \node [supergraph_f] (end_b_bar) [below = \dist of print.south west,anchor = west] {\textsf{end}$_\texttt{B.bar}$};
    
    \node [n,label=\texttt{s}] (st_b_bar_s) [left=\nd of st_b_bar.west] {};
    \node [n] (print_s) [left=\nd of print.west] {};
    \node [n] (end_b_bar_s) [left=\nd of end_b_bar.west] {};
    \node [n,label=$\Lambda$] (st_b_bar_z) [left=\nd of st_b_bar_s] {};
    \node [n] (print_z) [left=\nd of print_s] {};
    \node [n] (end_b_bar_z) [left=\nd of end_b_bar_s] {};
%B.bar method edges
    \path[->](st_b_bar_z) edge (print_z);
    \path[->](print_z) edge (end_b_bar_z);
    \path[->](st_b_bar_s) edge (print_s);
    \path[->](print_s) edge (end_b_bar_s);
    \path[->,dashed](call_bar_z) edge [out=40,in=140,color=lightsalmonpink] (st_b_bar_z);
    \path[->,dashed](call_bar_v) edge [out=20,in=160,color=lightsalmonpink] node[arrowtext,above right=-.4cm and .4cm,rotate=-10]{$\lambda m.m[\texttt a\to m(\texttt a)\cap\{\texttt B\}]$} (st_b_bar_s);
    \path[->,dashed](end_b_bar_z) edge [out=210,in=-40,color=lightsalmonpink] (return_bar_z);    
}
\end{tikzpicture}
  \caption[An example program demonstrating correlated-call edge functions on the $\Lambda$-node path for Listing~\ref{list:ccexample}]{An example program demonstrating correlated-call edge functions on the $\Lambda$-node path for Listing~\ref{list:ccexample}. All non-labeled edges are implicitly labeled with identity functions $\id$. The variable corresponding to an initial secret value is denoted as $\psi$.}%
  \label{fig:cc_edgefn_example}%
  \end{figure}
\end{example}

\begin{definition}
  For an IFDS problem $P=(G^\#)$ and a set $S$, the correlated-calls transformation $\transCC_S$ is defined as
  \begin{equation}
    \transCC_S((G^\#))=\left(G^\#,\,\lcc_S,\,\ccedgefn S\right),
  \end{equation}
  where $\lcc_S:\,S\to2^T$.
\end{definition}

Then, for an edge $e$, the correlated-call micro functions can be defined as $\ccedgefn R$ and a correlated-calls transformation is defined as $\transCC_R$.

\paragraph{Converting IDE Results to IFDS Results}

Let $P$ be an IFDS problem. 
Let $E:\,N\times D$ be the domain of the IDE result $\result(Q)$.
To convert $\result(\transCC_R(P))$ to an IFDS result, we need to map the control-flow-supergraph nodes $n\in N^*$ to the corresponding facts $d\in D$. 
Unlike in~$\backEq$, we will only map each~$n$ to the facts~$d$ for~which~$\result(\transCC_R(P))(n,\,d)$ does not contain an~empty mapping for~any~receiver. 

For a node $n\in N^*$ and a correlated-calls IDE problem result $\rho=\result(\transCC_S(P))$, let $D_n^\Subset(\rho)$ be a set of data-flow facts defined as
\begin{equation}\label{eq:dnq}
  D_n^\Subset(\rho)=\left\{d\ |\ 
    d\in\mvp F(n)\,\wedge\,\forall r\in R:\ \rho(n,\ d)(r)\ne\top_T\right\}.
\end{equation}
Then, for a set $S\subseteq R$, the~correlated-calls-conversion function from a~correlated-calls IDE~result $\rho$ to~an~IFDS~result looks as~follows:
\begin{equation}\label{eq:ucc}
  \backCC\left(\rho\right)=
    \left\{(n,\,D_n^\Subset(\rho)\ | \ n\in N^*\right\}.
\end{equation}

In the following lemma we show that the result of an IDE problem obtained through a correlated-calls transformation is a subset of the original IFDS result.

\begin{lemma}[Precision]\label{lem:subsetifds}
    For an IFDS problem $P$ and all ${n\in N^*}$,
    \begin{equation}\label{eq:correct}
      \backCC\left(\result(\transCC_R(P))\right)(n)
      \subseteq
      \result_{\text{IFDS}}(P)(n)\,.
    \end{equation}
\end{lemma}
\begin{proof}
  The transformation $\backCC$ is the same as $\backEq$, except that it can remove data-flow facts from the result:
  \begin{align*}
    \backCC\left(\result(\transCC_R(P))\right)(n)&=\{(n',\,D_n'^\Subset(\result(\transCC_R(P))))\ |\ n\in N^*\}(n)\\
      &=D_n^\Subset(\result(\transCC_R(P)))\\
      &\subseteq\mvp F(n)\\
      &=\result_\text{IFDS}(P)(n)\,.\qedhere
  \end{align*}
\end{proof}

We will next show, in Lemma~\ref{lem:sound}, that our analysis is sound, i.e. that the result of an IDE problem obtained through a correlated-calls transformation removes only facts that occur on infeasible paths. To prove the Soundness Lemma, we first introduce Lemmas~\ref{lem:sound1} and~\ref{lem:sound3}.

We will denote the top element in the environment lattice as
\begin{equation}
  \Omega=\lambda d\,.\,\topcc\,.
\end{equation}

For the purpose of the proofs, we will rewrite Equation~\eqref{eq:edgefndef} that defines an edge function as follows:
  \begin{equation}\label{eq:edgefnThroughDelta}
    \ccedgefn S=\lambda e\,.\,
    \begin{cases}
      \id&\text{if $d_1=d_2=\Lambda$,}\\
      \lambda m\,.\,\varepsilon(e)(\delta(m))&\text{otherwise},
    \end{cases}
  \end{equation}
  where $S\subseteq R$, $d_1$ and $d_2$ are the source and target facts, and for a map $m\in\lcc_U$, $\delta(m)$ is either $m$ or $\botcc$:
  \begin{equation}\label{eq:deltadef}
    \delta(m)=\begin{cases}
      \botcc&\text{if $d_1=\Lambda$}\\
      m&\text{otherwise.}
    \end{cases}
  \end{equation}

Additionally, for a path $p=[\startmain,\,\dots]$ and a fact $d\in D$, we will denote the lattice element that is mapped to $d$ according to the flow functions of path $p$ as follows:
\begin{equation}
  \mpd=\menv(p)(\Omega)(d)\,.
\end{equation}

The following Lemma shows that the lattice elements (receiver-to-types maps) of a correlated-calls IDE analysis correctly overapproximate the possible types of a receiver in a program execution.

\begin{lemma}\label{lem:sound1}
Let $p=[\startmain,\,\dots,\,n]$ be some concrete execution trace of the program, and let $r\in R$ be a receiver. If after the execution trace $p$, at node $n$, $r$ points to an object of runtime type~$t$, and $d\in D$ is a fact such that $d\in M_F(p)(\varnothing)$, then
  \begin{equation}
    t\in\mpd(r)\,.
  \end{equation}
\end{lemma}
\begin{proof}
  By induction on the length of the trace.
  
  \textit{Basis:} $p=[\startmain]$. Then there is no instruction at which a receiver $r$ could be instantiated, and the Lemma is trivially true.    

  \textit{Induction hypothesis:} Let $p=[\startmain,\,\dots,\,n_{k-1}]$, and let $\uptau$ be the set of types to which $\mpdkm$ maps $r$:
  \begin{equation}
    \uptau=\mpdkm(r)\,.
  \end{equation}
  Assume that for a concrete execution path $p=[\startmain,\,\dots,\,n_{k-1}]$, at node $(n_{k-1},\,d_{k-1})$, the Lemma holds, i.e. $t\in\uptau$.
  
  \textit{Induction step:} Let $p'=[\startmain,\,\dots,n_{k-1},\,n_k]$ and $t'\in T$ be the type to which $r$ is mapped at $n_k$.
  
  For each $i$, let $e_i$ be the edge $((n_{i-1},\,d_{i-1}),\,(n_i,\,d_i))$. Note that $$e_1=((\startmain,\,\Lambda),\,(n_1,\,d_1))\,.$$  

  Observe that
  \begin{align*}
    \mppd
    &=\menv(p')(\Omega)(d)\\
    &=\left(\menv(e_k)\circ\menv(e_{k-1})\circ\ldots\circ\menv(e_1)\right)(\Omega)(d)\\
    &=\menv(e_k)\left(\menv(p)(\Omega)\right)(d)\,.
  \end{align*}
  
  According to~\eqref{eq:envTransToEdgeFnEdge},
  \begin{align*}
    &\menv(e_k)\left(\menv(p)(\Omega)\right)(d)(r)\\
      =&\bigg(\ccedgefn R((n_{k-1},\,\Lambda),\,(n_k,\,d))(\topcc)\sqcap\\
       &\bigsqcap_{d'\in D}\ccedgefn R((n_{k-1},\,d'),\,(n_k,\,d))(\menv(p)(\Omega)(d'))\bigg)(r)\\
      \supseteq&
        \bigsqcap_{d'\in D}\ccedgefn R((n_{k-1},\,d'),\,(n_k,\,d))(\menv(p)(\Omega)(d'))(r)\\
      \supseteq&\,\ccedgefn R((n_{k-1},\,d_{k-1}),\,(n_k,\,d))(\mpdkm)(r)\,.
  \end{align*}
  Therefore, 
  \begin{equation}\label{eq:musubset}
    \efek\subseteq\mppd(r)\,.
  \end{equation}
  
  We will now show that
  \[
    t'\in\efek\,,
  \]
  which, due to~\eqref{eq:musubset}, means that the Lemma holds.
  
  According to~\eqref{eq:edgefnThroughDelta}, there are two cases in which $\ccedgefn R(e_k)$ could fall.

  If $d_{k-1}=d_k=\Lambda$, then $d_k\notin M_F(p)(\varnothing)$, since it does not belong to the set $D$, and the Lemma trivially holds.
  
  Otherwise, 
  \[
    \ccedgefn R(e_k)=\lambda m\,.\varepsilon(e_k)(\delta(m))\,.
  \]
  It follows that
  \begin{align}\label{eq:efek}
    \efek
    &=(\lambda m\,.\varepsilon(e_k)(\delta(m)))(\mpdkm)(r)\notag\\
    &=\varepsilon(e_k)(\delta(\mpdkm))(r).
  \end{align}
  Let us denote the lattice element $\delta(\mpdkm)$ with $\Delta$:
  \[
    \Delta=\delta(\mpdkm)\,.
  \]
  Note that since $\Delta$, according to~\eqref{eq:deltadef}, can be either $\botcc$ or $\mpdkm$, it always maps $r$ to a set containing~$t$:
  \begin{equation}\label{eq:deltaContainsT}
    t\in\Delta(r)\,.
  \end{equation}
  Note also that unless the instruction at $n_{k-1}$ contains an assignment for $r$, $r$ is mapped to the same object of type $t$ as at node $n_{k-1}$, and $t=t'$. Therefore, for the non-assignment instructions, it is sufficient to prove that $t\in\Delta(r)$.
 
   Depending on the instructions at the nodes $n_{k-1}$ and $n_k$, there are four cases:
  \begin{enumerate}
    \item\label{item:asgn} The instruction at $n_{k-1}$ is an assignment for a receiver $r'\in R$.
        Since $\varepsilon_R(e_k)=\lambda m\,.\,m[r'\to\bot_T]$,
        \begin{align*}
          \efek&=            
            (\lambda m\,.\,m[r'\to\bot_T])(\Delta)(r)\\
          &=\Delta[r'\to\bot_T](r)\,.
        \end{align*}
      In the resulting map, $r'$ is mapped to $\bot_T$. Then
      
      \begin{enumerate}
        \item if $r=r'$, then $\efek=\bot_T$, which contains $t'$.
        \item\label{item:defaultmap} If $r\ne r'$, then $r$ has not been reassigned a value, and still maps to the same object of type $t$. The receiver $r$ is mapped to $\Delta(r)$, which, according to~\eqref{eq:deltaContainsT}, contains $t$. Since $t=t'$, $\Delta(r)$ contains $t'$.
      \end{enumerate}
    \item\label{item:callstart} $e_k$ is a call-start edge with signature $s_\mathcal F$, and $f\in\mathcal F$ is the called procedure.
      Then
        \begin{align*}
          \efek
          &=(\lambda m\,.\,m[r'\to m(r')\cap\tau(s_\mathcal F,\,f)])(\Delta)(r)\\
          &=\Delta[r'\to\Delta(r')\cap\tau(s_\mathcal F,\,f)]\,,
        \end{align*}
      where $r'$ is the receiver of the call.
      \begin{itemize}
        \item If $r'=r$, then $\Delta(r')=\Delta(r)$ which contains $t$. Since $t\in\tau(s_\mathcal F,\,f)$, it follows that $t\in\Delta(r)\cap\tau(s_\mathcal F,\,f)$, and $t\in\efek$.
         \item If $r'\ne r$, see~(\ref{item:defaultmap}).
       \end{itemize}
    \item $e_k$ is an end-return edge, $r_1,\,\dots,\,r_k\in R$ are the local variables in the callee method, $r'$ is the receiver of the call site corresponding to the return node $n_k$, and $f\in\mathcal F$ is the called method with signature~$s_\mathcal F$.
      Then 
      \[
        \varepsilon_R(e_k)=\lambda m\,.\,m[r'\to m(r')\cap\tau(s_\mathcal F,\,f)][r_1\to\bot_T]\dots[r_k\to\bot_T].
      \]
      If $r\in\{r_1,\,\dots,\,r_k\}$, see~Case~\ref{item:asgn}. Otherwise, the case is analogous to Case~\ref{item:callstart}.
    \item\label{item:idcase} The node contains any other instruction.
      Then 
      \[
        \ccedgefn R(e_k)(\mpdkm)(r)=\id(\Delta)(r)=\Delta(r),
      \]
      which contains $t$ according to~\eqref{eq:deltaContainsT}.\qedhere
  \end{enumerate}
\end{proof}

We will now show that on a node of a concrete execution path, the correlated-calls analysis does not map receivers to $\top_T$. In other words, the analysis never considers nodes of a concrete execution path unreachable.

\begin{lemma}\label{lem:sound3}
  Let $p=[\startmain,\,\dots,\,n]$ be a concrete execution path, $r\in R$ a receiver, and $d\in D$ a data-flow fact. Then if $d\in M_F(p)(\varnothing)$,
  \begin{equation}
    \mpd(r)\ne\top_T\,.
  \end{equation}
\end{lemma}
\begin{proof}
  By induction on the length of the execution trace.
  
  \textit{Basis:} 
    Let $p=[\startmain]$. Since the only realizable path corresponding to $p$ is $[(\startmain,\,\Lambda)]$, there is no fact $d\in D$ such that $d\in M_F(p)(\varnothing)$, and the claim follows immediately.
  
  \textit{Induction hypothesis:} 
  Let $p=[\startmain,\,\dots,\,n_{k-1}]$. Let $\uptau$ be the set of types to which $r$ is mapped by $\mpdkm$:
  \begin{equation}
    \uptau=\mpdkm(r)\,.
  \end{equation}
  Assume the Lemma holds for that for a concrete execution path $$p=[\startmain,\,n_1,\,\dots,\,n_{k-1}]\,,$$ i.e. $\uptau\ne\top_T$ for an arbitrary $r\in R$ and $d_{k-1}\in D$.
  
  \textit{Induction step:}
    Let $p'=[\startmain,\,n_1,\,\dots,n_{k-1},\,n_k]$ be a concrete execution path.
    
    Let $e_k=((n_{k-1},\,d_{k-1}),\,(n_k,\,d))$. As shown in~\eqref{eq:musubset},
    \begin{align*}
      \mppd(r)
        &\supseteq\ccedgefn R(e_k)(\mpdkm)(r)\,.
    \end{align*}

    From Definition~\ref{def:edgefn}, we can see that unless $e_k$ is a call-start edge or an end-return edge, the result follows from the induction hypothesis.
    More formally, if $e_k$ is not a call-start or end-return edge, then for all $m\in\lcc_R$,
    \[
      \ccedgefn R(e_k)(m)\sqsubseteq m\,.
    \]
     The edge function corresponding to the call-start and end-return edges is the only place in which the set of types that a receiver maps to can be reduced.
    
    Assume that $e_k$ is a end-return edge with a call on the receiver $r'\in R$ with a signature $s_\mathcal F$ to a function $f\in\mathcal F$.
    \begin{align*}
      \ccedgefn R&(e_k)(\mpdkm)(r)\\
      &=\left(\lambda m\,.\,m[r'\to m(r)\cap\tau(s_\mathcal F,\,f)][r_1\to\bot_T]\dots[r_l\to\bot_T]\right)(\mpdkm)(r)\\
      &=\left(\mpdkm[r'\to \uptau\cap\tau(s_\mathcal F,\,f)][r_1\to\bot_T]\dots[r_l\to\bot_T]\right)(r)\,,
    \end{align*}
  where $r_1,\,\dots,r_l\in R$ are the local variables in the called method.
    
    If $r\in\{r_1,\,\dots,\,r_l\}$, then $\efek=\bot_T\ni t$\footnote{In the case of a recursive call, it is possible that both $r\in\{r_1,\,\dots,\,r_l\}$ and $r=r'$.
    In that case, the set to which $r$ will be mapped would be still ``overwritten'' by $\bot_T$.}.
    
    Otherwise, if $r=r'$, then $\efek=\uptau\cap\tau(s_\mathcal F,\,f)$.

    According to Lemma~\ref{lem:sound1} and by the induction hypothesis, the runtime type $t$ of $r$ must be contained in $\mpdkm(r)=\uptau$. At the same time, by definition, $t$ is part of $\tau(s_\mathcal F,\,f)$. Therefore, $t\in\uptau\cap\tau(s_\mathcal F,\,f)\subseteq\efek$, which means that $\efek\ne\top_T$.
    
    The same reasoning applies to the case where $e_k$ is a call-start edge.
\end{proof}

Finally, we will prove the soundness of the correlated-calls analysis: we will show that our analysis only considers a path infeasible if it cannot occur in a concrete execution of a program.

\begin{lemma}[Soundness]\label{lem:sound}
  Let $p=[\startmain,\,\dots,\,n]$ be a concrete execution path, and let $d\in D$.
  If $d\in M_F(p)(\varnothing),$
  then
  \begin{equation}
    d\in \backCC\left(\result(\transCC_R(P))\right)(n)\,.
  \end{equation}
\end{lemma}
\begin{proof} Let $\rho=\result(\transCC_R(P))$. Then
  \begin{align*}
    \backCC(\rho)(n)
    &=D_n^\Subset(\rho)\\
    &=\left\{d'\ |\ 
        d'\in\mvp F(n)\,\wedge\,\forall r\in R:\ \rho(n,\ d')(r)\ne\top_T\right\}.
  \end{align*}
  Since $\mvp F(n)=\bigsqcap_{q\in\ivp(n)}M_F(q)(\varnothing)$, and $p\in\ivp(n)$, it follows that
    \begin{align*}
      d&\in M_F(p)(\varnothing)\\&\subseteq\mvp F(n)\,.
    \end{align*}
  At the same time, for all receivers $r\in R$,
  \begin{align*}
    \rho(n,\,d)(r)
    &=\left(\bigsqcap_{q\in\ivp(n)}\mpddef q d\right)(r)\\
    &=\bigsqcap_{q\in\ivp(n)}\mpddef q d(r)\,.
  \end{align*}
  According to~Lemma~\ref{lem:sound3}, $\mpd(r)\ne\top_T$.
  Since $p\in\ivp(n)$,
  \begin{align*}
    \mpd(r)\subseteq\bigsqcap_{q\in\ivp(n)}\mpddef qd(r)\,.
  \end{align*}
  From $\bigsqcap_{q\in\ivp(n)}\mpddef qd(r)=\rho(n,\,d)(r)$ it follows that
  $\mpd(r)\subseteq\rho(n,\,d)(r)$.
  Therefore, $\rho(n,\,d)(r)\ne\top_T$, and $d\in D_n^\Subset(\rho)=\backCC(\rho)(n)$.
\end{proof}

\paragraph{Correlated-Call Receivers}\label{sec:ccreceivers}
We will now show that in a correlated-calls transformation, it is enough to consider only some of the receivers of set $R$.

\begin{definition}
Let $c_1$ and $c_2$ be two call sites on a receiver $r\in R$.
  If both call sites are polymorphic, then we say that $r$ is a \textit{correlated-call receiver}.
\end{definition}
In other words, a correlated-call receiver is a receiver that has at least two polymorphic call invocations.
We will denote the set of correlated-call receivers as $\rcc$.

We will describe a ``reduced'' correlated-calls transformation in which we only consider receivers from $\rcc$ and ignore other receivers of $R$. We will show that IDE problems obtained through ordinary and reduced correlated-calls transformations yield the same results.
  
The following Lemma shows that the types to which a given receiver is mapped in the result of the algorithm is not affected by other receivers and the types to which they are mapped.

\begin{lemma}\label{lem:recindepedgefn}
  Let $P$ be an IFDS problem. Let $N^*$ be the supergraph for $P$, $D$ the set of data-flow facts, $n\in N^*$ a node, and $p=[\startmain,\,\dots,\,n]$ a path in the supergraph. Let $d\in D\cup\{\Lambda\}$.
  Then for any realizable path $p'\in\textsf{RP}(p,\,d)$, set $S\subseteq R$, and receiver $r\in S$,
  \begin{equation}
    \ccedgefn S(p')(\topcc)(r)=
    \ccedgefn{\{r\}}(p')(\topcc)(r)\,.
  \end{equation}
\end{lemma}
\begin{proof}
  By induction on the length of $p$.
  
  \textit{Basis:} $p'=[(\startmain,\,\Lambda)]$. Then $\ccedgefn S(p')=\id=\ccedgefn{\{r\}}(p')$, and the Lemma follows directly.
  
  \textit{Induction hypothesis:} Suppose that for a path $q=[(\startmain,\,\Lambda),\,\dots,\,(n_{k-1},\,d_{k-1})]$, where $q\in\textsf{RP}(n,\,d)$, the Lemma holds, i.e. both edge functions map $r$ to the same set of types $\uptau$:
  \begin{align*}
    \uptau
    &=\ccedgefn S(q)(\topcc)(r)\\
    &=\ccedgefn{\{r\}}(q)(\topcc)(r)\,.
  \end{align*}
  
  \textit{Induction step:} Let $q'=[(\startmain,\,\Lambda),\,\dots,\,(n_{k-1},\,d_{k-1}),\,(n_k,\,d_k)]$ and the edge $e_k=((n_{k-1},\,d_{k-1}),\,(n_k,\,d_k))$.
  
  Observe that for any set $U\subseteq R$ such that $r\in U$,
  \begin{align}\label{eq:edgefnU}
    \ccedgefn U(q')(\topcc)(r)
    &=\ccedgefn U(e_k)(\ccedgefn U(q)(\topcc))(r)\,.
  \end{align}
  
  We can see from~\eqref{eq:edgefnThroughDelta} that there are two cases.  
  
  If $d_{k-1}=d_k=\Lambda$, $\ccedgefn S(e_k)=\id=\ccedgefn{\{r\}}(e_k)$, and, due to~\eqref{eq:edgefnU},
  \begin{align*}
    \ccedgefn S(q')(\topcc)(r)&=\uptau\\
    &=\ccedgefn{\{r\}}(q')(\topcc)(r)\,.
  \end{align*}
  
  Otherwise, there are four sub-cases.
  \begin{enumerate}
    \item $e_k$ is a call-start edge, $r'.c()$ is the call site at $n_{k-1}$ with signature $s_\mathcal F$, $f\in\mathcal F$ is the called procedure, and $r'\in U$.
    Then
    \[
      \ccedgefn U(e_k)=\lambda m\,.\,\delta(m)[r'\to\delta(m)(r)\cap\tau(s_\mathcal F,\,f)]\,.
    \]
    There are two sub-cases.
    \begin{enumerate}
      \item\label{item:callstartreceq} If $r=r'$, then, according to~\eqref{eq:edgefnU}, the resulting set of types 
        \[
          \ccedgefn U(q')(\topcc)(r)=\delta(\ccedgefn U(q)(\topcc))(r)\cap\tau(s_\mathcal F,\,f).
        \]
        If $d_{k-1}=\Lambda$, then $\delta(\ccedgefn U(q)(\topcc))(r)=\botcc(r)=\bot_T$. If $d_{k-1}\ne\Lambda$, then $\delta(\ccedgefn U(q)(\topcc))(r)=\ccedgefn U(q)(\topcc)(r)=\uptau$. The set $\tau(s_\mathcal F,\,f)$ is the same for either case.
    
        Therefore, the value of $\ccedgefn U(q')(\topcc)(r)$ has the same result regardless of $U$,
        which means that $\ccedgefn S(q')(\topcc)(r)=\ccedgefn{\{r\}}(q')(\topcc)(r)$, and the Lemma holds.
      \item\label{item:callstartrecneq} If $r\ne r'$, then
        \begin{equation}
          \ccedgefn U(q')(\topcc)(r)=\delta(\ccedgefn U(q)(\topcc))(r)\,,
        \end{equation}
        which, as we have seen in~Case~\eqref{item:callstartreceq}, does not depend on~$U$, and the Lemma holds.
    \end{enumerate}
    \item $e_k$ is an end-return edge, $r_1,\,\dots,\,r_l\in U$ are the local variables in the callee method, $r'.c()$ is the call corresponding to the return node at $n_k$, $f\in\mathcal F$ is the called method with signature $s_\mathcal F$, and $r'\in U$.
    Then
    \[
      \ccedgefn U(e_k)=\lambda m\,.\,\delta(m)
      [r'\to\delta(m)(r)\cap\tau(s_\mathcal F,\,f)]
      [r_1\to\bot_T]\ldots[r_l\to\bot_T]\,.
    \]
    There are three sub-cases.
    \begin{enumerate}
      \item\label{item:localvarrec} If $r\in\{r_1,\,\dots,\,r_l\}$, then regardless of the value of $U$,
      \[
        \ccedgefn U(q')(\topcc)(r)=\bot_T\,,
      \]
      and the Lemma holds.
      \item Otherwise, if $r=r'$, the case is analogous to Case~\eqref{item:callstartreceq}.
      \item If $r\notin\{r',\,r_1,\,\dots,\,r_l\}$, then see Case~\eqref{item:callstartrecneq}.
    \end{enumerate}
    \item $n_{k-1}$ contains an assignment for $r'\in U$. Then
    \[
      \ccedgefn U(e_k)=\lambda m\,.\,\delta(m)[r'\to\bot_T]\,.
    \]
    If $r=r'$, see Case~\eqref{item:localvarrec}. If $r\ne r'$, see Case~\eqref{item:callstartrecneq}.
    \item Otherwise,
    \[
      \ccedgefn U(e_k)=\lambda m\,.\,\delta(m)\,,
    \]
    and the case is analogous to Case~\eqref{item:callstartrecneq}.\qedhere
  \end{enumerate}
\end{proof}

The following Lemma shows that the correlated-calls analysis computes the results for each receiver independently, or separately. To compute the set of types to which a receiver~$r$ is mapped at each exploded-graph node, we can exclude all other receivers in the program from the analysis (recall from~\eqref{eq:edgefndef} that the set of receivers that are considered in the analysis is specified by the set $S$ in a correlated-calls transformation $\transCC_S$). Therefore, for a given receiver $r$, the results of a $\transCC_S$- and a $\transCC_{\{r\}}$-analysis are the same.

\begin{lemma}\label{lem:recindep} Let $P$ be an IFDS problem. Let $N^*$ be the supergraph for $P$, $D$ the set of data-flow facts, and $S\subseteq R$ a set of receivers.
  Then for any $n\in N^*$, $d\in D$, and receiver $r\in S$,
  \begin{equation}
    \result\left(\transCC_S(P)\right)(n,\,d)(r)=
    \result(\transCC_{\{r\}}(P))(n,\,d)(r)\,.
  \end{equation}
\end{lemma}
\begin{proof}
  According to~\eqref{eq:ideresult}, \eqref{eq:mvpdef}, and~\eqref{eq:envTransToEdgefn},
  \begin{align}
    \result\left(\transCC_S(P)\right)(n,\,d)(r)
    &=\mvp{\textsf{Env}}(n,\,d)(r)\notag\\
    &=\left(\bigsqcap_{q\in\ivp(n)}M_\textsf{Env}(q)(\Omega)(d)\right)(r)\notag\\
    &=\left(\bigsqcap_{q\in\ivp(n)}\bigsqcap_{q'\in\mathsf{RP}(q,\,d)}\ccedgefn S(q')(\topcc)\right)(r)\notag\\
    &=\bigcup_{q\in\ivp(n)}\bigcup_{q'\in\mathsf{RP}(q,\,d)}\ccedgefn S(q')(\topcc)(r)\,.\label{eq:resultThroughEdgefn}
  \end{align}
  Then from Lemma~\ref{lem:recindepedgefn},
  \begin{align*}
    \result\left(\transCC_S(P)\right)(n,\,d)(r)
    &=\bigcup_{q\in\ivp(n)}\bigcup_{q'\in\mathsf{RP}(q,\,d)}\ccedgefn{\{r\}}(q')(\topcc)(r)\\
    &=\result\left(\transCC_{\{r\}}(P)\right)(n,\,d)(r)\,.\qedhere
  \end{align*}
\end{proof}

The next lemma shows that the set of types to which a receiver is mapped in a correlated-calls lattice element can be represented as an intersection of static-type function applications $\tau(s_{\mathcal F_i},\,f_i)$.
\begin{lemma}\label{lem:edgefnThroughTaus}
  For an IFDS problem $P$, a node $n\in N^*$, and fact $d\in D$, let $p\in\mathsf{RP}(n,\,d)$ be a realizable path and $r\in R$ a receiver. Then there exists a non-negative number $\gamma$ of calls on the receiver $r$ with signatures $s_{\mathcal F_\gamma}$ to the functions $f_\gamma\in\mathcal F_\gamma$, for which
  \[
    \ccedgefn{\{r\}}(p)(\topcc)(r)=
      \bigcap_{\gamma\ge0}\tau(s_{\mathcal F_\gamma},\,f_\gamma)\,.
  \]
\end{lemma}
\begin{proof}
  Let $p$ have the following form\footnote{It can be shown from the definition of a pointwise representation in~Sagiv et al.~\cite{sagiv1996precise} that in a realizable path, there is never an edge from a fact of the set $D$ to a $\Lambda$ fact. Therefore, we can represent $p$ as a sequence of nodes that has a prefix of $\Lambda$-fact nodes, after which all nodes are non-$\Lambda$ facts.}:
  \[
    p=[(\startmain,\,\Lambda),\,(n_1,\,\Lambda),\,\dots,\,(n_k,\,\Lambda),
       (n_{k+1},\,d_{k+1}),\,\dots,\,(n_{k+l},\,d_{k+l})]\,,
  \]
  where $l\ge1$ and the facts for all nodes up to $n_k$ are equal to $\Lambda$ and $d_{k+i}\in D$ for $0<i\le l$.
  
  As previously, for all $i$, we will denote the edge $(n_i,\,n_{i+1})$ by $e_i$.  
  
  From~\eqref{eq:edgefndef} we can infer that
  \[
    \ccedgefn{\{r\}}(p)=
    \ccedgefn{\{r\}}(e_{k+l})
    \circ\ldots
    \circ\ccedgefn{\{r\}}(e_{k+2})
    \circ(\lambda m\,.\,\beta)
    \circ\id
    \circ\ldots
    \circ\id\,,
  \]
  where
  \[
    \beta=
    \begin{cases}
      \botcc[r\to\tau(s_\mathcal F,\,f)]&\text{if $(n_k,\,n_{k+1})$ is a call-start or end-return edge, and}\\&\text{the call site $r.c()$ with signature $s_\mathcal F$ to the function}\\
      &\text{$f\in\mathcal F$ corresponds to the call-start or end-return edge,}\\
      \botcc&\text{otherwise\footnotemark.}
    \end{cases}
  \]
  \footnotetext{Since $d_k=\Lambda$ and $d_{k+1}\ne\Lambda$, the micro function for the edge $e_{k+1} $ is equal to $\lambda m\,.\,\varepsilon_{\{r\}}(e_{k+1})(\botcc)$. From the definition of $\varepsilon_S$~\eqref{eq:varepsilon} we can see that the only case where $\varepsilon_{\{r\}}(e_{k+1})(m)$ would not be equal to $\botcc$ is when $e_{k+1}$ is call-start or end-return edge.}
  
  Therefore,
  \begin{align}\label{eq:edgefnbeta}
    \ccedgefn{\{r\}}(p)(\topcc)
      &=\left(\ccedgefn{\{r\}}(e_{k+l})\circ\ldots\circ
        \ccedgefn{\{r\}}(e_{k+2})\right)((\lambda m\,.\,\beta)(\topcc))\notag\\
      &=\left(\ccedgefn{\{r\}}(e_{k+l})\circ\ldots\circ
        \ccedgefn{\{r\}}(e_{k+2})\circ\id\right)(\beta)\,.
  \end{align}

We can now prove the lemma by induction on $l$.

\textit{Basis:}
If $l=1$, then $\ccedgefn{\{r\}}(p)(\topcc)=\id(\beta)=\beta$.
There are two cases.

If $\beta=\botcc$, then 
\begin{align*}
  \ccedgefn{\{r\}}(p)(\topcc)(r)&=\beta(r)\\&=\bot_T\,,
\end{align*} and $\gamma=0$.

If $\beta=\botcc[r\to\tau(s_\mathcal F,\,f)]$, then 
\[
  \ccedgefn{\{r\}}(p)(\topcc)(r)=\tau(s_\mathcal F,\,f)\,,
\]
and $\gamma=1$.

\textit{Induction hypothesis:}
Assume that for a path $p=[(\startmain,\,\Lambda),\,\dots,\,(n_{k+l},\,d_{k+l})]$, the Lemma holds for $\gamma=N$, where $N\ge0$.

\textit{Induction step:}
Let $p'=[(\startmain,\,\Lambda),\,\dots,\,(n_{k+l},\,d_{k+l}),\,(n_{k+l+1},\,d_{k+l+1})]$.

Recall that
\begin{align*}
  \ccedgefn{\{r\}}(p')(\topcc)(r)
  &=\ccedgefn{\{r\}}(e_{k+l+1})\left(\ccedgefn{\{r\}}(p)(\topcc)\right)(r)\,.
\end{align*}

From~\eqref{eq:varepsilon} we can see that unless $e_{k+l+1}$ is a call-start or end-return edge corresponding to a call on the receiver $r$, then $\ccedgefn{\{r\}}(e_{k+l+1})(r)$ must be equal to either $\bot_T$ or $m(r)$, where $m=\ccedgefn{\{r\}}(p)(\topcc)$. 

If $\ccedgefn{\{r\}}(e_{k+l+1})(r)=\bot_T$, then the Lemma holds for $\gamma=0$. 

Otherwise,
\begin{align*}
  \ccedgefn{\{r\}}(e_{k+l+1})(\topcc)(r)&=\ccedgefn{\{r\}}(p)(\topcc)(r)\\
  &=\bigcap_{N}\tau(s_{\mathcal F_N},\,f_N)\,,
\end{align*}
and therefore $\gamma=N$.

Suppose that $e_{k+l+1}$ is a call-start edge with a call on the receiver $r$ with signature $s_\mathcal G$ to a function $g\in\mathcal G$. Then, according to~\eqref{eq:varepsilon}, 
\[
  \ccedgefn{\{r\}}(e_{k+l+1})=\lambda m\,.\,m[r\to m(r)\cap\tau(s_\mathcal G,\,g)]\,.
\]
Therefore,
\begin{align*}
  \ccedgefn{\{r\}}&(p')(\topcc)(r)\\
  &=\lambda m\,.\,m[r\to m(r)\cap\tau(s_\mathcal G,\,g)]\left(\ccedgefn{\{r\}}(p)(\topcc)\right)(r)\\
  &=\ccedgefn{\{r\}}(p)(\topcc)(r)\cap\tau(s_\mathcal G,\,g)\\
  &=\left(\bigcap_{N}\tau(s_{\mathcal F_N},\,f_N)\right)\cap\tau(s_\mathcal G,\,g)\,,
\end{align*}
and the Lemma holds for $\gamma=N+1$.

The case where $e_{k+l+1}$ is an end-return edge is analogous to the previous case.
\end{proof}

We now show that a receiver will be only mapped to $\topcc$ if it is the receiver of a correlated call.

\begin{lemma}\label{lem:ccrectop}
   For an IFDS problem $P$, let $n\in N^*$ be a node, and $d\in D$ a dataflow fact such that there exists a realizable path $p\in\textsf{RP}(n,\,d)$. Let $T$ be the set of all types in the program.
   If there exists a receiver $r\in R$ such that
  \[
    \ccedgefn{\{r\}}(p)(\topcc)(r)=\top_T\,,                       
  \]
  then $r\in \rcc$.
\end{lemma}
\begin{proof}
 According to Lemma~\ref{lem:edgefnThroughTaus},
 \[
   \ccedgefn{\{r\}}(p)(\topcc)(r)=
      \bigcap_{\gamma\ge0}\tau(s_{\mathcal F_\gamma},\,f_\gamma).
 \]
 
 Let $\tau_i=\tau(s_{\mathcal F_i},\,f_i)$. 
 For a given $k$, let $r.m_k()$ be the call site corresponding to $\tau_k$, and $T'$ the set of types compatible with the static type of $r$.
 Recall from Section~\ref{sec:ef} that 
 \begin{itemize}
   \item $\tau_k\ne\top_T$;
   \item if $\tau_k=T'$ then the corresponding call site is monomorphic;
   \item if $\tau_k\subset T'$ then the call site is polymorphic.
 \end{itemize}
 
 From the conditions of the Lemma, 
 \begin{equation}
   \bigcap_{\gamma\ge 0}\tau_\gamma=\top_T\,.
 \end{equation} 
 
 If all $\tau_k=T'$, then $\bigcap_{\gamma\ge 0}\tau_\gamma$ is also equal to $T'$. Since $T'\ne\top_T$, this is a contradiction. 
 
 If exactly one $\tau_k\subset T'$ and the rest are equal to $T'$, then $\bigcap_{\gamma\ge 0}\tau_\gamma$ is equal to $\tau_k$, which cannot be $\top_T$ either.
 
 Therefore, there are at least two sets, $\tau_i$ and $\tau_j$, which are strict subsets of $T'$. Since both $\tau_i$ and $\tau_j$ are non-empty and their intersection equals $\top_T$, $\tau_i$ and $\tau_j$ must be disjoint. If $\tau_i$ and $\tau_j$ are disjoint, they must correspond to different call sites.
 
 In other words, there are at least two calls on the same receiver for which the static-type function is a strict subset of the set of types compatible with a given receiver $r$. It follows that both calls have to be polymorphic. Therefore,~${r\in\rcc}$.
\end{proof}

We will now show that if a receiver ever gets mapped to top, then it is a correlated-calls receiver.

\begin{lemma}\label{lem:ccrectop}
   For an IFDS problem $P$, let $n\in N^*$ be a node, and $d\in D$ a dataflow fact such there exists a realizable path $p\in\textsf{RP}(n,\,d)$.
   Then, if there exists a receiver $r\in R$, such that
  \[
    \result\left(\transCC_{\{r\}}(P)\right)(n,\,d)(r)=\top_T\,,
  \]
  then $r\in \rcc$.
\end{lemma}
\begin{proof}
  As shown in~\eqref{eq:resultThroughEdgefn},
  \[
    \result\left(\transCC_{\{r\}}(P)\right)(n,\,d)(r)=\bigcup_{q\in\ivp(n)}\bigcup_{q'\in\mathsf{RP}(q,\,d)}\ccedgefn{\{r\}}(q')(\topcc)(r)\,.
  \]
  Since the latter is equal to $\top_T$, it follows that for each realizable path $p'$ to node $n$, $\ccedgefn{\{r\}}(p')(\top)(r)=\top_T$. According to~Lemma~\ref{lem:ccrectop}, this is only possible if $r\in\rcc$.
\end{proof}

Finally, we show that if a correlated calls analysis considers only correlated-call receivers, no precision is lost. A correlated-calls analysis that considers all receivers computes the same result as an analysis that considers only correlated-call receivers.

\begin{lemma}\label{lem:onlycorrrec}
  Let $P$ be an IFDS problem. Then
  \begin{equation}
    \backCC\left(\result\left(\transCC_\rcc(P)\right)\right)=\backCC(\result\left(\transCC_R(P)\right))\,.
  \end{equation}
\end{lemma}
\begin{proof}
  From~\eqref{eq:ucc} we know that
  \begin{align*}
    \backCC(\result\left(\transCC_R(P)\right))
    &=\left\{(n,\,D_n^\Subset(\result(\transCC_R(P))))\ |\ n\in N^*\right\}.
  \end{align*}
  According to~\eqref{eq:dnq} and~Lemma~\ref{lem:recindep}, for a given $n\in N^*$,
  \begin{align*}
    D_n^\Subset&(\result(\transCC_R(P))))\\
    &=\left\{d\ |\ d\in\mvp F(n)\,\wedge\,\forall r\in R:\ \left\{(r,\,\result(\transCC_{\{r\}}(P))(n,\,d)(r))\ |\ r\in R\right\}(r)\ne\top_T\right\}\\
    &=\left\{d\ |\ d\in\mvp F(n)\,\wedge\,\forall\bm{r\in R}:\ \result(\transCC_{\{r\}}(P))(n,\,d)(r)\ne\top_T\right\}.
  \end{align*}

Since, according to Lemma~\ref{lem:ccrectop}, $\result(\transCC_{\{r\}}(P))(n,\,d)(r)$ can only be equal to $\top_T$ when $r\in\rcc$, we can conclude that
  \begin{align*}
    D_n^\Subset&(\result(\transCC_R(P))))\\
    &=\left\{d\ |\ d\in\mvp F(n)\,\wedge\,\forall\bm{r\in\rcc}:\ \result(\transCC_{\{r\}}(P))(n,\,d)(r)\ne\top_T\right\}\\
    &=D_n^\Subset(\result(\transCC_\rcc(P)))).
  \end{align*}
  
  Therefore,
  \begin{align*}
    \backCC(\result\left(\transCC_R(P)\right))
    &=\left\{(n,\,D_n^\Subset(\result(\transCC_\rcc(P))))\ |\ n\in N^*\right\}\\
    &=\backCC(\result\left(\transCC_\rcc(P)\right))\,.\qedhere
  \end{align*}
\end{proof}

To summarize, Lemma~\ref{lem:sound} shows that the result $\resultCC$ of a correlated-calls analysis is sound since it overapproximates the data flow of all possible concrete execution paths.
We have also shown in Lemma~\ref{lem:subsetifds} that the correlated-calls analysis improves the precision of the original IFDS result $\result_\text{IFDS}$, because the correlated-calls result $\resultCC$ underapproximates an equivalence-IDE result $\resultEq=\result_\text{IFDS}$.
Finally, we showed that a correlated-call transformation to IDE that considers only correlated-call receivers $\rcc$ achieves the same result $\resultCC$ that is obtained when considering all receivers $R$.

This is the general idea of the correlated-calls analysis. The analysis involves a transformation from IFDS to IDE problems. To implement an IDE problem, it is necessary to define a representation of lattice elements and micro functions. An efficient representation of those data structures for the correlated-calls analysis is presented in the next section.

\section{Correlated Calls Representations}\label{sec:ccdatastr}

In order to define a correlated-calls transformation, we need to represent 
 lattice elements $\lcc_\rcc:\ \rcc\to 2^T$ of the target IDE problem, which are functions from receivers to sets of types,
and micro functions $\lcc_\rcc\to\lcc_\rcc$.

As defined in~Sagiv et al.~\cite{sagiv1996precise}, a representation of micro functions is efficient if the following conditions hold:
\begin{enumerate}
	\item There is a representation for the identity and top functions.
  \item The representation is closed under the meet and composition operations.
  \item The micro functions form a finite-height lattice.
  \item The apply, meet, composition, and equality-check operations can be computed in constant time.
  \item There is a constant bound on the storage space for a micro function representation.
\end{enumerate}

We will distinguish the representation of a concept from its denotation. For a concept $c$, we will write $\denote c$ for its denotation and just $c$ for its representation. For example, if we want to represent a constant function $g$ with the constant value $v$ that it returns, we will write for $g$'s representation, $g=v$, and for $g$'s denotation, $\denote g=\lambda x\,.\,v$.

\subsection{Lattice Elements}Elements of the $\lcc_\rcc$ lattice can be represented with a map from receivers to sets of types. The bottom element maps each receiver to the set~$T$ of all types:
\begin{align}
  \botcc&=\left\{(r,\,T)\ |\ r\in \rcc\right\}
\intertext{and the top element maps each receiver to the empty set:}
  \topcc&=\left\{(r,\,\varnothing)\ |\ r\in \rcc\right\}.
\end{align}

\subsection{Micro Functions}
In the context of the correlated-calls transformation to an IDE problem, a lattice element is a map from receivers to sets of types. Thus, a micro function transforms, or updates, an existing receiver-to-types map with new information about the types of a receiver.

We will represent micro functions with \textit{update maps} which we describe in the next section.

\subsubsection{Update Maps}
To represent micro functions that transform maps from receivers to sets of types, we use \textit{update maps}. To define update maps, we first introduce the notions of \textit{update functions} and \textit{normalization}.

Let $f$ be a micro function, $r\in\rcc$ a correlated-call receiver, and $T$ the set of types in a program.

\begin{definition}
  A non-normalized update function $\updfun f^*$ is a pair of sets
  \begin{equation}
    \updfun f^*=\setpair{I_{f,\,r}}{U_{f,\,r}},
  \end{equation}
  where $I_{f,\,r}\subseteq T$ is called the intersection set and $U_{f,\,r}\subseteq T$ the union set of the update function.
\end{definition}

\begin{definition}
  Let $\setpair IU$ be a pair of sets. The normalization function $\mathcal N$ is defined as
    \begin{equation}
        \mathcal N(\setpair I U)=\setpair{I\cup U} U.
    \end{equation}
\end{definition}

\begin{definition}\label{def:updfun}
  An update function $\updfun f$ is a normalized pair of sets
  \begin{equation}
    \updfun f=\mathcal N(\updfun f^*)=\mathcal N(\setpair{I_{f,\,r}}{U_{f,\,r}})\,.
  \end{equation}
\end{definition}

\begin{definition}
  The update map of $f$ is a map from receivers to update functions:
  \begin{equation}
    \updmap f=\{(r,\,\updfun f)\,|\,r\in\rcc\}\,.
  \end{equation}
\end{definition}
Thus, each micro function $f$ is represented with an update map:
\begin{equation}
  f=\denote{\updmap f}.
\end{equation}

\subsubsection{Denotation of Update Maps}
Intuitively, the meaning of a micro function is the update that it performs on a receiver-to-types map. To represent a micro function, it is enough to specify how the set of types for a given receiver has to be transformed. This is what the update map does.

For a micro function $f$, an update map takes a receiver and returns an update function:
\begin{equation}
  \denote{\updmap f}=\lambda r\,.\,\denote{\updfun f}.
\end{equation}

Given a micro function $f:(\rcc\to 2^T)\to(\rcc\to2^T)$, an update map is defined so that
\begin{equation}
    f(m)=\left\{\left(r,\,\denote{\updmap f}(r)(m(r)\right)\ |\ r\in\rcc\right\},
\end{equation}
where the update map $\denote{\updmap f}$ has type $\rcc\to(2^T\to2^T)$, and the update function  $\denote{\updfun f}=\denote{\updmap f}(r)$ has type $2^T\to2^T$.

For any receiver-to-types map $m$, an update function specifies two things:
\begin{itemize}
    \item which elements of $m(r)$ should be preserved, and 
    \item which new elements should be added to $m(r)$.
\end{itemize}
This can be achieved by maintaining  the intersection $I_{f,\,r}$ and union set $U_{f,\,r}$, where
\begin{equation}
    \denote{\updfun f}(m(r))=(m(r)\cap I_{f,\,r})\cup U_{f,\,r}\,.
\end{equation}

However, as we will see in Section~\ref{sec:equality}, we need to be able to check update functions for equality, which is difficult to do with non-normalized update functions.

\begin{example}
 For a non-empty set of types $T$, consider two update functions
 $$u_1=\setpair T T$$ 
and 
$$u_2=\setpair\varnothing T.$$
The denotations of the functions look as follows:
$$\denote{u_1}=\lambda t\,.\,(t\cap T)\cup T$$ and $$\denote{u_2}=\lambda t\,.\,(t\cap\varnothing)\cup T\,.$$
We can see that both $\denote{u_1}$ and $\denote{u_2}$ are equal to the function $\lambda t\,.\,T$.

Therefore, the same function can have more than one non-normalized representation. This means that to check two functions for equality, it is not enough to compare their non-normalized representations.
\end{example}

This is why Definition~\ref{def:updfun} requires update functions to be normalized. Normalization makes the union set of the update function a subset of the intersection set.
As we show later, normalization guarantees that each update function has a unique representation.

\subsubsection{Equality of Micro Functions}\label{sec:equality}
We will now show that micro functions can be checked for equality if their representations use normalized update functions.

First, let us show that normalization does not change the behaviour of an update function. This means that the normalized and non-normalized versions of an update function always denote the same function.
   
\begin{lemma}\label{lem:normeq}
    $\denote{\mathcal N(\updfun f^*)}=\denote{\updfun f^*}.$
\end{lemma}


Let us show that two update functions are equal if and only if their normalized representations are equal.

It is obvious that two functions represented with the same pairs of sets are equal. The following lemma states that two different pairs of sets represent different update functions.

\begin{lemma}\label{lem:normuneq}
    Let $\setpair IU$ and $\setpair{I'}{U'}$ be two normalized update functions such that $\setpair IU\ne\setpair {I'}{U'}$. Then $\denote{\setpair IU}\ne\denote{\setpair {I'}{U'}}$.
\end{lemma}

We have shown that transfer functions can be represented using update maps. 

\subsubsection{Operations on Update Maps}\label{sec:opsTransRep}

Let us now define the apply, compose, meet, and equals functions on micro function representations.

For an update map $f$ and  a receiver $r\in \rcc$, let the update function $f(r)=\setpair{I_{f,\,r}}{U_{f,\,r}}$.

Then the operations on the update maps $f_1$ and $f_2$ are defined as follows:
        \begin{align}
    %apply
            \textsf{apply}_{f_1}=&\lambda m\,.\,
              \left\{
                \left(
                  r,\,
                  (m(r)\cap I_{f,\,r})\cup U_{f,\,r}
                \right)\,|\,
                r\in\rcc
              \right\},\\\notag \\
    %compose
            f_1\circ f_2=&
            \left\{
              \left(
                r,\,
                  \mathcal N(\langle
                    I_{f_1,\,r}\cap I_{f_2,\,r},\ 
                    (I_{f_1,\,r}\cap U_{f_2,\,r})\cup U_{f_1,\,r}
                  \rangle)
              \right)\,|\,
              r\in\rcc
            \right\},\\\notag \\
    %meet
            f_1\sqcap f_2=&
            \left\{
              \left(
                r,\,
                  \langle
                    I_{f_1,\,r}\cup I_{f_2,\,r},\ 
                    U_{f_1,\,r}\cup U_{f_2,\,r}
                  \rangle
              \right)\,|\,
              r\in\rcc
            \right\},\\\notag \\
    %equals
            \textsf{equals}(f_1,\,f_2)=&
              \begin{cases}
                 \mathsf{true}&\text{ if }f_1\text{ and }f_2\text{ are structurally equal,}\\
                 \mathsf{false}&\text{otherwise.}
              \end{cases}
        \end{align}


The denotation of the operations on update maps can be explained in the following way.

The apply function of a micro function $f$ maps over all receivers. For each receiver $r\in \rcc$, \textsf{update}$_f(r)$ transforms the argument 
receiver-to-types map $m$. It returns a new map in which $r$ is mapped to a new set of types, $\updfun f(m(r))$:
\begin{align*}
    \denote{\textsf{apply}_f}
    &=\denote{\lambda m\,.\,\left\{
                \left(
                  r,\,
                  (m(r)\cap I_{f,\,r})\cup U_{f,\,r}
                \right)\,|\,
                r\in\rcc
              \right\}}\\
    &=\lambda m\,.\,\left\{\left(r,\,\denote{\updfun f}(m(r))\right)\,|\,r\in \rcc\right\}.
\end{align*}
Note that in the beginning of the algorithm, $m$ maps each receiver to $\bot_T$ (all types).

Composing two micro functions means to compose their update-map denotations:\mtodo{Should this be somehow moved to the appendix? Or should we just say ``It can be shown that ...'' and remove the whole computations at all?}
\begin{align}
    \denote{f_1\circ f_2}
    &=\denote{\left\{
              \left(
                r,\,
                  \mathcal N(\langle
                    \ione\cap\itwo,\ 
                    (\ione\cap\utwo)\cup\uone
                  \rangle)
              \right)\,|\,
              r\in\rcc
            \right\}}\notag\\
    &=\footnotemark\lambda m\,.\,\left\{
              \left(
                r,\,
                  \denote{\langle
                    \ione\cap\itwo,\ 
                    (\ione\cap\utwo)\cup\uone
                  \rangle}(m(r))
              \right)\,|\,
              r\in\rcc
            \right\}\notag\\
    &=\lambda m\,.\,\left\{\left(r,\,(\ione\cap m(r)\cap\itwo)\cup(\ione\cap\utwo)\cup\uone\right)\,|\,r\in\rcc\right\}\notag\\
    &=\lambda m\,.\,\left\{\left(r,\,(((m(r)\cap\itwo)\cup\utwo)\cap\ione)\cup\uone\right)\,|\,r\in\rcc\right\}\notag\\
    &=\left(\lambda m\,.\,\{(r,\,(m(r)\cap\ione)\cup\uone)\,|\,r\in\rcc\}\right)\notag\\
      &\quad\circ\left((\lambda m\,.\,\{(r,\,(m(r)\cap\itwo)\cup\utwo)\,|\,r\in\rcc\}\right)\notag\\
    &=\denote{\{(r,\,\setpair{\ione}{\itwo})\}}\circ\denote{\{(r,\,\setpair\itwo\utwo)\}}\notag\\
    &=\denote{f_1}\circ\denote{f_2}.\label{eq:compclosed}
\end{align}
\footnotetext{See Lemma~\ref{lem:normeq}.}
\indent Finally, the meet operation on two micro functions is the union of their update maps:
\begin{align}
    \denote{f_1\sqcap f_2}
    &=\denote{\left\{
              \left(
                r,\,
                  \langle
                    \ione\cup\itwo,\ 
                    \uone\cup\utwo
                  \rangle
              \right)\,|\,
              r\in\rcc
            \right\}}\notag\\
    &=\lambda m\,.\,\left\{\left(r,\,m(r)\cap(\ione\cup\itwo)\cup\uone\cup\utwo\right)\,|\,r\in\rcc\right\}\notag\\
    &=\lambda m\,.\,\left\{\left(r,\,((m(r)\cap\ione)\cup\uone)\cup((m(r)\cap\itwo)\cup\utwo)\right)\,|\,r\in\rcc\right\}\notag\\
    &=\lambda m\,.\,\left\{\left(r,\,(m(r)\cap\ione)\cup\uone\right)\,|\,r\in\rcc\right\}\notag\\
      &\quad\sqcap\lambda m\,.\,\left\{\left(r,\,(m(r)\cap\itwo)\cup\utwo\right)\,|\,r\in\rcc\right\}\notag\\
    &=\denote{f_1}\sqcap\denote{f_2}.\label{eq:meetclosed}
\end{align}

We can now show how the correlated-calls definitions of the meet and composition operations on micro functions allow us to detect infeasible paths in a program.
\begin{example}\label{ex:cc}
  The edges of the exploded supergraph in Figure~\ref{fig:cc_edgefn_example} correspond to the edges of an IFDS taint analysis. We can see that there is a path from the node $(\highlight{\startmain}{greyblue},\,\Lambda)$ to $(\highlight{\texttt{print(s)}}{lightsalmonpink},\,\texttt s)$. This means that the IFDS taint analysis considers \texttt s to be a secret value that is leaked at the print statement.
  
  The correlated-calls analysis, on the other hand, detects that the path to $(\highlight{\texttt{print(s)}}{lightsalmonpink},\,\texttt s)$ is infeasible: at the print node, the lattice element corresponding to the fact \verb's' contains a mapping $\texttt a\to\top_T$.
  
  The lattice element for the print statement is evaluated as follows:
  \begin{align*}
    &((\lambda m\,.\,m[\texttt a\to m(\texttt a)\cap\{\texttt B\}])\circ
    \id\circ
    (\lambda m\,.\,m[\texttt a\to m(\texttt a)\cap\{\texttt A\}])\\
    &\circ
    (\lambda m\,.\,\botcc)\circ
    \id\circ\ldots\circ\id)(\topcc)
    \\
    =&(\botcc[\texttt a\to m(\texttt a)\cap(\{\texttt A\}\cap\{\texttt B\})])
    \\
    =&(\botcc[\texttt a\to\top_T])\,.
  \end{align*}
  
  Therefore, the path to the print statement will be considered infeasible, and the analysis does not claim that the program leaks a secret value.
\end{example}

\begin{mdelete}
The following Lemma states that the representation of micro functions is efficient according to the definition  of efficiency discussed in Section~\ref{sec:ccdatastr}.
\end{mdelete}
\begin{lemma}\label{lem:efficient}
  The correlated-call representation of a micro function is efficient.
\end{lemma}

\begin{mdelete}
\paragraph{Final Remarks}

A straightforward solution to representing micro functions would be to use the function constructs that are provided by many programming languages. The efficiency requirement prohibits us from doing so.
For most programming languages, equality for functions is either defined as reference equality (as in Scala), or is not defined at all (as in Haskell). Even if we were to define our own definition of equality for functions, we would have to iterate over the whole domain of the functions and compare the results of the function applications, which would be inefficient. Additionally, the equality check would be non-terminating if the domain of the functions were infinite, and undecidable if the language for defining the functions were Turing-complete.

Second, a composition $f$ of two functions $f_1$ and $f_2$ would have to store both $f_1$ and $f_2$. For instance, if $f_1=\lambda x\,.\,x+1$ and $f_2=\lambda x\,.\,x+2$, then $f=f_2\circ f_1$ would be represented as 
\[
  f=\lambda x\,.\,(\lambda y\,.\,y+2)((\lambda z\,.\,z+1)\,x)
\]
instead of
\[
  f=\lambda x\,.\,x+3.
\]
Having a compact representation for function composition is especially important for the first phase of the IDE algorithm, in the computation of jump functions~\cite{sagiv1996precise}. The same argument applies to computing function meets.
\end{mdelete}

\subsubsection{Edge Function Representation}  \label{sec:edgefnrep}
We will now show the representations for the correlated-call micro functions $\ccedgefn{\rcc}(e)$, described in Definition~\ref{def:edgefn}. Let $\textsf{identity}=\setpair{\bot_T}{\top_T}$ represent the identity function $\id$ and $\textsf{bottom}=\setpair{\bot_T}{\bot_T}$ represent the function $\lambda t\,.\,\bot_T$.

On the call-start edge,
\begin{align}
  &m\left[r\to (m(r)\cap\tau(s_\mathcal F,\,f))\right]\notag\\
  =&\denote{\{(r,\,\langle\tau(s_\mathcal F,\,f),\,\top_T\rangle)\}
    \cup
    \{(r',\,\textsf{identity})\,|\,r'\in \rcc,\,r'\ne r\}}.
\end{align}

On the end-return edge,
\begin{align}
  \lambda m\,.\,m&\left[v_1\to\bot_T\right]\dots[v_k\to\bot_T][r\to (m(r)\cap\tau(s_\mathcal F,\,f))]\notag
  \\=&\denote{
    \{(r,\,\langle\tau(s_\mathcal F,\,f),\,\top_T\rangle)\}
    \cup
    \left\{(r',\,w(r'))\,|\,r\in\rcc,\,r'\ne r
  \right\}},
\end{align}
where
\[
w(r)=\begin{cases}
      \textsf{bottom}&\text{if $r$ is a local variable in the exiting method,}\\
      \textsf{identity}&\text{otherwise}.
    \end{cases}
\]

For assignments in the source node of $e$,
\begin{align}
  \lambda m.m\left[r\to \bot_T\right]
  =\denote{\left\{(r,\,\textsf{bottom})\}
    \cup
    \{(r',\,\textsf{identity})\,|\,r'\in \rcc,\,r'\ne r\right\}}.
\end{align}%

In the default case,
\[
  \id
  =\denote{\{(r,\,\textsf{identity}),\,r\in \rcc\}}.
\]

We have shown how IDE problems that account for correlated calls can be represented in an efficient way. In the next section, we address the implementation and present an evaluation of the correlated-calls analysis.

\section{Evaluation}\label{chapter:eval}

This section discusses implementation aspects of the correlated-calls analysis and presents experimental results.

\subsection{Implementation of the Analysis}
The correlated-calls analysis was implemented in the Scala programming language~\cite{odersky2004overview}. 
We chose Java as the target language for client programs of the analysis.
To retrieve information about an input program, such as its control-flow supergraph or the set of receivers and their types, we used the WALA framework for static analysis on Java bytecode~\cite{fink2012wala}.

Since WALA currently only contains an implementation of IFDS, we implemented IDE from scratch. Instead of using WALA's IFDS implementation, to run an IFDS problem, we converted it to an IDE problem and used our own IDE solver.

\begin{odelete}
\subsubsection{IFDS}
\otodo{We should incorporate the four functions into the explanation of IFDS in the background. Then we don't need them here.}
As described in Section~\ref{sec:ifdsdef}, an IFDS problem is defined in terms of an exploded supergraph. The control-flow supergraph of an input program can be retrieved using WALA. Hence, our implementation of an IFDS problem should be able to convert a control-flow supergraph into an exploded supergraph.

We represent an IFDS problem with a trait, or protocol, that contains declarations of four \textit{flow functions}. Each function has type 
\[
  F:\,(N\times D\times N)\to2^D
\]
and defines a set of edges on the exploded graph. 
Given an edge $(n_1,\,n_2)$ of the control-flow supergraph and the fact $d_1$ that 
corresponds to the source node $n_1$, $F(n_1,\,d_1,\,n_2)$ returns the set of all facts $d_2\in D_2$ such that $((n_1,\,d_1),\,(n_2,\,d_2))\in E^\#$\footnote{In each invocation of a flow function, the fact $d_1$ is provided by the IDE algorithm.}.
The four functions are:
\begin{itemize}
  \item \textsf{call-start}, for inter-procedural edges from a call node to the start node of the target method;
  \item \textsf{call-return}, for intra-procedural edges from a call node to its return node;
  \item \textsf{end-return}, for inter-procedural edges from the end node of a method to the return node of the callee;
  \item \textsf{default}, for all other intra-procedural edges.
\end{itemize}
\end{odelete}

\paragraph{Taint Analysis}
Using this representation of an IFDS problem, we implemented an IFDS problem instance for taint analysis. We used it as a sample IFDS problem on which to evaluate the correlated-calls-IDE construction.

Let $N^*$ be the control-flow supergraph of a program and $D$ the set of the program variables.
Let \textsf{encl}$(n)$ be a function that returns the enclosing method of a node $n\in N^*$. Finally, let the function $r_m:\,D\to 2^D$ be defined as follows:
\begin{equation}
  r_m(d)=\begin{cases}
                \varnothing&\text{if }d\text{ is a local variable in method }m,\\
                \{d\}&\text{otherwise.}
              \end{cases}
\end{equation}
When defining the flow functions for a taint analysis, we will use $r_m$ to avoid the propagation of local variables, as shown below.

For a fact $d_1\in D\cup\{\mathbf0\}$ and two nodes $n_1,\,n_2\in N^*$, the simplified\footnote{For simplicity, the shown flow functions do not account for different Java-specific features such as arrays, fields, operations on strings, etc.} version of flow functions for a taint-analysis looks as follows.

If $n_1$ is a call node that calls method $m$, and $n_2$ is $m$'s start node,
\begin{align*}
  \textsf{call-start}(n_1,\,d_1,\,n_2)&=
    \begin{cases}
      r_{\textsf{encl}(n_1)}(d_1)\cup\{v\}
        &\text{if $a$ is the $i$th argument of the call,}\\
        &\text{$d_1=a$, and $v$ is the $i$th parameter}\\
        &\text{of $m$;}\\
      r_{\textsf{encl}(n_1)}(d_1)&\text{otherwise.}
    \end{cases}
\intertext{If $n_1$ is a call node with corresponding return node $n_2$,}
  \textsf{call-return}(n_1,\,d_1,\,n_2)&=
    \begin{cases}
      \{d_1\}&\text{if $d_1$ is a local variable in \textsf{encl}$(n_1)$,}\\
      \varnothing&\text{otherwise.}
    \end{cases}
\intertext{If $c$ is a call node calling method $m$, $n_1$ is $m$'s end node, and $n_2$ is $c$'s return node,}
  \textsf{end-return}(n_1,\,d_1,\,n_2)&=
    \begin{cases}
        r_{\textsf{encl}(n_1)}(d_1)\cup\{x\}
          &\text{if $n_1$ is a return statement}\\
          &\text{returning $v$, $n_2$ is an assignment}\\
          &\text{with left-hand side $x$, and $d_1=v$;}
        \\
        r_{\textsf{encl}(n_1)}(d_1)&\text{otherwise.}
    \end{cases}
\intertext{Otherwise,}
  \textsf{default}(n_1,\,d_1,\,n_2)&=\{d_1\}.
\end{align*}

\begin{odelete}
\begin{example}
  Consider the supergraph in Figure~\ref{fig:exampleexploded}. The call-to-start flow function from method \verb'main' to \verb'f' looks as follows:
  \begin{align*}
    \textsf{call-start}(\highlight{\textsf{call}_{\texttt{A.f}}}{greyblue},\,\texttt a,\,\highlight{\textsf{start}_\texttt{f}}{lightsalmonpink})
    &=r_\texttt{main}(\texttt a)\cup\{\texttt s\}\\
    &=\{\texttt s\}.
  \end{align*}
  We can see that correspondingly, the exploded supergraph contains an edge from $(\highlight{\textsf{call}_{\texttt{A.f}}}{greyblue},\,\texttt a)$ to $(\highlight{\textsf{start}_\texttt{f}}{lightsalmonpink},\,\texttt s)$.
\end{example}
\end{odelete}

\subsubsection{IDE}
The correlated-calls analysis was implemented as an IDE problem instance.

We defined an IDE problem in the same way as an IFDS problem, except that the IDE flow functions are of type
\[
  (N\times D\times N)\to2^{D\times (L\to L)}.
\]
With the new flow functions, we can implement a labeled exploded supergraph, since the new flow functions return a set of facts that are paired with micro functions.

For example, if $Q$ is an IDE problem, then the call-to-start flow function for $Q$ is defined as follows:
\begin{align*}
  &\textsf{call-start}^Q(n_1,\,d_1,\,n_2)\\
  =&\left\{(d_2,\,f)\,|\,d_2\in D,\,f\in L^Q\to L^Q:\,\edgefn^Q((n_1,\,d_1),\,(n_2,\,d_2))=f\right\}.
\end{align*}
The other flow functions are defined analogously.

\subsection{Testing}
In this section we assess the correctness and effectiveness of the correlated-calls analysis.

\begin{odelete}
\subsubsection{Conversion from IFDS to IDE}
We implemented the equivalence transformation $\transEq$ and the correlated-calls transformation $\transCC_\rcc$ from IFDS to IDE described in Section~\ref{sec:equivtrans}. To run an IFDS problem, we converted it to an IDE problem using $\transEq$ and $\transCC_\rcc$ and used our IDE analysis algorithm to run the latter.

Given an IFDS problem described with IFDS flow functions, an equivalence transformation creates an IDE problem described with the following IDE flow functions:
\begin{align*}
  \textsf{call-start}^\equiv(n_1,\,d_1,\,n_2)=&\{(d_2,\,\epsilon(d_1,\,d_2))\,|\,d_2\in\textsf{call-start}(n_1,\,d_1,\,n_2)\}\\
  \textsf{call-return}^\equiv(n_1,\,d_1,\,n_2)=&\{(d_2,\,\epsilon(d_1,\,d_2))\,|\,d_2\in\textsf{call-return}(n_1,\,d_1,\,n_2)\}\\
  \textsf{end-return}^\equiv(n_1,\,d_1,\,n_2)=&\{(d_2,\,\epsilon(d_1,\,d_2))\,|\,d_2\in\textsf{end-return}(n_1,\,d_1,\,n_2)\}\\
  \textsf{default}^\equiv(n_1,\,d_1,\,n_2)=&\{(d_2,\,\epsilon(d_1,\,d_2))\,|\,d_2\in\textsf{default}(n_1,\,d_1,\,n_2)\}\,,
\end{align*}
where $\epsilon$ is the bottom function on an edge from a $\Lambda$-fact to a non-$\Lambda$-fact, and the identity function otherwise:
\[
  \epsilon(d_1,\,d_2)=\begin{cases}
    \lambda l\,.\,\bot&\text{if $d_1=\Lambda$ and $d_2\ne\Lambda$;}\\
    \id&\text{otherwise.}
  \end{cases}
\]

We also implemented a correlated-call transformation from IFDS into IDE problems that consider correlated calls. This transformation is described in Section~\ref{sec:cctrans}.
\end{odelete}

\subsubsection{Regression Testing}
We used regression tests to assess the correctness of the implemented analyses. Each test involves running a certain analysis on one input Java program.

\begin{mdelete}
\paragraph{IDE-Implementation Correctness}
To test the correctness of the IDE algorithm implementation, we implemented a copy-constant-propagation IDE problem~\cite{sagiv1996precise}.
In a copy-constant propagation analysis, a variable is considered constant if it is assigned a constant literal or another variable that is also a constant. For example, in a program
\inputMinted{java}{correct.java}
\texttt a and \texttt b are considered constant, but \texttt c and \texttt d are not (although \texttt d would be considered constant in linear-constant propagation).

We tested the propagation of constants on different intra- and inter-procedural data-flow paths, in parameter passing,
and in conditional branches. Each regression test contained assertions of the form ``at the end of method $m$, variable with name $x$ should be (not) constant''.

We also tested the implementation of the IDE algorithm on an IDE problem generated by conversion from an IFDS problem.

To do that, we implemented an IFDS instance for taint analysis.

Recall from Section~\ref{sec:bgifds} that taint analysis aims to discover variables that are secret at a given program point called a sink.

We used assertions of the form ``at program statement $n$, variable $x$ should be (not) secret'' by defining the sink of a secret value through special \verb'isSecret' and \verb'notSecret' methods.
Those methods asserted that the parameter passed to them is secret and not secret, respectively.
To define a source secret value we created a static \verb'secret()' method that returned a string. 

\begin{example}
  Listing~\ref{list:assertions} illustrates the use of the \verb'isSecret' and \verb'notSecret' assertions.
\begin{figure}
  \centering
  \begin{minipage}{\textwidth}
    \inputMinted{java}{assertions.java}
  \end{minipage}
  \caption{Example usage of \texttt{isSecret} and \texttt{notSecret} assertions in regression tests}
  \label{list:assertions}
\end{figure}
\end{example}

We tested data flow through
\begin{itemize}
  \item method calls and returns;
  \item conditional branches and loops, including nested constructions, the ternary operator, and \verb'switch' statements;
  \item arrays and fields\footnote{%
    In Java, arrays are allocated on the heap, and array elements can be aliases of each other. 
    Hence, if any array element gets assigned a secret value, we considered all elements of any \texttt{String} or \texttt{Object} array in the program secret. 
    For the same reason, if a field \texttt{f} of an object of class \texttt{A} is assigned a secret value, then we considered the field \texttt{f} of any object of class \texttt{A} secret.%  
  };
  \item static and instance class members;
  \item classes and interfaces that involve inheritance, overriding, and overloading;
  \item recursion;
  \item library calls\footnote{%
    We created a specification for library functions that allowed us to indicate under which conditions a library function returned a secret value. This let us avoid the expensive analysis of library functions.%
  };
  \item string concatenation and usage of the \texttt{StringBuffer} and \texttt{StringBuilder} classes\footnote{%
    Using mutation, objects of these classes can be converted into wrappers around secret strings. This is why we added a special handling for \texttt{StringBuffer} and \texttt{StringBuilder} objects. For instance, if a field had the  \texttt{StringBuilder} type, it was considered secret.%
  };
  \item generics, type conversions through castings, and exception handling.
\end{itemize}

Our taint analysis implementation becomes unsound in the presence of static initializers. If a static field is initialized to a secret value, our analysis will not detect it as such.

A static initializer is invoked only once, before the instance creation of a class or the access of a static member of that class.
Static initializers are invoked lazily by the Java Virtual Machine~\cite{lindholmjava}. This makes finding out at which program point a static initializer is invoked undecidable~\cite{hubert2009soundly}. To account for static initializers in the analysis would require modifying WALA's control-flow supergraph (which does not have edges to static initializers) or using a data-flow analysis for static initialization. Since the primary purpose of the taint-analysis implementation was to test the correlated-call analysis, we did not include a static-initializer analysis in this work.
\end{mdelete}

\paragraph{Correlated-Calls-Analysis Correctness}
We tested the implementation of the correlated-calls analysis by converting the taint analysis into an IDE problem with an implementation of $\transCC_\rcc$.

Since none of the test cases in the previous section contained correlated calls, we used the same tests with the same assertions to ensure that the correlated-calls analysis produces the same results as an IFDS-equivalent analysis in the absence of correlated calls.

We then added test cases that contained correlated calls. We added a new assertion method, \verb'notSecretCC'. For the IFDS-equivalent analysis, the method asserted that the argument passed to it was secret, and for the correlated-calls analysis, it asserted that the argument was not secret.

Separately, we used unit tests to check the implementation correctness of micro functions. We wrote assertions for the results of the equality, meet, and composition operations on all possible combinations of the identity, top, bottom, and constant functions.

\subsubsection{Benchmark Testing}
\otodo{We need to decide what exactly we want to include in the evaluation.
    The numbers of correlated calls are interesting. But I think we want
    to leave out all mention of the taint analysis. Perhaps we should
    have no empirical results at all (and only the theory/proofs)?}
To assess the benefit of the correlated-calls analysis, we counted the frequencies of correlated-call occurrences in the Dacapo benchmarks~\cite{blackburn2006dacapo}. We then ran the normal- and correlated-call-taint analysis on the Dacapo benchmarks to see what improvement we would get from the correlated-calls analysis.

\paragraph{Occurrences of Correlated Calls}
Our goal was to obtain an upper bound on the number of redundant IFDS-result nodes that could be potentially removed by our analysis.
We counted the number of correlated calls that occurred in programs of the Dacapo benchmarks, as shown in Table~\ref{tab:dacapostat}.

In the table, the number of all call sites in a program is denoted as $C$. 
Polymorphic call sites are denoted as $C_P$, and correlated call sites as $C^\Subset$. 
The first four columns indicate the overall number of various call sites and correlated-call receivers in a program. 
The last three columns indicate the ratio of polymorphic to all call sites, the ratio of correlated to polymorphic call sites, and the ratio of correlated call sites to correlated-call receivers.

\begin{table}
\caption{Frequencies of correlated-call occurrences in the Dacapo benchmarks}\label{tab:dacapostat}
\centering
%\resizebox{\textwidth}{!}{%
\begin{tabular}{@{}lrrrr
>{\columncolor[HTML]{FFFFFF}}r 
>{\columncolor[HTML]{FFFFFF}}l r@{}}
\toprule
\textbf{Benchmark}  &
  \multicolumn{1}{c}{\textbf{$|C|$}} & 
  \multicolumn{1}{c}{\textbf{$|C_P|$}} & 
  \multicolumn{1}{c}{\textbf{$|C^\Subset|$}} & 
  \multicolumn{1}{c}{\textbf{$|\rcc|$}} & 
  {\color[HTML]{000000} \textbf{\begin{tabular}[c]{@{}l@{}}$\cfrac{|C_P|}{|C|}$\end{tabular}}} & 
  {\color[HTML]{000000} \textbf{\begin{tabular}[c]{@{}l@{}}$\cfrac{|C^\Subset|}{|C_P|}$\end{tabular}}} & 
  \textbf{\begin{tabular}[c]{@{}l@{}}$\cfrac{|C^\Subset|}{|\rcc|}$\end{tabular}} \\ \midrule
\textbf{antlr}      & 7,610                             & 428                                   & 299                              & 70                                   & {\color[HTML]{000000} \textbf{6\%}}                                                      & {\color[HTML]{000000} \textbf{70\%}}                                                     & 4                                                                \\
\textbf{bloat}      & 18,157                            & 933                                   & 429                              & 119                                  & {\color[HTML]{000000} \textbf{5\%}}                                                      & {\color[HTML]{000000} \textbf{46\%}}                                                     & 4                                                                \\
\textbf{chart}      & 18,101                            & 466                                   & 195                              & 61                                   & {\color[HTML]{000000} \textbf{3\%}}                                                      & {\color[HTML]{000000} \textbf{42\%}}                                                     & 3                                                                \\
\textbf{eclipse}    & 3,222                             & 100                                   & 35                               & 10                                   & {\color[HTML]{000000} \textbf{3\%}}                                                      & {\color[HTML]{000000} \textbf{35\%}}                                                     & 4                                                                \\
\textbf{fop}        & 4,831                             & 129                                   & 40                               & 12                                   & {\color[HTML]{000000} \textbf{3\%}}                                                      & {\color[HTML]{000000} \textbf{31\%}}                                                     & 3                                                                \\
\textbf{hsqldb}     & 3,573                             & 81                                    & 35                               & 10                                   & {\color[HTML]{000000} \textbf{2\%}}                                                      & {\color[HTML]{000000} \textbf{43\%}}                                                     & 4                                                                \\
\textbf{jython}     & 12,149                            & 487                                   & 129                              & 54                                   & {\color[HTML]{000000} \textbf{4\%}}                                                      & {\color[HTML]{000000} \textbf{26\%}}                                                     & 2                                                                \\
\textbf{luindex}    & 7,190                             & 188                                   & 79                               & 29                                   & {\color[HTML]{000000} \textbf{3\%}}                                                      & {\color[HTML]{000000} \textbf{42\%}}                                                     & 3                                                                \\
\textbf{lusearch}   & 9,043                             & 350                                   & 126                              & 47                                   & {\color[HTML]{000000} \textbf{4\%}}                                                      & {\color[HTML]{000000} \textbf{36\%}}                                                     & 3                                                                \\
\textbf{pmd}        & 10,972                            & 219                                   & 68                               & 23                                   & {\color[HTML]{000000} \textbf{2\%}}                                                      & {\color[HTML]{000000} \textbf{31\%}}                                                     & 3                                                                \\
\textbf{xalan}      & 3,889                             & 110                                   & 35                               & 10                                   & {\color[HTML]{000000} \textbf{3\%}}                                                      & {\color[HTML]{000000} \textbf{32\%}}                                                     & 4                                                                \\
\textbf{Geom. mean} & \textbf{7,572}                    & \textbf{240}                          & \textbf{91}                      & \textbf{29}                          & {\color[HTML]{000000} \textbf{3\%}}                                                      & {\color[HTML]{000000} \textbf{38\%}}                                                     & \textbf{3}                                                       \\ \bottomrule
\end{tabular}
%}
\end{table}

We can see that on average, 3\% of all call sites $C$ are polymorphic call sites $C_P$. Out of those call sites, 38\% are correlated call sites $C^\Subset$. We also see that for one correlated-call receiver, there are on average three correlated calls. 

\paragraph{Experiments}
We ran the analysis on the Dacapo benchmarks to test if the taint analysis would benefit from the improved, correlated-calls based, analysis. 
We defined any user input string to be considered a secret source and compared the overall number of results in the original and correlated-call taint analyses. If the number of secret values in the original result were larger than in the correlated-call result, we would see a practical benefit from our analysis.

However, even when we considered each program point as a sink, the ``improved'' analysis revealed the same number of secret values as the original taint analysis.

A correlated call that could affect a taint-analysis result could most likely occur in the following scenario:
\begin{itemize}
  \item there is a receiver with at least two polymorphic calls;
  \item at least one of the calls $c_1$ returns a string~--- this would mean that the method potentially returns a secret value;
  \item at least one of the calls $c_2$ takes a string parameter~--- this would mean that a secret value could potentially be propagated to the method as an argument.
\end{itemize}
Then, if the correlated call occurred on an invocation $c_2(c_1())$, there might be a possibility of benefiting from the correlated-calls analysis. 
Given the relatively rare occurrence of correlated calls, this situation is not likely to appear often. 
This is illustrated in Table~\ref{tab:stringstat} which shows how often correlated calls would invoke methods that either take a string as a parameter \textit{or} return a string.
The set of receivers on which there are invocations of such methods is denoted as $\rcc_S$. 
A situation where one correlated call returned a string, \textit{and} another correlated call on the same receiver took a string parameter, appeared in only one case in the \verb'antlr' benchmark. However, the strings invoked were not designated as secret.

This explains why, specifically for a taint analysis as the client analysis, and specifically for the Dacapo benchmarks, the correlated call analysis did not make a difference.

\begin{table}
\caption{Frequency of correlated-call receivers for which at least one of the correlated calls takes a string as a parameter or returns a string}

\centering
\begin{tabular}{@{}lrrr@{}}
\toprule
\textbf{Benchmark}   & 
  \textbf{$|\rcc_S|$} & 
  \textbf{$|\rcc|$} & 
  \textbf{$\cfrac{|\rcc_S|}{|\rcc|}$} \\ \midrule
\textbf{antlr}       & 43                     & 70               & 62\%                            \\
\textbf{bloat}       & 0                      & 119              & 0\%                             \\
\textbf{chart}       & 1                      & 61               & 2\%                             \\
\textbf{eclipse}     & 0                      & 10               & 0\%                             \\
\textbf{fop}         & 0                      & 12               & 0\%                             \\
\textbf{hsqldb}      & 0                      & 10               & 0\%                             \\
\textbf{jython}      & 6                      & 54               & 23\%                            \\
\textbf{luindex}     & 0                      & 29               & 0\%                             \\
\textbf{lusearch}    & 2                      & 47               & 6\%                             \\
\textbf{pmd}         & 1                      & 23               & 3\%                             \\
\textbf{xalan}       & 0                      & 10               & 0\%                             \\
\textbf{Geom. mean} & \textbf{3}             & \textbf{29}      & \textbf{9}                    \\ \bottomrule
\end{tabular}
\label{tab:stringstat}
\end{table}

\begin{odelete}
\subsection{Future Work}
\otodo{We should convert this section into two or three sentences to be added to the conclusion. They should focus especially on the interprocedurally-correlated receivers.}
In this section we point out the limitations of the correlated-calls analysis and suggest improvements to the analysis for future work. 

One limitation of the analysis is that it only works for IFDS problems like taint analysis, reachable definitions, or available expressions. The correlated-call analysis is not applicable to IDE problems like copy- or linear-constant propagation. Therefore, a possible direction for future work is to create a correlated-calls analysis that transforms an original IDE problem into one that considers correlated calls (with a modified lattice and edge function definition), and then transforms the correlated-calls result into a more precise result of the original IDE problem.

\begin{figure}
  \centering
  \begin{minipage}{\textwidth}
    \inputMinted{java}{interprocRec.java}
  \end{minipage}
  \caption{Inter-procedurally-correlated calls}
  \label{list:interProcRec}
\end{figure}

Another constraint of the algorithm is that it only accounts for intra-procedurally-correlated receivers, or receivers on which correlated calls occur within one method. For example, in Listing~\ref{list:interProcRec}, \verb'a' is a correlated-call receiver, since there are two polymorphic method invocations on \verb'a'. However, the first one, \verb'a.setString()', is inside method \verb'main', and the second one, \verb'a.printString()', is inside method \verb'propagate'. Therefore, we do not treat \verb'a' as a correlated-call receiver, and the analysis would not improve the original IFDS result for this program.

Finally, correlated calls can occur on multiple receivers and other scenarios discussed in~\cite{DBLP:journals/scp/Tip15} that are not handled in this work.

\end{odelete}

\section{Related Work}
IFDS is a version of the functional approach to data-flow analysis developed by M.\,Sharir and A.\,Pnueli~\cite{pnueli1981two}.
\commentout{
Their algorithm is based on computing \textit{summary functions} that return the data-flow value at the end of a procedure, given the data-flow value at the start of the procedure. IFDS problems form a more restricted set of data-flow problems: unlike in the functional approach, IFDS flow functions have to be distributive, and the set of data-flow facts $D$ has to be finite. However, the IFDS algorithm is more general than Sharir's and Pnueli's algorithm in that it can handle programs containing local variables and parameters in recursive methods.
}
IFDS has been used to encode a variety of data-flow problems, for~example, typestate analysis%
\commentout{(determining which operations can be performed on an object at a given program point)}~\cite{naeem2008typestate,DBLP:conf/pldi/ZhangMGNY14} or shape analysis%
\commentout{(detecting errors and validating properties of programs at compile time)}
~\cite{DBLP:conf/birthday/KreikerRRSWY13}. IFDS has been broadly used~\cite{DBLP:conf/pldi/ArztRFBBKTOM14,tripp2009taj} and extended~\cite{DBLP:conf/sigsoft/LerchHBM14} to solve taint-analysis problems.

IFDS is implemented for two popular static-analysis frameworks for Java bytecode, the T.J.~Watson Libraries for~Analysis (WALA)~\cite{fink2012wala} and Heros~\cite{bodden2012inter}.

\commentout{
WALA is a framework for static analysis on Java bytecode developed by the IBM T.J.~Watson Research Center.
In the implementation of our work, we use WALA to build and traverse the supergraph (a special kind of control-flow graph) of a Java program\footnote{However, we do not use WALA's IFDS implementation, as explained in~Section~\ref{chapter:eval}.}.

Soot is a framework for program analysis and optimization on Java bytecode, developed by the Sable Research Group at McGill University. Unlike WALA, Soot also has an implementation of the IDE algorithm. The IFDS and IDE implementations for Soot are part of the Heros project~\cite{bodden2012inter}.

Whereas one advantage of Soot's IFDS implementation (and other static analysis tools) is ease of use and extensibility, WALA's primary focus is efficiency. For example, WALA uses bit-vectors to represent some of the analysis data types, like local variables and parameters. Another difference is that WALA's intermediate representation of a program uses static single assignment (SSA) form~\cite{cytron1991efficiently}. SSA form is a representation of the program in which each variable has only one definition (assignment). SSA can make dataflow analysis simpler and more efficient~\cite{appel1998ssa}.
}

Work on improving the IFDS algorithm includes the Practical Extensions to the IFDS algorithm~\cite{naeem2010pei}.
\commentout{The paper presents four extensions to the IFDS algorithm.}
Two of the four extensions improve the efficiency of the IFDS analysis for certain classes of IFDS problems. Another extension widens the class of problems applicable for the IFDS analysis. 
\commentout{However, those extensions do not affect the precision of IFDS problems.}
Our analysis, in contrast, does not improve the efficiency or generality of IFDS, but it allows us to solve IFDS problems more precisely.
The fourth extension is targeted towards programs that are represented in SSA form. Executing the IFDS analysis on such programs results in loss of precision in the presence of control-flow constructs (e.g. conditionals and loops), compared to programs in non-SSA form.
The extension makes the IFDS analysis on programs in SSA form as precise as on programs that are not represented in SSA form. In contrast, the correlated-calls analysis is applicable to programs in both SSA and non-SSA forms. Even if applied to a program in SSA form, our analysis and the extension improve the precision of IFDS in unrelated situations: the first analysis handles correlated calls, and the latter handles control-flow constructs. Thus, an IFDS analysis could benefit from both precision improvements independently.

Another work on improving the efficiency of the IFDS algorithm is E.\,Bodden et~al.'s framework for the analysis of software products lines~\cite{bodden2013spl}. Their paper uses transformations from IFDS to IDE problems, a technique we also employ. Finally, J.\,Rodriguez and O.\,Lhot\'ak implemented a concurrent version of the IFDS algorithm using actors~\cite{rodriguez2010concurrent}. However, neither of those works is concerned with improving the precision of IFDS results.

\commentout{
The correlated-calls analysis improves the precision of a data-flow analysis by eliminating a special type of infeasible paths. This is similar to the idea of context-sensitive analysis: just as a context-sensitive analysis eliminates infeasible paths from the end of a procedure to the call sites that do not match the given procedure call, the correlated-calls analysis eliminates infeasible paths caused by correlated method calls.
}

The idea of using correlated calls to remove infeasible paths in data-flow analyses of object-oriented programs was introduced by F.\,Tip~\cite{DBLP:journals/scp/Tip15}. The possibility of using IDE to achieve this is mentioned, but not elaborated upon. Our work presents a concrete solution to the problem and an implementation of that solution.

The idea of eliminating infeasible paths caused by correlated calls is similar to M.~Sridharan et~al.'s work on improving the precision of pointer analysis for JavaScript programs~\cite{DBLP:conf/ecoop/SridharanDCST12}. For each pointer, a pointer analysis determines the possible set of objects (the \textit{points-to} set) that the pointer can reference at a given program point. In JavaScript, it is challenging to compute the points-to set of fields because in general, field names can be derived from arbitrary expressions and bound at runtime.
As a result, an imprecise data-flow analysis will include infeasible paths between values of the form \verb'o[p]' (access of a property \verb'p' of object \verb'o'), where at compile time, \verb'p' can be bound to different values.
The idea of the paper is to track all dynamic property accesses (reads and writes) on an object \verb'o' with property name \verb'p'. The code snippets containing the references \verb'o[p]' are then extracted into a separate function $f$. The analysis is then run so that for each possible value of \verb'p', $f$ is analyzed separately; therefore, for a given property name, all correlated objects with that name are analyzed together.

The differences between this method of tracking correlated calls and our analysis are the following.
\begin{itemize}
  \item \textit{Type of target data-flow analysis} whose precision is to be improved. Our analysis improves the precision of IFDS data-flow analyses, whereas the JavaScript analysis improves the precision of pointer analysis.
  \item \textit{Target language}. Our analysis is for object-oriented languages where polymorphic methods, and not property names (which are known at compile time), cause infeasible paths.
  \item \textit{Different handling of correlated calls}. Extracting code that contains correlated calls into separate methods would not prevent infeasible paths. Instead, our analysis uses IDE flow functions to detect and eliminate infeasible paths caused by correlated calls.
\end{itemize}

\section{Conclusions}\label{chapter:concl}
We presented a technique to improve the precision of solutions to IFDS problems in the presence of correlated calls. Correlated calls occur when there are multiple polymorphic method invocations on the same receiver. Such method calls cause a data-flow analysis to consider infeasible paths, which makes the data-flow analysis less precise.

Our method of eliminating infeasible paths caused by correlated calls works by transforming an existing IFDS problem into a specialized IDE problem. In this way, we are able to track the classes to which method invocations get dispatched. After solving the specialized IDE problem, we convert its result into an IFDS result that is potentially more precise than the solution to the original IFDS problem. The increase in precision can occur for programs that contain correlated calls. Specifically, if, on a certain data-flow path, there are two polymorphic method invocations on the same receiver that dispatch to incompatible classes, the IDE analysis will consider the path infeasible.

We proved that the correlated-calls analysis is sound and that it improves the precision of IFDS results.

Our Scala implementation of the correlated-calls analysis includes
\begin{itemize}
  \item an implementation of the IDE analysis, which is based on the WALA static program analysis framework;
  \item a taint-analysis implementation as an IFDS problem instance;
  \item a transformer of IFDS problems to equivalent IDE problems, and a second transformer that accounts for correlated calls.
\end{itemize}

We tested the correlated-calls analysis on our taint analysis implementation by comparing the number of secret values that were leaked when using an IFDS taint analysis and a taint analysis that accounts for correlated calls. We used the Dacapo benchmarks as input programs. Although the benchmarks contained a number of correlated calls, we were not able to improve the precision of the taint analysis, because the correlated calls did not occur on paths of secret information leaks.

We are hopeful that other analyses can benefit from the extra information provided by the correlated-calls analysis, and plan to test this hypothesis in the future.


\bibliographystyle{plain}
\bibliography{bib/ref}

\section*{Appendix}
In this appendix we present the proofs to the Lemmas introduced in Section~\ref{sec:cca}.

\subsection*{Representation of Micro Functions}
\otodo{Repeat the lemmas here before each proof.}
We start by presenting the proofs to the lemmas about the representation of micro functions.

\begin{proof}[\textbf{Proof of Lemma~\ref{lem:equalEq}}]
    OL: Can't we just say the following?
    \begin{align*}
        \denote{\setpair {I}{U}}\\
       & = \lambda m. \lambda r. (m(r) \cap I(r))\cup U(r))\\
       & = \lambda m. \lambda r. (m(r) \cap I'(r))\cup U'(r))\\
       & = 
        \denote{\setpair {I'}{U'}}\\
    \end{align*}
    

First, we need to show that for all $r\in R$ and ${m\in R\to2^T}$, if $I(r)=I'(r)$ and $U(r)=U'(r)$, then $\denote{\setpair IU}(m)(r)=\denote{\setpair{I'}{U'}}(m)(r)$. Indeed, we can see that
\begin{align*}
  \denote{\setpair IU}(m)(r)
  &=\left(\lambda m'\,.\,\lambda r'\,.\,(m'(r')\cap I(r'))\cup U(r')\right)(m)(r)\\
  &=(m(r)\cap I(r))\cup U(r)\\
  &=(m(r)\cap I'(r))\cup U'(r)\\
  &=\left(\lambda m'\,.\,\lambda r'\,.\,(m'(r')\cap I'(r'))\cup U'(r')\right)(m)(r)\\
  &=\denote{\setpair{I'}{U'}}(m)(r).
\end{align*}

OL: Again, we can keep it simpler:
For the other direction:
\begin{align*}
&&
    \denote{\setpair{I}{U}} &= \denote{\setpair{I'}{U'}}\\
&\implies &
    \denote{\setpair{I}{U}}(\lambda r. \emptyset) &= \denote{\setpair{I'}{U'}}(\lambda r. \emptyset)\\
&\implies &
(\lambda m.\lambda r.(m(r)\cap I(r))\cup U(r))(\lambda r.\emptyset) &=
(\lambda m.\lambda r.(m(r)\cap I'(r))\cup U'(r))(\lambda r.\emptyset) \\
&\implies &
\lambda r.(\emptyset \cap I(r))\cup U(r) &=
\lambda r.(\emptyset \cap I'(r))\cup U'(r) \\
&\implies &
\lambda r.U(r) &=
\lambda r.U'(r) \\
&\implies &
U &=
U' \\
\end{align*}
Similarly:
\begin{align*}
&&
    \denote{\setpair{I}{U}} &= \denote{\setpair{I'}{U'}}\\
&\implies &
\denote{\setpair{I}{U}}(I) &= \denote{\setpair{I'}{U'}}(I)\\
&\implies &
(\lambda m.\lambda r.(m(r)\cap I(r))\cup U(r))(I) &=
(\lambda m.\lambda r.(m(r)\cap I'(r))\cup U'(r))(I) \\
&\implies &
\lambda r.(I(r) \cap I(r))\cup U(r) &=
\lambda r.(I(r) \cap I'(r))\cup U'(r) \\
&\implies &
\lambda r.(I(r) \cap I(r)) &=
\lambda r.(I(r) \cap I'(r)) \textrm{ since $\forall r.U(r) \subseteq I(r)$} \\
&\implies \forall r.&
I(r) &= I(r) \cap I'(r)\\
&\implies \forall r.&
I(r) &\subseteq I'(r)\\
\end{align*}
Symmetrically, we can also establish that $\forall r.I'(r) \subseteq I(r)$ by applying the functions to $I'$ instead of to $I$. Therefore, $I = I'$.


For the other direction, we need to show that if for any $r\in R$ and ${m\in R\to2^T}$, $(m(r)\cap I(r))\cup U(r)=(m(r)\cap I'(r))\cup U'(r)$, then for all $r'\in R$, $I(r')=I'(r')$ and $U(r')=U'(r')$.

To prove that $U(r')=U'(r')$, we will pick $m(r)=\varnothing$.
Then by our assumption, for all $r$, $U(r)=U'(r)$, and hence $U(r')=U'(r')$.

To prove that $I(r')=I'(r')$, let us pick $m=I$. Then the assumption
is $I(r)\cup U(r)=(I(r)\cap I'(r))\cup U'(r)$. By our definition, $U(r)\subseteq I(r)$, and we just proved that $U(r')=U'(r)$. Theorefore, $I(r)=(I(r)\cap I'(r))\cup U'(r)$.
Additionally, $U'(r)\subseteq I'(r)$ and $U'(r)\subseteq I(r)$ (because $U'(r)=U(r)\subseteq I(r)$). This means that $U'(r)\subseteq I(r)\cap I'(r)$. Therefore, $I(r)=I(r)\cap I'(r)$,
and we can conclude that $I(r)=I'(r)$.
\end{proof}

\otodo{The (m)(r) does not seem necessary. Can we replace $r'$ with $r$ and $m'$ with $m$?}
\begin{proof}[\textbf{Proof of Lemma~\ref{lem:equalComp}}]
For all $r\in R$ and $m\in R\to2^T$,
\begin{align*}
  &\denote{\setpair IU\circ\setpair{I'}{U'}}(m)(r)\\
  &=\denote{\setpair{\lambda r'\,.\,(I(r')\cap I'(r'))\cup U(r')}
                    {\lambda r'\,.\,(I(r')\cap U'(r'))\cup U(r')}}(m)(r)\\
  &=(m(r)\cap(I(r)\cap I'(r))\cup U(r))\cup(I(r)\cap U'(r))\cup U(r)\\
  &=(m(r)\cap I'(r))\cup I(r)\cup (U'(r)\cap I(r))\cup U(r)\\
  &=(((m(r)\cap I'(r))\cup U'(r))\cap I(r))\cup U(r)\\
  &=\left(\lambda r'.(((m(r')\cap I'(r'))\cup U'(r'))\cap I(r'))\cup U(r')\right)(r)\\
  &=\left(\lambda m'.\lambda r'.(m'(r')\cap I(r'))\cup U(r')\right)
    \left(\left(\lambda m'.\lambda r'.(m'(r')\cap I'(r'))\cup U'(r')\right)  (m)\right)(r)\\
  &=\left(\denote{\setpair IU}\circ\denote{\setpair{I'}{U'}}\right)(m)(r).\qedhere
\end{align*}
\end{proof}

\begin{proof}[\textbf{Proof of Lemma~\ref{lem:equalMeet}}]
  For all $r\in R$ and $m\in R\to2^T$,
\begin{align*}
  \left(\denote{\setpair{I}{U} \sqcap \setpair{I'}{U'}}\right)&(m)(r)\\
  &=(\denote{\setpair{\lambda r\,.\,I(r) \cup I'(r)}{\lambda r\,.\,U(r) \cup U'(r)}})(m)(r)\\
  &=(m(r)\cap(I(r)\cup I'(r))\cup U(r)\cup U'(r)\\
  &=(m(r)\cap I(r))\cup U(r)\cup(m(r)\cap I'(r))\cup U'(r)\\
  &=\denote{\setpair{I}{U}}(m)(r)\cup\denote{\setpair{I'}{U'}}(m)(r)\\
  &=\left(\denote{\setpair{I}{U}} \sqcap \denote{\setpair{I'}{U'}}\right)(m)(r).\qedhere
\end{align*}
\end{proof}

\commentout{
\begin{proof}[\textbf{Proof of Lemma~\ref{lem:efficient}}]
  \begin{enumerate}
    \item The identity function is represented as
      \[
        \denote{\id}=\{(r,\,\langle \bot_T,\,\top_T\rangle)\,|\,r\in \rcc\}\,;
      \]
      the top function is represented as
      \[
        \denote{\lambda m\,.\,\topcc}=\{(r,\,\langle \top_T,\,\top_T\rangle)\,|\,r\in \rcc\}\,.
      \]
    \item Equations~\eqref{eq:compclosed} and~\eqref{eq:meetclosed} show that the representation of micro functions is closed under composition and meet.
    \item To show that our representation for micro functions forms a lattice with finite height, let us first show that $\lcc_\rcc:\,\rcc\to2^T$ forms a lattice. Since $T$ is a finite set, $(2^T,\,\subseteq)$ is a finite-height lattice. $\rcc$ is a finite set. Hence, the mapping
    \[
      \rcc\mapsto2^T=\{(r,\,t)\,|\,r\in \rcc,\,t\in 2^T\}=\lcc_\rcc
    \]        
    also forms a finite-height lattice \cite{nielson1999principles}. 
    
    Furthermore, $\lcc_\rcc$ is a finite set. 
    Every element of $\lcc_\rcc$ can be applied to $|\rcc|$ receivers, where each receiver is mapped to a set of types. There are $|\rcc|\cdot2^{|T|}$ different possibilities to form those mappings, so
    \[
      |\lcc_\rcc|=|\rcc|\cdot2^{|T|}.
    \]
    Therefore, $\lcc_\rcc\mapsto \lcc_\rcc$ also forms a finite-height lattice.
    \item All operations can be computed in $O(\rcc\times T)$ time. Note that the $\rcc$ and $T$ sets are an input to the correlated-calls analysis, and the time it takes to compute the meet or composition of micro functions is independent of the representation of the specific operand micro functions.
    \item The space bound is $O(\rcc\times T)$.
  \end{enumerate}
\end{proof}
}

\subsection*{Soundness and Precision}

In this part of the Appendix we prove the Lemmas of Soundness and Precision of the correlated-calls analysis.

\begin{proof}[\textbf{Proof of Lemma~\ref{lem:subsetifds}}]
    \otodo{It's Section 3.2.}
  Let $P$ be an IFDS problem. Recall from Section~\ref{sec:ideToIfds} that the result of an IFDS analysis $\resultIFDS(P)$ maps supergraph
  nodes $n\in N^*$ to sets of data-flow facts $\delta\in2^D$. Specifically,
  \begin{equation}
    \resultIFDS(P)=\lambda n\,.\,\mvp F(n)\,.
  \end{equation}
  \otodo{Where do we mention $\backEq$ in this lemma?}
  The transformation $\backCC$ is the same as $\backEq$, except that it can remove data-flow facts from the result:
  \begin{align*}
    \backCC\left(\resultIDE(\transCC_R(P))\right)(n)
    &=\left(\lambda n'\,.\,\left\{d\,|\,\forall r\in R\,.\,\resultIDE(\transCC_R(P))(n')(d)(r)\ne\top_T\right\}\right)(n)\\
    &=\left\{d\,|\,\forall r\in R\,.\,\resultIDE(\transCC_R(P))(n')(d)(r)\ne\top_T\right\}\\
    &\subseteq\left\{d\,|\,\resultIDE(\transCC_R(P))(n)(d)\ne\topcc\right\}\\
    &=\backEq\left(\resultIDE(\transCC_R(P))(n)\right).\qedhere
  \end{align*}
\end{proof}

To prove the Soundness Lemma, we first introduce Lemmas~\ref{lem:sound1} and~\ref{lem:sound3}.

We will denote the top element in the environment lattice as
$\Omega=\lambda d\,.\,\topcc$.
\otodo{I've been using $\top_\textsf{Env}$ for clarity.}

For the purpose of the proofs, we will rewrite Equation~\eqref{eq:edgefndef} that defines an edge function as follows:
  \begin{equation}\label{eq:edgefnThroughDelta}
    \ccedgefn S=\lambda e\,.\,
    \begin{cases}
      \id&\text{if $d_1=d_2=\Lambda$,}\\
      \lambda m\,.\,\varepsilon(e)(\delta(m))&\text{otherwise},
    \end{cases}
  \end{equation}
  where $S\subseteq R$, $d_1$ and $d_2$ are the source and target facts, and for a map $m\in\lcc_U$, $\delta(m)$ is either $m$ or $\botcc$:
  \begin{equation}\label{eq:deltadef}
    \delta(m)=\begin{cases}
      \botcc&\text{if $d_1=\Lambda$}\\
      m&\text{otherwise.}
    \end{cases}
  \end{equation}

Additionally, for a path $p=[\startmain,\,\dots]$ and a fact $d\in D$, we will denote the lattice element that is mapped to $d$ according to the flow functions of path $p$ as follows:
\begin{equation}
  \mpd=\menv(p)(\Omega)(d)\,.
\end{equation}

The following Lemma shows that the lattice elements (receiver-to-types maps) of a correlated-calls IDE analysis correctly overapproximate the possible types of a receiver in a program execution.

\begin{lemma}\label{lem:sound1}
Let $p=[\startmain,\,\dots,\,n]$ be some concrete execution trace of the program, and let $r\in R$ be a receiver. If after the execution trace $p$, at node $n$, $r$ points to an object of runtime type~$t$, and $d\in D$ is a fact such that $d\in M_F(p)(\varnothing)$, then
  \begin{equation}
    t\in\mpd(r)\,.
  \end{equation}
\end{lemma}
\begin{proof}
  By induction on the length of the trace.
  
  \textit{Basis:} $p=[\startmain]$. Then there is no instruction at which a receiver $r$ could be instantiated, and the Lemma is trivially true.    

  \textit{Induction hypothesis:} Let $p=[\startmain,\,\dots,\,n_{k-1}]$, and let $\uptau$ be the set of types to which $\mpdkm$ maps $r$:
  \begin{equation}
    \uptau=\mpdkm(r)\,.
  \end{equation}
  Assume that for a concrete execution path $p=[\startmain,\,\dots,\,n_{k-1}]$, at node $(n_{k-1},\,d_{k-1})$, the Lemma holds, i.e. $t\in\uptau$.
  
  \textit{Induction step:} Let $p'=[\startmain,\,\dots,n_{k-1},\,n_k]$ and $t'\in T$ be the type to which $r$ is mapped at $n_k$.
  
  For each $i$, let $e_i$ be the edge $((n_{i-1},\,d_{i-1}),\,(n_i,\,d_i))$. Note that $$e_1=((\startmain,\,\Lambda),\,(n_1,\,d_1))\,.$$  

  Observe that
  \begin{align*}
    \mppd
    &=\menv(p')(\Omega)(d)\\
    &=\left(\menv(e_k)\circ\menv(e_{k-1})\circ\ldots\circ\menv(e_1)\right)(\Omega)(d)\\
    &=\menv(e_k)\left(\menv(p)(\Omega)\right)(d)\,.
  \end{align*}
  
As shown in~Sagiv et al.~\cite{sagiv1996precise}, the relationship between environment transformers and edge functions can be described with the following equation. For an edge $(n_1,\,n_2)\in E^*$ an environment $\textsf{env}$ that maps $D$ to $L$, and a fact $d\in D$,
\begin{align}\label{eq:envTransToEdgeFnEdge}
  \menv&((n_1,\,n_2))(\textsf{env})(d)\notag\\
  &=\edgefn((n_1,\,\Lambda),\,(n_2,\,d))(\top)\sqcap\bigsqcap_{d'\in D}\edgefn((n_1,\,d'),\,(n_2,\,d))(\textsf{env}(d'))\,.
\end{align}  
Then, according to~\eqref{eq:envTransToEdgeFnEdge},\mtodo{Is it okay that in~\eqref{eq:envTransToEdgeFnEdge}, we use $\edgefn$, and below we use $\ccedgefn R$?}
  \begin{align*}
    &\menv(e_k)\left(\menv(p)(\Omega)\right)(d)(r)\\
      =&\bigg(\ccedgefn R((n_{k-1},\,\Lambda),\,(n_k,\,d))(\topcc)\sqcap\\
       &\bigsqcap_{d'\in D}\ccedgefn R((n_{k-1},\,d'),\,(n_k,\,d))(\menv(p)(\Omega)(d'))\bigg)(r)\\
      \supseteq&
        \bigsqcap_{d'\in D}\ccedgefn R((n_{k-1},\,d'),\,(n_k,\,d))(\menv(p)(\Omega)(d'))(r)\\
      \supseteq&\,\ccedgefn R((n_{k-1},\,d_{k-1}),\,(n_k,\,d))(\mpdkm)(r)\,.
  \end{align*}
  Therefore, 
  \begin{equation}\label{eq:musubset}
    \efek\subseteq\mppd(r)\,.
  \end{equation}
  
  We will now show that
  \[
    t'\in\efek\,,
  \]
  which, due to~\eqref{eq:musubset}, means that the Lemma holds.
  
  According to~\eqref{eq:edgefnThroughDelta}, there are two cases in which $\ccedgefn R(e_k)$ could fall.

  If $d_{k-1}=d_k=\Lambda$, then $d_k\notin M_F(p)(\varnothing)$, since it does not belong to the set $D$, and the Lemma trivially holds.
  
  Otherwise, 
  \[
    \ccedgefn R(e_k)=\lambda m\,.\varepsilon(e_k)(\delta(m))\,.
  \]
  It follows that
  \begin{align}\label{eq:efek}
    \efek
    &=(\lambda m\,.\varepsilon(e_k)(\delta(m)))(\mpdkm)(r)\notag\\
    &=\varepsilon(e_k)(\delta(\mpdkm))(r).
  \end{align}
  Let us denote the lattice element $\delta(\mpdkm)$ with $\Delta$:
  \[
    \Delta=\delta(\mpdkm)\,.
  \]
  Note that since $\Delta$, according to~\eqref{eq:deltadef}, can be either $\botcc$ or $\mpdkm$, it always maps $r$ to a set containing~$t$:
  \begin{equation}\label{eq:deltaContainsT}
    t\in\Delta(r)\,.
  \end{equation}
  Note also that unless the instruction at $n_{k-1}$ contains an assignment for $r$, $r$ is mapped to the same object of type $t$ as at node $n_{k-1}$, and $t=t'$. Therefore, for the non-assignment instructions, it is sufficient to prove that $t\in\Delta(r)$.
 
   Depending on the instructions at the nodes $n_{k-1}$ and $n_k$, there are four cases:
  \begin{enumerate}
    \item\label{item:asgn} The instruction at $n_{k-1}$ is an assignment for a receiver $r'\in R$.
        Since $\varepsilon_R(e_k)=\lambda m\,.\,m[r'\to\bot_T]$,
        \begin{align*}
          \efek&=            
            (\lambda m\,.\,m[r'\to\bot_T])(\Delta)(r)\\
          &=\Delta[r'\to\bot_T](r)\,.
        \end{align*}
      In the resulting map, $r'$ is mapped to $\bot_T$. Then
      
      \begin{enumerate}
        \item if $r=r'$, then $\efek=\bot_T$, which contains $t'$.
        \item\label{item:defaultmap} If $r\ne r'$, then $r$ has not been reassigned a value, and still maps to the same object of type $t$. The receiver $r$ is mapped to $\Delta(r)$, which, according to~\eqref{eq:deltaContainsT}, contains $t$. Since $t=t'$, $\Delta(r)$ contains $t'$.
      \end{enumerate}
    \item\label{item:callstart} $e_k$ is a call-start edge with signature $s$, and $f$ is the called procedure.
      Then
        \begin{align*}
          \efek
          &=(\lambda m\,.\,m[r'\to m(r')\cap\tau(s,\,f)])(\Delta)(r)\\
          &=\Delta[r'\to\Delta(r')\cap\tau(s,\,f)]\,,
        \end{align*}
      where $r'$ is the receiver of the call.
      \begin{itemize}
        \item If $r'=r$, then $\Delta(r')=\Delta(r)$ which contains $t$. Since $t\in\tau(s,\,f)$, it follows that $t\in\Delta(r)\cap\tau(s,\,f)$, and $t\in\efek$.
         \item If $r'\ne r$, see~(\ref{item:defaultmap}).
       \end{itemize}
    \item $e_k$ is an end-return edge, $r_1,\,\dots,\,r_k\in R$ are the local variables in the callee method, $r'$ is the receiver of the call site corresponding to the return node $n_k$, and $f$ is the called method with signature~$s$.
      Then 
      \[
        \varepsilon_R(e_k)=\lambda m\,.\,m[r'\to m(r')\cap\tau(s,\,f)][r_1\to\bot_T]\dots[r_k\to\bot_T].
      \]
      If $r\in\{r_1,\,\dots,\,r_k\}$, see~Case~\ref{item:asgn}. Otherwise, the case is analogous to Case~\ref{item:callstart}.
    \item\label{item:idcase} The node contains any other instruction.
      Then 
      \[
        \ccedgefn R(e_k)(\mpdkm)(r)=\id(\Delta)(r)=\Delta(r),
      \]
      which contains $t$ according to~\eqref{eq:deltaContainsT}.\qedhere
  \end{enumerate}
\end{proof}

We will now show that on a node of a concrete execution path, the correlated-calls analysis does not map receivers to $\top_T$. In other words, the analysis never considers nodes of a concrete execution path unreachable.

\begin{lemma}\label{lem:sound3}
  Let $p=[\startmain,\,\dots,\,n]$ be a concrete execution path, $r\in R$ a receiver, and $d\in D$ a data-flow fact. Then if $d\in M_F(p)(\varnothing)$,
  \begin{equation}
    \mpd(r)\ne\top_T\,.
  \end{equation}
\end{lemma}
\begin{proof}
  By induction on the length of the execution trace.
  
  \textit{Basis:} 
    Let $p=[\startmain]$. Since the only realizable path corresponding to $p$ is $[(\startmain,\,\Lambda)]$, there is no fact $d\in D$ such that $d\in M_F(p)(\varnothing)$, and the claim follows immediately.
  
  \textit{Induction hypothesis:} 
  Let $p=[\startmain,\,\dots,\,n_{k-1}]$. Let $\uptau$ be the set of types to which $r$ is mapped by $\mpdkm$:
  \begin{equation}
    \uptau=\mpdkm(r)\,.
  \end{equation}
  Assume the Lemma holds for that for a concrete execution path $$p=[\startmain,\,n_1,\,\dots,\,n_{k-1}]\,,$$ i.e. $\uptau\ne\top_T$ for an arbitrary $r\in R$ and $d_{k-1}\in D$.
  
  \textit{Induction step:}
    Let $p'=[\startmain,\,n_1,\,\dots,n_{k-1},\,n_k]$ be a concrete execution path.
    
    Let $e_k=((n_{k-1},\,d_{k-1}),\,(n_k,\,d))$. As shown in~\eqref{eq:musubset},
    \begin{align*}
      \mppd(r)
        &\supseteq\ccedgefn R(e_k)(\mpdkm)(r)\,.
    \end{align*}

    From Definition~\ref{def:edgefn}, we can see that unless $e_k$ is a call-start edge or an end-return edge, the result follows from the induction hypothesis.
    More formally, if $e_k$ is not a call-start or end-return edge, then for all $m\in\lcc_R$,
    \[
      \ccedgefn R(e_k)(m)\sqsubseteq m\,.
    \]
     The edge function corresponding to the call-start and end-return edges is the only place in which the set of types that a receiver maps to can be reduced.
    
    Assume that $e_k$ is a end-return edge with a call on the receiver $r'\in R$ with a signature $s$ to a function $f$.
    \begin{align*}
      \ccedgefn R&(e_k)(\mpdkm)(r)\\
      &=\left(\lambda m\,.\,m[r'\to m(r)\cap\tau(s,\,f)][r_1\to\bot_T]\dots[r_l\to\bot_T]\right)(\mpdkm)(r)\\
      &=\left(\mpdkm[r'\to \uptau\cap\tau(s,\,f)][r_1\to\bot_T]\dots[r_l\to\bot_T]\right)(r)\,,
    \end{align*}
  where $r_1,\,\dots,r_l\in R$ are the local variables in the called method.
    
    If $r\in\{r_1,\,\dots,\,r_l\}$, then $\efek=\bot_T\ni t$\footnote{In the case of a recursive call, it is possible that both $r\in\{r_1,\,\dots,\,r_l\}$ and $r=r'$.
    In that case, the set to which $r$ will be mapped would be still ``overwritten'' by $\bot_T$.}.
    
    Otherwise, if $r=r'$, then $\efek=\uptau\cap\tau(s,\,f)$.

    According to Lemma~\ref{lem:sound1} and by the induction hypothesis, the runtime type $t$ of $r$ must be contained in $\mpdkm(r)=\uptau$. At the same time, by definition, $t$ is part of $\tau(s,\,f)$. Therefore, $t\in\uptau\cap\tau(s,\,f)\subseteq\efek$, which means that $\efek\ne\top_T$.
    
    The same reasoning applies to the case where $e_k$ is a call-start edge.
\end{proof}

For the following proofs, recall from Section~\ref{sec:ideToIfds} that the result of an IDE analysis maps a lattice element to each node in the exploded supergraph. Specifically, for an IDE problem $Q$, the result $\resultIDE(Q):\ N^*\to(D\to L)$ maps nodes of the supergraph to pairs of data-flow facts and lattice elements~\cite{sagiv1996precise}:
\begin{equation}\label{eq:ideresult}
  \resultIDE(Q)=\lambda n\,.\,\lambda d\,.\,\mvp{\textsf{Env}}(n,\,d))\,.
\end{equation}

Finally, we can prove the Soundness Lemma.

\begin{proof}[\textbf{Proof of Lemma~\ref{lem:sound}}] 
According to~\eqref{eq:ideresult}, we can rewrite~\eqref{eq:transCcDef} as
\begin{align*}\label{eq:transCcResultCc}
  \backCC(\resultIDE(\transCC_R(P)))(n)
  &=\{d'\,|\,\forall r\in R\,.\,\mvp{\textsf{Env}}(n,\,d'))(r)\ne\top_T\}\\
  &=\left\{d'\,|\,\forall r\in R\,.\bigsqcap_{q\in\textsf{VP}(n)}\menv(q)(\Omega)(d')(r)\ne\top_T\right\}\\
  &=\left\{d'\,|\,\forall r\in R\,.\bigsqcap_{q\in\textsf{VP}(n)}\mpddef q {d'}(r)\ne\top_T\right\}.
\end{align*}
  According to~Lemma~\ref{lem:sound3}, since $d\in M_F(p)(\varnothing)$, then for any $r\in R$, $\mpd(r)\ne\top_T$. 
  Since $\mpd(r)$ is a non-empty set that is contained in $\bigsqcap_{q\in\textsf{VP}(n)}\mpddef q {d}(r)$, it follows that $$\bigsqcap_{q\in\textsf{VP}(n)}\mpddef q {d}(r)\ne\top_T.$$ Therefore, $d\in \backCC(\resultIDE(\transCC_R(P)))(n)$.
\end{proof}

\subsection*{Correlated Call Receivers}
We will now present the proof for Lemma~\ref{lem:onlycorrrec} which shows that in a correlated-calls analysis, it is enough to consider only correlated-call receivers~$\rcc$.

In this section, we will denote the set of realizable paths corresponding to a valid path $p$ and a fact $d$ as $\textsf{RP}(p,\,d)$.

First, we introduce a Lemma showing that the types to which a given receiver is mapped in the result of the algorithm is not affected by other receivers and the types to which they are mapped.

\begin{lemma}\label{lem:recindepedgefn}
  Let $P$ be an IFDS problem. Let $N^*$ be the supergraph for $P$, $D$ the set of data-flow facts, $n\in N^*$ a node, and $p=[\startmain,\,\dots,\,n]$ a path in the supergraph. Let $d\in D\cup\{\Lambda\}$.
  Then for any realizable path $p'\in\textsf{RP}(p,\,d)$, set $S\subseteq R$, and receiver $r\in S$,
  \begin{equation}
    \ccedgefn S(p')(\topcc)(r)=
    \ccedgefn{\{r\}}(p')(\topcc)(r)\,.
  \end{equation}
\end{lemma}
\begin{proof}
  By induction on the length of $p$.
  
  \textit{Basis:} $p'=[(\startmain,\,\Lambda)]$. Then $\ccedgefn S(p')=\id=\ccedgefn{\{r\}}(p')$, and the Lemma follows directly.
  
  \textit{Induction hypothesis:} Suppose that for a path $q=[(\startmain,\,\Lambda),\,\dots,\,(n_{k-1},\,d_{k-1})]$, where $q\in\textsf{RP}(n,\,d)$, the Lemma holds, i.e. both edge functions map $r$ to the same set of types $\uptau$:
  \begin{align*}
    \uptau
    &=\ccedgefn S(q)(\topcc)(r)\\
    &=\ccedgefn{\{r\}}(q)(\topcc)(r)\,.
  \end{align*}
  
  \textit{Induction step:} Let $q'=[(\startmain,\,\Lambda),\,\dots,\,(n_{k-1},\,d_{k-1}),\,(n_k,\,d_k)]$ and the edge $e_k=((n_{k-1},\,d_{k-1}),\,(n_k,\,d_k))$.
  
  Observe that for any set $U\subseteq R$ such that $r\in U$,
  \begin{align}\label{eq:edgefnU}
    \ccedgefn U(q')(\topcc)(r)
    &=\ccedgefn U(e_k)(\ccedgefn U(q)(\topcc))(r)\,.
  \end{align}
  
  We can see from~\eqref{eq:edgefnThroughDelta} that there are two cases.  
  
  If $d_{k-1}=d_k=\Lambda$, $\ccedgefn S(e_k)=\id=\ccedgefn{\{r\}}(e_k)$, and, due to~\eqref{eq:edgefnU},
  \begin{align*}
    \ccedgefn S(q')(\topcc)(r)&=\uptau\\
    &=\ccedgefn{\{r\}}(q')(\topcc)(r)\,.
  \end{align*}
  
  Otherwise, there are four sub-cases.
  \begin{enumerate}
    \item $e_k$ is a call-start edge, $r'.c()$ is the call site at $n_{k-1}$ with signature $s$, $f$ is the called procedure, and $r'\in U$.
    Then
    \[
      \ccedgefn U(e_k)=\lambda m\,.\,\delta(m)[r'\to\delta(m)(r)\cap\tau(s,\,f)]\,.
    \]
    There are two sub-cases.
    \begin{enumerate}
      \item\label{item:callstartreceq} If $r=r'$, then, according to~\eqref{eq:edgefnU}, the resulting set of types 
        \[
          \ccedgefn U(q')(\topcc)(r)=\delta(\ccedgefn U(q)(\topcc))(r)\cap\tau(s,\,f).
        \]
        If $d_{k-1}=\Lambda$, then $\delta(\ccedgefn U(q)(\topcc))(r)=\botcc(r)=\bot_T$. If $d_{k-1}\ne\Lambda$, then $\delta(\ccedgefn U(q)(\topcc))(r)=\ccedgefn U(q)(\topcc)(r)=\uptau$. The set $\tau(s,\,f)$ is the same for either case.
    
        Therefore, the value of $\ccedgefn U(q')(\topcc)(r)$ has the same result regardless of $U$,
        which means that $\ccedgefn S(q')(\topcc)(r)=\ccedgefn{\{r\}}(q')(\topcc)(r)$, and the Lemma holds.
      \item\label{item:callstartrecneq} If $r\ne r'$, then
        \begin{equation}
          \ccedgefn U(q')(\topcc)(r)=\delta(\ccedgefn U(q)(\topcc))(r)\,,
        \end{equation}
        which, as we have seen in~Case~\eqref{item:callstartreceq}, does not depend on~$U$, and the Lemma holds.
    \end{enumerate}
    \item $e_k$ is an end-return edge, $r_1,\,\dots,\,r_l\in U$ are the local variables in the callee method, $r'.c()$ is the call corresponding to the return node at $n_k$, $f$ is the called method with signature $s$, and $r'\in U$.
    Then
    \[
      \ccedgefn U(e_k)=\lambda m\,.\,\delta(m)
      [r'\to\delta(m)(r)\cap\tau(s,\,f)]
      [r_1\to\bot_T]\ldots[r_l\to\bot_T]\,.
    \]
    There are three sub-cases.
    \begin{enumerate}
      \item\label{item:localvarrec} If $r\in\{r_1,\,\dots,\,r_l\}$, then regardless of the value of $U$,
      \[
        \ccedgefn U(q')(\topcc)(r)=\bot_T\,,
      \]
      and the Lemma holds.
      \item Otherwise, if $r=r'$, the case is analogous to Case~\eqref{item:callstartreceq}.
      \item If $r\notin\{r',\,r_1,\,\dots,\,r_l\}$, then see Case~\eqref{item:callstartrecneq}.
    \end{enumerate}
    \item $n_{k-1}$ contains an assignment for $r'\in U$. Then
    \[
      \ccedgefn U(e_k)=\lambda m\,.\,\delta(m)[r'\to\bot_T]\,.
    \]
    If $r=r'$, see Case~\eqref{item:localvarrec}. If $r\ne r'$, see Case~\eqref{item:callstartrecneq}.
    \item Otherwise,
    \[
      \ccedgefn U(e_k)=\lambda m\,.\,\delta(m)\,,
    \]
    and the case is analogous to Case~\eqref{item:callstartrecneq}.\qedhere
  \end{enumerate}
\end{proof}

The following Lemma shows that the correlated-calls analysis computes the results for each receiver independently, or separately. To compute the set of types to which a receiver~$r$ is mapped at each exploded-graph node, we can exclude all other receivers in the program from the analysis (recall from~\eqref{eq:edgefndef} that the set of receivers that are considered in the analysis is specified by the set $S$ in a correlated-calls transformation $\transCC_S$). Therefore, for a given receiver $r$, the results of a $\transCC_S$- and a $\transCC_{\{r\}}$-analysis are the same.

\begin{lemma}\label{lem:recindep} Let $P$ be an IFDS problem. Let $N^*$ be the supergraph for $P$, $D$ the set of data-flow facts, and $S\subseteq R$ a set of receivers.
  Then for any $n\in N^*$, $d\in D$, and receiver $r\in S$,
  \begin{equation}
    \resultIDE\left(\transCC_S(P)\right)(n)(d)(r)=
    \resultIDE(\transCC_{\{r\}}(P))(n)(d)(r)\,.
  \end{equation}
\end{lemma}
\begin{proof} Recall from Section~\ref{sec:bgide} that
\begin{equation}\label{eq:mvpdef}
  \mvp{\textsf{Env}}(n)=\bigsqcap_{q\in\ivp(n)}M_\textsf{Env}(q)(\top)
\end{equation}

  According to~\eqref{eq:ideresult}, \eqref{eq:mvpdef}, and~\eqref{eq:envTransToEdgeFnEdge},
  \begin{align}
    \resultIDE\left(\transCC_S(P)\right)(n)(d)(r)
    &=\mvp{\textsf{Env}}(n,\,d)(r)\notag\\
    &=\left(\bigsqcap_{q\in\ivp(n)}M_\textsf{Env}(q)(\Omega)(d)\right)(r)\notag\\
    &=\left(\bigsqcap_{q\in\ivp(n)}\bigsqcap_{q'\in\mathsf{RP}(q,\,d)}\ccedgefn S(q')(\topcc)\right)(r)\notag\\
    &=\bigcup_{q\in\ivp(n)}\bigcup_{q'\in\mathsf{RP}(q,\,d)}\ccedgefn S(q')(\topcc)(r)\,.\label{eq:resultThroughEdgefn}
  \end{align}
  Then from Lemma~\ref{lem:recindepedgefn},
  \begin{align*}
    \resultIDE\left(\transCC_S(P)\right)(n)(d)(r)
    &=\bigcup_{q\in\ivp(n)}\bigcup_{q'\in\mathsf{RP}(q,\,d)}\ccedgefn{\{r\}}(q')(\topcc)(r)\\
    &=\resultIDE\left(\transCC_{\{r\}}(P)\right)(n)(d)(r)\,.\qedhere
  \end{align*}
\end{proof}

The next lemma shows that the set of types to which a receiver is mapped in a correlated-calls lattice element can be represented as an intersection of static-type function applications $\tau(s_i,\,f_i)$.
\begin{lemma}\label{lem:edgefnThroughTaus}
  For an IFDS problem $P$, a node $n\in N^*$, and fact $d\in D$, let $p\in\mathsf{RP}(n,\,d)$ be a realizable path and $r\in R$ a receiver. Then there exists a non-negative number $\gamma$ of calls on the receiver $r$ with signatures $s_\gamma$ to the functions~$f_\gamma$, for which
  \[
    \ccedgefn{\{r\}}(p)(\topcc)(r)=
      \bigcap_{\gamma\ge0}\tau(s_\gamma,\,f_\gamma)\,.
  \]
\end{lemma}
\begin{proof}
  Let $p$ have the following form\footnote{It can be shown from the definition of a pointwise representation in~Sagiv et al.~\cite{sagiv1996precise} that in a realizable path, there is never an edge from a fact of the set $D$ to a $\Lambda$ fact. Therefore, we can represent $p$ as a sequence of nodes that has a prefix of $\Lambda$-fact nodes, after which all nodes are non-$\Lambda$ facts.}:
  \[
    p=[(\startmain,\,\Lambda),\,(n_1,\,\Lambda),\,\dots,\,(n_k,\,\Lambda),
       (n_{k+1},\,d_{k+1}),\,\dots,\,(n_{k+l},\,d_{k+l})]\,,
  \]
  where $l\ge1$ and the facts for all nodes up to $n_k$ are equal to $\Lambda$ and $d_{k+i}\in D$ for $0<i\le l$.
  
  As previously, for all $i$, we will denote the edge $(n_i,\,n_{i+1})$ by $e_i$.  
  
  From~\eqref{eq:edgefndef} we can infer that
  \[
    \ccedgefn{\{r\}}(p)=
    \ccedgefn{\{r\}}(e_{k+l})
    \circ\ldots
    \circ\ccedgefn{\{r\}}(e_{k+2})
    \circ(\lambda m\,.\,\beta)
    \circ\id
    \circ\ldots
    \circ\id\,,
  \]
  where
  \[
    \beta=
    \begin{cases}
      \botcc[r\to\tau(s,\,f)]&\text{if $(n_k,\,n_{k+1})$ is a call-start or end-return edge, and}\\&\text{the call site $r.c()$ with signature $s$ to the function}\\
      &\text{$f$ corresponds to the call-start or end-return edge,}\\
      \botcc&\text{otherwise\footnotemark.}
    \end{cases}
  \]
  \footnotetext{Since $d_k=\Lambda$ and $d_{k+1}\ne\Lambda$, the micro function for the edge $e_{k+1} $ is equal to $\lambda m\,.\,\varepsilon_{\{r\}}(e_{k+1})(\botcc)$. From the definition of $\varepsilon_S$~\eqref{eq:varepsilon} we can see that the only case where $\varepsilon_{\{r\}}(e_{k+1})(m)$ would not be equal to $\botcc$ is when $e_{k+1}$ is call-start or end-return edge.}
  
  Therefore,
  \begin{align}\label{eq:edgefnbeta}
    \ccedgefn{\{r\}}(p)(\topcc)
      &=\left(\ccedgefn{\{r\}}(e_{k+l})\circ\ldots\circ
        \ccedgefn{\{r\}}(e_{k+2})\right)((\lambda m\,.\,\beta)(\topcc))\notag\\
      &=\left(\ccedgefn{\{r\}}(e_{k+l})\circ\ldots\circ
        \ccedgefn{\{r\}}(e_{k+2})\circ\id\right)(\beta)\,.
  \end{align}

We can now prove the lemma by induction on $l$.

\textit{Basis:}
If $l=1$, then $\ccedgefn{\{r\}}(p)(\topcc)=\id(\beta)=\beta$.
There are two cases.

If $\beta=\botcc$, then 
\begin{align*}
  \ccedgefn{\{r\}}(p)(\topcc)(r)&=\beta(r)\\&=\bot_T\,,
\end{align*} and $\gamma=0$.

If $\beta=\botcc[r\to\tau(s,\,f)]$, then 
\[
  \ccedgefn{\{r\}}(p)(\topcc)(r)=\tau(s,\,f)\,,
\]
and $\gamma=1$.

\textit{Induction hypothesis:}
Assume that for a path $p=[(\startmain,\,\Lambda),\,\dots,\,(n_{k+l},\,d_{k+l})]$, the Lemma holds for $\gamma=N$, where $N\ge0$.

\textit{Induction step:}
Let $p'=[(\startmain,\,\Lambda),\,\dots,\,(n_{k+l},\,d_{k+l}),\,(n_{k+l+1},\,d_{k+l+1})]$.

Recall that
\begin{align*}
  \ccedgefn{\{r\}}(p')(\topcc)(r)
  &=\ccedgefn{\{r\}}(e_{k+l+1})\left(\ccedgefn{\{r\}}(p)(\topcc)\right)(r)\,.
\end{align*}

From~\eqref{eq:varepsilon} we can see that unless $e_{k+l+1}$ is a call-start or end-return edge corresponding to a call on the receiver $r$, then $\ccedgefn{\{r\}}(e_{k+l+1})(r)$ must be equal to either $\bot_T$ or $m(r)$, where $m=\ccedgefn{\{r\}}(p)(\topcc)$. 

If $\ccedgefn{\{r\}}(e_{k+l+1})(r)=\bot_T$, then the Lemma holds for $\gamma=0$. 

Otherwise,
\begin{align*}
  \ccedgefn{\{r\}}(e_{k+l+1})(\topcc)(r)&=\ccedgefn{\{r\}}(p)(\topcc)(r)\\
  &=\bigcap_{N}\tau(s_N,\,f_N)\,,
\end{align*}
and therefore $\gamma=N$.

Suppose that $e_{k+l+1}$ is a call-start edge with a call on the receiver $r$ with signature $s$ to a function $g$. Then, according to~\eqref{eq:varepsilon}, 
\[
  \ccedgefn{\{r\}}(e_{k+l+1})=\lambda m\,.\,m[r\to m(r)\cap\tau(s,\,g)]\,.
\]
Therefore,
\begin{align*}
  \ccedgefn{\{r\}}&(p')(\topcc)(r)\\
  &=\lambda m\,.\,m[r\to m(r)\cap\tau(s,\,g)]\left(\ccedgefn{\{r\}}(p)(\topcc)\right)(r)\\
  &=\ccedgefn{\{r\}}(p)(\topcc)(r)\cap\tau(s,\,g)\\
  &=\left(\bigcap_{N}\tau(s_N,\,f_N)\right)\cap\tau(s,\,g)\,,
\end{align*}
and the Lemma holds for $\gamma=N+1$.

The case where $e_{k+l+1}$ is an end-return edge is analogous to the previous case.
\end{proof}

We now show that a receiver will be only mapped to $\topcc$ if it is the receiver of a correlated call.

\begin{lemma}\label{lem:ccrectop}
   For an IFDS problem $P$, let $n\in N^*$ be a node, and $d\in D$ a dataflow fact such that there exists a realizable path $p\in\textsf{RP}(n,\,d)$. Let $T$ be the set of all types in the program.
   If there exists a receiver $r\in R$ such that
  \[
    \ccedgefn{\{r\}}(p)(\topcc)(r)=\top_T\,,                       
  \]
  then $r\in \rcc$.
\end{lemma}
\begin{proof} 
Observe that if there is a supergraph path from a method call with signature $s$ to the start of $f$, then the set $\tau(s,\,f)$ is always non-empty.
  Let $r.c()$ be a call on a receiver $r\in R$ with a method signature $s$ to a function $f$.
  If the call site is monomorphic, then $\tau(s,\,f)$ contains all types $T'\subseteq T$ that are compatible with the static type of $r$.
  If the call site is polymorphic, then $\tau(s,\,f)\subset T'$, since some types $t\in T'$ cause dispatch to a method other than~$f$.
 
 According to Lemma~\ref{lem:edgefnThroughTaus},
 \[
   \ccedgefn{\{r\}}(p)(\topcc)(r)=
      \bigcap_{\gamma\ge0}\tau(s_\gamma,\,f_\gamma).
 \]
 
Let $\tau_i=\tau(s_i,\,f_i)$. 
 For a given $k$, let $r.m_k()$ be the call site corresponding to $\tau_k$, and $T'$ the set of types compatible with the static type of $r$. Then the following is true:
 \begin{itemize}
   \item $\tau_k\ne\top_T$;
   \item if $\tau_k=T'$ then the corresponding call site is monomorphic;
   \item if $\tau_k\subset T'$ then the call site is polymorphic.
 \end{itemize}  
 
 From the conditions of the Lemma, 
 \begin{equation}
   \bigcap_{\gamma\ge 0}\tau_\gamma=\top_T\,.
 \end{equation} 
 
 If all $\tau_k=T'$, then $\bigcap_{\gamma\ge 0}\tau_\gamma$ is also equal to $T'$. Since $T'\ne\top_T$, this is a contradiction. 
 
 If exactly one $\tau_k\subset T'$ and the rest are equal to $T'$, then $\bigcap_{\gamma\ge 0}\tau_\gamma$ is equal to $\tau_k$, which cannot be $\top_T$ either.
 
 Therefore, there are at least two sets, $\tau_i$ and $\tau_j$, which are strict subsets of $T'$. Since both $\tau_i$ and $\tau_j$ are non-empty and their intersection equals $\top_T$, $\tau_i$ and $\tau_j$ must be disjoint. If $\tau_i$ and $\tau_j$ are disjoint, they must correspond to different call sites.
 
 In other words, there are at least two calls on the same receiver for which the static-type function is a strict subset of the set of types compatible with a given receiver $r$. It follows that both calls have to be polymorphic. Therefore,~${r\in\rcc}$.
\end{proof}

We will now show that if a receiver ever gets mapped to top, then it is a correlated-calls receiver.

\begin{lemma}\label{lem:ccrectop}
   For an IFDS problem $P$, let $n\in N^*$ be a node, and $d\in D$ a dataflow fact such there exists a realizable path $p\in\textsf{RP}(n,\,d)$.
   Then, if there exists a receiver $r\in R$, such that
  \[
    \resultIDE\left(\transCC_{\{r\}}(P)\right)(n)(d)(r)=\top_T\,,
  \]
  then $r\in \rcc$.
\end{lemma}
\begin{proof}
  As shown in~\eqref{eq:resultThroughEdgefn},
  \[
    \resultIDE\left(\transCC_{\{r\}}(P)\right)(n)(d)(r)=\bigcup_{q\in\ivp(n)}\bigcup_{q'\in\mathsf{RP}(q,\,d)}\ccedgefn{\{r\}}(q')(\topcc)(r)\,.
  \]
  Since the latter is equal to $\top_T$, it follows that for each realizable path $p'$ to node $n$, $\ccedgefn{\{r\}}(p')(\top)(r)=\top_T$. According to~Lemma~\ref{lem:ccrectop}, this is only possible if $r\in\rcc$.
\end{proof}

Finally, we present the proof for Lemma~\ref{lem:onlycorrrec} which states that a correlated-calls analysis that considers all receivers computes the same result as an analysis that considers only correlated-call receivers.

\begin{proof}[\textbf{Proof of Lemma~\ref{lem:onlycorrrec}}]
  By definition of $\backCC$,
  \begin{align*}
    \backCC(\resultIDE\left(\transCC_R(P)\right))
    &=\{d\,|\,\resultIDE(\transCC_R(P))(n)(d)=\ell,\,\forall r\,.\,\ell(r)\ne\top_T\}\\
    &=\{d\,|\,\forall r\in R,\,\resultIDE(\transCC_R(P))(n)(d)(r)\ne\top_T\}.
  \end{align*}
  
  \commentout{
  \begin{align*}
    \backCC(\resultIDE\left(\transCC_R(P)\right))
    &=\left\{(n,\,D_n^\Subset(\resultIDE(\transCC_R(P))))\ |\ n\in N^*\right\}.
  \end{align*}
  According to~\eqref{eq:dnq} and~Lemma~\ref{lem:recindep}, for a given $n\in N^*$,
  \begin{align*}
    D_n^\Subset&(\resultIDE(\transCC_R(P))))\\
    &=\left\{d\ |\ d\in\mvp F(n)\,\wedge\,\forall r\in R:\ \left\{(r,\,\resultIDE(\transCC_{\{r\}}(P))(n,\,d)(r))\ |\ r\in R\right\}(r)\ne\top_T\right\}\\
    &=\left\{d\ |\ d\in\mvp F(n)\,\wedge\,\forall\bm{r\in R}:\ \resultIDE(\transCC_{\{r\}}(P))(n,\,d)(r)\ne\top_T\right\}.
  \end{align*}
  }
Since, according to Lemma~\ref{lem:ccrectop}, $\resultIDE(\transCC_{\{r\}}(P))(n)(d)(r)$ can only be equal to $\top_T$ when $r\in\rcc$, we can conclude that
  \begin{align*}
    \backCC(\resultIDE\left(\transCC_R(P)\right))
    &=\{d\,|\,\bm{\forall r\in\rcc},\,\resultIDE(\transCC_{\bm\rcc}(P))(n)(d)(r)\ne\top_T\}\\
    &=\backCC(\resultIDE\left(\transCC_\rcc(P)\right)).
    \qedhere
  \end{align*}
\end{proof}

\end{document}
