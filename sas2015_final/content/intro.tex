\section{Introduction}

A control-flow graph (CFG) is an over-approximation of the possible flows 
of control in concrete executions of a program. It may contain \textit{infeasible} 
paths that cannot occur at runtime. The precision of a data-flow analysis algorithm
depends on its ability to detect and disregard such infeasible paths.
The \textit{Interprocedural Finite Distributive Subset} (IFDS) algorithm~\cite{reps1995precise}
is a general data-flow analysis algorithm that 
avoids infeasible interprocedural paths in which calls 
and returns to/from functions are not properly matched. 
The \textit{Interprocedural Distributive Environment} (IDE) algorithm~\cite{sagiv1996precise}
has the same property, but
supports a broader range of data-flow problems.

This paper presents an approach to data-flow analysis that avoids a type
of infeasible path that arises in object-oriented programs when two or more
methods are dynamically dispatched on the same receiver object.
If the
method calls are polymorphic
(i.e., the method invoked depends on the run-time type of the receiver),
then their dispatch 
behaviours are correlated, and some of the paths between them are infeasible. A recent paper~\cite{DBLP:journals/scp/Tip15} made this observation
but did not present any concrete algorithm to take advantage of it.


Our approach transforms an IFDS problem into an
IDE problem that precisely accounts for infeasible paths due to correlated calls. 
The results of this IDE problem can be mapped back to the data-flow domain of the 
original IFDS problem, but are more precise than the results of directly applying
the IFDS algorithm to the original problem.
We present a formalization of the transformation and prove its correctness:
specifically, we prove it still soundly considers all paths that are feasible,
and that it avoids flow along all paths that are infeasible due to
correlated calls.

We implemented the correlated-calls transformation and the IDE algorithm in Scala,
on top of the WALA framework for static analysis of JVM bytecode~\cite{fink2012wala}.
Our prototype implementation was tested extensively by using it to transform an IFDS-based 
taint analysis into a more precise IDE-based taint analysis, and applying the latter
to small example programs with correlated calls. Our prototype along with all tests
will be made available to the artifact evaluation committee.


The remainder of this paper is organized as follows.
Section~\ref{sec:MotivatingExample} presents a motivating example.
Section~\ref{sec:bg} reviews the IFDS and IDE algorithms.
Section~\ref{sec:cca} presents the correlated-calls transformation, 
states the correctness properties%
\footnote{
  Detailed proofs of our lemmas and theorems can be found in
  the \reportOrAppendix.
}, and discusses our implementation.
Related work is discussed in Section~\ref{sec:Related}.
Finally, Section~\ref{sec:Conclusions} presents conclusions and directions for future work.

 
 
