\section{Motivation}
  \label{sec:MotivatingExample}
  
We illustrate our approach using a small example that
applies our technique to improve the precision of taint analysis.
A taint analysis computes how string values
may flow from ``sources'', which are typically statements that read untrusted input, 
to ``sinks'', which are typically security-sensitive
operations such as calls to 
a database. In previous research~\cite{DBLP:conf/issta/GuarnieriPTDTB11,DBLP:conf/pldi/ArztRFBBKTOM14}, 
taint analysis algorithms have been formulated as IFDS problems.     

\begin{figure}
  \centering
  
  \begin{minipage}{\textwidth}
    \inputMinted{java}{ccexample.java}
  \end{minipage}
    \vspace*{-4mm}
  \caption{Example program containing correlated calls}
  \label{list:ccexample}
   
   \begin{minipage}{\textwidth}
    \tikzstyle{supergraph}=[
      fill=greyblue,
      rounded corners,
      font=\footnotesize,
      align=right,
      text=charcoal,
      drop shadow={ashadow, color=greyblue}
    ]
    \tikzstyle{supergraph_y}=[
      fill=bisque,
      rounded corners,
      font=\footnotesize,
      align=left,
      text=charcoal,
      drop shadow={ashadow, color=greyblue}
    ]
    \tikzstyle{supergraph_f}=[
      fill=lightsalmonpink,
      rounded corners,
      font=\footnotesize,
      align=left,
      text=charcoal,
      drop shadow={ashadow, color=greyblue}
    ]
    \tikzstyle{description}=[fill=white]
    \tikzstyle{n}=[fill=black,circle,inner sep=1.5pt]
    \tikzstyle{arrowtext}=[font=\tiny,color=black,above]
    
\hspace*{-10pt}
\begin{tikzpicture}
\scalebox{.8}{
% main method nodes
    \node [supergraph] (st_main) {\textsf{start}$_{\texttt{main}}$};
    \node [supergraph] (asgn_a) [below = \dist of st_main.south] {\texttt{a = args==null\,?}\\\texttt{new\,A()\,:\,new\,B()}};
    \node [supergraph] (call_foo) [below = \dist of asgn_a.south] {\textsf{call}$_\texttt{foo}$};
    \node [supergraph] (return_foo) [below = \dist of call_foo.south] {\textsf{return}$_\texttt{foo}$\\\texttt{v = a.foo()}};
    \node [supergraph] (call_bar) [below = \dist of return_foo.south] {\textsf{call}$_\texttt{bar}$};
    \node [supergraph] (return_bar) [below = \dist of call_bar.south] {\textsf{return}$_\texttt{bar}$};
    \node [supergraph] (end_main) [below = \dist of return_bar.south] {\textsf{end}$_{\texttt{main}}$};
    
% main method edges 
    \path[->,ultra thick] (st_main) edge (asgn_a);
    \path[->,ultra thick] (asgn_a) edge (call_foo);
    \path[->] (call_foo) edge (return_foo);
    \path[->,ultra thick] (return_foo) edge (call_bar);
    \path[->] (call_bar) edge (return_bar);
    \path[->,ultra thick] (return_bar) edge (end_main);
    
%A.foo method nodes
    \node [supergraph_y] (st_a_foo) [right = 5*\dist of st_main] {\textsf{start}$_{\texttt{A.foo}}$};
    \node [supergraph_y] (return_secret) [below = \dist of st_a_foo.south] {\texttt{return secret()}};
    \node [supergraph_y] (end_a_foo) [below = \dist of return_secret.south] {\textsf{end}$_\texttt{A.foo}$};

%A.foo method edges
    \path[->] (st_a_foo) edge (return_secret);
    \path[->] (return_secret) edge (end_a_foo);
    
%A.bar method nodes
    \node [supergraph_y] (st_a_bar) [right = 5*\dist of st_a_foo] {\textsf{start}$_{\texttt{A.bar}}$};
    \node [supergraph_y] (end_a_bar) [below = \dist of st_a_bar.south] {\textsf{end}$_\texttt{A.bar}$};
    
%A.bar method edges
    \path[->] (st_a_bar) edge (end_a_bar);

%B.foo method nodes
    \node [supergraph_f] (st_b_foo) [right = 5*\dist of call_bar] {\textsf{start}$_{\texttt{B.foo}}$};
    \node [supergraph_f] (return_not_secret) [below = \dist of st_b_foo.south] {\texttt{return "not secret"}};
    \node [supergraph_f] (end_b_foo) [below = \dist of return_not_secret.south] {\textsf{end}$_\texttt{B.foo}$};
    
%B.foo method edges
  \path[->] (st_b_foo) edge (return_not_secret);
  \path[->] (return_not_secret) edge (end_b_foo);
    
%B.bar method nodes
    \node [supergraph_f] (st_b_bar) [right = 5*\dist of st_b_foo] {\textsf{start}$_{\texttt{B.bar}}$};
    \node [supergraph_f] (print) [below = \dist of st_b_bar.south] {\texttt{print(s)}};
    \node [supergraph_f] (end_b_bar) [below = \dist of print.south] {\textsf{end}$_\texttt{B.bar}$};
    
%B.bar method edges
  \path[->] (st_b_bar) edge (print);
  \path[->] (print) edge (end_b_bar);
  
%interprocedural edges
  \path[->,dashed,ultra thick] (call_foo) edge[out=20,in=190] (st_a_foo);
  \path[->,dashed] (call_foo) edge[color=lightsalmonpink,out=-20,in=135] (st_b_foo);
  \path[->,dashed] (call_bar) edge[out=25,in=220] (st_a_bar);
  \path[->,dashed,ultra thick] (call_bar) edge[color=lightsalmonpink,out=18,in=160] (st_b_bar);
  \path[->,dashed,ultra thick] (end_a_foo) edge[out=220,in=15] (return_foo);
  \path[->,dashed] (end_b_foo) edge[color=lightsalmonpink,out=160,in=-60] (return_foo);
  \path[->,dashed] (end_a_bar) edge[out=210,in=40] (return_bar);
  \path[->,dashed,ultra thick] (end_b_bar) edge[color=lightsalmonpink,out=200,in=-40] (return_bar);
    
%arrow description
    \node [description] (intra_arrow_left) [above left = 2*\dist and 3.5*\dist of st_a_bar] {};
    \node [description] (intra_arrow_right) [right = \dist of intra_arrow_left] {};
    \node [description] (inter_arrow_left) [below = .5*\dist of intra_arrow_left] {};
    \node [description] (inter_arrow_right) [below = .5*\dist of intra_arrow_right] {};
    \path[->](intra_arrow_left) edge node[right = .5*\dist]{intraprocedural edge} (intra_arrow_right);
    \path[dashed,->](inter_arrow_left) edge node[right = .5*\dist]{interprocedural edge} (inter_arrow_right); 
}
\end{tikzpicture}
   \end{minipage}
   \vspace*{-25mm}
  \caption{
    Control flow supergraph for the example program of Figure~\ref{list:ccexample}.
    An infeasible path is shown in bold. 
  }%
  \label{fig:examplesupergraph}%
\end{figure} 

Figure~\ref{list:ccexample} shows a small Java program.
The program
declares a class \code{A} with a subclass \code{B},  where \code{A} defines methods 
\code{foo()} and \code{bar()} that are overridden in \code{B}.  
We assume  that secret values are created by 
an unspecified function \code{secret()}, which is called in \code{A.foo()} on 
line~\ref{line:Afoo}. 
Any write to standard output 
is assumed to be a sink (e.g., the call to \code{System.out.println()} in \code{B.bar()}).
Depending on the number of 
arguments passed to the program, the \code{main()} method of the example program creates 
either an \code{A}-object or a \code{B}-object. The 
program then calls \code{foo()} on this object
on line~\ref{line:callfoo}, which is 
followed by a call to \code{bar()} on the same object.  

 
We wish to answer the following question: Is it possible for the untrusted value 
that is read on line~\ref{line:Afoo} to flow to the print statement?
Consider the control-flow
supergraph for the example program that is shown in Figure~\ref{fig:examplesupergraph}.
 The nodes in this graph correspond to statements, method entry points (start nodes) and 
method exit points (end nodes). For each method call, the graph contains a 
distinct call-node and a return-node. Edges in the graph reflect intraprocedural control flow, 
flow of control from a caller to a callee (edges from call-nodes to start-nodes), or
flow of control from a callee back to a caller (edges from end-nodes to return-nodes). 

In our example, the control flow within each method is straightforward and
all interesting issues arise from interprocedural control flow. In particular,
since  \code{a} may point to either an \code{A}-object or a \code{B}-object, 
the call on line~\ref{line:callfoo} may dispatch to either \code{A.foo()} or to \code{B.foo()},
as is reflected by edges  
  from the node labeled $\highlight{\code{call}_\code{foo}}{greyblue}$ to the nodes labeled
  $\highlight{\code{start}_\code{A.foo()}}{bisque}$ and $\highlight{\code{start}_\code{B.foo()}}{lightsalmonpink}$
and by edges
  from the nodes labeled $\highlight{\code{end}_\code{A.foo}}{bisque}$ and $\highlight{\code{end}_\code{B.foo}}{lightsalmonpink}$ 
  to the node labeled  $\highlight{\code{return}_\code{foo}}{greyblue}$.  
Similarly, there are edges from the node labeled $\highlight{\code{call}_\code{bar}}{greyblue}$ to the nodes 
$\highlight{\code{start}_\code{A.bar()}}{bisque}$ and $\highlight{\code{start}_\code{B.bar()}}{lightsalmonpink}$, and 
edges
  from the nodes labeled $\highlight{\code{end}_\code{A.bar}}{bisque}$ and $\highlight{\code{end}_\code{B.bar}}{lightsalmonpink}$ 
  to the node labeled  $\highlight{\code{return}_\code{bar}}{greyblue}$. 

 
 
An IFDS analysis propagates data-flow facts along the edges
of a control flow supergraph such as the one in Figure~\ref{fig:examplesupergraph}. The
IFDS algorithm already avoids flow along infeasible paths from one call site, through a target method,
and returning to a different call site of the target method.
However, in this example, all methods are
called in exactly one place, so IFDS is unable to eliminate data flow along any of the
paths shown in the figure. As a result, IFDS-based taint analysis algorithms such as
\cite{DBLP:conf/issta/GuarnieriPTDTB11,DBLP:conf/pldi/ArztRFBBKTOM14} would report 
that the secret value read on line~\ref{line:Afoo}
might flow to the print statement on line~\ref{line:Bbar}. 

As we discussed previously, the calls to \code{foo()} and \code{bar()} may dispatch
to the implementations in classes \code{A} and \code{B}, 
because the receiver variable \code{a} may be bound to
 objects of type \code{A} or \code{B} at run time. 
However, the methods \code{foo()} and \code{bar()} are invoked
on \textit{the same object}. Thus the behaviours of the method calls
are \textit{correlated}: if the call to \code{foo()} dispatches to \code{A.foo()},
then the call to \code{bar()} must dispatch to \code{A.bar()}, and analogously
for \code{B.foo()} and \code{B.bar()}.
Consequently, paths such as the one shown in bold in
Figure~\ref{fig:examplesupergraph} where the calls dispatch to
\code{A.foo()} and \code{B.bar()}
are infeasible.  

Our main contribution is an algorithm for transforming an IFDS problem 
into an IDE problem that expresses the feasibility of paths
in light of correlated calls.
The approach associates with each interprocedural CFG edge 
a function that records the types of 
variables that are used as the receiver of correlated method calls. Paths that 
are composed of edges in which the same receiver expression has different types
are infeasible, and the propagation of data-flow facts along such paths is
prevented. Applying our technique to an IFDS-based taint analysis would enable
the resulting IDE-based taint analysis to determine that no secret value can flow from
line~\ref{line:Afoo} to the print statement on line~\ref{line:Bbar}. 

While the discussion in this section has focused on the specific problem of taint analysis,
our technique generally applies to \textit{any}
data-flow-analysis problem that can be expressed in the IFDS framework. This includes
many common analysis tasks such as reaching definitions, constant propagation, slicing,
typestate analysis, pointer analysis, and lightweight
shape analysis.

\subsection{Occurrences of Correlated Calls}\label{sec:occur}\vspace{-.5mm}
How often do correlated calls occur in practice? To assess the benefit of the correlated-calls analysis, we counted the number of correlated calls that occur in programs of the Dacapo benchmarks~\cite{blackburn2006dacapo}, using the WALA framework~\cite{fink2012wala}.
Our goal was to obtain an upper bound on the number of redundant IFDS-result nodes that could be potentially removed by our analysis. The results are shown in\techreport{ Table~\ref{tab:dacapostat} in } the~\reportOrAppendix.

In these programs, on average, 3\% of all call sites $C$ are polymorphic call sites $C_P$.
Out of these polymorphic call sites, a significant fraction (39\%) are correlated 
call sites $C^\Subset$. We also see that, on average,  each correlated-call receiver is involved in approximately
three correlated calls. 

\subsection{An Example from the Scala Collections Library}\label{sec:collex}
The Scala collections library contains the trait \code{TraversableOnce} that is
shared by both collections and iterators over them. 
The \code{toArray} method of this trait
creates an array and copies the contents of the collection or iterator into it:
\inputMintedNoLNs{scala}{collectionsExample.scala}
When \code{this} refers to an iterator rather than a collection, the call to \code{this.size}
extracts all elements of the iterator to count them.
 At the call to \code{copyToArray},
the iterator is already empty, so nothing is copied to the newly created array.
One could design an IFDS analysis to detect this kind of bug.

However, the implementation of \code{TraversableOnce.toArray} is actually correct because the above
code is guarded with a test: \code{if (this.isTraversableAgain) ...}
When the \code{isTraversableAgain} method returns false, as it does for
an iterator, the \code{toArray} method uses a different
(less efficient) implementation. The bug report would therefore be a false positive. 
The \code{isTraversableAgain} method is easy to analyze: it returns the constant
true in a collection and the constant false in an iterator. However, in order
to eliminate the false positive bug report, an analysis would need to rule out infeasible
paths using correlated calls. Specifically, the following path triggers the bug, but
is infeasible: first, call \code{isTraversableAgain} on a collection, returning
true, then call \code{size} and \code{copyToArray} on an iterator. Our correlated
calls analysis could determine that this path is infeasible because it calls the
collection version of \code{isTraversableAgain} but the iterator versions of
\code{size} and \code{copyToArray}.
The relevant code from \textsf{TraversableOnce} and other related traits is shown
\techreport{in Figure~\ref{fig:traversableonce} }in the \reportOrAppendix.
