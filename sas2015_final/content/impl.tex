\section{Correlated Calls Analysis}\label{sec:cca}

\subsection{Transformations from IFDS to IDE}\label{seq:transIfdsIde}

Let $G^\#$ be the exploded supergraph of an arbitrary IFDS problem. A \emph{transformation}
$\mathcal T:\ (G^\#)\to (G^\#,\,L,\,\edgefn)$
converts the IFDS problem into an IDE problem.
We consider two IFDS-to-IDE transformations: an \textit{equivalence transformation} $\transEq$ (pronounced ``t-equiv'') and
a \textit{correlated-calls transformation} $\transCC_S$ (pronounced ``t-c-c'') for a set of receivers $S$.
Both transformations keep the exploded supergraph $G^\#$ the same, and only generate different edge functions.
The solution of the IDE problem
can be mapped back to an IFDS solution.
If the equivalence transformation was used, then this solution is identical to the
solution that would be computed by the IFDS algorithm for the original IFDS problem.
If the correlated-calls transformation was used, then this solution is more precise
because it excludes flow along infeasible paths due to correlated calls.

\subsubsection{Equivalence Transformation}\label{sec:equivtrans}

The lattice
for the equivalence transformation $\transEq$ 
is the two-point lattice
    $L^\equiv=\{\bot,\ \top\}$,
where $\bot$ means ``reachable'', and $\top$ means ``not reachable''. 
The edge functions $\edgefn^\equiv$ are defined as
\begin{equation}
    \edgefn^\equiv=
    \begin{cases}
        \lambda e\,.\,\lambda m\,.\,\bot&\text{if $e=(n_1,\mathbf0)\to(n_2,d_2)$, where $d_2 \neq \mathbf0$;}\\
      \lambda e\,.\,\id&\text{otherwise.}
    \end{cases}
\end{equation}
At a~``diagonal'' edge from a $\mathbf0$-fact to a non-$\mathbf0$-fact
$d$, the micro function returns $\bot$
to make
the fact $d$ reachable. All other micro-functions are the identity function.

\subsubsection{Correlated-Calls Transformation}\label{sec:cctrans}

In the correlated-calls transformation $\transCC_R$, the 
lattice elements are maps from receivers to sets of types:
    $\lcc=\left\{\,m:\ R\to2^T\right\}$,
where $R$ is the set of considered receivers and $T$ is the set of all types.
For each receiver~$r$, the map gives an overapproximation of the possible
runtime types of $r$.
Sets of types are ordered by the superset relation, and this is lifted to maps from receivers
to sets of types, so the bottom element $\botcc$ maps every receiver to the set of all types,
and the top element $\topcc$ maps every receiver to the empty set of types.
During an actual execution, every receiver $r$ points to an object of some
runtime type. Therefore, a data-flow fact
is unreachable along a given path if its corresponding lattice element
maps any receiver to the empty set of types.

A micro-function $f\in\lcc\to \lcc$ defines how the map from receivers to types should
be updated when an instruction is executed.
The micro-function for most kinds of instructions is the identity.
On a call to and return from a specific method $m$ called on receiver $r$,
the micro-function restricts the receiver-to-type map to map $r$ only to
types consistent with the polymorphic dispatch to method $m$.
Finally, when an instruction assigns an object of unknown type to a receiver
$r$, the corresponding micro-function updates the map to map $r$ to the
set of all types. This is made precise by the following definition:
\begin{definition}\label{def:microfn}
Given a previously fixed set $S \subseteq R$ of receivers, the micro-function $\varepsilon_S(e)$
of a supergraph edge $e$ is defined as:

  \begin{align}\label{eq:varepsilon}
      &\varepsilon_S(e)= \lambda m\,.\,\\
        &\begin{cases}
            m[r\to m(r)\cap \tau(s,\,f)],
            &\text{\parbox{75mm}{
                    if $e$ is a call-start edge $r.c() \to \textsf{start}_f$ that calls
                procedure $f$ with signature $s$,
                and $r\in S$;}}\\[3ex]
        \parbox{30mm}{\[\begin{array}{l}
                m[r\to m(r)\cap \tau(s,\,f)]\\
            [v_1\to\bot_T] \ldots [v_k\to\bot_T],
        \end{array}
        \]
        }
        &
        \text{\parbox{75mm}{
                if $e$ is an end-return edge
                $\textsf{end}_f \to \textsf{return}_{r.c()}$
                from method $f$ with signature $s$ to the return node corresponding
                to the call $r.c()$,
                $v_1,\,\dots,\,v_k\in S$ are the local variables
                in $f$,
                and $r\in S$;}}\\[5ex]
            m\left[r\to \bot_T\right],
            &\text{\parbox{75mm}{if $e = n_1 \to n_2$ and $n_1$ contains an assignment
                    to $r\in S$;}}\\[2ex]
            m
                &\text{otherwise.}
        \end{cases}
  \end{align}
\end{definition}

In the above definition, the purpose of the set $S$ is to limit the set of considered receivers. We will use $S$ in Section~\ref{sec:ccreceivers}.

We can now define $\edgefn$, which assigns a micro-function to each edge in
the exploded supergraph. 
Along a $\mathbf0$-edge, the micro function is the identity. 
On a ``diagonal'' edge from $\mathbf0$ to a non-$\mathbf0$ fact that corresponds
to some data-flow fact becoming reachable, $\varepsilon_S(e)$ is applied to
$\botcc$ that maps every receiver to an object of every possible type.
On all other edges, $\varepsilon_S(e)$ is applied to the existing map before the edge.
The is formalized in the following definition.
\begin{definition}\label{def:edgefn}
For each edge $e = (n_1,d_1) \to (n_2,d_2)$, $\ccedgefn S(e)$ is defined as follows:
  \begin{equation}\label{eq:edgefndef}
      \ccedgefn S(e)=
        \begin{cases}
          \id  &\text{if $d_1=d_2=\mathbf0$},\\
          \lambda m\,.\,\varepsilon_S(e)(\botcc) &\text{if $d_1=\mathbf0$ and $d_2\ne\mathbf0$},\\
          \lambda m\,.\,\varepsilon_S(e)(m)  &\text{otherwise.}
        \end{cases}
  \end{equation}
\end{definition}


\begin{example}
  Consider the program from Figure~\ref{list:ccexample}, whose exploded supergraph appeared in Figure~\ref{fig:cc_edgefn_example}.
  Returning a secret value in method \code{A.foo} creates a ``diagonal'' edge from the $\mathbf0$-fact to the method's return value $r$. 
  The diagonal edge is labeled with $\lambda m\,.\,\botcc$, so every receiver is mapped to the set of all types~$\bot_T$.
On the end-return edge from \code{A.foo} to \code{main}, the set of types of \code{a} is restricted by the micro function $\lambda m\,.\,m[\code a\to m(\code a)\cap\{\code A\}]$ corresponding to the assignment of the return value $r$ to \code v.
On the call-start edge from \code{main} to \code{B.bar}, the possible types
of \code{a} are further restricted by the micro-function
$\lambda m\,.\,m[\code a\to m(\code a)\cap\{\code B\}]$
on the edge that passes the argument \code{v} to the parameter \code{s}.
The composition of these micro functions results in the empty set as the possible
types of \code{a}, indicating that this
path 
is 
infeasible.
\end{example}

\subsection{Converting IDE Results to IFDS Results}\label{sec:ideToIfds}

An IFDS solution $\resultIFDS$ has type $N^* \to 2^D$: it maps each program point $n$
to a set of facts $d$ that may be reached at $n$.
An IDE solution $\resultIDE$
pairs each such fact~$d$ with a lattice element $\ell$, so its type
is $N^* \to (D \to L)$.


In the equivalence transformation lattice $L^\equiv$, $\bot$ means reachable
and $\top$ means unreachable.
Therefore, an IDE solution $\rho$ computed using $\transEq$ is converted to
an IFDS solution as:
  $\backEq(\rho) = \lambda n.\{d\mid\rho(n)(d)\ne\top\}$.
In the correlated-calls transformation lattice $\lcc$, a map that maps any receiver
to the empty set of possible types means that the corresponding data-flow path is
infeasible. Therefore, an IDE solution $\rho$ computed using
$\transCC_S$ is converted to an IFDS solution as
\vspace{-.4cm}
\begin{equation}
	\backCC(\rho) = \lambda n.\{d\mid\forall r\in S\,.\,\rho(n)(d)(r)\neq\top_T\}.\label{eq:transCcDef}
\end{equation}

\subsection{Implementation of Correlated Calls Micro-Functions}

Conceptually, micro-functions are functions from $L$ to $L$, where $L$
is the IDE lattice, either $L^\equiv$ or $\lcc$ in our context. 
The IDE algorithm requires an efficient representation of micro-functions.
The representation must support the basic micro-functions that
we presented in Section~\ref{seq:transIfdsIde}, and it must
support function application, comparison, and be closed under function composition
and meet. We now propose such a representation for the correlated-calls micro-functions.

The representation of a micro-function is a map from receivers to pairs of sets of
types $I(r)$ and $U(r)$, where $U(r)$ is required to be a subset of $I(r)$. 
We use the notation
$\setpair{I}{U}$ to represent such a map, and $I(r)$ and $U(r)$ to look up the
sets corresponding to a particular receiver $r$.
The micro-function takes the existing set of possible types of the
receiver $r$, intersects it with $I(r)$, then unions it with $U(r)$:
$\denote{\setpair{I}{U}} = \lambda m\,.\,\lambda r\,.\,(m(r) \cap I(r)) \cup U(r)$.

All of the basic micro-functions defined in Definition~\ref{def:microfn} can be expressed
in this representation. The following lemmas show how function comparison, composition, and meet can be implemented using basic set operations
on $I$ and $U$.
The proofs of all of the lemmas and theorems are in the \reportOrAppendix.


\begin{restatable}{lemma}{equalEq}
\label{lem:equalEq}
    For any pair of micro-function representations $\setpair{I}{U}$, $\setpair{I'}{U'}$,
    \begin{equation}
        \forall r\,.\,I(r) = I'(r) \land U(r) = U'(r)
        \iff
        \denote{\setpair{I}{U}} = \denote{\setpair{I'}{U'}}.
    \end{equation}
\end{restatable}

\begin{restatable}{lemma}{equalComp}\label{lem:equalComp}
    For any pair of micro-function representations $\setpair{I}{U}$, $\setpair{I'}{U'}$,
    \[
        \denote{\setpair{I}{U} \circ \setpair{I'}{U'}}
        = 
        \denote{\setpair{I}{U}} \circ \denote{\setpair{I'}{U'}},
    \]
    where the composition of two micro-function representations is defined as follows:
$$\setpair{I}{U} \circ \setpair{I'}{U'} = \setpair{\lambda r\,.\,(I(r) \cap I'(r))\cup U(r)}{\lambda r\,.\,(I(r)\cap U'(r))\cup U(r)}.$$
\end{restatable}

\begin{restatable}{lemma}{equalMeet}\label{lem:equalMeet}
\label{lem:equalMeet}
    Let $\denote{\setpair{I}{U}} \sqcap \denote{\setpair{I'}{U'}}
     =\lambda m.\lambda r.\denote{\setpair{I}{U}}(m)(r)\cup\denote{\setpair{I'}{U'}}(m)(r)$.
    For any pair of micro-function representations $\setpair{I}{U}$, $\setpair{I'}{U'}$,
    \begin{equation}
        \denote{\setpair{I}{U} \sqcap \setpair{I'}{U'}}
        = 
        \denote{\setpair{I}{U}} \sqcap \denote{\setpair{I'}{U'}},
    \end{equation}
    where the meet of two micro-function representations is defined as follows:
$$\setpair{I}{U} \sqcap \setpair{I'}{U'} = \setpair{\lambda r\,.\,I(r) \cup I'(r)}{\lambda r\,.\,U(r) \cup U'(r)}.$$
\end{restatable}

\subsection{Theoretical Results}

The following lemma shows that our analysis is sound, i.e. 
that the resulting IDE problem still considers all data-flow
paths that are actually feasible.

\begin{restatable}[Soundness]{lemma}{sound}\label{lem:sound}
  Let $P$ be an IFDS problem and $p=[\startmain,\,\dots,\,n]$ a concrete execution path, and let $d\in D$.
  If $d\in M_F(p)(\varnothing),$
  then
  $$
    d\in \backCC\left(\resultIDE(\transCC_R(P))\right)(n)\,.
  $$
\end{restatable}

We also show that the result of an IDE problem obtained through a correlated-calls transformation is a subset of the original IFDS result.

\begin{restatable}[Precision]{lemma}{precision}\label{lem:subsetifds}
    For an IFDS problem $P$ and all ${n\in N^*}$,
    \begin{equation}
      \backCC\left(\resultIDE(\transCC_R(P))\right)(n)
      \subseteq
      \resultIFDS(P)(n)\,.
    \end{equation}
\end{restatable}

\subsection{Correlated-Call Receivers}\label{sec:ccreceivers}
We will now show that in a correlated-calls transformation, it is enough to consider only some of the receivers of set $R$.

\begin{definition}
    If $r\in R$ is the receiver of at least two polymorphic
    call sites, then we call $r$ a \textit{correlated-call receiver}, and we define $\rcc$ as the set of all
  such receivers.
\end{definition}

We will show that it is sufficient for the correlated-calls
micro-functions to be defined only on correlated-call receivers.
Specifically, a ``reduced'' correlated-calls transformation that
considers only correlated-call receivers in the micro-functions
yields the same solution as the full correlated-calls transformation
(i.e. no precision is lost).

\begin{restatable}{lemma}{onlycorrrec}\label{lem:onlycorrrec}
Let $P$ be an IFDS problem. Then  
  \begin{equation}
    \backCC\left(\resultIDE\left(\transCC_\rcc(P)\right)\right)=\backCC(\resultIDE\left(\transCC_R(P)\right))\,.
  \end{equation}
\end{restatable}

\subsection{Efficiency}

Both the IFDS and IDE algorithms have been proven
to run in $O(ED^3)$ time~\cite{reps1995precise,sagiv1996precise},
where $E$ is the number of edges in the (non-exploded) supergraph, and $D$ is the size
of the set of facts.
The IDE algorithm may evaluate micro-functions up to $O(ED^3)$ times,
so this running time must be multiplied by the cost of evaluating a
micro-function. We show that the micro-functions in the correlated-calls
IDE analysis can be evaluated in time $O(\rcc T)$, where $\rcc$ is the number
of correlated-call receivers $\rcc$ and the $T$ is the number of run-time
types. Therefore, the overall worst-case cost of the correlated-calls IDE analysis
is $O(ED^3\rcc T)$. In practice, $\rcc$ is much smaller
than $R$, so Lemma~\ref{lem:onlycorrrec} is significant for 
performance.

Specifically, the complexity proof for the IDE algorithm requires
the implementation of the micro-functions to be \textit{efficient}
according to a list of specific criteria. Our micro-function
implementation does satisfy the criteria:

\begin{restatable}{lemma}{efficient}\label{lem:efficient}
  The correlated-call representation of a micro function is efficient according to the IDE criteria~\cite{sagiv1996precise} and the required operations
  on micro-functions can be computed in time $O(\rcc T)$.
\end{restatable}

\subsection{Implementation of the Correlated-Calls Analysis}

We implemented the correlated-calls analysis in Scala~\cite{odersky2004overview}. 
Our implementation analyzes JVM bytecode compiled from input programs written
in Java.
We use WALA~\cite{fink2012wala} to retrieve information about an input program, such 
as its control-flow supergraph and the set of receivers and their types.
Since WALA does not contain an implementation of the IDE algorithm,
we implemented it from scratch;
we are working on contributing our infrastructure to WALA. 
 
We tested our correlated-calls analysis using an IFDS taint-analysis  
as a client analysis. To this end, we converted the IFDS taint analysis into an IDE problem with an 
implementation of $\transCC_\rcc$. We extensively tested the correlated-calls analysis to ensure that, in 
the absence of correlated calls, the analysis produces the same results as an IFDS-equivalent analysis, and that 
it produces more precise results in the presence of correlated calls as expected.

To evaluate the practicality of our approach, we applied two variants of the IFDS taint analysis
to the SPEC JVM98 benchmarks: 
(i) an equivalent IDE taint analysis obtained using $\transEq$, and
(ii) an IDE taint analysis obtained using   $\transCC_\rcc$ that avoids imprecision due to correlated method
calls.

The equivalence analysis is there for two reasons: 
(i)  to explain how a correlated-calls-IDE problem can be derived from an IDE problem that has the same meaning as the original IFDS problem, and 
(ii) to provide a base line against which to compare the efficiency of the correlated-calls analysis.
We compare the efficiency of the correlated-calls analysis against the equivalence-IDE analysis instead of the IFDS analysis because the time complexities of an IFDS and an equivalent IDE analysis are the same: an equivalent IDE analysis is just an IFDS analysis in which all edges are labeled with identity micro functions, and all operations on those functions are optimized to be constant-time.

The running times $t_\Subset$ of the correlated-calls and $t_\equiv$ equivalence analyses are shown in Table~\ref{tab:runtimes}. In the table, $N^*_r$ is the number of reachable nodes in the control-flow supergraph, and $N^\#_r$ the number of reachable nodes in the exploded supergraph.
\vspace{-.7cm}
\begin{table}[h]
\caption{Running times of the analyses}\label{tab:runtimes}
\centering
\setlength{\tabcolsep}{2ex}
\begin{tabular}{lrrrrr}
\toprule
\textbf{Benchmark}   & $\bm{N^*}$ & $\bm{N^*_r}$ & $\bm{N^\#_r}$ & $\bm{t_\equiv}$ & $\bm{t_{\Subset}}$ \\ \midrule
\textbf{compress}    & 36,420     & 2,155        & 24,730        & 0:00:03         & 0:00:02            \\
\textbf{db}          & 38,258     & 2,285        & 22,938        & 0:00:05         & 0:00:07            \\
\textbf{jack}        & 55,286     & 17,602       & 284,625       & 0:10:55         & 0:10:32            \\
\textbf{jess}        & 62,453     & 14,448       & 316,418       & 0:12:11         & 0:11:51            \\
\textbf{javac}       &            &              &               &                 & 2:00:54            \\
\textbf{mpegaudio}   & 68,425     & 11,959       & 224,886       & 0:00:54         & 0:00:54            \\
\textbf{mtrt}        &            &              &               &                 &                    \\
\textbf{raytrace}    &            &              &               &                 &                    \\ \bottomrule
\end{tabular}
\end{table}


The results suggest that the overhead of tracking correlated calls is acceptable. In particular, the correlated-calls analysis takes at most twice as long as the equivalence analysis.
The absolute times range from 
a few seconds on the smaller SPEC programs to about two hours on \texttt{javac}.

Our implementation is a research prototype and many opportunities for
optimization remain.
For the specific combination of this IFDS client analysis and these benchmark
programs, tracking correlated calls did not impact precision. 
