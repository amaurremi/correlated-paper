\section{Related Work}
  \label{sec:Related}
The IFDS algorithm is an instance of the functional approach to data-flow analysis developed by Sharir and Pnueli~\cite{pnueli1981two}.
IFDS has been used to encode a variety of data-flow problems such as typestate analysis~\cite{naeem2008typestate,DBLP:conf/pldi/ZhangMGNY14} and shape analysis%
~\cite{DBLP:conf/birthday/KreikerRRSWY13}. IFDS has been used~\cite{DBLP:conf/pldi/ArztRFBBKTOM14,tripp2009taj} and extended~\cite{DBLP:conf/sigsoft/LerchHBM14} to solve taint-analysis problems.

Naeem and Lhot\'ak~\cite{naeem2010pei} proposed several extensions of IFDS.
In particular, they propose several techniques for improving the algorithm's 
efficiency, as well as a technique that improves expressiveness by extending
applicability to a wider class of dataflow analysis problems. 
These extensions are orthogonal to, and could be combined with the approach
presented in this paper. Our work differs from theirs by targeting
analysis precision, not efficiency or expressiveness.

Bodden et~al.~\cite{bodden2013spl} presents a framework
for applying IFDS analyses to software product lines. Their approach enables the analysis of
all possible products derived from a product line in a single analysis pass. Like our approach,
their approach transforms IFDS problems to IDE problems. The micro-functions keep track
of the possible program variations specified by the product line.
Rodriguez and Lhot\'ak evaluate a parallelized implementation of the IFDS algorithm using actors~\cite{rodriguez2010concurrent} that can take advantage of multiple processors.

The idea of using correlated calls to remove infeasible paths in data-flow analyses of object-oriented programs 
was introduced by Tip~\cite{DBLP:journals/scp/Tip15}. The possibility of using IDE to achieve this is mentioned, 
but not elaborated upon. Our work is the first to present and implement a concrete solution.

Recent work on correlation tracking for JavaScript~\cite{DBLP:conf/ecoop/SridharanDCST12} 
also eliminates infeasible paths. Instead of infeasible paths
between dynamically dispatched method calls, their approach eliminates
infeasible paths between reads and writes of different properties
of an object. 
The approach differs from ours in that it targets points-to analysis rather
than IFDS analyses, in that it targets infeasible paths due to different
property names rather than different dynamically dispatched methods,
and in that it employs context sensitivity to improve precision.

Our approach superficially resembles, but is orthogonal to, context
sensitivity, including the CPA algorithm~\cite{Agesen:95} and such variations as object sensitivity~\cite{DBLP:journals/tosem/MilanovaRR05}.
Context-sensitive
points-to analysis is orthogonal to our work because it analyzes the
flow of data (pointers), whereas we analyze control flow paths.
Also, object-sensitive points-to analysis is flow-insensitive, while
IFDS and IDE are flow-sensitive analyses. Note that our transformation only makes sense
in a flow-sensitive setting since a flow-insensitive analysis already
introduces many infeasible control flow paths.

It would be possible to simulate the effect of our correlated calls
transformation in the following way inspired by context-sensitivity:
we could re-analyze each method in a number
of contexts. There would be a separate context for every possible
assignment of concrete types to all of the pointers in the method that
are used as receivers at a call site. The number of such contexts for
each method would be $O(R^T)$, where $R$ is the number of receiver pointers
in the method and $T$ is the number of possible concrete types that could
be assigned to a receiver pointer. Our approach computes equally precise
analysis results but avoids this exponential cost.

