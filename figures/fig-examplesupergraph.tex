\begin{figure}
  \centering
  
  \begin{minipage}{\textwidth}
    \inputMinted{java}{ccexample.java}
  \end{minipage}
    \vspace*{-4mm}
  \caption{Example program containing correlated calls}
  \label{list:ccexample}
   
   \begin{minipage}{\textwidth}
    \tikzstyle{supergraph}=[
      fill=greyblue,
      rounded corners,
      font=\footnotesize,
      align=right,
      text=charcoal,
      drop shadow={ashadow, color=greyblue}
    ]
    \tikzstyle{supergraph_y}=[
      fill=bisque,
      rounded corners,
      font=\footnotesize,
      align=left,
      text=charcoal,
      drop shadow={ashadow, color=greyblue}
    ]
    \tikzstyle{supergraph_f}=[
      fill=lightsalmonpink,
      rounded corners,
      font=\footnotesize,
      align=left,
      text=charcoal,
      drop shadow={ashadow, color=greyblue}
    ]
    \tikzstyle{description}=[fill=white]
    \tikzstyle{n}=[fill=black,circle,inner sep=1.5pt]
    \tikzstyle{arrowtext}=[font=\tiny,color=black,above]
    
\hspace*{-10pt}
\begin{tikzpicture}
\scalebox{.8}{
% main method nodes
    \node [supergraph] (st_main) {\textsf{start}$_{\texttt{main}}$};
    \node [supergraph] (asgn_a) [below = \dist of st_main.south] {\texttt{a = args==null\,?}\\\texttt{new\,A()\,:\,new\,B()}};
    \node [supergraph] (call_foo) [below = \dist of asgn_a.south] {\textsf{call}$_\texttt{foo}$};
    \node [supergraph] (return_foo) [below = \dist of call_foo.south] {\textsf{return}$_\texttt{foo}$\\\texttt{v = a.foo()}};
    \node [supergraph] (call_bar) [below = \dist of return_foo.south] {\textsf{call}$_\texttt{bar}$};
    \node [supergraph] (return_bar) [below = \dist of call_bar.south] {\textsf{return}$_\texttt{bar}$};
    \node [supergraph] (end_main) [below = \dist of return_bar.south] {\textsf{end}$_{\texttt{main}}$};
    
% main method edges 
    \path[->,ultra thick] (st_main) edge (asgn_a);
    \path[->,ultra thick] (asgn_a) edge (call_foo);
    \path[->] (call_foo) edge (return_foo);
    \path[->,ultra thick] (return_foo) edge (call_bar);
    \path[->] (call_bar) edge (return_bar);
    \path[->,ultra thick] (return_bar) edge (end_main);
    
%A.foo method nodes
    \node [supergraph_y] (st_a_foo) [right = 5*\dist of st_main] {\textsf{start}$_{\texttt{A.foo}}$};
    \node [supergraph_y] (return_secret) [below = \dist of st_a_foo.south] {\texttt{return secret()}};
    \node [supergraph_y] (end_a_foo) [below = \dist of return_secret.south] {\textsf{end}$_\texttt{A.foo}$};

%A.foo method edges
    \path[->] (st_a_foo) edge (return_secret);
    \path[->] (return_secret) edge (end_a_foo);
    
%A.bar method nodes
    \node [supergraph_y] (st_a_bar) [right = 5*\dist of st_a_foo] {\textsf{start}$_{\texttt{A.bar}}$};
    \node [supergraph_y] (end_a_bar) [below = \dist of st_a_bar.south] {\textsf{end}$_\texttt{A.bar}$};
    
%A.bar method edges
    \path[->] (st_a_bar) edge (end_a_bar);

%B.foo method nodes
    \node [supergraph_f] (st_b_foo) [right = 5*\dist of call_bar] {\textsf{start}$_{\texttt{B.foo}}$};
    \node [supergraph_f] (return_not_secret) [below = \dist of st_b_foo.south] {\texttt{return "not secret"}};
    \node [supergraph_f] (end_b_foo) [below = \dist of return_not_secret.south] {\textsf{end}$_\texttt{B.foo}$};
    
%B.foo method edges
  \path[->] (st_b_foo) edge (return_not_secret);
  \path[->] (return_not_secret) edge (end_b_foo);
    
%B.bar method nodes
    \node [supergraph_f] (st_b_bar) [right = 5*\dist of st_b_foo] {\textsf{start}$_{\texttt{B.bar}}$};
    \node [supergraph_f] (print) [below = \dist of st_b_bar.south] {\texttt{print(s)}};
    \node [supergraph_f] (end_b_bar) [below = \dist of print.south] {\textsf{end}$_\texttt{B.bar}$};
    
%B.bar method edges
  \path[->] (st_b_bar) edge (print);
  \path[->] (print) edge (end_b_bar);
  
%inter-procedural edges
  \path[->,dashed,ultra thick] (call_foo) edge[out=20,in=190] (st_a_foo);
  \path[->,dashed] (call_foo) edge[color=lightsalmonpink,out=-20,in=135] (st_b_foo);
  \path[->,dashed] (call_bar) edge[out=25,in=220] (st_a_bar);
  \path[->,dashed,ultra thick] (call_bar) edge[color=lightsalmonpink,out=18,in=160] (st_b_bar);
  \path[->,dashed,ultra thick] (end_a_foo) edge[out=220,in=15] (return_foo);
  \path[->,dashed] (end_b_foo) edge[color=lightsalmonpink,out=160,in=-60] (return_foo);
  \path[->,dashed] (end_a_bar) edge[out=210,in=40] (return_bar);
  \path[->,dashed,ultra thick] (end_b_bar) edge[color=lightsalmonpink,out=200,in=-40] (return_bar);
    
%arrow description
    \node [description] (intra_arrow_left) [above left = 2*\dist and 3.5*\dist of st_a_bar] {};
    \node [description] (intra_arrow_right) [right = \dist of intra_arrow_left] {};
    \node [description] (inter_arrow_left) [below = .5*\dist of intra_arrow_left] {};
    \node [description] (inter_arrow_right) [below = .5*\dist of intra_arrow_right] {};
    \path[->](intra_arrow_left) edge node[right = .5*\dist]{intra-procedural edge} (intra_arrow_right);
    \path[dashed,->](inter_arrow_left) edge node[right = .5*\dist]{inter-procedural edge} (inter_arrow_right); 
}
\end{tikzpicture}
   \end{minipage}
   \vspace*{-25mm}
  \caption{
    Control flow supergraph for the example program of Figure~\ref{list:ccexample}.
    An infeasible path is shown in bold. 
  }%
  \label{fig:examplesupergraph}%
\end{figure} 