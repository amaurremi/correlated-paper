\begin{figure}
  \begin{minipage}{.5\textwidth}
    \inputMinted{java}{ccexample.java}
%    \vspace*{-4mm}
    \caption{Example program containing\\ correlated calls}
  \label{list:ccexample}
  \end{minipage}
   \begin{minipage}{.5\textwidth}
    \tikzstyle{supergraph}=[
      fill=greyblue,
      rounded corners,
      font=\footnotesize,
      align=right,
      text=charcoal,
      drop shadow={ashadow, color=greyblue}
    ]
    \tikzstyle{supergraph_y}=[
      fill=bisque,
      rounded corners,
      font=\footnotesize,
      align=left,
      text=charcoal,
      drop shadow={ashadow, color=greyblue}
    ]
    \tikzstyle{supergraph_f}=[
      fill=lightsalmonpink,
      rounded corners,
      font=\footnotesize,
      align=left,
      text=charcoal,
      drop shadow={ashadow, color=greyblue}
    ]
    \tikzstyle{description}=[fill=white]
    \tikzstyle{n}=[fill=black,circle,inner sep=1.5pt]
    \tikzstyle{arrowtext}=[font=\tiny,color=black,above]
    
\hspace*{-20pt}
\begin{tikzpicture}
\scalebox{.8}{
% main method nodes
    \node [supergraph] (st_main) {\textsf{start}$_{\texttt{main}}$};
    \node [supergraph] (asgn_a) [below = \dist of st_main.south] {\texttt{a = args==null\,?}\\\texttt{new\,A()\,:\,new\,B()}};
    \node [supergraph] (call_foo) [below = \dist of asgn_a.south] {\textsf{call}$_\texttt{foo}$};
    \node [supergraph] (return_foo) [below = \dist of call_foo.south] {\textsf{return}$_\texttt{foo}$\\\texttt{v = a.foo()}};
    \node [supergraph] (call_bar) [below = \dist of return_foo.south] {\textsf{call}$_\texttt{bar}$};
    \node [supergraph] (return_bar) [below = \dist of call_bar.south] {\textsf{return}$_\texttt{bar}$};
    \node [supergraph] (end_main) [below = \dist of return_bar.south] {\textsf{end}$_{\texttt{main}}$};
    
% main method edges 
    \path[->,ultra thick] (st_main) edge (asgn_a);
    \path[->,ultra thick] (asgn_a) edge (call_foo);
    \path[->] (call_foo) edge (return_foo);
    \path[->,ultra thick] (return_foo) edge (call_bar);
    \path[->] (call_bar) edge (return_bar);
    \path[->,ultra thick] (return_bar) edge (end_main);
    
%A.foo method nodes
    \node [supergraph_y] (st_a_foo) [right = 5*\dist of st_main] {\textsf{start}$_{\texttt{A.foo}}$};
    \node [supergraph_y] (return_secret) [below = \dist of st_a_foo.south] {\texttt{return secret()}};
    \node [supergraph_y] (end_a_foo) [below = \dist of return_secret.south] {\textsf{end}$_\texttt{A.foo}$};

%A.foo method edges
    \path[->] (st_a_foo) edge (return_secret);
    \path[->] (return_secret) edge (end_a_foo);
    
%A.bar method nodes
    \node [supergraph_y] (st_a_bar) [right = 5*\dist of st_a_foo] {\textsf{start}$_{\texttt{A.bar}}$};
    \node [supergraph_y] (end_a_bar) [below = \dist of st_a_bar.south] {\textsf{end}$_\texttt{A.bar}$};
    
%A.bar method edges
    \path[->] (st_a_bar) edge (end_a_bar);

%B.foo method nodes
    \node [supergraph_f] (st_b_foo) [right = 5*\dist of call_bar] {\textsf{start}$_{\texttt{B.foo}}$};
    \node [supergraph_f] (return_not_secret) [below = \dist of st_b_foo.south] {\texttt{return "not secret"}};
    \node [supergraph_f] (end_b_foo) [below = \dist of return_not_secret.south] {\textsf{end}$_\texttt{B.foo}$};
    
%B.foo method edges
  \path[->] (st_b_foo) edge (return_not_secret);
  \path[->] (return_not_secret) edge (end_b_foo);
    
%B.bar method nodes
    \node [supergraph_f] (st_b_bar) [right = 5*\dist of st_b_foo] {\textsf{start}$_{\texttt{B.bar}}$};
    \node [supergraph_f] (print) [below = \dist of st_b_bar.south] {\texttt{print(s)}};
    \node [supergraph_f] (end_b_bar) [below = \dist of print.south] {\textsf{end}$_\texttt{B.bar}$};
    
%B.bar method edges
  \path[->] (st_b_bar) edge (print);
  \path[->] (print) edge (end_b_bar);
  
%interprocedural edges
  \path[->,dashed,ultra thick] (call_foo) edge[out=-5,in=198] (st_a_foo);
  \path[->,dashed] (call_foo) edge[color=lightsalmonpink,out=-20,in=135] (st_b_foo);
  \path[->,dashed] (call_bar) edge[out=25,in=220] (st_a_bar);
  \path[->,dashed,ultra thick] (call_bar) edge[color=lightsalmonpink,out=18,in=160] (st_b_bar);
  \path[->,dashed,ultra thick] (end_a_foo) edge[out=220,in=15] (return_foo);
  \path[->,dashed] (end_b_foo) edge[color=lightsalmonpink,out=180,in=-50] (return_foo);
  \path[->,dashed] (end_a_bar) edge[out=210,in=40] (return_bar);
  \path[->,dashed,ultra thick] (end_b_bar) edge[color=lightsalmonpink,out=190,in=-40] (return_bar);
    
%arrow description
    \node [description] (intra_arrow_left) [above left = 2*\dist and 3.5*\dist of st_a_bar] {};
    \node [description] (intra_arrow_right) [right = \dist of intra_arrow_left] {};
    \node [description] (inter_arrow_left) [below = .5*\dist of intra_arrow_left] {};
    \node [description] (inter_arrow_right) [below = .5*\dist of intra_arrow_right] {};

%    \path[->](intra_arrow_left) edge node[right = .5*\dist]{intraprocedural edge} (intra_arrow_right);
%    \path[dashed,->](inter_arrow_left) edge node[right = .5*\dist]{interprocedural edge} (inter_arrow_right); 
}
\end{tikzpicture}
\vspace*{-18mm}
  \caption{
    Control flow supergraph for the example program of Figure~\ref{list:ccexample}.
    Dashed lines depict interprocedural edges.
    An infeasible path is shown in bold. 
  }%
  \label{fig:examplesupergraph}%
\end{minipage}

\begin{minipage}{\textwidth}
\begin{figure}[h!]%
  \centering
    \tikzstyle{supergraph}=[
      fill=greyblue,
      rounded corners,
      font=\footnotesize,
      align=right,
      text=charcoal,
      drop shadow={ashadow, color=greyblue}
    ]
    \tikzstyle{supergraph_y}=[
      fill=bisque,
      rounded corners,
      font=\footnotesize,
      align=left,
      text=charcoal,
      drop shadow={ashadow, color=greyblue}
    ]
    \tikzstyle{supergraph_f}=[
      fill=lightsalmonpink,
      rounded corners,
      font=\footnotesize,
      align=left,
      text=charcoal,
      drop shadow={ashadow, color=greyblue}
    ]
    \tikzstyle{description}=[fill=white]
    \tikzstyle{n}=[fill=black,circle,inner sep=1.5pt]
    \tikzstyle{arrowtext}=[font=\tiny,color=black,above]
    
\hspace*{-10pt}
\begin{tikzpicture}
\scalebox{.8}{
% main method nodes
    \node [supergraph] (st_main) {\textsf{start}$_{\texttt{main}}$};
    \node [supergraph] (asgn_a) [below = \dist of st_main.south east, anchor=east] {\texttt{a = args==null\,?}\\\texttt{new\,A()\,:\,new\,B()}};
    \node [supergraph] (call_foo) [below = \dist of asgn_a.south east, anchor=east] {\textsf{call}$_\texttt{foo}$};
    \node [supergraph] (return_foo) [below = \dist of call_foo.south east,anchor=east] {\textsf{return}$_\texttt{foo}$\\\texttt{v = a.foo()}};
    \node [supergraph] (call_bar) [below = \dist of return_foo.south east, anchor=east] {\textsf{call}$_\texttt{bar}$};
    \node [supergraph] (return_bar) [below = \dist of call_bar.south east, anchor=east] {\textsf{return}$_\texttt{bar}$};
    \node [supergraph] (end_main) [below = \dist of return_bar.south east, anchor=east] {\textsf{end}$_{\texttt{main}}$};
    
    \node [n,label=$\mathbf0$] (st_main_z) [right=\nd of st_main.east] {};
    \node [n,label=\texttt{v}] (st_main_v) [right=\nd of st_main_z] {};
    \node [n] (asgn_a_z) [right=\nd of asgn_a.east] {};
    \node [n] (call_foo_z) [right=\nd of call_foo.east] {};
    \node [n] (return_foo_z) [right=\nd of return_foo.east] {};
    \node [n] (call_bar_z) [right=\nd of call_bar.east] {};
    \node [n] (return_bar_z) [right=\nd of return_bar.east] {};
    \node [n] (end_main_z) [right=\nd of end_main.east] {};
    \node [n] (asgn_a_v) [right=\nd of asgn_a_z] {};
    \node [n] (call_foo_v) [right=\nd of call_foo_z] {};
    \node [n] (return_foo_v) [right=\nd of return_foo_z] {};
    \node [n] (call_bar_v) [right=\nd of call_bar_z] {};
    \node [n] (return_bar_v) [right=\nd of return_bar_z] {};
    \node [n] (end_main_v) [right=\nd of end_main_z] {};
% main method edges
    \path[->, ultra thick](st_main_z) edge (asgn_a_z);
    \path[->, ultra thick](asgn_a_z) edge (call_foo_z);
    \path[->](call_foo_z) edge (return_foo_z);
    \path[->](return_foo_z) edge (call_bar_z);
    \path[->](call_bar_z) edge (return_bar_z);
    \path[->](return_bar_z) edge (end_main_z);  
    \path[->](st_main_v) edge (asgn_a_v);
    \path[->](asgn_a_v) edge (call_foo_v);
    \path[->](call_foo_v) edge (return_foo_v);
    \path[->, ultra thick](return_foo_v) edge (call_bar_v);
    \path[->](call_bar_v) edge (return_bar_v);
    \path[->](return_bar_v) edge (end_main_v);  
    
%A.foo method nodes
    \node [supergraph_y] (st_a_foo) [right = 6*\dist of st_main] {\textsf{start}$_{\texttt{A.foo}}$};
    \node [supergraph_y] (return_secret) [below = \dist of st_a_foo.south west, anchor=west] {\texttt{return secret()}};
    \node [supergraph_y] (end_a_foo) [below = \dist of return_secret.south west,anchor = west] {\textsf{end}$_\texttt{A.foo}$};
    
    \node [n,label=$r$] (st_a_foo_p) [left=\nd of st_a_foo.west] {};
    \node [n] (return_secret_p) [left=\nd of return_secret.west] {};
    \node [n] (end_a_foo_p) [left=\nd of end_a_foo.west] {};
    \node [n,label=$\mathbf0$] (st_a_foo_z) [left=\nd of st_a_foo_p] {};
    \node [n] (return_secret_z) [left=\nd of return_secret_p] {};
    \node [n] (end_a_foo_z) [left=\nd of end_a_foo_p] {};
%A.foo method edges
    \path[->, ultra thick](st_a_foo_z) edge (return_secret_z); 
    \path[->](return_secret_z) edge node[arrowtext,near start,above right=0.4cm and -.15cm,rotate=-60]{$\lambda m.\botcc$} (end_a_foo_z); 
    \path[->,dashed, ultra thick](call_foo_z) edge [out=38,in=238]  (st_a_foo_z);
    \path[->,dashed](end_a_foo_z) edge [out=200,in=43] (return_foo_z);
    \path[->,dashed, ultra thick](end_a_foo_p) edge  [out=200,in=43] node[arrowtext,below,rotate=30]{$\lambda m.m[\texttt a\to m(\texttt a)\cap\{\texttt A\}]$} (return_foo_v);
    \path[->](st_a_foo_p) edge (return_secret_p); 
    \path[->](return_secret_p) edge (end_a_foo_p);
    \path[->, ultra thick](return_secret_z) edge (end_a_foo_p);
    
%A.bar method nodes
    \node [supergraph_y] (st_a_bar) [right = 6*\dist of st_a_foo] {\textsf{start}$_{\texttt{A.bar}}$};
    \node [supergraph_y] (end_a_bar) [below = \dist of st_a_bar.south west,anchor = west] {\textsf{end}$_\texttt{A.bar}$};
    
    \node [n,label=\texttt{s}] (st_a_bar_s) [left=\nd of st_a_bar.west] {};
    \node [n,label=$\mathbf0$] (st_a_bar_z) [left=\nd of st_a_bar_s] {};
    \node [n] (end_a_bar_s) [left=\nd of end_a_bar.west] {};    
    \node [n] (end_a_bar_z) [left=\nd of end_a_bar_s] {};
%A.bar method edges
    \path[->](st_a_bar_z) edge (end_a_bar_z);
    \path[->,dashed](call_bar_z) edge [out=20,in=229] (st_a_bar_z);
    \path[->,dashed](end_a_bar_z) edge [out=220,in=35] (return_bar_z);
    \path[->](st_a_bar_s) edge (end_a_bar_s);
    \path[->,dashed](call_bar_v) edge [out=20,in=230] node[arrowtext,near end,above left=-.1cm and -.15cm,rotate=36]{$\lambda m.m[\texttt a\to m(\texttt a)\cap\{\texttt A\}]$} (st_a_bar_s);


%B.foo method nodes
    \node [supergraph_f] (st_b_foo) [right = 6*\dist of call_bar] {\textsf{start}$_{\texttt{B.foo}}$};
    \node [supergraph_f] (return_not_secret) [below = \dist of st_b_foo.south west, anchor=west] {\texttt{return "not secret"}};
    \node [supergraph_f] (end_b_foo) [below = \dist of return_not_secret.south west,anchor = west] {\textsf{end}$_\texttt{B.foo}$};
    
    \node [n,label=$\mathbf0$] (st_b_foo_z) [left=\nd of st_b_foo.west] {};
    \node [n] (return_not_secret_z) [left=\nd of return_not_secret.west] {};
    \node [n] (end_b_foo_z) [left=\nd of end_b_foo.west] {};
    
%B.foo method edges
    \path[->](st_b_foo_z) edge (return_not_secret_z); 
    \path[->](return_not_secret_z) edge (end_b_foo_z); 
    \path[->,dashed](call_foo_z) edge [out=-60,in=160,color=lightsalmonpink] (st_b_foo_z);
    \path[->,dashed](end_b_foo_z) edge [out=145,in=-65,color=lightsalmonpink] (return_foo_z);
    
%B.bar method nodes
    \node [supergraph_f] (st_b_bar) [right = 6*\dist of st_b_foo] {\textsf{start}$_{\texttt{B.bar}}$};
    \node [supergraph_f] (print) [below = \dist of st_b_bar.south west,anchor = west] {\texttt{print(s)}};
    \node [supergraph_f] (end_b_bar) [below = \dist of print.south west,anchor = west] {\textsf{end}$_\texttt{B.bar}$};
    
    \node [n,label=\texttt{s}] (st_b_bar_s) [left=\nd of st_b_bar.west] {};
    \node [n] (print_s) [left=\nd of print.west] {};
    \node [n] (end_b_bar_s) [left=\nd of end_b_bar.west] {};
    \node [n,label=$\mathbf0$] (st_b_bar_z) [left=\nd of st_b_bar_s] {};
    \node [n] (print_z) [left=\nd of print_s] {};
    \node [n] (end_b_bar_z) [left=\nd of end_b_bar_s] {};
%B.bar method edges
    \path[->](st_b_bar_z) edge (print_z);
    \path[->](print_z) edge (end_b_bar_z);
    \path[->, ultra thick](st_b_bar_s) edge (print_s);
    \path[->](print_s) edge (end_b_bar_s);
    \path[->,dashed](call_bar_z) edge [out=40,in=140,color=lightsalmonpink] (st_b_bar_z);
    \path[->,dashed, ultra thick](call_bar_v) edge [out=20,in=160,color=lightsalmonpink] node[arrowtext,above right=-.4cm and .4cm,rotate=-10]{$\lambda m.m[\texttt a\to m(\texttt a)\cap\{\texttt B\}]$} (st_b_bar_s);
    \path[->,dashed](end_b_bar_z) edge [out=210,in=-40,color=lightsalmonpink] (return_bar_z);    
}
\end{tikzpicture}
\caption[An example program demonstrating correlated-call edge functions on the $\mathbf0$-node path for Listing~\ref{list:ccexample}]{An example program demonstrating correlated-call edge functions on the $\mathbf0$-node path for Listing~\ref{list:ccexample}. All non-labeled edges are implicitly labeled with identity functions $\id$. The variable $\psi$ denotes the return value of the \texttt{A.foo} method.}%
  \label{fig:cc_edgefn_example}%  
  \end{figure}

\end{minipage}
\end{figure} 