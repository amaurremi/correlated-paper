\section{Background}\label{sec:bg}
%The purpose of the correlated-calls analysis is to solve IFDS problems more precisely than using the standard IFDS algorithm by ruling out some infeasible paths. The correlated-calls analysis works by transforming an IFDS problem to an IDE problem, solving the IDE problem, and transforming the IDE result to a solution to the original IFDS problem. 
This section defines basic terminology and presents the IFDS and IDE algorithms.
%\ftodo{
  %Shall we add a sentence saying that readers familiar with these algorithms may skip this section?
%}

\subsection{Terminology and Notation}
%We will start by introducing several concepts used by the IFDS and IDE analyses.
%\begin{mdelete}\mtodo{We already say (but less precisely) what a CFG is in the introduction.}
The \textit{control-flow graph} of a procedure is a directed graph whose nodes are instructions, which contains an edge from $n_1$ to $n_2$ whenever
$n_2$ may execute immediately after $n_1$. A CFG has a distinguished \textit{start node} \textsf{start}$_p$ and \textit{end node} \textsf{end}$_p$.
Following the presentation of Reps~et~al.~\cite{reps1995precise,sagiv1996precise}, we follow every
call instruction with a no-op instruction, so that every \textit{call node} is immediately
followed by a \textit{return node} in the CFG.
The \textit{control-flow supergraph} of a program contains the CFGs of all of the procedures as
subgraphs. In addition, for each call instruction $c$, the supergraph contains a \textit{call-to-start} edge to the start node of every procedure that
may be called from $c$, and an \textit{end-to-return} edge from the end node of the procedure back to the call instruction.

A call site is \textit{monomorphic} if it always calls the same procedure. In an object-oriented language, 
a call site $r.m(\ldots)$ can dynamically dispatch to multiple methods depending on the runtime
type of the object pointed to by the receiver $r$.
A call site that calls multiple procedures is called
\textit{polymorphic}.
We define a function $\textsf{lookup}$ to specify the dynamic dispatch: 
if $s$ is the signature of $m$ and $t$ is the runtime type of the object
pointed to by $r$, $\textsf{lookup}(s,t)$ gives the procedure that will
be invoked by the call $r.m(\ldots)$. We also define a function $\tau$
that may be viewed as the inverse
of $\textsf{lookup}$: given a signature $s$ and a specific invoked procedure
$f$, $\tau(s,f)$ gives the set of all runtime types of $r$ that cause $r.m(\ldots)$
to dispatch to $f$: $\tau(s,f) = \{t\mid \textsf{lookup}(s,t)=f\}$.

A path in the control-flow supergraph is \textit{valid} if it follows the usual stack-based calling 
discipline: every end-to-return edge on the path returns
to the site of the most recent call that has not yet been matched by a return. 
%\ftodo{Can we explain/define the notion of the ``most recent unmatched call'' more precisely?}
The set of all valid paths from the program entry point to
a node $n$ is denoted $\ivp(n)$.

A \textit{lattice}%
\footnote{The definitions that we give here are of \textit{complete} lattices and semilattices.
Since all of the (semi)lattices discussed in this paper are required to be complete, we omit the
\textit{complete} qualifier.}
is a partially ordered set $(S,\sqsubseteq)$ in which every subset has a least upper bound, called \textit{join} or $\sqcup$, and a greatest lower bound, called \textit{meet} or $\sqcap$.
A \textit{meet semilattice} is a partially ordered set in which every subset only has a greatest lower bound.
The symbols $\bot$ and $\top$ are used to denote the greatest lower bound of $S$ and of the empty set, respectively.

In this paper, we denote a map $m$ as a set of pairs of keys and values, in which each key appears at most once.
For any map $m$,  $m(k)$ is the value paired with the key $k$ in $m$. We denote by $m[x\to y]$ a map that maps $x$ to $y$, and every other key $k$ to $m(k)$.

\subsection{IFDS}\label{sec:bgifds}
The IFDS framework~\cite{reps1995precise} is a precise and efficient algorithm for data-flow analysis
%. IFDS was developed in 1995 by T.\,Reps, S.\,Horwitz, and M.\,Sagiv at the University of Wisconsin and
that has been used to solve a variety of data-flow analysis problems~\cite{bodden2013spl,naeem2008typestate,DBLP:conf/birthday/KreikerRRSWY13,tripp2009taj}.
The IFDS framework is an instance of the \textit{functional approach} to data-flow analysis~\cite{pnueli1981two}
because it constructs summaries of the effects of called procedures.
The IFDS framework is applicable to \textit{interprocedural} data-flow problems whose domain consists of \textit{subsets} of a \textit{finite} set $D$,
and whose data-flow functions are \textit{distributive}.
A function $f$ is distributive if 
$f(x_1\sqcap x_2)=f(x_1)\sqcap f(x_2)$.

The IFDS algorithm is notable because it computes a meet-over-valid paths solution in polynomial time.
Most other interprocedural analysis algorithms are either: 
  (i) imprecise due to invalid paths,
  (ii) general but do not run in polynomial time~\cite{knoop1992interprocedural,pnueli1981two}, or 
  (iii) handle a very specific set of problems~\cite{knoop1993efficient}.

The input to the IFDS algorithm is specified as
%Formally, a data-flow analysis problem is suitable for an IFDS analysis if it can be encoded as an IFDS problem
  $(G^*,\,D,\,F,\,M_F,\,\sqcap)$,
where $G^*=(N^*,\,E^*)$ is the supergraph of the input program with nodes $N^*$ and edges $E^*$,
$D$ is a finite set of \textit{data-flow facts},
$F$ is a set of distributive data-flow functions of type $2^D\to2^D$,
$M_F:\,E^*\to F$ assigns a data-flow function to each supergraph edge,
and $\sqcap$ is the \textit{meet operator} on the powerset $2^D$, either union or intersection.
In our presentation, the meet operator will always be union, but all of the results apply dually when the
meet is intersection.

The output of the IFDS algorithm is, for each node $n$ in the supergraph, the \textit{meet-over-all-valid-paths} solution
 $\mvp{F}(n)=\bigsqcap_{q\in\ivp(n)}M_F(q)(\top)$, where $M_F$ is extended from edges to paths by composition.

\subsubsection{Overview of the IFDS Algorithm}\label{sec:overviewifds}
The key idea behind the IFDS algorithm is that it is possible to represent any distributive
function $f$ from $2^D$ to $2^D$ by a \textit{representation relation} $R_f \subseteq 
(D \cup \{0\})
\times
(D \cup \{0\})$. The representation relation can be visualized as a bipartite graph
with edges from one instance of $D \cup \{0\}$ to another instance of $D \cup \{0\}$.
The IFDS algorithm uses such graphs to efficiently represent both the input data-flow functions
and the summary functions that it computes for called procedures.
Specifically, the representation relation $R_f$ of a function $f$ is defined as:
\[
R_f=\{(\mathbf0,\,\mathbf0)\}\cup
      \{(\mathbf0,\,d_j)\,|\,d_j\in f(\varnothing)\}\cup
      \{(d_i,\,d_j)\,|\,d_j\in f(\{d_i\}) \setminus f(\varnothing)\}.
\]
\vspace{-5mm}
\begin{example}\label{ex:flowfun}
Given $D=\{u,\,v,\,w\}$ and $f(S) = S\setminus\{v\} \cup \{u\}$,
    the representation
 relation 
  $
    R_f=\{(\mathbf0,\,\mathbf0),\,(\mathbf0,\,u),\,(w,\,w)\}
$, which is depicted in Figure~\ref{fig:minigraph1}.
\end{example}

The representation relation decomposes a flow function into functions
(edges) that operate on each fact individually. This is possible due
to distributivity: applying the flow function to a set of facts is
equivalent to applying it on each fact individually and then
taking the union of the results.

The meet of two functions can be computed as simply the union of their representation
functions: $R_{f\sqcap f'} = R_f \cup R_{f'}$. The composition of two functions can be
computed by combining their representation graphs, merging the range nodes of the
first function with the corresponding domain nodes of the second function, and finding
paths in the resulting graph.
  \centering
    \tikzstyle{n}=[fill=black,circle,inner sep=1.5pt]
\begin{tikzpicture}[>=stealth,on grid, auto]
    \node [n,label=$\mathbf{0}$] (zero){};
    \node [n] (zero_) [below=\edge of zero] {};
    \node [n,label=$u$] (u) [right=\nd of zero] {};
    \node [n] (u_) [below=\edge of u] {};
    \node [n,label=$v$] (v) [right=\nd of u] {};
    \node [n] (v_) [below=\edge of v] {};
    \node [n,label=$w$] (w) [right=\nd of v] {};
    \node [n] (w_) [below=\edge of w] {};
    \node [n] (zero__) [below=\edge of zero_] {};
    \node [n] (u__) [below=\edge of u_] {};
    \node [n] (v__) [below=\edge of v_] {};
    \node [n] (w__) [below=\edge of w_] {};

    \path[->](zero) edge node[left] {$R_f$} (zero_);
    \path[->](zero) edge (u_);
    \path[->](w) edge (w_);
    \path[->](zero_) edge node[left] {$R_g$} (zero__);
    \path[->](u_) edge (u__);
    \path[->](v_) edge (v__);
    
    \node [n,label=$\mathbf{0}$] (zeroC) [below = 1.5\edge of zero__] {};
    \node [n] (zeroC_) [below=\edge of zeroC] {};
    \node [n,label=$u$] (uC) [right=\nd of zeroC] {};
    \node [n] (uC_) [below=\edge of uC] {};
    \node [n,label=$v$] (vC) [right=\nd of uC] {};
    \node [n] (vC_) [below=\edge of vC] {};
    \node [n,label=$w$] (wC) [right=\nd of vC] {};
    \node [n] (wC_) [below=\edge of wC] {};

    \path[->](zeroC) edge node[left] {$R_g\circ R_f$} (zeroC_);
    \path[->](zeroC) edge (uC_);
  \end{tikzpicture}

\begin{example}
    If $g(S)=S\setminus\{w\}$ and $f(S) = S\setminus\{v\} \cup \{u\}$, then
    $
    R_g\circ R_f=\{(\mathbf0,\,\mathbf0),\,(\mathbf0,\,u)\},
    $
  as illustrated in Figure~\ref{fig:minigraph2}.
\end{example}

Composition of two distributive
functions $f$ and $f'$ corresponds to finding reachable nodes in a graph composed from their representation
relations $R_f$ and $R_{f'}$. Therefore, evaluating the composed data-flow function for a control
flow path corresponds to finding reachable nodes in a graph composed from the representation relations
of the data-flow functions for individual instructions.

It is this graph of representation relations that the IFDS algorithm operates on.
In this graph, called the \textit{exploded supergraph}, each node is a pair $(n,d)$, where
$n\in N^*$ is a node of the control-flow supergraph and $d$ is an element of $D\cup\{0\}$.
For each edge $(n \to n') \in E^*$, the exploded supergraph contains a set of edges
$(n,d_i) \to (n',d_j)$, which form the representation relation of the data-flow function
$M_F(n \to n')$. The IFDS algorithm finds all exploded supergraph edges that are reachable by
\textit{realizable} paths in the exploded supergraph. A path
is \textit{realizable} if its projection to the (non-exploded) supergraph is a valid
path (i.e., if it is of the form $(n_0, d_0) \to (n_1,d_1) \to \cdots \to (n_m,d_m)$
and where $n_0 \to n_1 \to \cdots \to n_m$ is a valid path).

\begin{figure}[h!]%
  \centering
    \tikzstyle{supergraph}=[
      fill=greyblue,
      rounded corners,
      font=\footnotesize,
      align=right,
      text=charcoal,
      drop shadow={ashadow, color=greyblue}
    ]
    \tikzstyle{supergraph_y}=[
      fill=bisque,
      rounded corners,
      font=\footnotesize,
      align=left,
      text=charcoal,
      drop shadow={ashadow, color=greyblue}
    ]
    \tikzstyle{supergraph_f}=[
      fill=lightsalmonpink,
      rounded corners,
      font=\footnotesize,
      align=left,
      text=charcoal,
      drop shadow={ashadow, color=greyblue}
    ]
    \tikzstyle{description}=[fill=white]
    \tikzstyle{n}=[fill=black,circle,inner sep=1.5pt]
    \tikzstyle{arrowtext}=[font=\tiny,color=black,above]
    
\hspace*{-10pt}
\begin{tikzpicture}
\scalebox{.8}{
% main method nodes
    \node [supergraph] (st_main) {\textsf{start}$_{\texttt{main}}$};
    \node [supergraph] (asgn_a) [below = \dist of st_main.south east, anchor=east] {\texttt{a = args==null\,?}\\\texttt{new\,A()\,:\,new\,B()}};
    \node [supergraph] (call_foo) [below = \dist of asgn_a.south east, anchor=east] {\textsf{call}$_\texttt{foo}$};
    \node [supergraph] (return_foo) [below = \dist of call_foo.south east,anchor=east] {\textsf{return}$_\texttt{foo}$\\\texttt{v = a.foo()}};
    \node [supergraph] (call_bar) [below = \dist of return_foo.south east, anchor=east] {\textsf{call}$_\texttt{bar}$};
    \node [supergraph] (return_bar) [below = \dist of call_bar.south east, anchor=east] {\textsf{return}$_\texttt{bar}$};
    \node [supergraph] (end_main) [below = \dist of return_bar.south east, anchor=east] {\textsf{end}$_{\texttt{main}}$};
    
    \node [n,label=$\mathbf0$] (st_main_z) [right=\nd of st_main.east] {};
    \node [n,label=\texttt{v}] (st_main_v) [right=\nd of st_main_z] {};
    \node [n] (asgn_a_z) [right=\nd of asgn_a.east] {};
    \node [n] (call_foo_z) [right=\nd of call_foo.east] {};
    \node [n] (return_foo_z) [right=\nd of return_foo.east] {};
    \node [n] (call_bar_z) [right=\nd of call_bar.east] {};
    \node [n] (return_bar_z) [right=\nd of return_bar.east] {};
    \node [n] (end_main_z) [right=\nd of end_main.east] {};
    \node [n] (asgn_a_v) [right=\nd of asgn_a_z] {};
    \node [n] (call_foo_v) [right=\nd of call_foo_z] {};
    \node [n] (return_foo_v) [right=\nd of return_foo_z] {};
    \node [n] (call_bar_v) [right=\nd of call_bar_z] {};
    \node [n] (return_bar_v) [right=\nd of return_bar_z] {};
    \node [n] (end_main_v) [right=\nd of end_main_z] {};
% main method edges
    \path[->, ultra thick](st_main_z) edge (asgn_a_z);
    \path[->, ultra thick](asgn_a_z) edge (call_foo_z);
    \path[->](call_foo_z) edge (return_foo_z);
    \path[->](return_foo_z) edge (call_bar_z);
    \path[->](call_bar_z) edge (return_bar_z);
    \path[->](return_bar_z) edge (end_main_z);  
    \path[->](st_main_v) edge (asgn_a_v);
    \path[->](asgn_a_v) edge (call_foo_v);
    \path[->](call_foo_v) edge (return_foo_v);
    \path[->, ultra thick](return_foo_v) edge (call_bar_v);
    \path[->](call_bar_v) edge (return_bar_v);
    \path[->](return_bar_v) edge (end_main_v);  
    
%A.foo method nodes
    \node [supergraph_y] (st_a_foo) [right = 6*\dist of st_main] {\textsf{start}$_{\texttt{A.foo}}$};
    \node [supergraph_y] (return_secret) [below = \dist of st_a_foo.south west, anchor=west] {\texttt{return secret()}};
    \node [supergraph_y] (end_a_foo) [below = \dist of return_secret.south west,anchor = west] {\textsf{end}$_\texttt{A.foo}$};
    
    \node [n,label=$r$] (st_a_foo_p) [left=\nd of st_a_foo.west] {};
    \node [n] (return_secret_p) [left=\nd of return_secret.west] {};
    \node [n] (end_a_foo_p) [left=\nd of end_a_foo.west] {};
    \node [n,label=$\mathbf0$] (st_a_foo_z) [left=\nd of st_a_foo_p] {};
    \node [n] (return_secret_z) [left=\nd of return_secret_p] {};
    \node [n] (end_a_foo_z) [left=\nd of end_a_foo_p] {};
%A.foo method edges
    \path[->, ultra thick](st_a_foo_z) edge (return_secret_z); 
    \path[->](return_secret_z) edge node[arrowtext,near start,above right=0.4cm and -.15cm,rotate=-60]{$\lambda m.\botcc$} (end_a_foo_z); 
    \path[->,dashed, ultra thick](call_foo_z) edge [out=38,in=238]  (st_a_foo_z);
    \path[->,dashed](end_a_foo_z) edge [out=200,in=43] (return_foo_z);
    \path[->,dashed, ultra thick](end_a_foo_p) edge  [out=200,in=43] node[arrowtext,below,rotate=30]{$\lambda m.m[\texttt a\to m(\texttt a)\cap\{\texttt A\}]$} (return_foo_v);
    \path[->](st_a_foo_p) edge (return_secret_p); 
    \path[->](return_secret_p) edge (end_a_foo_p);
    \path[->, ultra thick](return_secret_z) edge (end_a_foo_p);
    
%A.bar method nodes
    \node [supergraph_y] (st_a_bar) [right = 6*\dist of st_a_foo] {\textsf{start}$_{\texttt{A.bar}}$};
    \node [supergraph_y] (end_a_bar) [below = \dist of st_a_bar.south west,anchor = west] {\textsf{end}$_\texttt{A.bar}$};
    
    \node [n,label=\texttt{s}] (st_a_bar_s) [left=\nd of st_a_bar.west] {};
    \node [n,label=$\mathbf0$] (st_a_bar_z) [left=\nd of st_a_bar_s] {};
    \node [n] (end_a_bar_s) [left=\nd of end_a_bar.west] {};    
    \node [n] (end_a_bar_z) [left=\nd of end_a_bar_s] {};
%A.bar method edges
    \path[->](st_a_bar_z) edge (end_a_bar_z);
    \path[->,dashed](call_bar_z) edge [out=20,in=229] (st_a_bar_z);
    \path[->,dashed](end_a_bar_z) edge [out=220,in=35] (return_bar_z);
    \path[->](st_a_bar_s) edge (end_a_bar_s);
    \path[->,dashed](call_bar_v) edge [out=20,in=230] node[arrowtext,near end,above left=-.1cm and -.15cm,rotate=36]{$\lambda m.m[\texttt a\to m(\texttt a)\cap\{\texttt A\}]$} (st_a_bar_s);


%B.foo method nodes
    \node [supergraph_f] (st_b_foo) [right = 6*\dist of call_bar] {\textsf{start}$_{\texttt{B.foo}}$};
    \node [supergraph_f] (return_not_secret) [below = \dist of st_b_foo.south west, anchor=west] {\texttt{return "not secret"}};
    \node [supergraph_f] (end_b_foo) [below = \dist of return_not_secret.south west,anchor = west] {\textsf{end}$_\texttt{B.foo}$};
    
    \node [n,label=$\mathbf0$] (st_b_foo_z) [left=\nd of st_b_foo.west] {};
    \node [n] (return_not_secret_z) [left=\nd of return_not_secret.west] {};
    \node [n] (end_b_foo_z) [left=\nd of end_b_foo.west] {};
    
%B.foo method edges
    \path[->](st_b_foo_z) edge (return_not_secret_z); 
    \path[->](return_not_secret_z) edge (end_b_foo_z); 
    \path[->,dashed](call_foo_z) edge [out=-60,in=160,color=lightsalmonpink] (st_b_foo_z);
    \path[->,dashed](end_b_foo_z) edge [out=145,in=-65,color=lightsalmonpink] (return_foo_z);
    
%B.bar method nodes
    \node [supergraph_f] (st_b_bar) [right = 6*\dist of st_b_foo] {\textsf{start}$_{\texttt{B.bar}}$};
    \node [supergraph_f] (print) [below = \dist of st_b_bar.south west,anchor = west] {\texttt{print(s)}};
    \node [supergraph_f] (end_b_bar) [below = \dist of print.south west,anchor = west] {\textsf{end}$_\texttt{B.bar}$};
    
    \node [n,label=\texttt{s}] (st_b_bar_s) [left=\nd of st_b_bar.west] {};
    \node [n] (print_s) [left=\nd of print.west] {};
    \node [n] (end_b_bar_s) [left=\nd of end_b_bar.west] {};
    \node [n,label=$\mathbf0$] (st_b_bar_z) [left=\nd of st_b_bar_s] {};
    \node [n] (print_z) [left=\nd of print_s] {};
    \node [n] (end_b_bar_z) [left=\nd of end_b_bar_s] {};
%B.bar method edges
    \path[->](st_b_bar_z) edge (print_z);
    \path[->](print_z) edge (end_b_bar_z);
    \path[->, ultra thick](st_b_bar_s) edge (print_s);
    \path[->](print_s) edge (end_b_bar_s);
    \path[->,dashed](call_bar_z) edge [out=40,in=140,color=lightsalmonpink] (st_b_bar_z);
    \path[->,dashed, ultra thick](call_bar_v) edge [out=20,in=160,color=lightsalmonpink] node[arrowtext,above right=-.4cm and .4cm,rotate=-10]{$\lambda m.m[\texttt a\to m(\texttt a)\cap\{\texttt B\}]$} (st_b_bar_s);
    \path[->,dashed](end_b_bar_z) edge [out=210,in=-40,color=lightsalmonpink] (return_bar_z);    
}
\end{tikzpicture}
\caption[An example program demonstrating correlated-call edge functions on the $\mathbf0$-node path for Listing~\ref{list:ccexample}]{An example program demonstrating correlated-call edge functions on the $\mathbf0$-node path for Listing~\ref{list:ccexample}. All non-labeled edges are implicitly labeled with identity functions $\id$. The variable $\psi$ denotes the return value of the \texttt{A.foo} method.}%
  \label{fig:cc_edgefn_example}%  
  \end{figure}
 

\begin{example}
  The exploded supergraph for Listing~\ref{list:ccexample} is shown in Figure~\ref{fig:cc_edgefn_example}.
  The labels on the edges will be explained in Section~\ref{sec:bgide}
  We can see that there is a realizable path, highlighted in bold,
  from the start node of the exploded graph to the variable \verb's' at~the~node~$\textsf{print(s)}$ in the \verb'B.bar' method. This means that \verb's' is considered secret at that node.
\end{example}

\subsection{IDE}\label{sec:bgide}

The IDE algorithm~\cite{sagiv1996precise} extends IFDS to
\textit{interprocedural} \textit{distributive} \textit{environment}
problems. An \textit{environment} problem is an analysis whose data-flow
lattice is the lattice $\textsf{Env}(D,L)$ of maps from a finite set $D$ to a meet semilattice
$L$ of finite height, ordered pointwise. Like IFDS, IDE requires the data-flow functions to be distributive.

The input to the IDE algorithm is
  $(G^*,\,D,\,L,\,M_\textsf{Env})$
where $G^*$ is a control-flow supergraph,
$D$ is a set of data-flow facts,
$L$ is a meet semilattice of finite height,
and $M_\textsf{Env}:\,E^*\to(\textsf{Env}(D,\,L)\to \textsf{Env}(D,\,L))$ assigns a data-flow function
to each supergraph edge.

The output of the IDE algorithm is, for each node $n$ in the supergraph,
the \textit{meet-over-all-valid-paths} solution
$\mvp{\textsf{Env}}(n)=\bigsqcap_{q\in\ivp(n)}M_\textsf{Env}(q)(\top_\textsf{Env})$,
%\mtodo{Should it be $\Omega$ instead of $\top$?}
where $\top_\textsf{Env} = \lambda d.\top$ is the top element of the lattice of environments,
and $M_\textsf{Env}$ is extended from edges to paths by composition.

\subsubsection{Overview of the IDE Algorithm}\label{sec:ideoverview}
Just as any distributive function from $2^D$ to $2^D$ can be represented with a
representation relation, it is also possible to represent any distributive function from
$\textsf{Env}(D,L)$
to
$\textsf{Env}(D,L)$
with a \textit{pointwise representation}. A pointwise representation is a bipartite graph
with the same nodes~\footnote{The IDE literature uses the symbol $\Lambda$ for the node that
    is denoted $\mathbf0$ in the IFDS literature. We use $\mathbf0$ throughout this paper for consistency.}
and edges as a representation relation, except that each edge is labelled
with a \textit{micro-function}, which is a function from $L$ to $L$. 

Thanks to distributivity, every environment transformer 
$t : \textsf{Env}(D,L) \to \textsf{Env}(D,L)$
can be decomposed into its effect on $\top_{\textsf{Env}}$ and on a set of environments $\top_{\textsf{Env}}[d_i\to l]$
that map every element to $\top$ except one ($d_i$):
\[
    t(m)(d_j) = \lambda l. t(\top_{\textsf{Env}})(d_j) \sqcap \bigsqcap_{d_i\in D} \lambda l. t(\top_{\textsf{Env}}[d_i\to l])(d_j).
\]
The functions $\lambda l. \cdots$ in this decomposition are the micro-functions that
appear on the edges of the pointwise representation edges from $\mathbf0$ to each $d_j$ and from
each $d_i$ to each $d_j$.%
\footnote{The IDE paper defines a more complicated but equivalent set of micro-functions
that eliminate some duplication of computation.}
The absence of an edge in the pointwise representation from some $d_i$ to some $d_j$ is
equivalent to an edge with micro-function $\lambda l.\top$.

\begin{example}
    In the exploded supergraph in Figure~\ref{fig:cc_edgefn_example},
    the micro-functions are shown as labels on the graph edges.
    Every edge without an explicit label has the identity as its
    micro-function. The micro-functions on the three edges from
    the node \verb+return secret()+ to the node $\texttt{end}_\texttt{A.foo}$
    together represent the environment transformer 
    $\lambda e.e[r \to \lambda m.\bot \sqcap \lambda m.m]$.

    To eliminate infeasible paths due to correlated calls, we encode the taint
    analysis using environments $\textsf{Env}(D,L)$, where $D$ is the set of
    variables and $L$ is a map from receiver variables to sets of possible
    types. The interpretation of such an environment $e$ is that a given variable
    $v \in D$ may contain a secret value in an execution in which the runtime
    types of the objects pointed to by the receiver variables are in the sets
    specified by $e(v)$.
\end{example}

The meet of two environment transformers $t_1, t_2$ can be computed by taking the union of the edges
in their pointwise representations. When the same edge appears in the pointwise representations of
both $t_1$ and $t_2$, the micro-function for that edge in $t_1 \sqcap t_2$ is the meet of the
micro-functions for that same edge in $t_1$ and in $t_2$.

The composition of two environment transformers can be computed by combining their pointwise
representation graphs in the same fashion as IFDS representation relations, and 
computing the composition of the micro-functions appearing along each path in the resulting graph.

The IDE algorithm operates on the same exploded supergraph as the IFDS algorithm
(except that the edges are labelled with micro-functions). For each pair $(n,d)$ of
node and fact, the algorithm computes a micro-function equal to the meet of the micro-functions
of all the realizable paths from the program entry point to the pair.

In order to do this efficiently, the IDE algorithm requires a representation of micro-functions
that is general enough to express the basic micro-functions of the data-flow functions for individual
instructions, and that supports computing the meet and composition of micro-functions.

A practical implementation of the IDE algorithm requires
the input data-flow functions to be provided in their pointwise representation
as exploded supergraph edges labelled with micro-functions.
%\mtodo{Should we cite the ``Practical extensions'' paper here that talks about the on-demand IFDS algorithm?}
Specifically,
the input is generally provided as a function $\textsf{EdgeFn}: (N^* \times D) \times (N^* \times D) \to F$,
where $F$ is the set of representations of micro-functions from $L$ to $L$.
Given an exploded supergraph edge $e=(n,\,d) \to (n',\,d')$,
$\textsf{EdgeFn}((n,\,d),\,(n',\,d'))$ returns the micro-function that appears on the exploded supergraph edge $e$.
%Like in the IFDS algorithm,
In an implementation, it can be convenient to split the function $\textsf{EdgeFn}$ into separate functions that handle the cases when $n\to n'$ is an intraprocedural edge, a call-to-return edge, a call-to-start edge, or an end-to-return edge.
