% brackets for denotations
\newcommand{\denote}[1]{\left\llbracket#1\right\rrbracket}

%\renewcommand{\theFancyVerbLine}{\sffamily \textcolor[rgb]{0.5,0.5,0.5}{\scriptsize \oldstylenums{\arabic{FancyVerbLine}}}}

\newcommand{\inputMinted}[2]{\lstinputlisting[language=#1]{code/#2}}
\newcommand{\inputMintedNoLNs}[2]{\lstinputlisting[language=#1,numbers=none]{code/#2}}

%\newcommand{\inputMinted}[2]{\par\inputminted[fontsize=%\footnotesize,linenos=true,bgcolor=white,numberblanklines=false,mathescape,samepage=true]{#1}{code/#2}\\ \ \\}
\definecolor{bg}{rgb}{0.95,0.95,0.95}
%%\usemintedstyle{manni}     % comci comca
%%\usemintedstyle{borland}   % pretty strict style
%%\usemintedstyle{vim}       % together with fruity and native only with dark bg!
%\usemintedstyle{bw}         % strict, too

\counterwithout{footnote}{chapter}

\newlength{\dist}
\setlength{\dist}{.6cm}

\newlength{\distt}
\setlength{\distt}{1cm}

\newlength{\disth}
\setlength{\disth}{1cm}

\newlength{\edge}
\setlength{\edge}{.7cm}

\newlength{\longedge}
\setlength{\longedge}{1.7cm}

\newlength{\nd}
\setlength{\nd}{.6cm}

\newlength{\ndd}
\setlength{\ndd}{.5cm}

%\newtheorem{theorem}{Theorem}[section]
%\newtheorem{sublemma}[theorem]{Sub-Lemma}
%\newtheorem{lemma}[theorem]{Lemma}
%\newtheorem{proposition}[theorem]{Proposition}
%\newtheorem{corollary}[theorem]{Corollary}
%\newtheorem{definition}[theorem]{Definition}
%\newtheorem{example}[theorem]{Example}
\let\oldexample\example
\renewcommand{\example}{\oldexample\normalfont}

\tikzset{>=latex,on grid, auto}

\tikzset{
  ashadow/.style={
    opacity=.25,
    shadow xshift=0.07,
    shadow yshift=-0.07
  }
}
\tikzstyle{supergraph}=[
  fill=greyblue,
  rounded corners,
  font=\small,
  align=right,
  text=charcoal,
  drop shadow={ashadow, color=greyblue}
]

\newcommand{\backEq}{\mathcal U^\equiv}
\newcommand{\backCC}{\mathcal U^\Subset}
\newcommand{\transEq}{\mathcal T^\equiv}
\newcommand{\transCC}{\mathcal T^\Subset}

\newcommand{\rcc}{{R^\Subset}}
\newcommand{\lcc}{L^\Subset}

\newcommand{\result}[1]{\mathcal R_{\text{#1}}}
\newcommand{\resultIFDS}{\result{IFDS}}
\newcommand{\resultIDE}{\result{IDE}}

\newcommand{\updmap}[1]{\textsf{update}_{#1}}
\newcommand{\updfun}[1]{\textsf{update}_{#1,\,r}}

\newcommand{\setpair}[2]{\left\langle #1,\,#2\right\rangle}

\newcommand{\highlight}[2]{\setlength{\fboxsep}{1pt}\colorbox{#2}{$\displaystyle #1$}}

\newcommand{\ione}{I_{f_1,\,r}}
\newcommand{\itwo}{I_{f_2,\,r}}
\newcommand{\uone}{U_{f_1,\,r}}
\newcommand{\utwo}{U_{f_2,\,r}}

\newcommand{\mvp}[1]{\textsf{MVP}_{#1}}
\newcommand{\ivp}{\textsf{VP}}

\newcommand{\startmain}{\textsf{start}_\texttt{main}}

\newcommand{\edgefn}{\textsf{EdgeFn}}
\newcommand{\ccedgefn}[1]{\edgefn^\Subset_{#1}}

\newcommand{\menv}{M_\textsf{Env}}

\newcommand{\topcc}{\top_\Subset}
\newcommand{\botcc}{\bot_\Subset}

\newcommand{\mpddef}[2]{\xi(#1,\,#2)}
\newcommand{\mpd}{\mpddef p d}
\newcommand{\mppd}{\mpddef{p'} d}
\newcommand{\mpdkm}{\mpddef{p}{d_{k-1}}}
\newcommand{\efek}{\ccedgefn R(e_k)(\mpdkm)(r)}

\newcommand{\id}{\textsf{id}}

\newcommand{\src}[1]{\mathsf{src}(#1)}
\newcommand{\target}[1]{\mathsf{end}(#1)}

\newcommand{\comment}[3]{\todo[size=\tiny,color=#3]{#1 \textbf{(#2)}}}
\newcommand{\mtodo}[1]{\comment{#1}{M}{mcomment}}
\newcommand{\ftodo}[1]{\comment{#1}{F}{fcomment}}
\newcommand{\otodo}[1]{\comment{#1}{O}{ocomment}}

\makeatletter
\newcommand*{\savecbcolor}{%
  \let\saved@cb@current@color\cb@current@color
}
\newcommand*{\restorecbcolor}{%
  \global\let\cb@current@color\saved@cb@current@color
}
\makeatother

\newenvironment{mdelete}{%
  \savecbcolor
  \begin{changebar}%
  \cbcolor{mcomment}%
  \addtolength{\changebarsep}{2mm}%
  \setlength{\changebarwidth}{2mm}%
}{
  \end{changebar}%
  \restorecbcolor
}

\newenvironment{odelete}{%
  \savecbcolor
  \begin{changebar}%
  \cbcolor{ocomment}%
  \addtolength{\changebarsep}{2mm}%
}{%
  \end{changebar}%
  \restorecbcolor
}

\newenvironment{fdelete}{%
  \savecbcolor
  \begin{changebar}%
  \cbcolor{fcomment}%
  \addtolength{\changebarsep}{2mm}%
}{%
  \end{changebar}%
  \restorecbcolor
}

\newcommand{\commentout}[1]{}

\newcommand{\TODO}[1]{\textcolor{darkred}{\textbf{$\blacktriangleright$#1$\blacktriangleleft$}}}

\newcommand{\code}[1]{\textsf{#1}}  
\lstdefinelanguage{scala}[]{Java}{
   morekeywords={trait,def,object,with,override,val,type,var} 
}

\lstdefinestyle{Eclipse}{
  xleftmargin=0pt,
  language = scala,
  basicstyle=\sffamily\small,
  stringstyle=\color{sh_string},
  keywordstyle = \color{sh_keyword}\bfseries,  %}
  lineskip=-0.0em,
  commentstyle=\color{sh_comment}\itshape,  
  escapeinside={/*@}{@*/},
  numbersep=5pt,
  captionpos=b,
  xleftmargin=0.4cm, xrightmargin=0.5cm,
   morekeywords={invokestatic,invokeinterface,invokevirtual,invokespecial},
}


\lstset{
  showspaces=false,showtabs=false,tabsize=2,columns=flexible,
  morekeywords={trait,def,object,with},
  language={scala},
  style=Eclipse,
  numbers=left,
  numberstyle=\scriptsize\color{CommentColor},
  firstnumber=last,
  showstringspaces=true
}  

\abovedisplayshortskip=0pt
\belowdisplayshortskip=0pt
\abovedisplayskip=0pt
\belowdisplayskip=0pt

%\makeatletter
\setlength{\parskip}{0pt}

\setlength\headsep   {12\p@}

\setlength\footnotesep{0\p@}
\setlength\textfloatsep{0mm\@plus 1\p@ \@minus 1\p@}
\setlength\intextsep   {0mm\@plus 1\p@ \@minus 1\p@}

\renewcommand\section{\@startsection{section}{1}{\z@}%
                       {-4\p@ \@plus 0\p@ \@minus 0\p@}%
                       {4\p@ \@plus 0\p@ \@minus 0\p@}%
                       {\normalfont\large\bfseries\boldmath
                        \rightskip=\z@ \@plus 8em\pretolerance=10000 }}
\renewcommand\subsection{\@startsection{subsection}{2}{\z@}%
                       {-2\p@ \@plus 0\p@ \@minus 0\p@}%
                       {2\p@ \@plus 0\p@ \@minus 1\p@}%
                       {\normalfont\normalsize\bfseries\boldmath
                        \rightskip=\z@ \@plus 8em\pretolerance=10000 }}
\renewcommand\subsubsection{\@startsection{subsubsection}{3}{\z@}%
                       {-2\p@ \@plus 0\p@ \@minus 0\p@}%
                       {-0.5em \@plus -0.22em \@minus -0.1em}%
                       {\normalfont\normalsize\bfseries\boldmath}}
\renewcommand\paragraph{\@startsection{paragraph}{4}{\z@}%
                       {-8\p@ \@plus -2\p@ \@minus -2\p@}%
                       {-0.5em \@plus -0.22em \@minus -0.1em}%
                       {\normalfont\normalsize\itshape}}

\def\@spthm#1#2#3#4{\topsep 0\p@ \@plus0\p@ \@minus0\p@
\refstepcounter{#1}%
\@ifnextchar[{\@spythm{#1}{#2}{#3}{#4}}{\@spxthm{#1}{#2}{#3}{#4}}}

\setlength  \labelsep  {.1em}
\def\@listI{\leftmargin\leftmargini
            \parsep 0\p@ \@plus1\p@ \@minus\p@
            \topsep 0\p@ \@plus1\p@ \@minus0\p@
            \itemsep0\p@}
\let\@listi\@listI
\@listi
\def\@listii {\leftmargin\leftmarginii
              \labelwidth\leftmarginii
              \advance\labelwidth-\labelsep
              \topsep    0\p@ \@plus2\p@ \@minus\p@}
\def\@listiii{\leftmargin\leftmarginiii
              \labelwidth\leftmarginiii
              \advance\labelwidth-\labelsep
              \topsep    0\p@ \@plus\p@\@minus\p@
              \parsep    \z@
              \partopsep \p@ \@plus\z@ \@minus\p@}

\setlength\arraycolsep{1.0\p@}
\setlength\tabcolsep{1.0\p@}

\setlength\abovecaptionskip{0\p@}%
\setlength\belowcaptionskip{0\p@}%

\renewenvironment{table}
               {\setlength\abovecaptionskip{0\p@}%
                \setlength\belowcaptionskip{0\p@}%
                \@float{table}}
               {\end@float}
\renewenvironment{table*}
               {\setlength\abovecaptionskip{0\p@}%
                \setlength\belowcaptionskip{0\p@}%
                \@dblfloat{table}}
               {\end@dblfloat}


\def\@maketitle{\newpage
 \markboth{}{}%
 \def\lastand{\ifnum\value{@inst}=2\relax
                 \unskip{} \andname\
              \else
                 \unskip \lastandname\
              \fi}%
 \def\and{\stepcounter{@auth}\relax
          \ifnum\value{@auth}=\value{@inst}%
             \lastand
          \else
             \unskip,
          \fi}%
 \begin{center}%
 {\Large \bfseries\boldmath
  \pretolerance=10000
  \@title \par}\vskip .4cm
\if!\@subtitle!\else {\large \bfseries\boldmath
  \vskip -.65cm
  \pretolerance=10000
  \@subtitle \par}\vskip .8cm\fi
 \setbox0=\vbox{\setcounter{@auth}{1}\def\and{\stepcounter{@auth}}%
 \def\thanks##1{}\@author}%
 \global\value{@inst}=\value{@auth}%
 \global\value{auco}=\value{@auth}%
 \setcounter{@auth}{1}%
{\lineskip .5em
\noindent\ignorespaces
\@author\vskip.15cm}
 {\small\institutename}\vskip-.35cm
 \end{center}%
 }


\renewenvironment{abstract}{%
      \list{}{\advance\topsep by0.20cm\relax\small
      \leftmargin=1cm
      \labelwidth=\z@
      \listparindent=\z@
      \itemindent\listparindent
      \rightmargin\leftmargin}\item[\hskip\labelsep
                                    \bfseries\abstractname]}
    {\endlist}

\makeatother


