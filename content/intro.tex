\section{Introduction}

Data-flow analysis computes an approximation of how values may flow  
through a program, and has applications in compiler optimization, 
programming tools, and computer security, and many other areas.
Data-flow analyses operate on \textit{control-flow graphs} (CFGs)
that model 
the order in which the instructions of a program are executed. This is
typically done by associating \textit{flow functions} that represent how 
data is propagated with the edges of the CFG.  A 
\textit{confluence operator} specifies how the data facts that have been 
computed along different paths should be merged when the paths join.

Since a CFG is an over-approximation of the possible flows 
of control in concrete executions of a program, it may contain \textit{infeasible} 
paths that cannot occur at runtime. The precision of a data-flow analysis algorithm
depends critically on its ability to detect and disregard such infeasible paths.
The popular \textit{Interprocedural Finite Distributive Subset} (IFDS) algorithm 
by  Reps, Horwitz, and Sagiv \cite{reps1995precise} is a general data-flow analysis
algorithm for computing solutions to standard finite distributive data-flow problems such 
as reaching definitions, available expressions, and taint analysis.
A distinguishing 
characteristic of IFDS is that it avoids infeasible interprocedural paths in which calls 
and returns to/from functions are not properly matched. Sagiv, Reps, and Horwitz
also presented the \textit{Interprocedural Distributive Environment} (IDE) algorithm
\cite{sagiv1996precise} that similarly only considers properly matched call/return
edges, but that supports a broader range of dataflow problems by expanding the 
domain of flow functions to \textit{environments} that go beyond the data-flow facts 
considered by IFDS.

This paper presents an approach to dataflow analysis that avoids a type
of infeasible path that arises in object-oriented programs when two or more
methods are dynamically dispatched on the same receiver object.
%\mtodo{Should we define the term ``receiver''?}
In such cases, if the
method calls are polymorphic (i.e., if they dispatch to different method definitions 
depending on the type of the receiver expression at run time), then their dispatch 
behaviours will be correlated. A recent paper \cite{DBLP:journals/scp/Tip15} identified
this problem but did not present a concrete solution or algorithm, and we are not aware 
of any existing dataflow analysis that is capable of avoiding infeasible paths that arise 
in the presence of correlated method calls.

\begin{figure}
  \centering
    \tikzset{
  ashadow/.style={opacity=.25, shadow xshift=0.07, shadow yshift=-0.07},
}
  \tikzstyle{problem}=[fill=greyblue,text width=1.8cm,rounded corners,font=\scriptsize,text=charcoal,drop shadow={ashadow, color=greyblue}]
  \tikzstyle{result}=[fill=bisque,text width=1.5cm,rounded corners,font=\scriptsize,text=charcoal,drop shadow={ashadow, color=greyblue}]
\begin{tikzpicture}
    \node [problem]
      (ifds) {IFDS problem};
    \node [problem]
      (equiv) [above right=0.5cm and .7\dist of ifds.east] {Equivalent IDE problem};
    \node [problem]
      (ccide)[below=2cm of equiv.west, anchor=west] {Correlated-calls IDE problem};
    \node [result]
      (equivres) [right=\dist of equiv.east] {Equivalence-IDE result};
    \node [result]
      (ccres) [below=2cm of equivres.west,anchor=west] {Correlated-calls IDE result};
    \node [result,text height=1.5cm,text width=2cm]
      (ifdsres) [below right=0cm and \dist of equivres.east] {};
    \node [below,font=\scriptsize] at (ifdsres.north) {IFDS result};
    \node [rounded corners,font=\scriptsize,text width=1.5cm,fill=lightsalmonpink,text=charcoal]
      (improved) [above right=.6cm and (\dist+.3cm) of ccres.east] {Improved IFDS result};
    
    \path[->] (equivres) edge[out=0,in=150] node[above,font=\scriptsize]{$\backEq$} (ifdsres);
    \path[->] (ccres) edge[out=0,in=230] node[above left,font=\scriptsize] {$\backCC$} (improved);
    %\path[->] (ccres) edge[out=25,in=210] node[above,font=\scriptsize] {$\backEq$} (ifdsres);
    \path[->] (ifds) edge[out=43,in=183] node[above,font=\scriptsize] {$\transEq$} (equiv);
    \path[->] (ifds) edge[out=-45,in=173] node[below,font=\scriptsize] {$\transCC$} (ccide);
    \path[->] (equiv) edge node[above,font=\scriptsize]{$\result$} (equivres);
    \path[->] (ccide) edge node[above,font=\scriptsize]{$\result$} (ccres);
    \path[->,dashed] (ifds) edge[out=-3,in=183] node[above,font=\scriptsize]{$\result_{\text{IFDS}}$} (ifdsres);
\end{tikzpicture}
  \caption{Transformations between IFDS and IDE problems and their results}%
  \label{fig:transformations}%
\end{figure}

The approach taken in our work is to transform a standard IFDS problem into an
IDE problem that precisely accounts for infeasible paths due to correlated calls. 
The results of this IDE problem can be mapped back to the dataflow domain of the 
original IFDS problem.

We present a formalization of the transformation and prove its correctness
as follows. 
First, we derive an IDE problem that is equivalent to the original IFDS problem,
as described by Reps et al.~\cite{sagiv1996precise}.
Solving this ``Equivalence-IDE Problem'' with the IDE algorithm yields a result that
is equivalent to the one obtained by applying the IFDS algorithm to the original IFDS problem.
Then, we derive a ``Correlated Calls 
IDE Problem'' from the original IFDS problem and show 
that the solution to this problem, when converted to an IFDS result, is more precise
than the original IFDS result.
We also show that the correlated-calls analysis is sound, i.e., that it never considers concrete execution paths as infeasible.
This is illustrated schematically
in Figure~\ref{fig:transformations}%
\footnote{ 
  The labels %$\resultIDE$, $\resultIFDS$, $\backEq$, $\backCC$, $\transEq$, and $\transCC$
  on edges in 
  Figure~\ref{fig:transformations} reflect a number of mappings and projections 
  that will be defined in Section~\ref{sec:cca} and that can be ignored here.
}.

We implemented the correlated-calls transformation and the IDE algorithm in Scala,
on top of the WALA framework for static analysis of JVM bytecode~\cite{fink2012wala}.
Our prototype implementation was tested extensively by using it to transform an IFDS-based 
taint analysis into a more precise IDE-based taint analysis, and applying the latter
to small example programs with correlated calls. Our prototype along with all tests
will be made available to the artifact evaluation committee.

%We also report on preliminary experiments in which our correlated-calls transformation
%is applied to an IFDS formulation of a simple taint analysis. Our results show that
%solving the resulting IDE problem avoids infeasible paths due to correlated calls as
%expected.
%
%In summary, the contributions of this paper are as follows:
%\begin{itemize}
%  \item
%    We present a general approach for transforming IFDS problems into corresponding
%    IDE problems that avoid infeasible paths due to correlated method calls and
%    prove its correctness. 
%  \item
%    We implemented the approach in Scala, on top of the WALA program analysis framework
%    and report on preliminary experiments. 
%\end{itemize}

The remainder of this paper is organized as follows.
%
Section~\ref{sec:MotivatingExample} presents a motivating example.
%
Section~\ref{sec:bg} reviews the IFDS and IDE algorithms.
%
Section~\ref{sec:cca} presents the correlated-calls transformation
and sketches a proof of its correctness%
\footnote{
  Detailed proofs of our lemmas and theorems can be found in
  the supplemental materials that are included with this submission.
}.
%
Related work is discussed in Section~\ref{sec:Related}.
%
Finally, conclusions and directions for future work are presented in Section~\ref{sec:Conclusions}.

 
 
