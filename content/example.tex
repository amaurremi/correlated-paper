\section{Motivating Example}
  \label{sec:MotivatingExample}
  
In this section, we illustrate our approach using a small motivating example that
shows how our technique can be used to improve the precision of taint analysis.
A taint analysis is a specialized dataflow analysis for computing how string values
may flow from ``sources'', which are typically statements that read untrusted input, 
to ``sinks'', which are typically securitive-sensitive operations such as calls to 
a database. In previous research \TODO{cite}, taint analysis algorithms have been 
formulated as IFDS problems.     
  
\begin{figure}
  \centering
  
  \begin{minipage}{\textwidth}
    \inputMinted{java}{ccexample.java}
  \end{minipage}
    \vspace*{-4mm}
  \caption{Example program containing correlated calls}
  \label{list:ccexample}
   
   \begin{minipage}{\textwidth}
    \tikzstyle{supergraph}=[
      fill=greyblue,
      rounded corners,
      font=\footnotesize,
      align=right,
      text=charcoal,
      drop shadow={ashadow, color=greyblue}
    ]
    \tikzstyle{supergraph_y}=[
      fill=bisque,
      rounded corners,
      font=\footnotesize,
      align=left,
      text=charcoal,
      drop shadow={ashadow, color=greyblue}
    ]
    \tikzstyle{supergraph_f}=[
      fill=lightsalmonpink,
      rounded corners,
      font=\footnotesize,
      align=left,
      text=charcoal,
      drop shadow={ashadow, color=greyblue}
    ]
    \tikzstyle{description}=[fill=white]
    \tikzstyle{n}=[fill=black,circle,inner sep=1.5pt]
    \tikzstyle{arrowtext}=[font=\tiny,color=black,above]
    
\hspace*{-10pt}
\begin{tikzpicture}
\scalebox{.8}{
% main method nodes
    \node [supergraph] (st_main) {\textsf{start}$_{\texttt{main}}$};
    \node [supergraph] (asgn_a) [below = \dist of st_main.south] {\texttt{a = args==null\,?}\\\texttt{new\,A()\,:\,new\,B()}};
    \node [supergraph] (call_foo) [below = \dist of asgn_a.south] {\textsf{call}$_\texttt{foo}$};
    \node [supergraph] (return_foo) [below = \dist of call_foo.south] {\textsf{return}$_\texttt{foo}$\\\texttt{v = a.foo()}};
    \node [supergraph] (call_bar) [below = \dist of return_foo.south] {\textsf{call}$_\texttt{bar}$};
    \node [supergraph] (return_bar) [below = \dist of call_bar.south] {\textsf{return}$_\texttt{bar}$};
    \node [supergraph] (end_main) [below = \dist of return_bar.south] {\textsf{end}$_{\texttt{main}}$};
    
% main method edges 
    \path[->,ultra thick] (st_main) edge (asgn_a);
    \path[->,ultra thick] (asgn_a) edge (call_foo);
    \path[->] (call_foo) edge (return_foo);
    \path[->,ultra thick] (return_foo) edge (call_bar);
    \path[->] (call_bar) edge (return_bar);
    \path[->,ultra thick] (return_bar) edge (end_main);
    
%A.foo method nodes
    \node [supergraph_y] (st_a_foo) [right = 5*\dist of st_main] {\textsf{start}$_{\texttt{A.foo}}$};
    \node [supergraph_y] (return_secret) [below = \dist of st_a_foo.south] {\texttt{return secret()}};
    \node [supergraph_y] (end_a_foo) [below = \dist of return_secret.south] {\textsf{end}$_\texttt{A.foo}$};

%A.foo method edges
    \path[->] (st_a_foo) edge (return_secret);
    \path[->] (return_secret) edge (end_a_foo);
    
%A.bar method nodes
    \node [supergraph_y] (st_a_bar) [right = 5*\dist of st_a_foo] {\textsf{start}$_{\texttt{A.bar}}$};
    \node [supergraph_y] (end_a_bar) [below = \dist of st_a_bar.south] {\textsf{end}$_\texttt{A.bar}$};
    
%A.bar method edges
    \path[->] (st_a_bar) edge (end_a_bar);

%B.foo method nodes
    \node [supergraph_f] (st_b_foo) [right = 5*\dist of call_bar] {\textsf{start}$_{\texttt{B.foo}}$};
    \node [supergraph_f] (return_not_secret) [below = \dist of st_b_foo.south] {\texttt{return "not secret"}};
    \node [supergraph_f] (end_b_foo) [below = \dist of return_not_secret.south] {\textsf{end}$_\texttt{B.foo}$};
    
%B.foo method edges
  \path[->] (st_b_foo) edge (return_not_secret);
  \path[->] (return_not_secret) edge (end_b_foo);
    
%B.bar method nodes
    \node [supergraph_f] (st_b_bar) [right = 5*\dist of st_b_foo] {\textsf{start}$_{\texttt{B.bar}}$};
    \node [supergraph_f] (print) [below = \dist of st_b_bar.south] {\texttt{print(s)}};
    \node [supergraph_f] (end_b_bar) [below = \dist of print.south] {\textsf{end}$_\texttt{B.bar}$};
    
%B.bar method edges
  \path[->] (st_b_bar) edge (print);
  \path[->] (print) edge (end_b_bar);
  
%inter-procedural edges
  \path[->,dashed,ultra thick] (call_foo) edge[out=20,in=190] (st_a_foo);
  \path[->,dashed] (call_foo) edge[color=lightsalmonpink,out=-20,in=135] (st_b_foo);
  \path[->,dashed] (call_bar) edge[out=25,in=220] (st_a_bar);
  \path[->,dashed,ultra thick] (call_bar) edge[color=lightsalmonpink,out=18,in=160] (st_b_bar);
  \path[->,dashed,ultra thick] (end_a_foo) edge[out=220,in=15] (return_foo);
  \path[->,dashed] (end_b_foo) edge[color=lightsalmonpink,out=160,in=-60] (return_foo);
  \path[->,dashed] (end_a_bar) edge[out=210,in=40] (return_bar);
  \path[->,dashed,ultra thick] (end_b_bar) edge[color=lightsalmonpink,out=200,in=-40] (return_bar);
    
%arrow description
    \node [description] (intra_arrow_left) [above left = 2*\dist and 3.5*\dist of st_a_bar] {};
    \node [description] (intra_arrow_right) [right = \dist of intra_arrow_left] {};
    \node [description] (inter_arrow_left) [below = .5*\dist of intra_arrow_left] {};
    \node [description] (inter_arrow_right) [below = .5*\dist of intra_arrow_right] {};
    \path[->](intra_arrow_left) edge node[right = .5*\dist]{intra-procedural edge} (intra_arrow_right);
    \path[dashed,->](inter_arrow_left) edge node[right = .5*\dist]{inter-procedural edge} (inter_arrow_right); 
}
\end{tikzpicture}
   \end{minipage}
   \vspace*{-25mm}
  \caption{
    Control flow supergraph for the example program of Figure~\ref{list:ccexample}.
    An infeasible path is shown in bold. 
  }%
  \label{fig:examplesupergraph}%
\end{figure} 
  
Figure~\ref{list:ccexample} shows a small Java program that we will use to illustrate
how our technique can improve the precision of a simple taint analysis. The program
declares a class \code{A} with a subclass \code{B},  where \code{A} defines methods 
\code{foo()} and \code{bar()} that are overridden in \code{B}.  In order to keep the 
example small and self-contained, it is assumed  that secret values are created by 
an unspecified function \code{secret()}, which is called in method \code{A.foo()} on 
line~\ref{line:Afoo}. Furthermore, it is assumed that any write to the standard output 
stream via calls \code{System.out.println()} such as the one in method \code{B.bar()} 
on line~\ref{line:Bbar} is a security-sensitive operation. Depending on the number of 
arguments passed to the program, the \code{main()} method of the example program creates 
either an \code{A}-object or a \code{B}-object on line~\ref{line:createObject}. The 
program then calls \code{foo()} on this object on line~\ref{line:callfoo}, which is 
followed by a call to \code{bar()} on the same object on line~\ref{line:callbar}.  
 
The specific question that we would like to answer for the example program of
Figure~\ref{list:ccexample} is the following: Is it possible for the untrusted value 
that is read on line~\ref{line:Afoo} to flow to the print statement on line~\ref{line:Bbar}? 
%
To determine whether such a flow of tainted data is possible, consider the control-flow
supergraph for the example program that is shown in Figure~\ref{fig:examplesupergraph}.
 The nodes in this graph correspond to statements, method entry points (start nodes) and 
method exit points (end nodes). Note that for each method call, the graph contains a 
distinct call-node and a return-node. Edges in the graph reflect intraprocedural control flow, 
flow of control from a caller to a callee (edges from call-nodes to start-nodes), or
flow of control from a callee back to a caller (edges from end-nodes to return-nodes). 

In the case of our example, the control flow within each of the methods is straightforward and
all interesting issues arise from interprocedural control flow. In particular,
note that, since the variable \code{a} may point to either an \code{A}-object or a \code{B}-object, 
the call on line~\ref{line:callfoo} may dispatch to either \code{A.foo()} or to \code{B.foo()},
and this is reflected by edges  
  from the node labeled $\code{call}_\code{foo}$ to the nodes labeled
  $\code{Start}_\code{A.foo()}$ and $\code{Start}_\code{B.foo()}$
and by edges
  from the nodes labeled $\code{End}_\code{A.foo}$ and $\code{End}_\code{B.foo}$ 
  to the node labeled  $\code{return}_\code{foo}$.  
Similarly, there are edges from the node labeled $\code{call}_\code{bar}$ to the nodes 
$\code{Start}_\code{A.bar()}$ and $\code{Start}_\code{B.bar()}$, and 
edges
  from the nodes labeled $\code{End}_\code{A.bar}$ and $\code{End}_\code{B.bar}$ 
  to the node labeled  $\code{return}_\code{bar}$. 
 
Informally speaking, an IFDS-based dataflow analysis propagates dataflow facts along the edges
of a control flow supergraph such as the one in Figure~\ref{fig:examplesupergraph}. The
IFDS algorithm was carefully designed to avoid infeasible paths that may arise in the
presence of multiple calls to the same method. However, in this example all methods are
called in exactly one place, so IFDS is unable to eliminate dataflow along any of the
paths shown in the figure. As a result, an IFDS-based taint analysis such as the
one of \TODO{cite} would report that the secret value read on line~\ref{line:Afoo}
might flow to the print statement on line~\ref{line:Bbar}. 

As we discussed previously, the calls to \code{foo()} and \code{bar()} may dispatch
to different implementations of these methods in classes \code{A} and \code{B}, 
because the variable \code{a} that serves as the receiver for the call may be bound to
 objects of type \code{A} or type \code{B} at run time. 
However, consider the fact that the methods \code{foo()} and \code{bar()} are invoked
on \textit{the same object}. This means that the behaviors of the two method calls
are \textit{correlated}: if the call to \code{foo()} dispatches to \code{A.foo()},
then the call to \code{bar()} must dispatch to \code{A.bar()}, and if the call
to \code{foo()} dispatches to \code{B.foo()}, then the call to \code{bar()} must 
dispatch to \code{B.bar()}.  Consequently, paths such as the one highlighted in
Figure~\ref{fig:examplesupergraph} where the call to \code{foo()} dispatches 
to \code{A.foo()} and where the call to \code{bar()} dispatches to \code{B.bar()}
are infeasible.  

The main contribution of this paper is an algorithm for transforming an IFDS problem 
into an IDE problem that expresses the feasibility of paths
in light of correlated calls.
Informally speaking, our approach associates with each edge in an
interprocedural control flow graph a function that records the types of 
variables that are used as the receiver of correlated method calls. Paths that 
are composed of edges in which the same receiver expression has different types
are infeasible, and the propagation of dataflow facts along such paths is
prevented. Applying our technique to an IFDS-based taint analysis would enable
the resulting IDE-based taint analysis to determine that no secret value can flow from
line~\ref{line:Afoo} to the print statement on line~\ref{line:Bbar}. 
 


  
\commentout{

Consider a call site~$r.m()$ in an object-oriented programming language, where the variable~$r$ is the \textit{receiver} variable of the call site and $m$ is the name of the invoked method\footnote{We assume an internal representation of the program in which for each call site $e_r.m()$, the expression~$e_r$ has been evaluated to the variable~$r$.}. In the rest of the paper, we use the general term \textit{receiver} to mean a receiver variable.
At runtime, the actual method that will be invoked by the call site depends on the runtime type of the object referenced by $r$. If the call site $r.m()$ can be associated with more than one method at compile time, we will say that the call site is \textit{polymorphic}.

For example, in Listing~\ref{list:ccexample}, it is not possible to infer statically whether the runtime type of the variable \verb'a' in the \texttt{main} method is \verb'A' or \verb'B'.
The call \verb'a.foo()' can be dispatched to either \verb'A.foo' or \verb'B.foo', and \verb'a.bar(v)' can be dispatched to either \verb'A.bar' or \verb'B.bar'.
A concrete execution path for the main method might therefore go through \verb'A.foo' and \verb'A.bar', or through \verb'B.foo' and \verb'B.bar'.
However, there cannot be an execution path through \verb'A.foo' and \verb'B.bar' or through \verb'B.foo' and \verb'A.bar'.



We call the invocations to methods \verb'foo' and \verb'bar' \textit{correlated}.
More generally, correlated calls occur when more than one polymorphic call is invoked on the same receiver variable.

Suppose we wanted to perform a taint analysis on the program in Listing~\ref{list:ccexample}.
Most dataflow-analysis algorithms, including IFDS, would conservatively assume that the call \verb'a.bar' could be dispatched to both \verb'A.bar' and \verb'B.bar', independently of what \verb'a.foo' had been dispatched to in the previous line.

As a result, 
such an analysis would consider a path through \verb'A.foo' and \verb'B.bar' feasible. This means that the variable \verb'v' would be considered secret. We would conclude that a secret value is passed to \verb'B.bar' and printed to the user. In other words, we would consider the program to leak secret information, which it does not do in any concrete execution.

\commentout{
Our technique for improving the precision of an IFDS result is based on transforming the original IFDS problem into a more expressive \textit{Inter-procedural Distributive Environment} (IDE) problem. IDE problems can be solved with the IDE algorithm which is a generalization of IFDS~\cite{sagiv1996precise}. The IDE algorithm can, for instance, solve certain versions of the constant propagation problem that IFDS cannot.

To improve the precision of IFDS results, given an IFDS problem $P$, we convert it into an IDE problem $Q$ that accounts for correlated calls. We then use the IDE algorithm to obtain a solution to $Q$. Finally, we convert the IDE result into a IFDS result. In the presence of correlated calls, the obtained IFDS result can be more precise than the solution that the IFDS algorithm would compute for $P$.

The goal of the correlated-calls analysis presented in this work is to modify the output of an IFDS analysis to account for correlated calls. Specifically, the correlated-calls analysis improves the precision of IFDS problem results by eliminating infeasible execution paths caused by correlated calls. This is done by converting the input-IFDS problem to an IDE problem that detects infeasible paths, and converting the IDE result back to a more precise IFDS result.

The contributions of this work are:
\begin{itemize}
  \item A transformation from IFDS to IDE problems that considers correlated calls.
  \item An implementation in Scala of the correlated-calls transformation and the IDE algorithm which is based on the WALA framework for static analysis on Java bytecode~\cite{fink2012wala}.
\end{itemize}\mtodo{Reference the figure here.}

  We prove that the solution to an IDE problem that considers correlated calls is more precise than the solution to the original IFDS problem.
  We also show that the correlated-calls analysis is sound, i.e. that it never considers concrete execution paths as infeasible.\mtodo{As discussed on the phone, we want to talk more about the results here.}

 Finally, we evaluate the effectiveness of the correlated-calls analysis using an implementation of taint analysis as the source IFDS problem.

The remainder of this paper is organized as follows. In the next section, we describe the IFDS and IDE analyses in detail. In Section~\ref{chapter:cca} we present the correlated-calls analysis as a transformation of IFDS problems into a special kind of IDE problem. Section~\ref{sec:ccdatastr} describes an efficient representation of the data structures that are required to define a correlated-calls IDE transformation.
In Section~\ref{chapter:eval} we address some implementation aspects of the correlated-calls analysis and present an evaluation of its results. Section~\ref{chapter:concl} contains concluding remarks. The proofs for the lemmas in the paper are provided in the Appendix.\mtodo{Reference appendix like this?}
}
}
