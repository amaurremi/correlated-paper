\section*{Appendix}
In this appendix we present the proofs to the Lemmas introduced in Section~\ref{sec:cca}.

\subsection*{Representation of Micro Functions}
We start by presenting the proofs to the lemmas about the representation of micro functions.

\equalEq*
\begin{proof}
   % OL: Can't we just say the following?
   First, we need to show that for all $r\in R$, if $I(r)=I'(r)$ and $U(r)=U'(r)$, then $\denote{\setpair IU}=\denote{\setpair{I'}{U'}}$. Indeed, we can see that
    \begin{align*}
        \denote{\setpair {I}{U}}
       & = \lambda m. \lambda r. (m(r) \cap I(r))\cup U(r))\\
       & = \lambda m. \lambda r. (m(r) \cap I'(r))\cup U'(r))\\
       & = 
        \denote{\setpair {I'}{U'}}.
    \end{align*}

%OL: Again, we can keep it simpler:
For the other direction:
\begin{align*}
&&
    \denote{\setpair{I}{U}} &= \denote{\setpair{I'}{U'}}\\
&\implies &
    \denote{\setpair{I}{U}}(\lambda r. \emptyset) &= \denote{\setpair{I'}{U'}}(\lambda r. \emptyset)\\
&\implies &
(\lambda m.\lambda r.(m(r)\cap I(r))\cup U(r))(\lambda r.\emptyset) &=
(\lambda m.\lambda r.(m(r)\cap I'(r))\cup U'(r))(\lambda r.\emptyset) \\
&\implies &
\lambda r.(\emptyset \cap I(r))\cup U(r) &=
\lambda r.(\emptyset \cap I'(r))\cup U'(r) \\
&\implies &
\lambda r.U(r) &=
\lambda r.U'(r) \\
&\implies &
U &=
U' \\
\end{align*}
Similarly:
\begin{align*}
&&
    \denote{\setpair{I}{U}} &= \denote{\setpair{I'}{U'}}\\
&\implies &
\denote{\setpair{I}{U}}(I) &= \denote{\setpair{I'}{U'}}(I)\\
&\implies &
(\lambda m.\lambda r.(m(r)\cap I(r))\cup U(r))(I) &=
(\lambda m.\lambda r.(m(r)\cap I'(r))\cup U'(r))(I) \\
&\implies &
\lambda r.(I(r) \cap I(r))\cup U(r) &=
\lambda r.(I(r) \cap I'(r))\cup U'(r) \\
&\implies &
\lambda r.(I(r) \cap I(r)) &=
\lambda r.(I(r) \cap I'(r)) \textrm{ since $\forall r.U(r) \subseteq I(r)$} \\
&\implies \forall r.&
I(r) &= I(r) \cap I'(r)\\
&\implies \forall r.&
I(r) &\subseteq I'(r)\\
\end{align*}
\mtodo{I agree that this is much better than what I had. I think from $I(r) = I(r) \cap I'(r)$ we can conclude that $I(r)=I'(r)$.}
Symmetrically, we can also establish that $\forall r.I'(r) \subseteq I(r)$ by applying the functions to $I'$ instead of to $I$. Therefore, $I = I'$.
\end{proof}

\equalComp*
\otodo{The (m)(r) does not seem necessary. Can we replace $r'$ with $r$ and $m'$ with $m$?}
\begin{proof}
\begin{align*}
  &\denote{\setpair IU\circ\setpair{I'}{U'}}\\
  =&\denote{\setpair{\lambda r\,.\,(I(r')\cap I'(r))\cup U(r)}
                    {\lambda r\,.\,(I(r)\cap U'(r))\cup U(r)}}\\
  =&\lambda m\,.\,\lambda r\,.\,(m(r)\cap(I(r)\cap I'(r))\cup U(r))\cup(I(r)\cap U'(r))\cup U(r)\\
  =&\lambda m\,.\,\lambda r\,.\,(m(r)\cap I'(r))\cup I(r)\cup (U'(r)\cap I(r))\cup U(r)\\
  =&\lambda m\,.\,\lambda r\,.\,(((m(r)\cap I'(r))\cup U'(r))\cap I(r))\cup U(r)\\
  =&\lambda m\,.\,\lambda r\,.\,(((m(r)\cap I'(r))\cup U'(r))\cap I(r))\cup U(r)\\
  =&\lambda m\,.\,\left(\lambda m'\,.\,\lambda r\,.\,(m'(r)\cap I(r))\cup U(r)\right)
    \left(
    \left(\lambda r\,.\,(m(r)\cap I'(r))\cup U'(r)\right)
    \right)\\
  =&\left(\lambda m\,.\,\lambda r\,.\,(m(r)\cap I(r))\cup U(r)\right)\circ
    \left(\lambda m\,.\,\lambda r\,.\,(m(r)\cap I'(r))\cup U'(r)\right)\\
  =&\denote{\setpair IU}\circ\denote{\setpair{I'}{U'}}.\qedhere
\end{align*}
\end{proof}

\equalMeet*
\begin{proof}
\begin{align*}
  &\denote{\setpair{I}{U} \sqcap \setpair{I'}{U'}}\\
  =&\denote{\setpair{\lambda r\,.\,I(r) \cup I'(r)}{\lambda r\,.\,U(r) \cup U'(r)}}\\
  =&\lambda m\,.\,\lambda r\,.\,(m(r)\cap(I(r)\cup I'(r))\cup U(r)\cup U'(r)\\
  =&\lambda m\,.\,\lambda r\,.\,(m(r)\cap I(r))\cup U(r)\cup(m(r)\cap I'(r))\cup U'(r)\\
  =&\lambda m\,.\,\lambda r\,.\,\denote{\setpair{I}{U}}(m)(r)\cup\denote{\setpair{I'}{U'}}(m)(r)\\
  =&\denote{\setpair{I}{U}} \sqcap \denote{\setpair{I'}{U'}}.\qedhere
\end{align*}
\end{proof}

\commentout{
\begin{proof}[\textbf{Proof of Lemma~\ref{lem:efficient}}]
  \begin{enumerate}
    \item The identity function is represented as
      \[
        \denote{\id}=\{(r,\,\langle \bot_T,\,\top_T\rangle)\,|\,r\in \rcc\}\,;
      \]
      the top function is represented as
      \[
        \denote{\lambda m\,.\,\topcc}=\{(r,\,\langle \top_T,\,\top_T\rangle)\,|\,r\in \rcc\}\,.
      \]
    \item Equations~\eqref{eq:compclosed} and~\eqref{eq:meetclosed} show that the representation of micro functions is closed under composition and meet.
    \item To show that our representation for micro functions forms a lattice with finite height, let us first show that $\lcc_\rcc:\,\rcc\to2^T$ forms a lattice. Since $T$ is a finite set, $(2^T,\,\subseteq)$ is a finite-height lattice. $\rcc$ is a finite set. Hence, the mapping
    \[
      \rcc\mapsto2^T=\{(r,\,t)\,|\,r\in \rcc,\,t\in 2^T\}=\lcc_\rcc
    \]        
    also forms a finite-height lattice \cite{nielson1999principles}. 
    
    Furthermore, $\lcc_\rcc$ is a finite set. 
    Every element of $\lcc_\rcc$ can be applied to $|\rcc|$ receivers, where each receiver is mapped to a set of types. There are $|\rcc|\cdot2^{|T|}$ different possibilities to form those mappings, so
    \[
      |\lcc_\rcc|=|\rcc|\cdot2^{|T|}.
    \]
    Therefore, $\lcc_\rcc\mapsto \lcc_\rcc$ also forms a finite-height lattice.
    \item All operations can be computed in $O(\rcc\times T)$ time. Note that the $\rcc$ and $T$ sets are an input to the correlated-calls analysis, and the time it takes to compute the meet or composition of micro functions is independent of the representation of the specific operand micro functions.
    \item The space bound is $O(\rcc\times T)$.
  \end{enumerate}
\end{proof}
}

\subsection*{Soundness and Precision}

In this part of the Appendix we prove the Lemmas of Soundness and Precision of the correlated-calls analysis.

To prove the Soundness Lemma, we first introduce Lemmas~\ref{lem:sound1} and~\ref{lem:sound3}.

As previously we will denote the top element in the environment lattice as~$\top_\textsf{Env}$.
%\otodo{I've been using $\top_\textsf{Env}$ for clarity.}

For the purpose of the proofs, we will rewrite Equation~\eqref{eq:edgefndef} that defines an edge function as follows:
  \begin{equation}\label{eq:edgefnThroughDelta}
    \ccedgefn S=\lambda e\,.\,
    \begin{cases}
      \id&\text{if $d_1=d_2=\Lambda$,}\\
      \lambda m\,.\,\varepsilon(e)(\delta(m))&\text{otherwise},
    \end{cases}
  \end{equation}
  where $S\subseteq R$, $d_1$ and $d_2$ are the source and target facts, and for a map $m\in\lcc_U$, $\delta(m)$ is either $m$ or $\botcc$:
  \begin{equation}\label{eq:deltadef}
    \delta(m)=\begin{cases}
      \botcc&\text{if $d_1=\Lambda$}\\
      m&\text{otherwise.}
    \end{cases}
  \end{equation}

Additionally, for a path $p=[\startmain,\,\dots]$ and a fact $d\in D$, we will denote the lattice element that is mapped to $d$ according to the flow functions of path $p$ as follows:
\begin{equation}
  \mpd=\menv(p)(\top_\textsf{Env})(d)\,.
\end{equation}

The following Lemma shows that the lattice elements (receiver-to-types maps) of a correlated-calls IDE analysis correctly overapproximate the possible types of a receiver in a program execution.

\begin{lemma}\label{lem:sound1}
Let $p=[\startmain,\,\dots,\,n]$ be some concrete execution trace of the program, and let $r\in R$ be a receiver. If after the execution trace $p$, at node $n$, $r$ points to an object of runtime type~$t$, and $d\in D$ is a fact such that $d\in M_F(p)(\varnothing)$, then
  \begin{equation}
    t\in\mpd(r)\,.
  \end{equation}
\end{lemma}
\begin{proof}
  By induction on the length of the trace.
  
  \textit{Basis:} $p=[\startmain]$. Then there is no instruction at which a receiver $r$ could be instantiated, and the Lemma is trivially true.    

  \textit{Induction hypothesis:} Let $p=[\startmain,\,\dots,\,n_{k-1}]$, and let $\uptau$ be the set of types to which $\mpdkm$ maps $r$:
  \begin{equation}
    \uptau=\mpdkm(r)\,.
  \end{equation}
  Assume that for a concrete execution path $p=[\startmain,\,\dots,\,n_{k-1}]$, at node $(n_{k-1},\,d_{k-1})$, the Lemma holds, i.e. $t\in\uptau$.
  
  \textit{Induction step:} Let $p'=[\startmain,\,\dots,n_{k-1},\,n_k]$ and $t'\in T$ be the type to which $r$ is mapped at $n_k$.
  
  For each $i$, let $e_i$ be the edge $((n_{i-1},\,d_{i-1}),\,(n_i,\,d_i))$. Note that $$e_1=((\startmain,\,\Lambda),\,(n_1,\,d_1))\,.$$  

  Observe that
  \begin{align*}
    \mppd
    &=\menv(p')(\top_\textsf{Env})(d)\\
    &=\left(\menv(e_k)\circ\menv(e_{k-1})\circ\ldots\circ\menv(e_1)\right)(\top_\textsf{Env})(d)\\
    &=\menv(e_k)\left(\menv(p)(\top_\textsf{Env})\right)(d)\,.
  \end{align*}
  
As shown in~Sagiv et al.~\cite{sagiv1996precise}, the relationship between environment transformers and edge functions can be described with the following equation. For an edge $(n_1,\,n_2)\in E^*$ an environment $\textsf{env}$ that maps $D$ to $L$, and a fact $d\in D$,
\begin{align}\label{eq:envTransToEdgeFnEdge}
  \menv&((n_1,\,n_2))(\textsf{env})(d)\notag\\
  &=\edgefn((n_1,\,\Lambda),\,(n_2,\,d))(\top)\sqcap\bigsqcap_{d'\in D}\edgefn((n_1,\,d'),\,(n_2,\,d))(\textsf{env}(d'))\,.
\end{align}  
Then, according to~\eqref{eq:envTransToEdgeFnEdge},\mtodo{Is it okay that in~\eqref{eq:envTransToEdgeFnEdge}, we use $\edgefn$, and below we use $\ccedgefn R$?}
  \begin{align*}
    &\menv(e_k)\left(\menv(p)(\top_\textsf{Env})\right)(d)(r)\\
      =&\bigg(\ccedgefn R((n_{k-1},\,\Lambda),\,(n_k,\,d))(\topcc)\sqcap\\
       &\bigsqcap_{d'\in D}\ccedgefn R((n_{k-1},\,d'),\,(n_k,\,d))(\menv(p)(\top_\textsf{Env})(d'))\bigg)(r)\\
      \supseteq&
        \bigsqcap_{d'\in D}\ccedgefn R((n_{k-1},\,d'),\,(n_k,\,d))(\menv(p)(\top_\textsf{Env})(d'))(r)\\
      \supseteq&\,\ccedgefn R((n_{k-1},\,d_{k-1}),\,(n_k,\,d))(\mpdkm)(r)\,.
  \end{align*}
  Therefore, 
  \begin{equation}\label{eq:musubset}
    \efek\subseteq\mppd(r)\,.
  \end{equation}
  
  We will now show that
  \[
    t'\in\efek\,,
  \]
  which, due to~\eqref{eq:musubset}, means that the Lemma holds.
  
  According to~\eqref{eq:edgefnThroughDelta}, there are two cases in which $\ccedgefn R(e_k)$ could fall.

  If $d_{k-1}=d_k=\Lambda$, then $d_k\notin M_F(p)(\varnothing)$, since it does not belong to the set $D$, and the Lemma trivially holds.
  
  Otherwise, 
  \[
    \ccedgefn R(e_k)=\lambda m\,.\varepsilon(e_k)(\delta(m))\,.
  \]
  It follows that
  \begin{align}\label{eq:efek}
    \efek
    &=(\lambda m\,.\varepsilon(e_k)(\delta(m)))(\mpdkm)(r)\notag\\
    &=\varepsilon(e_k)(\delta(\mpdkm))(r).
  \end{align}
  Let us denote the lattice element $\delta(\mpdkm)$ with $\Delta$:
  \[
    \Delta=\delta(\mpdkm)\,.
  \]
  Note that since $\Delta$, according to~\eqref{eq:deltadef}, can be either $\botcc$ or $\mpdkm$, it always maps $r$ to a set containing~$t$:
  \begin{equation}\label{eq:deltaContainsT}
    t\in\Delta(r)\,.
  \end{equation}
  Note also that unless the instruction at $n_{k-1}$ contains an assignment for $r$, $r$ is mapped to the same object of type $t$ as at node $n_{k-1}$, and $t=t'$. Therefore, for the non-assignment instructions, it is sufficient to prove that $t\in\Delta(r)$.
 
   Depending on the instructions at the nodes $n_{k-1}$ and $n_k$, there are four cases:
  \begin{enumerate}
    \item\label{item:asgn} The instruction at $n_{k-1}$ is an assignment for a receiver $r'\in R$.
        Since $\varepsilon_R(e_k)=\lambda m\,.\,m[r'\to\bot_T]$,
        \begin{align*}
          \efek&=            
            (\lambda m\,.\,m[r'\to\bot_T])(\Delta)(r)\\
          &=\Delta[r'\to\bot_T](r)\,.
        \end{align*}
      In the resulting map, $r'$ is mapped to $\bot_T$. Then
      
      \begin{enumerate}
        \item if $r=r'$, then $\efek=\bot_T$, which contains $t'$.
        \item\label{item:defaultmap} If $r\ne r'$, then $r$ has not been reassigned a value, and still maps to the same object of type $t$. The receiver $r$ is mapped to $\Delta(r)$, which, according to~\eqref{eq:deltaContainsT}, contains $t$. Since $t=t'$, $\Delta(r)$ contains $t'$.
      \end{enumerate}
    \item\label{item:callstart} $e_k$ is a call-start edge with signature $s$, and $f$ is the called procedure.
      Then
        \begin{align*}
          \efek
          &=(\lambda m\,.\,m[r'\to m(r')\cap\tau(s,\,f)])(\Delta)(r)\\
          &=\Delta[r'\to\Delta(r')\cap\tau(s,\,f)]\,,
        \end{align*}
      where $r'$ is the receiver of the call.
      \begin{itemize}
        \item If $r'=r$, then $\Delta(r')=\Delta(r)$ which contains $t$. Since $t\in\tau(s,\,f)$, it follows that $t\in\Delta(r)\cap\tau(s,\,f)$, and $t\in\efek$.
         \item If $r'\ne r$, see~(\ref{item:defaultmap}).
       \end{itemize}
    \item $e_k$ is an end-return edge, $r_1,\,\dots,\,r_k\in R$ are the local variables in the callee method, $r'$ is the receiver of the call site corresponding to the return node $n_k$, and $f$ is the called method with signature~$s$.
      Then 
      \[
        \varepsilon_R(e_k)=\lambda m\,.\,m[r'\to m(r')\cap\tau(s,\,f)][r_1\to\bot_T]\dots[r_k\to\bot_T].
      \]
      If $r\in\{r_1,\,\dots,\,r_k\}$, see~Case~\ref{item:asgn}. Otherwise, the case is analogous to Case~\ref{item:callstart}.
    \item\label{item:idcase} The node contains any other instruction.
      Then 
      \[
        \ccedgefn R(e_k)(\mpdkm)(r)=\id(\Delta)(r)=\Delta(r),
      \]
      which contains $t$ according to~\eqref{eq:deltaContainsT}.\qedhere
  \end{enumerate}
\end{proof}

We will now show that on a node of a concrete execution path, the correlated-calls analysis does not map receivers to $\top_T$. In other words, the analysis never considers nodes of a concrete execution path unreachable.

\begin{lemma}\label{lem:sound3}
  Let $p=[\startmain,\,\dots,\,n]$ be a concrete execution path, $r\in R$ a receiver, and $d\in D$ a data-flow fact. Then
  \begin{equation}
    d\in M_F(p)(\varnothing)\iff\mpd(r)\ne\top_T\,.
  \end{equation}\mtodo{Prove other direction}
\end{lemma}
\begin{proof}
  By induction on the length of the execution trace.
  
  \textit{Basis:} 
    Let $p=[\startmain]$. Since the only realizable path corresponding to $p$ is $[(\startmain,\,\Lambda)]$, there is no fact $d\in D$ such that $d\in M_F(p)(\varnothing)$, and the claim follows immediately.
  
  \textit{Induction hypothesis:} 
  Let $p=[\startmain,\,\dots,\,n_{k-1}]$. Let $\uptau$ be the set of types to which $r$ is mapped by $\mpdkm$:
  \begin{equation}
    \uptau=\mpdkm(r)\,.
  \end{equation}
  Assume the Lemma holds for a concrete execution path $$p=[\startmain,\,n_1,\,\dots,\,n_{k-1}]\,,$$ i.e. $\uptau\ne\top_T$ for an arbitrary $r\in R$ and $d_{k-1}\in D$.
  
  \textit{Induction step:}
    Let $p'=[\startmain,\,n_1,\,\dots,n_{k-1},\,n_k]$ be a concrete execution path.
    
    Let $e_k=((n_{k-1},\,d_{k-1}),\,(n_k,\,d))$. As shown in~\eqref{eq:musubset},
    \begin{align*}
      \mppd(r)
        &\supseteq\ccedgefn R(e_k)(\mpdkm)(r)\,.
    \end{align*}

    From Definition~\ref{def:edgefn}, we can see that unless $e_k$ is a call-start edge or an end-return edge, the result follows from the induction hypothesis.
    More formally, if $e_k$ is not a call-start or end-return edge, then for all $m\in\lcc_R$,
    \[
      \ccedgefn R(e_k)(m)\sqsubseteq m\,.
    \]
     The edge function corresponding to the call-start and end-return edges is the only place in which the set of types that a receiver maps to can be reduced.
    
    Assume that $e_k$ is a end-return edge with a call on the receiver $r'\in R$ with a signature $s$ to a function $f$.
    \begin{align*}
      \ccedgefn R&(e_k)(\mpdkm)(r)\\
      &=\left(\lambda m\,.\,m[r'\to m(r)\cap\tau(s,\,f)][r_1\to\bot_T]\dots[r_l\to\bot_T]\right)(\mpdkm)(r)\\
      &=\left(\mpdkm[r'\to \uptau\cap\tau(s,\,f)][r_1\to\bot_T]\dots[r_l\to\bot_T]\right)(r)\,,
    \end{align*}
  where $r_1,\,\dots,r_l\in R$ are the local variables in the called method.
    
    If $r\in\{r_1,\,\dots,\,r_l\}$, then $\efek=\bot_T\ni t$\footnote{In the case of a recursive call, it is possible that both $r\in\{r_1,\,\dots,\,r_l\}$ and $r=r'$.
    In that case, the set to which $r$ will be mapped would be still ``overwritten'' by $\bot_T$.}.
    
    Otherwise, if $r=r'$, then $\efek=\uptau\cap\tau(s,\,f)$.

    According to Lemma~\ref{lem:sound1} and by the induction hypothesis, the runtime type $t$ of $r$ must be contained in $\mpdkm(r)=\uptau$. At the same time, by definition, $t$ is part of $\tau(s,\,f)$. Therefore, $t\in\uptau\cap\tau(s,\,f)\subseteq\efek$, which means that $\efek\ne\top_T$.
    
    The same reasoning applies to the case where $e_k$ is a call-start edge.
\end{proof}

For the following proofs, recall from Section~\ref{sec:ideToIfds} that the result of an IDE analysis maps a lattice element to each node in the exploded supergraph. Specifically, for an IDE problem $Q$, the result $\resultIDE(Q):\ N^*\to(D\to L)$ maps nodes of the supergraph to pairs of data-flow facts and lattice elements~\cite{sagiv1996precise}:
\begin{equation}\label{eq:ideresult}
  \resultIDE(Q)=\lambda n\,.\,\lambda d\,.\,\mvp{\textsf{Env}}(n,\,d))\,.
\end{equation}

We can now prove the Soundness Lemma.

\sound*
\begin{proof}
According to~\eqref{eq:ideresult}, we can rewrite~\eqref{eq:transCcDef} as
\begin{align*}\label{eq:transCcResultCc}
  \backCC(\resultIDE(\transCC_R(P)))(n)
  &=\{d'\,|\,\forall r\in R\,.\,\mvp{\textsf{Env}}(n,\,d'))(r)\ne\top_T\}\\
  &=\left\{d'\,|\,\forall r\in R\,.\bigsqcap_{q\in\textsf{VP}(n)}\menv(q)(\top_\textsf{Env})(d')(r)\ne\top_T\right\}\\
  &=\left\{d'\,|\,\forall r\in R\,.\bigsqcap_{q\in\textsf{VP}(n)}\mpddef q {d'}(r)\ne\top_T\right\}.
\end{align*}
  According to~Lemma~\ref{lem:sound3}, since $d\in M_F(p)(\varnothing)$, then for any $r\in R$, $\mpd(r)\ne\top_T$. 
  Since $\mpd(r)$ is a non-empty set that is contained in $\bigsqcap_{q\in\textsf{VP}(n)}\mpddef q {d}(r)$, it follows that $$\bigsqcap_{q\in\textsf{VP}(n)}\mpddef q {d}(r)\ne\top_T.$$ Therefore, $d\in \backCC(\resultIDE(\transCC_R(P)))(n)$.
\end{proof}

\precision*
\begin{proof}
    \otodo{It's Section 3.2.}
    \mtodo{Section 3.2 talks about the IFDS algorithm, but it doesn't explicitly say how the result of an IFDS analysis looks like and doesn't introduce the notation $\resultIFDS$.}
  Let $P$ be an IFDS problem. Recall from Section~\ref{sec:ideToIfds} that the result of an IFDS analysis $\resultIFDS(P)$ maps supergraph
  nodes $n\in N^*$ to sets of data-flow facts $\delta\in2^D$. Specifically,
  \begin{align*}
    \resultIFDS(P)&=\lambda n\,.\,\mvp F(n)\\
    &=\bigsqcap_{q\in\textsf{VP}(n)}M_F(q)(\top).
  \end{align*}
  At the same time,
  \begin{align*}
    \backCC\left(\resultIDE(\transCC_R(P))\right)(n)
    &=\left(\lambda n\,.\,\left\{d\,|\,\forall r\in R\,.\,\resultIDE(\transCC_R(P))(n)(d)(r)\ne\top_T\right\}\right)(n)\\
    &=\left\{d\,|\,\forall r\in R\,.\,\resultIDE(\transCC_R(P))(n)(d)(r)\ne\top_T\right\}\\
    &=\left\{d\,|\,\forall r\in R\,.\,\bigsqcap_{q\in\textsf{VP}(n)}\xi(q,\,d)(r)\ne\top_T\right\}.
  \end{align*}
  This means that for each $d\in\backCC(\resultIDE(\transCC R(P)))(n)$ and for each path $q\in\textsf{VP}(n)$, $\xi(q,\,d)\ne\top_T$. Therefore, according to~Lemma~\ref{lem:sound3}, for each $q$, $d\in M_F(q)(\top)$. It follows that $d\in\bigsqcap_{q\in\textsf{VP}(n)}M_F(q)(\top)=\resultIFDS(P)(n)$. 
  In other words, if $d\in\backCC(\resultIDE(\transCC R(P)))(n)$, then $d\in\resultIFDS(P)(n)$.
\end{proof}

\subsection*{Correlated Call Receivers}
We will now present the proof for Lemma~\ref{lem:onlycorrrec} which shows that in a correlated-calls analysis, it is enough to consider only correlated-call receivers~$\rcc$.

In this section, we will denote the set of realizable paths corresponding to a valid path $p$ and a fact $d$ as $\textsf{RP}(p,\,d)$.

First, we introduce a Lemma showing that the types to which a given receiver is mapped in the result of the algorithm is not affected by other receivers and the types to which they are mapped.

\begin{lemma}\label{lem:recindepedgefn}
  Let $P$ be an IFDS problem. Let $N^*$ be the supergraph for $P$, $D$ the set of data-flow facts, $n\in N^*$ a node, and $p=[\startmain,\,\dots,\,n]$ a path in the supergraph. Let $d\in D\cup\{\Lambda\}$.
  Then for any realizable path $p'\in\textsf{RP}(p,\,d)$, set $S\subseteq R$, and receiver $r\in S$,
  \begin{equation}
    \ccedgefn S(p')(\topcc)(r)=
    \ccedgefn{\{r\}}(p')(\topcc)(r)\,.
  \end{equation}
\end{lemma}
\begin{proof}
  By induction on the length of $p$.
  
  \textit{Basis:} $p'=[(\startmain,\,\Lambda)]$. Then $\ccedgefn S(p')=\id=\ccedgefn{\{r\}}(p')$, and the Lemma follows directly.
  
  \textit{Induction hypothesis:} Suppose that for a path $q=[(\startmain,\,\Lambda),\,\dots,\,(n_{k-1},\,d_{k-1})]$, where $q\in\textsf{RP}(n,\,d)$, the Lemma holds, i.e. both edge functions map $r$ to the same set of types $\uptau$:
  \begin{align*}
    \uptau
    &=\ccedgefn S(q)(\topcc)(r)\\
    &=\ccedgefn{\{r\}}(q)(\topcc)(r)\,.
  \end{align*}
  
  \textit{Induction step:} Let $q'=[(\startmain,\,\Lambda),\,\dots,\,(n_{k-1},\,d_{k-1}),\,(n_k,\,d_k)]$ and the edge $e_k=((n_{k-1},\,d_{k-1}),\,(n_k,\,d_k))$.
  
  Observe that for any set $U\subseteq R$ such that $r\in U$,
  \begin{align}\label{eq:edgefnU}
    \ccedgefn U(q')(\topcc)(r)
    &=\ccedgefn U(e_k)(\ccedgefn U(q)(\topcc))(r)\,.
  \end{align}
  
  We can see from~\eqref{eq:edgefnThroughDelta} that there are two cases.  
  
  If $d_{k-1}=d_k=\Lambda$, $\ccedgefn S(e_k)=\id=\ccedgefn{\{r\}}(e_k)$, and, due to~\eqref{eq:edgefnU},
  \begin{align*}
    \ccedgefn S(q')(\topcc)(r)&=\uptau\\
    &=\ccedgefn{\{r\}}(q')(\topcc)(r)\,.
  \end{align*}
  
  Otherwise, there are four sub-cases.
  \begin{enumerate}
    \item $e_k$ is a call-start edge, $r'.c()$ is the call site at $n_{k-1}$ with signature $s$, $f$ is the called procedure, and $r'\in U$.
    Then
    \[
      \ccedgefn U(e_k)=\lambda m\,.\,\delta(m)[r'\to\delta(m)(r)\cap\tau(s,\,f)]\,.
    \]
    There are two sub-cases.
    \begin{enumerate}
      \item\label{item:callstartreceq} If $r=r'$, then, according to~\eqref{eq:edgefnU}, the resulting set of types 
        \[
          \ccedgefn U(q')(\topcc)(r)=\delta(\ccedgefn U(q)(\topcc))(r)\cap\tau(s,\,f).
        \]
        If $d_{k-1}=\Lambda$, then $\delta(\ccedgefn U(q)(\topcc))(r)=\botcc(r)=\bot_T$. If $d_{k-1}\ne\Lambda$, then $\delta(\ccedgefn U(q)(\topcc))(r)=\ccedgefn U(q)(\topcc)(r)=\uptau$. The set $\tau(s,\,f)$ is the same for either case.
    
        Therefore, the value of $\ccedgefn U(q')(\topcc)(r)$ has the same result regardless of $U$,
        which means that $\ccedgefn S(q')(\topcc)(r)=\ccedgefn{\{r\}}(q')(\topcc)(r)$, and the Lemma holds.
      \item\label{item:callstartrecneq} If $r\ne r'$, then
        \begin{equation}
          \ccedgefn U(q')(\topcc)(r)=\delta(\ccedgefn U(q)(\topcc))(r)\,,
        \end{equation}
        which, as we have seen in~Case~\eqref{item:callstartreceq}, does not depend on~$U$, and the Lemma holds.
    \end{enumerate}
    \item $e_k$ is an end-return edge, $r_1,\,\dots,\,r_l\in U$ are the local variables in the callee method, $r'.c()$ is the call corresponding to the return node at $n_k$, $f$ is the called method with signature $s$, and $r'\in U$.
    Then
    \[
      \ccedgefn U(e_k)=\lambda m\,.\,\delta(m)
      [r'\to\delta(m)(r)\cap\tau(s,\,f)]
      [r_1\to\bot_T]\ldots[r_l\to\bot_T]\,.
    \]
    There are three sub-cases.
    \begin{enumerate}
      \item\label{item:localvarrec} If $r\in\{r_1,\,\dots,\,r_l\}$, then regardless of the value of $U$,
      \[
        \ccedgefn U(q')(\topcc)(r)=\bot_T\,,
      \]
      and the Lemma holds.
      \item Otherwise, if $r=r'$, the case is analogous to Case~\eqref{item:callstartreceq}.
      \item If $r\notin\{r',\,r_1,\,\dots,\,r_l\}$, then see Case~\eqref{item:callstartrecneq}.
    \end{enumerate}
    \item $n_{k-1}$ contains an assignment for $r'\in U$. Then
    \[
      \ccedgefn U(e_k)=\lambda m\,.\,\delta(m)[r'\to\bot_T]\,.
    \]
    If $r=r'$, see Case~\eqref{item:localvarrec}. If $r\ne r'$, see Case~\eqref{item:callstartrecneq}.
    \item Otherwise,
    \[
      \ccedgefn U(e_k)=\lambda m\,.\,\delta(m)\,,
    \]
    and the case is analogous to Case~\eqref{item:callstartrecneq}.\qedhere
  \end{enumerate}
\end{proof}

The following Lemma shows that the correlated-calls analysis computes the results for each receiver independently, or separately. To compute the set of types to which a receiver~$r$ is mapped at each exploded-graph node, we can exclude all other receivers in the program from the analysis (recall from~\eqref{eq:edgefndef} that the set of receivers that are considered in the analysis is specified by the set $S$ in a correlated-calls transformation $\transCC_S$). Therefore, for a given receiver $r$, the results of a $\transCC_S$- and a $\transCC_{\{r\}}$-analysis are the same.

\begin{lemma}\label{lem:recindep} Let $P$ be an IFDS problem. Let $N^*$ be the supergraph for $P$, $D$ the set of data-flow facts, and $S\subseteq R$ a set of receivers.
  Then for any $n\in N^*$, $d\in D$, and receiver $r\in S$,
  \begin{equation}
    \resultIDE\left(\transCC_S(P)\right)(n)(d)(r)=
    \resultIDE(\transCC_{\{r\}}(P))(n)(d)(r)\,.
  \end{equation}
\end{lemma}
\begin{proof} Recall from Section~\ref{sec:bgide} that
\begin{equation}\label{eq:mvpdef}
  \mvp{\textsf{Env}}(n)=\bigsqcap_{q\in\ivp(n)}M_\textsf{Env}(q)(\top)
\end{equation}

  According to~\eqref{eq:ideresult}, \eqref{eq:mvpdef}, and~\eqref{eq:envTransToEdgeFnEdge},
  \begin{align}
    \resultIDE\left(\transCC_S(P)\right)(n)(d)(r)
    &=\mvp{\textsf{Env}}(n,\,d)(r)\notag\\
    &=\left(\bigsqcap_{q\in\ivp(n)}M_\textsf{Env}(q)(\top_\textsf{Env})(d)\right)(r)\notag\\
    &=\left(\bigsqcap_{q\in\ivp(n)}\bigsqcap_{q'\in\mathsf{RP}(q,\,d)}\ccedgefn S(q')(\topcc)\right)(r)\notag\\
    &=\bigcup_{q\in\ivp(n)}\bigcup_{q'\in\mathsf{RP}(q,\,d)}\ccedgefn S(q')(\topcc)(r)\,.\label{eq:resultThroughEdgefn}
  \end{align}
  Then from Lemma~\ref{lem:recindepedgefn},
  \begin{align*}
    \resultIDE\left(\transCC_S(P)\right)(n)(d)(r)
    &=\bigcup_{q\in\ivp(n)}\bigcup_{q'\in\mathsf{RP}(q,\,d)}\ccedgefn{\{r\}}(q')(\topcc)(r)\\
    &=\resultIDE\left(\transCC_{\{r\}}(P)\right)(n)(d)(r)\,.\qedhere
  \end{align*}
\end{proof}

The next lemma shows that the set of types to which a receiver is mapped in a correlated-calls lattice element can be represented as an intersection of static-type function applications $\tau(s_i,\,f_i)$.
\begin{lemma}\label{lem:edgefnThroughTaus}
  For an IFDS problem $P$, a node $n\in N^*$, and fact $d\in D$, let $p\in\mathsf{RP}(n,\,d)$ be a realizable path and $r\in R$ a receiver. Then there exists a non-negative number $\gamma$ of calls on the receiver $r$ with signatures $s_\gamma$ to the functions~$f_\gamma$, for which
  \[
    \ccedgefn{\{r\}}(p)(\topcc)(r)=
      \bigcap_{\gamma\ge0}\tau(s_\gamma,\,f_\gamma)\,.
  \]
\end{lemma}
\begin{proof}
  Let $p$ have the following form\footnote{It can be shown from the definition of a pointwise representation in~Sagiv et al.~\cite{sagiv1996precise} that in a realizable path, there is never an edge from a fact of the set $D$ to a $\Lambda$ fact. Therefore, we can represent $p$ as a sequence of nodes that has a prefix of $\Lambda$-fact nodes, after which all nodes are non-$\Lambda$ facts.}:
  \[
    p=[(\startmain,\,\Lambda),\,(n_1,\,\Lambda),\,\dots,\,(n_k,\,\Lambda),
       (n_{k+1},\,d_{k+1}),\,\dots,\,(n_{k+l},\,d_{k+l})]\,,
  \]
  where $l\ge1$ and the facts for all nodes up to $n_k$ are equal to $\Lambda$ and $d_{k+i}\in D$ for $0<i\le l$.
  
  As previously, for all $i$, we will denote the edge $(n_i,\,n_{i+1})$ by $e_i$.  
  
  From~\eqref{eq:edgefndef} we can infer that
  \[
    \ccedgefn{\{r\}}(p)=
    \ccedgefn{\{r\}}(e_{k+l})
    \circ\ldots
    \circ\ccedgefn{\{r\}}(e_{k+2})
    \circ(\lambda m\,.\,\beta)
    \circ\id
    \circ\ldots
    \circ\id\,,
  \]
  where
  \[
    \beta=
    \begin{cases}
      \botcc[r\to\tau(s,\,f)]&\text{if $(n_k,\,n_{k+1})$ is a call-start or end-return edge, and}\\&\text{the call site $r.c()$ with signature $s$ to the function}\\
      &\text{$f$ corresponds to the call-start or end-return edge,}\\
      \botcc&\text{otherwise\footnotemark.}
    \end{cases}
  \]
  \footnotetext{Since $d_k=\Lambda$ and $d_{k+1}\ne\Lambda$, the micro function for the edge $e_{k+1} $ is equal to $\lambda m\,.\,\varepsilon_{\{r\}}(e_{k+1})(\botcc)$. From the definition of $\varepsilon_S$~\eqref{eq:varepsilon} we can see that the only case where $\varepsilon_{\{r\}}(e_{k+1})(m)$ would not be equal to $\botcc$ is when $e_{k+1}$ is call-start or end-return edge.}
  
  Therefore,
  \begin{align}\label{eq:edgefnbeta}
    \ccedgefn{\{r\}}(p)(\topcc)
      &=\left(\ccedgefn{\{r\}}(e_{k+l})\circ\ldots\circ
        \ccedgefn{\{r\}}(e_{k+2})\right)((\lambda m\,.\,\beta)(\topcc))\notag\\
      &=\left(\ccedgefn{\{r\}}(e_{k+l})\circ\ldots\circ
        \ccedgefn{\{r\}}(e_{k+2})\circ\id\right)(\beta)\,.
  \end{align}

We can now prove the lemma by induction on $l$.

\textit{Basis:}
If $l=1$, then $\ccedgefn{\{r\}}(p)(\topcc)=\id(\beta)=\beta$.
There are two cases.

If $\beta=\botcc$, then 
\begin{align*}
  \ccedgefn{\{r\}}(p)(\topcc)(r)&=\beta(r)\\&=\bot_T\,,
\end{align*} and $\gamma=0$.

If $\beta=\botcc[r\to\tau(s,\,f)]$, then 
\[
  \ccedgefn{\{r\}}(p)(\topcc)(r)=\tau(s,\,f)\,,
\]
and $\gamma=1$.

\textit{Induction hypothesis:}
Assume that for a path $p=[(\startmain,\,\Lambda),\,\dots,\,(n_{k+l},\,d_{k+l})]$, the Lemma holds for $\gamma=N$, where $N\ge0$.

\textit{Induction step:}
Let $p'=[(\startmain,\,\Lambda),\,\dots,\,(n_{k+l},\,d_{k+l}),\,(n_{k+l+1},\,d_{k+l+1})]$.

Recall that
\begin{align*}
  \ccedgefn{\{r\}}(p')(\topcc)(r)
  &=\ccedgefn{\{r\}}(e_{k+l+1})\left(\ccedgefn{\{r\}}(p)(\topcc)\right)(r)\,.
\end{align*}

From~\eqref{eq:varepsilon} we can see that unless $e_{k+l+1}$ is a call-start or end-return edge corresponding to a call on the receiver $r$, then $\ccedgefn{\{r\}}(e_{k+l+1})(r)$ must be equal to either $\bot_T$ or $m(r)$, where $m=\ccedgefn{\{r\}}(p)(\topcc)$. 

If $\ccedgefn{\{r\}}(e_{k+l+1})(r)=\bot_T$, then the Lemma holds for $\gamma=0$. 

Otherwise,
\begin{align*}
  \ccedgefn{\{r\}}(e_{k+l+1})(\topcc)(r)&=\ccedgefn{\{r\}}(p)(\topcc)(r)\\
  &=\bigcap_{N}\tau(s_N,\,f_N)\,,
\end{align*}
and therefore $\gamma=N$.

Suppose that $e_{k+l+1}$ is a call-start edge with a call on the receiver $r$ with signature $s$ to a function $g$. Then, according to~\eqref{eq:varepsilon}, 
\[
  \ccedgefn{\{r\}}(e_{k+l+1})=\lambda m\,.\,m[r\to m(r)\cap\tau(s,\,g)]\,.
\]
Therefore,
\begin{align*}
  \ccedgefn{\{r\}}&(p')(\topcc)(r)\\
  &=\lambda m\,.\,m[r\to m(r)\cap\tau(s,\,g)]\left(\ccedgefn{\{r\}}(p)(\topcc)\right)(r)\\
  &=\ccedgefn{\{r\}}(p)(\topcc)(r)\cap\tau(s,\,g)\\
  &=\left(\bigcap_{N}\tau(s_N,\,f_N)\right)\cap\tau(s,\,g)\,,
\end{align*}
and the Lemma holds for $\gamma=N+1$.

The case where $e_{k+l+1}$ is an end-return edge is analogous to the previous case.
\end{proof}

We now show that a receiver will be only mapped to $\topcc$ if it is the receiver of a correlated call.

\begin{lemma}\label{lem:ccrectop}
   For an IFDS problem $P$, let $n\in N^*$ be a node, and $d\in D$ a dataflow fact such that there exists a realizable path $p\in\textsf{RP}(n,\,d)$. Let $T$ be the set of all types in the program.
   If there exists a receiver $r\in R$ such that
  \[
    \ccedgefn{\{r\}}(p)(\topcc)(r)=\top_T\,,                       
  \]
  then $r\in \rcc$.
\end{lemma}
\begin{proof} 
Observe that if there is a supergraph path from a method call with signature $s$ to the start of $f$, then the set $\tau(s,\,f)$ is always non-empty.
  Let $r.c()$ be a call on a receiver $r\in R$ with a method signature $s$ to a function $f$.
  If the call site is monomorphic, then $\tau(s,\,f)$ contains all types $T'\subseteq T$ that are compatible with the static type of $r$.
  If the call site is polymorphic, then $\tau(s,\,f)\subset T'$, since some types $t\in T'$ cause dispatch to a method other than~$f$.
 
 According to Lemma~\ref{lem:edgefnThroughTaus},
 \[
   \ccedgefn{\{r\}}(p)(\topcc)(r)=
      \bigcap_{\gamma\ge0}\tau(s_\gamma,\,f_\gamma).
 \]
 
Let $\tau_i=\tau(s_i,\,f_i)$. 
 For a given $k$, let $r.m_k()$ be the call site corresponding to $\tau_k$, and $T'$ the set of types compatible with the static type of $r$. Then the following is true:
 \begin{itemize}
   \item $\tau_k\ne\top_T$;
   \item if $\tau_k=T'$ then the corresponding call site is monomorphic;
   \item if $\tau_k\subset T'$ then the call site is polymorphic.
 \end{itemize}  
 
 From the conditions of the Lemma, 
 \begin{equation}
   \bigcap_{\gamma\ge 0}\tau_\gamma=\top_T\,.
 \end{equation} 
 
 If all $\tau_k=T'$, then $\bigcap_{\gamma\ge 0}\tau_\gamma$ is also equal to $T'$. Since $T'\ne\top_T$, this is a contradiction. 
 
 If exactly one $\tau_k\subset T'$ and the rest are equal to $T'$, then $\bigcap_{\gamma\ge 0}\tau_\gamma$ is equal to $\tau_k$, which cannot be $\top_T$ either.
 
 Therefore, there are at least two sets, $\tau_i$ and $\tau_j$, which are strict subsets of $T'$. Since both $\tau_i$ and $\tau_j$ are non-empty and their intersection equals $\top_T$, $\tau_i$ and $\tau_j$ must be disjoint. If $\tau_i$ and $\tau_j$ are disjoint, they must correspond to different call sites.
 
 In other words, there are at least two calls on the same receiver for which the static-type function is a strict subset of the set of types compatible with a given receiver $r$. It follows that both calls have to be polymorphic. Therefore,~${r\in\rcc}$.
\end{proof}

We will now show that if a receiver ever gets mapped to top, then it is a correlated-calls receiver.

\begin{lemma}\label{lem:ccrectop}
   For an IFDS problem $P$, let $n\in N^*$ be a node, and $d\in D$ a dataflow fact such there exists a realizable path $p\in\textsf{RP}(n,\,d)$.
   Then, if there exists a receiver $r\in R$, such that
  \[
    \resultIDE\left(\transCC_{\{r\}}(P)\right)(n)(d)(r)=\top_T\,,
  \]
  then $r\in \rcc$.
\end{lemma}
\begin{proof}
  As shown in~\eqref{eq:resultThroughEdgefn},
  \[
    \resultIDE\left(\transCC_{\{r\}}(P)\right)(n)(d)(r)=\bigcup_{q\in\ivp(n)}\bigcup_{q'\in\mathsf{RP}(q,\,d)}\ccedgefn{\{r\}}(q')(\topcc)(r)\,.
  \]
  Since the latter is equal to $\top_T$, it follows that for each realizable path $p'$ to node $n$, $\ccedgefn{\{r\}}(p')(\top)(r)=\top_T$. According to~Lemma~\ref{lem:ccrectop}, this is only possible if $r\in\rcc$.
\end{proof}

Finally, we present the proof for Lemma~\ref{lem:onlycorrrec} which states that a correlated-calls analysis that considers all receivers computes the same result as an analysis that considers only correlated-call receivers.

\begin{proof}[\textbf{Proof of Lemma~\ref{lem:onlycorrrec}}]
  By definition of $\backCC$,
  \begin{align*}
    \backCC(\resultIDE\left(\transCC_R(P)\right))
    &=\{d\,|\,\resultIDE(\transCC_R(P))(n)(d)=\ell,\,\forall r\,.\,\ell(r)\ne\top_T\}\\
    &=\{d\,|\,\forall r\in R,\,\resultIDE(\transCC_R(P))(n)(d)(r)\ne\top_T\}.
  \end{align*}
  
  \commentout{
  \begin{align*}
    \backCC(\resultIDE\left(\transCC_R(P)\right))
    &=\left\{(n,\,D_n^\Subset(\resultIDE(\transCC_R(P))))\ |\ n\in N^*\right\}.
  \end{align*}
  According to~\eqref{eq:dnq} and~Lemma~\ref{lem:recindep}, for a given $n\in N^*$,
  \begin{align*}
    D_n^\Subset&(\resultIDE(\transCC_R(P))))\\
    &=\left\{d\ |\ d\in\mvp F(n)\,\wedge\,\forall r\in R:\ \left\{(r,\,\resultIDE(\transCC_{\{r\}}(P))(n,\,d)(r))\ |\ r\in R\right\}(r)\ne\top_T\right\}\\
    &=\left\{d\ |\ d\in\mvp F(n)\,\wedge\,\forall\bm{r\in R}:\ \resultIDE(\transCC_{\{r\}}(P))(n,\,d)(r)\ne\top_T\right\}.
  \end{align*}
  }
Since, according to Lemma~\ref{lem:ccrectop}, $\resultIDE(\transCC_{\{r\}}(P))(n)(d)(r)$ can only be equal to $\top_T$ when $r\in\rcc$, we can conclude that
  \begin{align*}
    \backCC(\resultIDE\left(\transCC_R(P)\right))
    &=\{d\,|\,\bm{\forall r\in\rcc},\,\resultIDE(\transCC_{\bm\rcc}(P))(n)(d)(r)\ne\top_T\}\\
    &=\backCC(\resultIDE\left(\transCC_\rcc(P)\right)).
    \qedhere
  \end{align*}
\end{proof}
