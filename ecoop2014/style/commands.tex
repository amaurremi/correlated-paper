% brackets for denotations
\newcommand{\denote}[1]{\left\llbracket#1\right\rrbracket}

%\renewcommand{\theFancyVerbLine}{\sffamily \textcolor[rgb]{0.5,0.5,0.5}{\scriptsize \oldstylenums{\arabic{FancyVerbLine}}}}

\newcommand{\inputMinted}[2]{\lstinputlisting[language=#1]{code/#2}}
\newcommand{\inputMintedNoLNs}[2]{\lstinputlisting[language=#1,numbers=none]{code/#2}}

%\newcommand{\inputMinted}[2]{\par\inputminted[fontsize=%\footnotesize,linenos=true,bgcolor=white,numberblanklines=false,mathescape,samepage=true]{#1}{code/#2}\\ \ \\}
\definecolor{bg}{rgb}{0.95,0.95,0.95}
%%\usemintedstyle{manni}     % comci comca
%%\usemintedstyle{borland}   % pretty strict style
%%\usemintedstyle{vim}       % together with fruity and native only with dark bg!
%\usemintedstyle{bw}         % strict, too

\counterwithout{footnote}{chapter}

\newlength{\dist}
\setlength{\dist}{1cm}

\newlength{\edge}
\setlength{\edge}{1.1cm}

\newlength{\longedge}
\setlength{\longedge}{1.7cm}

\newlength{\nd}
\setlength{\nd}{.6cm}

\newlength{\ndd}
\setlength{\ndd}{.5cm}

%\newtheorem{theorem}{Theorem}[section]
%\newtheorem{sublemma}[theorem]{Sub-Lemma}
%\newtheorem{lemma}[theorem]{Lemma}
%\newtheorem{proposition}[theorem]{Proposition}
%\newtheorem{corollary}[theorem]{Corollary}
%\newtheorem{definition}[theorem]{Definition}
%\newtheorem{example}[theorem]{Example}
\let\oldexample\example
\renewcommand{\example}{\oldexample\normalfont}

\tikzset{>=latex,on grid, auto}

\tikzset{
  ashadow/.style={
    opacity=.25,
    shadow xshift=0.07,
    shadow yshift=-0.07
  }
}
\tikzstyle{supergraph}=[
  fill=greyblue,
  rounded corners,
  font=\small,
  align=right,
  text=charcoal,
  drop shadow={ashadow, color=greyblue}
]

\newcommand{\backEq}{\mathcal U^\equiv}
\newcommand{\backCC}{\mathcal U^\Subset}
\newcommand{\transEq}{\mathcal T^\equiv}
\newcommand{\transCC}{\mathcal T^\Subset}

\newcommand{\rcc}{{R^\Subset}}
\newcommand{\lcc}{L^\Subset}

\newcommand{\result}[1]{\mathcal R_{\text{#1}}}
\newcommand{\resultIFDS}{\result{IFDS}}
\newcommand{\resultIDE}{\result{IDE}}

\newcommand{\updmap}[1]{\textsf{update}_{#1}}
\newcommand{\updfun}[1]{\textsf{update}_{#1,\,r}}

\newcommand{\setpair}[2]{\left\langle #1,\,#2\right\rangle}

\newcommand{\highlight}[2]{\setlength{\fboxsep}{1pt}\colorbox{#2}{$\displaystyle #1$}}

\newcommand{\ione}{I_{f_1,\,r}}
\newcommand{\itwo}{I_{f_2,\,r}}
\newcommand{\uone}{U_{f_1,\,r}}
\newcommand{\utwo}{U_{f_2,\,r}}

\newcommand{\mvp}[1]{\textsf{MVP}_{#1}}
\newcommand{\ivp}{\textsf{VP}}

\newcommand{\startmain}{\textsf{start}_\texttt{main}}

\newcommand{\edgefn}{\textsf{EdgeFn}}
\newcommand{\ccedgefn}[1]{\edgefn^\Subset_{#1}}

\newcommand{\menv}{M_\textsf{Env}}

\newcommand{\topcc}{\top_\Subset}
\newcommand{\botcc}{\bot_\Subset}

\newcommand{\mpddef}[2]{\xi(#1,\,#2)}
\newcommand{\mpd}{\mpddef p d}
\newcommand{\mppd}{\mpddef{p'} d}
\newcommand{\mpdkm}{\mpddef{p}{d_{k-1}}}
\newcommand{\efek}{\ccedgefn R(e_k)(\mpdkm)(r)}

\newcommand{\id}{\textsf{id}}

\newcommand{\src}[1]{\mathsf{src}(#1)}
\newcommand{\target}[1]{\mathsf{end}(#1)}

\newcommand{\comment}[3]{\todo[size=\tiny,color=#3]{#1 \textbf{(#2)}}}
\newcommand{\mtodo}[1]{\comment{#1}{M}{mcomment}}
\newcommand{\ftodo}[1]{\comment{#1}{F}{fcomment}}
\newcommand{\otodo}[1]{\comment{#1}{O}{ocomment}}

\makeatletter
\newcommand*{\savecbcolor}{%
  \let\saved@cb@current@color\cb@current@color
}
\newcommand*{\restorecbcolor}{%
  \global\let\cb@current@color\saved@cb@current@color
}
\makeatother

\newenvironment{mdelete}{%
  \savecbcolor
  \begin{changebar}%
  \cbcolor{mcomment}%
  \addtolength{\changebarsep}{2mm}%
  \setlength{\changebarwidth}{2mm}%
}{
  \end{changebar}%
  \restorecbcolor
}

\newenvironment{odelete}{%
  \savecbcolor
  \begin{changebar}%
  \cbcolor{ocomment}%
  \addtolength{\changebarsep}{2mm}%
}{%
  \end{changebar}%
  \restorecbcolor
}

\newenvironment{fdelete}{%
  \savecbcolor
  \begin{changebar}%
  \cbcolor{fcomment}%
  \addtolength{\changebarsep}{2mm}%
}{%
  \end{changebar}%
  \restorecbcolor
}

\newcommand{\commentout}[1]{}

\newcommand{\TODO}[1]{\textcolor{darkred}{\textbf{$\blacktriangleright$#1$\blacktriangleleft$}}}

\newcommand{\code}[1]{\textsf{#1}}  
\lstdefinelanguage{scala}[]{Java}{
   morekeywords={trait,def,object,with,override,val,type,var} 
}

\lstdefinestyle{Eclipse}{
  xleftmargin=0pt,
  language = scala,
  basicstyle=\sffamily\small,
  stringstyle=\color{sh_string},
  keywordstyle = \color{sh_keyword}\bfseries,  %}
  lineskip=-0.0em,
  commentstyle=\color{sh_comment}\itshape,  
  escapeinside={/*@}{@*/},
  numbersep=5pt,
  captionpos=b,
  xleftmargin=0.4cm, xrightmargin=0.5cm,
   morekeywords={invokestatic,invokeinterface,invokevirtual,invokespecial},
}


\lstset{
  showspaces=false,showtabs=false,tabsize=2,columns=flexible,
  morekeywords={trait,def,object,with},
  language={scala},
  style=Eclipse,
  numbers=left,
  numberstyle=\scriptsize\color{CommentColor},
  firstnumber=last,
  showstringspaces=true
}  
